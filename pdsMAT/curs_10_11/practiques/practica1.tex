\documentclass{article}
\usepackage{enumerate}

\title{Pr\`actiques Processament Digital del Senyal.
Pr\`actica 1. Processament de senyals 1D }
\date{}

\begin{document}

\maketitle

\noindent \textbf{Exercici 1}.\label{ex1}
\newline

Conceptes b\`asics.
\begin{enumerate}[a)]
\item \label{1a} Representar el senyal {\it ftest1} entre $0$ i
$62$ per a diferents valors de precissi\'o (utilitzau la funci\'o
{\it plot})

\item \label{1b} Mostrejar el senyal amb diferents per\'\i odes de
mostreig (provau amb $T=0.5, 1, 2, 3$ i $4$) i representar el
resultat amb la funci\'o {\it stem}.

\item \label{1c} Escriure una funci\'o de Matlab que permeti fer
la reconstrucci\'o d'un senyal continu a partir d'un senyal
discret utilitzant el Teorema de Shannon.

\item \label{1d} Provar la funcci\'o anterior per a la
reconstrucci\'o de les diferents versions mostrejades obtingudes a
l'apartat \ref{1b}). Comentar els resultats. Qu\`e passa si
reconstru\"\i m el senyal amb un per\'iode de mostreig diferent
del que li co\-rres\-pon?

\item \label{1e} Escriure una funci\'o de Matlab que calculi la
Transformada de Fourier d'un senyal mostrejat.

\item \label{1f} Calcular les Transformades de Fourier dels
senyals mostrejats a l'apartat \ref{1b}) i relacionar-les amb els
resultats obtinguts a l'apartat \ref{1d}).

\item \label{1g} Exercici te\`oric: relacionar la FFT amb la
Transformada de Fourier programada a l'apartat \ref{1e}).

\item \label{1h} Calcular la FFT (funci\'o {\it fft} de Matlab)
dels senyals obtinguts a l'apartat \ref{1b}) i comentar els
resultats.
\end{enumerate}

\vskip 0.5 cm

\noindent \textbf{Exercici 2}. \label{ex2}
\newline

An\`alisi de funcions peri\`odiques.
\newline

El senyal {\it ftest2} \'es un senyal peri\`odic de per\'\i ode
$4$. En aquest exercici es demana repetir els apartats \ref{1b}),
\ref{1d}) i \ref{1h}) per a la funci\'o {\it ftest2} i comentar
quines consideracions addicionals s'han de fer en analitzar un
senyal peri\`odic.

\vskip 0.5 cm

\noindent \textbf{Exercici 3}. \label{ex3}
\newline

An\`alisi de senyals de veu i filtratge.

\begin{enumerate}[a)]
\item \label{3a} Reproduir el senyal de veu `awake.wav' des d'una
finestra de Windows.

\item \label{3b} Obrir el senyal des de Matlab emprant la funci\'o
{\it wavread}.

\item \label{3c} Representar el senyal i calcular la seva FFT.
Representar-la.

\item \label{3d} Reconstruir el senyal continu original a partir
de la f\`ormula de reconstrucci\'o de Shannon.

\item \label{3e} Mostrejar el senyal reconstru\"\i t amb diferents
per\'\i odes de mostreig, guardar i reproduir el resultat.
Comentar els resultats.

\item \label{3f} Utilitzar la funci\'o {\it wavwrite} per guardar
el senyal amb per\'iodes de mostreig diferents de l'original i
reproduir els resultats.

\item \label{3g} Aplicar diferents filtres freq\"uencials al
senyal (passa-baix, passa-alt, passa-banda) i reproduir els
senyals resultants.

\item \label{3h} Filtratge per convoluci\'o: filtre de mitja.
Convolucionar el senyal original amb el senyal $u=[\frac{1}{3}
\frac{1}{3} \frac{1}{3}]$ (utilitzar la funci\'o {\it conv} de
Matlab). Relacionar l'operaci\'o de convoluci\'o amb la FFT dels
senyals convolucionats. Escriure una funci\'o de Matlab que
apliqui $n$ vegades el filtre de mitja al senyal original.
Reproduir els resultats del filtratge.

\item \label{3i} Filtratge no lineal: filtre de mediana. Escriure
una funci\'o de Matlab que calculi el filtratge de mediana d'un
senyal. Aplicar la funci\'o al senyal de veu original per a
diferents valors dels par`ametres i reproduir els resultats.
\end{enumerate}

\end{document}
