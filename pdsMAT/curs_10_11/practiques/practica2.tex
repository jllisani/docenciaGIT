\documentclass{article}
\usepackage{enumerate}
\usepackage{amsfonts, amscd, amsmath, amssymb}

\title{Pr\`actiques de Processament Digital del
Senyal\newline\newline
Pr\`actica 2. \\
Exercicis b\`asics de processament d'imatges }
\date{}

\begin{document}

\maketitle

\noindent \textbf{Exercici 1}.\label{ex1}
\newline

Zoom: augment d'escala.
\begin{enumerate}[a)]
\item \label{1a} Obriu i visualitzau la imatge `nina.png'
(utilitzau les funcions {\it imread} i {\it imshow} de Matlab).

\item \label{1b} Obriu i visualitzau la imatge `ull.png' qu\`e
\'es una subimatge de `nina.png'. En aquests exercici estudiarem
diferents m\`etodes per ampliar aquesta subimatge.

\item \label{1c} {\bf Zoom per duplicaci\'o de p\'\i xels}.
Constru\"iu una matriu de tamany doble a la imatge `ull.png' i
formada en afegir entre cada dues files (columnes) de la imatge
original una fila (columna) de zeros (imatge inter-zeros).
Utilitzau la funci\'o {\it mascara2D} per aplicar la seg\"uent
m\`ascara a la imatge inter-zeros:
$$
\left( \begin{array}{ccc} 1 & 1 & 0\\1 & 1 & 0\\0 & 0 &
0\end{array} \right)
$$
\noindent L'efecte d'aplicar aquesta m\`ascara \'es la
duplicaci\'o del p\'\i xel $(i,j)$ de la imatge original en les
posicions $(2i,2j)$, $(2i+1,2j)$, $(2i,2j+1)$ i $(2i+1,2j+1)$ de
la imatge inter-zeros.
\newline
Repetiu el proc\'es v\`aries vegades fins a obtenir una imatge de
tamany $208 \times 248$ (ampliaci\'o en un factor $8$ de la imatge
original). Comentau els resultats.

\item \label{1d} {\bf Zoom a partir de la reconstrucci\'o de
Shannon de la imatge}. Al igual que haviem fet per als senyals 1D,
\'es possible calcular la reconstrucci\'o de Shannon de la imatge
discreta. Si en fer la reconstrucci\'o empram un per\'\i ode de
mostreig major de l'original, aconseguirem canviar l'escala de la
imatge (recordau l'exercici 1d de la Pr\`actica 1).\newline
Utilitzau la funci\'o {\it fShannon2D} per reconstruir la imatge
amb per\'\i odes de mostreig horitzontals i verticals iguals a
$8$. Comentau els resultats.

\item \label{1e} {\bf Zoom per zero-padding}. La t\`ecnica de {\it
zero-padding} d\'ona uns resultats equivalents als de l'exercici
anterior i \'es la utilitzada en la pr\`actica per fer un canvi
d'escala basat en la Teoria de Shannon. Una manera d'aconseguir un
nombre major de mostres de la imatge continua original consisteix
en augmentar el suport de la seva transformada de Fourier discreta
i a continuaci\'o calcular la transformada inversa. La manera
d'augmentar el suport \'es afegir zeros a la part de l'espectre
corresponent a les altes freq\"u\`encies (d'aqu\'\i $ $ el nom de
zero-padding).
\newline
Calculau la transformada de Fourier de la imatge original ({\it
fft2}). Constru\"\i u una matriu del tamany desitjat ($208 \times
248$) formada per zeros i copiau els 4 cantons de la FFT original
en els cantons corresponents d'aquesta matriu: el rectangle
definit pels vertexos $(0,0)-(26,31)$ copiat en la posici\'o
$(0,0)-(26,31)$, el rectangle $(0,32)-(26,62)$ copiat en $(0,
218)-(26,248)$, etc.
\newline
Visualitzau, en escala logar\'\i
tmica, el m\`odul dels valors d'aquesta matriu. Calculau
l'antitransformada de Fourier ({\it ifft2}) de la matriu i
visualitzau la imatge resultant. Comentau els resultats.

\item \label{1f} {\bf Zoom per interpolaci\'o lineal}. Constru\"iu
la imatge inter-zeros de la imatge original.  Utilitzau la
funci\'o {\it mascara2D} per aplicar la seg\"uent m\`ascara a la
imatge inter-zeros:
$$
\left(
\begin{array}{ccc}
\frac{1}{4} & \frac{1}{2} & \frac{1}{4}\\
\frac{1}{2} & 1 & \frac{1}{2}\\
\frac{1}{4} &  \frac{1}{2} & \frac{1}{4}
\end{array}
\right)
$$
\noindent L'efecte d'aplicar aquesta m\`ascara \'es la
interpolaci\'o lineal del valor dels pixels diferents de zero en
la imatge inter-zeros. Per exemple, el nou valor del pixel que es
troba en la posici\'o $(i+1,j)$ ser\`a igual a la mitja dels
valors dels pixels que es troben en les posicions $(i,j)$ i
$(i+2,j)$.
\newline
Repetiu el process v\`aries vegades fins a obtenir una imatge de
tamany $208 \times 248$. Comentau els resultats.
\end{enumerate}

\vskip 0.5 cm

\noindent \textbf{Exercici 2}. \label{ex2}
\newline

Zoom invers: reducci\'o d'escala (sub-mostreig).
\begin{enumerate}[a)]
\item \label{2a} Obriu la imatge `camisa.png' i visualitzau-la.

\item \label{2b} Calculau la transformada de Fourier discreta de
la imatge i visualitzau el seu m\`odul en escala logar\'\i tmica.

\item \label{2c} Sub-mostrejau la imatge prenent un pixel de cada
dos, visualitzau la imatge resultant i calculau la seva
transformada de Fourier. Pensau que hi ha aliasing? A qu\`e \'es
degut?

\item \label{2d} Escriviu un programa de Matlab que implementi un
filtre antialiasing (filtre passa baix) adequat per al mostreig
efectuat en l'apartat anterior.

\item \label{2e} Aplicau el filtre a la transformada de Fourier de
la imatge original i obteniu la imatge filtrada calculant
l'antitransformada de Fourier. Visualitzau la imatge resultant.

\item \label{2f}  Repetiu els apartats \ref{2b} i \ref{2c} amb la
imatge obtinguda a l'apartat anterior. Comentau els resultats.
S'ha produ\"it el fen\`omen de Gibbs?

\end{enumerate}

\vskip 0.5 cm


\noindent \textbf{Exercici 3}. \label{ex3}
\newline

Filtres passa-baix i filtres de mitjana: aplicaci\'o a
l'eliminaci\'o de renou.
\begin{enumerate}[a)]
 \item \label{3a} Utilitzau el programa escrit a l'apartat \ref{2d}
 per fer un filtratge passa-baix de la imatge `nina.png' amb diferents
valors de la freq\"u\`encia de tall. Observau com els contorns de
les imatges filtrades se van difuminant i com cada vegada \'es
m\'es evident l'efecte de Gibbs.

\item \label{3b} \'Es possible construir un filtre passa-baix
mitjan\c{c}ant l'aplicaci\'o de la m\`ascara seg\"uent:
\[
A=\frac{1}{s^2}(a_{ij}), \qquad a_{ij}=1
\qquad 1 \leq i,j \leq s
\]
\noindent Utilitzau la funci\'o {\it mascara2D} per aplicar
aquesta m\`ascara, amb $s=7$, a la imatge `nina.png'.
\newline
\noindent Observau com la imatge queda difuminada. Calculau la
transformada de Fourier de la m\`ascara i comprovau com es tracta
d'un filtre passa-baix.

\item \label{3c} Utilitzau la instrucci\'o {\it imnoise} de Matlab
per afegir diferents tipus de renou a la imatge `nina.png'.

\item \label{3d} Filtrau les imatges amb renou amb un filtre
passa-baix. Comentau els resultats.

\item \label{3e} Repetiu l'apartat anterior utilitzant un filtre
de mitjana (emprau la funci\'o {\it medfilt2} de Matlab). Comentau
els resultats.

\end{enumerate}


\vskip 0.5 cm

\noindent \textbf{Exercici 4}. \label{ex4}
\newline

Filtres passa-alt i millora de contorns.
\begin{enumerate}[a)]
\item \label{4a} Obriu i visualitzau la imatge `ullZ.png'.
Observau com els seus contorns es troben difuminats.

\item \label{4b} \'Es possible construir un filtre passa-alt
mitjan\c{c}ant l'aplicaci\'o de la m\`ascara seg\"uent:
\[A=\frac{1}{s^2}(a_{ij}), \qquad a_{ij}=
\begin{cases}
s^2-1 & \mathrm{si} \quad  i=j=\frac{s}{2}\\
-1    & \mathrm{resta}
\end{cases}
\qquad 1 \leq i,j \leq s
\]
\noindent Utilitzau aquesta m\`ascara amb $s=15$ per filtrar la
imatge i visualitzau el resultat. Observau com els contorns de la
imatge han quedat resaltats. Calculau la transformada de Fourier
de la m\`ascara i verificau que es tracta, en efecte, d'un filtre
passa-alt.

\item \label{4c} Una aplicaci\'o dels filtres passa-alt \'es la
millora dels contorns de les imatges. Sumau, a la imatge inicial,
la sortida del filtre passa-alt ponderada amb diferents valors
(per exemple $0.5$, $1$ i $2$). Aquesta operaci\'o equival a
``afegir contorns" a la imatge original. Visualitzau i comentau
els resultats.
\end{enumerate}


\vskip 0.5 cm


\noindent \textbf{Exercici 5}. \label{ex5}
\newline

Informacio visual i transformada de Fourier.
\begin{enumerate}[a)]
\item \label{5a} Obriu i visualitzau les imatges `nina.png' i
`casa.png'.

\item \label{5b} Calculau el m\`odul i la fase de les
transformades de Fourier de les dues imatges i intercanviau-los:
la fase de la segona imatge amb el m\`odul de la primera i
vice-versa. Visualitzau les imatges resultants i comentau els
resultats.
\end{enumerate}



\end{document}
