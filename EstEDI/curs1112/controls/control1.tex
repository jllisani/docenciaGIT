\documentclass[a4paper,10pt]{article}
%\usepackage[active]{srcltx}      	%% necesario para pasar del dvi al tex%
\usepackage[spanish]{babel}
\usepackage[utf8]{inputenc}
%\usepackage[latin1]{inputenc}
%\usepackage{sw20res1}
\usepackage{amsmath}
\usepackage{amsfonts}
\usepackage{amssymb}
\usepackage{graphicx}
\usepackage{enumerate}
\setlength{\topmargin}{-2.5cm}	%%formato de pagina que ocupa todo
\setlength{\textwidth}{18.5cm}
\setlength{\textheight}{28cm}
\setlength{\oddsidemargin}{-1.5cm}

\pagestyle{empty}
\newcounter{prbcont}
\stepcounter{prbcont}
\setcounter{prbcont}{0}
\newtheorem{problema}[prbcont]{Problema}

\begin{document}

\noindent
{\large \bf Escola Politècnica Superior}

\noindent
{\large Grau en Enginyeria d'Edificació}

\vskip 0.3cm
\noindent
{\large \bf Assignatura: Aplicacions Estadístiques}

\hrule

\vskip 0.3cm

\noindent
Tipus d'activitat

\begin{tabular}{|l|c|c|c|c|}
\hline
 & Exercici & Treball / Pràctica & Examen & Altres \\
\hline
Puntuable & & & X & \\ \hline
No Puntuable & & & & \\ \hline
\end{tabular}

\vskip 0.3cm

\noindent
Competències específiques que es treballen

\begin{tabular}{|l|c|}
\hline
Capacitat per a utilitzar les tècniques i mètodes probabilístics i d'anàlisi estadística & X \\
\hline
\end{tabular}

\vskip 0.3cm

\noindent
Competències genèriques que es treballen

\begin{tabular}{|l|c|}
\hline
Resolució de problemes (CI-1) & X \\ \hline
Capacitat d'anàlisi i síntesi (CI-4) & X \\ \hline
Coneixement d'informàtica relatiu a l'àmbit d'estudis (CI-2) & \\ \hline
Aptitud per a la gestió de l'informació (CI-5) & \\ \hline
Compromís ètic (CP-1) & X \\ \hline
Raonament crític (CP-2) & X \\ \hline
Aptitud per al treball en equip (CP-3) & \\ \hline
Aprenentatge autònom (CP-9) & \\ \hline
\end{tabular}


\vskip 0.3 cm

\noindent
\textbf{Data: 15/03/2011}

\hrule

\vspace{0.3 cm}

\noindent
Considerau les següents dades corresponents al pes (Kg) i l'altura (cm) de 20 jugadors de la plantilla del RCD Mallorca 2011-2012
 (font www.rcdmallorca.es):

%\vskip 0.3 cm
%\begin{center}
%\begin{tabular}{|c|cccccccccccccccccccc|}
%\hline
%Dorsal & 1 & 13 & 26 & 2 & 4 & 12 & 15 & 16 & 21 & 22 & 29 & 3 & 5 & 7 & 11 & 19 & 27 & 8 & 9 & 18\\
%\hline
%Pes & 88 & 87 & 82 & 74 & 84 & 82 & 79 & 81 & 76 & 82 & 75 & 82 & 78 & 69 & 76 & 76 &  74 & 77 & 81 & 80 \\
%\hline
%Altura & 191 & 188 & 192 & 179 & 189 & 187 & 179 & 185 & 181 & 180 & 176 & 182 & 185 & 177 & 177 & 180 & 179 & 181 & 185 &  180\\
%\hline
%\end{tabular}
%\end{center}

\vskip 0.3 cm
\begin{center}
\begin{tabular}{|c|cccccccccc|}
\hline
Dorsal & 1 & 13 & 26 & 2 & 4 & 12 & 15 & 16 & 21 & 22 \\
\hline
Pes & 88 & 87 & 82 & 74 & 84 & 82 & 79 & 81 & 76 & 82  \\
\hline
Altura & 191 & 188 & 192 & 179 & 189 & 187 & 179 & 185 & 181 & 180\\
\hline
\end{tabular}
\end{center}

\vskip 0.3 cm
\begin{center}
\begin{tabular}{|c|cccccccccc|}
\hline
Dorsal & 29 & 3 & 5 & 7 & 11 & 19 & 27 & 8 & 9 & 18\\
\hline
Pes & 75 & 82 & 78 & 69 & 76 & 76 &  74 & 77 & 81 & 80 \\
\hline
Altura &  176 & 182 & 185 & 177 & 177 & 180 & 179 & 181 & 185 &  180\\
\hline
\end{tabular}
\end{center}

\begin{problema}
.

\begin{enumerate}[a)]
\item Representau els valors de la variable `Pes' en un diagrama de capsa, indicant 
tots els valors numèrics rellevants i quins són, si n'hi ha, els valors atípics i extrems.
\item Agrupau els valors de la variable `Pes' en els intervals següents: \textit{menys de 70}, $[70, 75)$,  $[75, 80)$, 
$[80, 85)$, \textit{igual o superior a 85}
i calculau:
\begin{enumerate}[1)]
\item Taula de freqüències (absolutes, relatives, acumulades i percentatges).
\item Moda, mitjana i mediana. 
\item Dibuixau l'histograma de freqüències absolutes.
\end{enumerate}
\end{enumerate}


\end{problema}

\vspace{0.3cm}

\begin{problema}

Considerant les variables brutes de `Pes' i Altura' calculau, \textbf{amb dues xifres decimals de precissió en els càlculs}:

\begin{enumerate}[a)]
\item Mitjana i desviació típica de la variable `Pes'
\item Mitjana i desviació típica de la variable `Altura'.
\item Diagrama de dispersió.
\item Covariància i coeficient de correlació entre les variables, donant una interpretació del valor trobat.
\item Si el coeficient de correlació és superior a $0.5$ calculau i representau la recta de regressió.
\end{enumerate}

\end{problema}


\vskip 1cm

\newpage



\vskip 5cm
\textbf{Formulari Estadística Descriptiva}
\vskip 0.2 cm

\begin{itemize}
\item Percentil $p$ de dades agrupades en intervals:
\[
P_p=L_p + (L_{p+1}-L_p) \frac{N \cdot p - N_{p-1}}{n_p}
\]

\item Coeficient de simetria: $g_1=\frac{m_3}{s^3}$, $s$: desviació típica
\begin{itemize}
\item Dades brutes
\[
m_3=\frac{(x_1-\bar{x})^3+(x_2-\bar{x})^3+\cdots+(x_N-\bar{x})^3}{N}
\]
\item Dades en taula de freqüències
\[
m_3=\frac{(x_1-\bar{x})^3 n_1+(x_2-\bar{x})^3 n_2+\cdots+(x_k-\bar{x})^3 n_k}{N}
\]
\end{itemize}

\item Coeficient d'apuntament: $g_2=\frac{m_4}{s^4} - 3$, $s$: desviació típica
\begin{itemize}
\item Dades brutes
\[
m_4=\frac{(x_1-\bar{x})^4+(x_2-\bar{x})^4+\cdots+(x_N-\bar{x})^4}{N}
\]
\item Dades en taula de freqüències
\[
m_4=\frac{(x_1-\bar{x})^4 n_1+(x_2-\bar{x})^4 n_2+\cdots+(x_k-\bar{x})^4 n_k}{N}
\]
\end{itemize}

\item Recta de regressió: $\hat{Y}=aX+b$
\[
a=\frac{\mathrm{Cov(X, Y)}}{\mathrm{Var(X)}} \qquad b=\bar{y}-a \bar{x}
\]

\item Coeficient de conting\`encia:
\[
0\leq C \leq \sqrt{1-\dfrac{1}{\min(k,l)}},\quad C=\sqrt{\dfrac {\chi^2}{\chi^2+N}},\quad \chi^2=\sum_i \sum_j \dfrac {(n_{ij}-e_{ij})^2}{e_{ij}},\quad e_{ij}=\dfrac {n_{i*}n_{*j}}{N}
\]

\end{itemize}



\end{document}

