\documentclass[a4paper,10pt]{article}
%\usepackage[active]{srcltx}      	%% necesario para pasar del dvi al tex%
\usepackage[spanish]{babel}
\usepackage[utf8]{inputenc}
%\usepackage[latin1]{inputenc}
%\usepackage{sw20res1}
\usepackage{amsmath}
\usepackage{amsfonts}
\usepackage{amssymb}
\usepackage{graphicx}
\usepackage{enumerate}
\setlength{\topmargin}{-2.5cm}	%%formato de pagina que ocupa todo
\setlength{\textwidth}{18.5cm}
\setlength{\textheight}{28cm}
\setlength{\oddsidemargin}{-1.5cm}

\pagestyle{empty}
\newcounter{prbcont}
\stepcounter{prbcont}
\setcounter{prbcont}{0}
\newtheorem{problema}[prbcont]{Problema}

\begin{document}

\noindent
{\large \bf Escola Politècnica Superior}

\noindent
{\large Grau en Enginyeria d'Edificació}

\vskip 0.3cm
\noindent
{\large \bf Assignatura: Aplicacions Estadístiques}

\hrule

\vskip 0.3cm

\noindent
Tipus d'activitat

\begin{tabular}{|l|c|c|c|c|}
\hline
 & Exercici & Treball / Pràctica & Examen & Altres \\
\hline
Puntuable & & & X & \\ \hline
No Puntuable & & & & \\ \hline
\end{tabular}

\vskip 0.3cm

\noindent
Competències específiques que es treballen

\begin{tabular}{|l|c|}
\hline
Capacitat per a utilitzar les tècniques i mètodes probabilístics i d'anàlisi estadística & X \\
\hline
\end{tabular}

\vskip 0.3cm

\noindent
Competències genèriques que es treballen

\begin{tabular}{|l|c|}
\hline
Resolució de problemes (CI-1) & X \\ \hline
Capacitat d'anàlisi i síntesi (CI-4) & X \\ \hline
Coneixement d'informàtica relatiu a l'àmbit d'estudis (CI-2) & \\ \hline
Aptitud per a la gestió de l'informació (CI-5) & \\ \hline
Compromís ètic (CP-1) & X \\ \hline
Raonament crític (CP-2) & X \\ \hline
Aptitud per al treball en equip (CP-3) & \\ \hline
Aprenentatge autònom (CP-9) & \\ \hline
\end{tabular}


\vskip 0.3 cm

\noindent
\textbf{Data: 16/05/2012}

\hrule

\vspace{0.1 cm}


\begin{problema}
Un minibús fa trasllats diaris, sota reserva prèvia, des d'un hotel a l'aeroport.
El minibús té places per a 18 persones, però en previsió que alguna de les persones que 
fan la reserva fallin, sempre es fan 20 reserves. Suposant que la probabilitat que una reserva
falli sigui del $10\%$, es demana:
\begin{enumerate}[a)]
\item Quina és la probabilitat d'\textit{overbooking} (s'hi presenten més persones amb reserva que
places disponibles)?
\item Quina és la probabilitat que el minibús no s'ompli?
\item Quin és el màxim de reserves que es poden fer per garantir que es produirà \textit{overbooking}
com a màxim un pic cada setmana (1 de cada 7 vegades)? 
\item Repetiu  l'apartat anterior per al cas que s'utilitzi un autobús amb 90 places i la probabilitat 
de que una reserva falli sigui $0.04$. (Nota: utilitzau l'aproximació de Poisson).
\end{enumerate}
\end{problema}

\vspace{0.2cm}

\begin{problema}
Un estudi estad\'\i stic mostra que la distribuci\'o dels habitatges d'una determinada localitat segons la seva antiguitat 
segueix una llei normal amb mitjana 30 anys i desviació típica 7 anys. 

Es demana:
\begin{enumerate}[a)]
\item Quina \'es la probabilitat que un habitatge triat a l'atzar tengui més de 39 anys?
\item Quina és la probabilitat que un habitatge triat a l'atzar tengui entre 20 i 35 anys?
\item Si definim com `vell' un edifici de més de 39 anys i consideram un conjunt de 15 edificis triats a l'atzar,
quina és la probabilitat que més de 5 edificis d'aquest grup siguin `vells'?
\end{enumerate}

\end{problema}




\textbf{Variables aleatòries usuals}
\vskip 0.1 cm

\begin{tabular}{|c|cl|c|c|l|}
V.A. (X) & $f_X(x)$ & & $E(X)$ & $Var(X)$ & Altres propietats \\
\hline
Binomial $B(n, p)$ & $\binom{n}{x} p^x (1-p)^{n-x}$ & si $x\in \Omega_X$ &
 $np$ & $np(1-p)$ & \\
$\Omega_X=\{ 0, 1, \cdots, n \}$ & $0$ & si $x \notin \Omega_X$ & & & \\ \hline
Poisson $Po(\lambda)$ & $\frac{\lambda^x}{x!} e^{-\lambda}$ & si $x\in \Omega_X$ & 
 $\lambda$ & $\lambda$ & $B(n, p) \approx Po(np)$ \\
$\Omega_X=\{ 0, 1, \cdots \}$ & $0$ & si $x \notin \Omega_X$ & & & ($n$ gran, $p$ petit)\\ \hline
Uniforme ${\cal U}(a, b)$ & $\frac{1}{b-a}$ & si $x \in [a, b]$ & 
$\frac{b+a}{2}$ & $\frac{(b-a)^2}{12}$ & 
$F_X(x)=\begin{cases} 
\frac{x-a}{b-a} & x \in [a, b] \\
0 & x < a \\
1 & x > b
\end{cases}$ \\
$\Omega_X=[a, b]$ & 0 & si $x \notin [a, b]$ &  & & \\ \hline
Gaussiana $X(\mu, \sigma^2)$ & & & $\mu$ & $\sigma^2$ & $Z\sim N(0, 1)$ normal est\'andar \\
$\Omega_X=\mathbb{R}$ & & & &  & $F_Z(-z)=1-F_Z(z)$ \\
 & & & &  & $F_X(x)=F_Z(\frac{x-\mu}{\sigma})$ \\ 
 & & & &  & $B(n, p) \approx N(np, np(1-p))$ \\ 
 & & & &  & ($n$ gran) \\
 & & & &  & $Po(\lambda) \approx N(\lambda, \lambda)$ \\ 
 & & & &  & ($\lambda$ gran) \\ \hline
\end{tabular}

\end{document}

