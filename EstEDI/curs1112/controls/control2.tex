\documentclass[a4paper,10pt]{article}
%\usepackage[active]{srcltx}      	%% necesario para pasar del dvi al tex%
\usepackage[spanish]{babel}
\usepackage[utf8]{inputenc}
%\usepackage[latin1]{inputenc}
%\usepackage{sw20res1}
\usepackage{amsmath}
\usepackage{amsfonts}
\usepackage{amssymb}
\usepackage{graphicx}
\usepackage{enumerate}
\setlength{\topmargin}{-2.5cm}	%%formato de pagina que ocupa todo
\setlength{\textwidth}{18.5cm}
\setlength{\textheight}{28cm}
\setlength{\oddsidemargin}{-1.5cm}

\pagestyle{empty}
\newcounter{prbcont}
\stepcounter{prbcont}
\setcounter{prbcont}{0}
\newtheorem{problema}[prbcont]{Problema}

\begin{document}

\noindent
{\large \bf Escola Politècnica Superior}

\noindent
{\large Grau en Enginyeria d'Edificació}

\vskip 0.3cm
\noindent
{\large \bf Assignatura: Aplicacions Estadístiques}

\hrule

\vskip 0.3cm

\noindent
Tipus d'activitat

\begin{tabular}{|l|c|c|c|c|}
\hline
 & Exercici & Treball / Pràctica & Examen & Altres \\
\hline
Puntuable & & & X & \\ \hline
No Puntuable & & & & \\ \hline
\end{tabular}

\vskip 0.3cm

\noindent
Competències específiques que es treballen

\begin{tabular}{|l|c|}
\hline
Capacitat per a utilitzar les tècniques i mètodes probabilístics i d'anàlisi estadística & X \\
\hline
\end{tabular}

\vskip 0.3cm

\noindent
Competències genèriques que es treballen

\begin{tabular}{|l|c|}
\hline
Resolució de problemes (CI-1) & X \\ \hline
Capacitat d'anàlisi i síntesi (CI-4) & X \\ \hline
Coneixement d'informàtica relatiu a l'àmbit d'estudis (CI-2) & \\ \hline
Aptitud per a la gestió de l'informació (CI-5) & \\ \hline
Compromís ètic (CP-1) & X \\ \hline
Raonament crític (CP-2) & X \\ \hline
Aptitud per al treball en equip (CP-3) & \\ \hline
Aprenentatge autònom (CP-9) & \\ \hline
\end{tabular}


\vskip 0.3 cm

\noindent
\textbf{Data: 03/04/2012}

\hrule

\vspace{0.3 cm}


\begin{problema}
Dues persones, Maria i Joan, juguen a b\`asquet de la següent manera: llancen de manera alternativa
des de la l\'\i nea de 3 punts fins que un d'ells encistella i guanya i acaba el joc. Si la primera en llançar \'es na Maria,
i suposant que les probabilitats d'encistellar d'en Joan i na Maria s\'on, respectivament, $0.3$ i $0.4$,
quina \'es la probabilitat que el nombre total de llançaments que fa en Joan fins que acaba el joc sigui 2?
\ \hfill{\textbf{ 4 pt.}}
\end{problema}

\vspace{0.5cm}

\begin{problema}
Un estudi estad\'\i stic mostra que la distribuci\'o dels habitatges d'una determinada localitat segons la seva antiguitat 
\'es la seg\"uent:

\begin{center}
\begin{tabular}{|c|c|}
\hline
Antiguitat (anys) & Quantitat \\
\hline
m\'es de 50 & 25 \\
de 50 a 40 & 75 \\
de 40 a 30 & 100 \\
de 30 a 20 & 200 \\
de 20 a 10 & 200 \\
menys de 10 & 400 \\
\hline
\end{tabular}
\end{center}

Aix\'i mateix, l'estudi mostra que un $15\%$ dels habitatges \textit{vells}  (30 o m\'es anys)  tenen defectes estructurals, mentre 
que nom\'es un $5\%$ dels edificis \textit{nous}  (menys de 30 anys)  en tenen.

Es demana:
\begin{enumerate}[a)]
\item Quina \'es la probabilitat que un habitatge triat a l'atzar tengui defectes estructurals?
\item Quina \'es la probabilitat que un habitatge triat a l'atzar sigui \textit{nou} i tengui defectes estructurals?
\item Quina \'es la probabilitat que un edifici que té defectes estructurals sigui \textit{nou}?
\end{enumerate}

\ \hfill{\textbf{ 6 pt.}}
\end{problema}


\vskip 1cm
\noindent
\textbf{Nota:} en tots els problemes s'han de definir correctament els successos i indicar quines propietats
s'apliquen per fer els c\`alculs.

\vskip 1cm
\noindent
\textbf{F\`ormules de combinat\`oria:}

\begin{tabular}{l}
$\displaystyle VR_n^k=n^k$ \\
$\displaystyle V_n^k=\frac{n!}{(n-k)!}$ \\
$\displaystyle P_n = n!$ \\
$\displaystyle PR_n^{k_1, k_2, \cdots, k_r} = \frac{n!}{k_1! k_2! \cdots k_r!}$ \\
$\displaystyle C_n^k=\binom{n}{k}=\frac{n!}{k! (n-k)!}$ \\
$\displaystyle CR_n^k=\binom{n+k-1}{k}=\frac{(n+k-1)!}{k! (n-1)!}$ \\
\end{tabular}


\end{document}

