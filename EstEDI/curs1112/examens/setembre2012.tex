\documentclass[a4paper,12pt]{article}
%\usepackage[active]{srcltx}      	%% necesario para pasar del dvi al tex%
\usepackage[spanish]{babel}
%\usepackage[latin1]{inputenc}
\usepackage[utf8]{inputenc}
%\usepackage{sw20res1}
\usepackage{amsmath}
\usepackage{amsfonts}
\usepackage{amssymb}
\usepackage{graphicx}
\usepackage{enumerate}
\setlength{\topmargin}{-2.5cm}	%%formato de pagina que ocupa todo
\setlength{\textwidth}{18.5cm}
\setlength{\textheight}{28cm}
\setlength{\oddsidemargin}{-1.5cm}

\pagestyle{empty}
\newcounter{prbcont}
\stepcounter{prbcont}
\setcounter{prbcont}{0}
\newtheorem{problema}[prbcont]{Problema}

\begin{document}
\noindent\textbf{\large{Escola Polit\`ecnica Superior}}\\
\noindent{\large{Grau en Enginyeria d'Edificaci\'o}}\\

\noindent\textbf{\underline{\large{Assignatura: Aplicacions Estad\'{i}stiques}\hspace{10cm}} }\\ 
\noindent\begin{small}\textbf{Tipus d'activitat:}\end{small}
\begin{center}
\begin{tabular}{|l|c|c|c|c|}\hline
 		&Exercici & Treball/Pr\`actica & Examen & Altres \\ \hline
Puntuable       & 	  & 			 & \textbf{X}  	  & \\ \hline
No Puntuable    &  	  &			 & 	  & \\ \hline
\end{tabular}
\end{center}
\noindent\begin{small}\textbf{Compet\`encies espec\'{\i}fiques que es treballen:}\end{small}\\
\begin{tabular}{|l|c|}\hline
Capacitat per a utilitzar les t\`ecniques i m\`etodes probabil\'{\i}stics i d'an\`alisi estad\'{\i}stica & \textbf{X} \\ \hline
\end{tabular}\vspace{0.25cm}

\noindent\begin{small}\textbf{Compet\`encies gen\`eriques que es treballen:}\end{small}\\
\begin{tabular}{|l|c|}\hline
Resoluci\'o de problemes (CI-1) & \textbf{X}\\ \hline
Capacitat d'an\`alisi i s\'{\i}ntesi (CI-4) & \textbf{X}  \\ \hline
Comprom\'{\i}s \`etic (CP-1)$\quad$ & \textbf{X} \\ \hline
\end{tabular}\vspace{0.25cm}

\noindent\begin{small}\textbf{Data: 10/09/2012 }\end{small}\\
\noindent\underline{\hspace{18cm}}


\begin{problema}
En la seg\"uent taula s'ha reflectit el nombre de formigoneres que entren en un taller de reparaci\'o especialitzat segons els anys d'antiguitat (X) i l'import total de la factura (Y).
\begin{center}
\begin{tabular}{|l||c|c|c|}\hline
Y \textbackslash X & $1$ & $2 $& $3$ \\ \hline \hline
$[50,100)$ & 8 & 2 & 10 \\ \hline
$[100,300)$ & 20 & 5 & 25 \\ \hline
$[300,600]$ & 12 & 3 & 15\\ \hline
\end{tabular}
\end{center}
\begin{itemize}
\item [a)] Per als vehicles de 3 anys, calculau l'import mitj\`a de la factura i la vari\`ancia.
\item [b)] Analitzau si existeix independ\`encia estad\'{\i}stica entre les variables.
\item [c)] Determinau el coeficient de correlaci\'o lineal i interpretau el resultat.
\end{itemize}
\end{problema}

\vskip 0.5cm

\begin{problema}
L'alineaci\'o entre el suport i el cap lector en un sistema d'emmagatzematge de dades, afecta a l'acompliment del sistema. Es considera que el 10\% de les operacions de lectura es veuen atenuades per una alineaci\'o obliqua, el 5\% de les operacions de lectura es veuen atenuades per una alineaci\'o descentrada i la resta de les operacions de lectura es realitzen de forma correcta. La probabilitat d'error en la lectura a causa d'una alineaci\'o obliqua \'es 0.01, per una alineaci\'o descentrada \'es 0.02 i quan l'alineaci\'o \'es correcta \'es 0.001.
\begin{itemize}
\item [a)]  Identificau i donau nom als successos que es descriuen a l'enunciat del problema.
\item [b)] Quina \'es la probabilitat de tenir un error de lectura?
\item [c)] Si es presenta un error de lectura, quina \'es la probabilitat que es degui a una alineaci\'o obliqua?
\end{itemize}
\end{problema}

\newpage

\begin{problema}
Una fabricant de vidres sap que, per terme mitjà, els vidres que fabrica tenen 10 imperfeccions per metre quadrat.

\vskip 0.3 cm
Es demana:

\begin{enumerate}[a)]
\item Quina és la probabilitat de trobar més de 20 imperfeccions en un vidre d'un metre quadrat?
\item Quina és la probabilitat de trobar més de 10 i menys de 15 imperfeccions en un vidre de mig metre quadrat?
\item Quina és la probabilitat de trobar menys de 40 imperfeccions en un vidre de 5 metres quadrats?
\item Si es fabriquen 10 vidres d'un metre quadrat cadascun, quina és la probabilitat que més de 7 d'aquests 
vidres presentin 20 o menys imperfeccions?
\end{enumerate}
\end{problema}

\vskip 0.5cm

\begin{problema}
En una publicaci\'o cient\'{\i}fica es descriu l'efecte de la p\`erdua de l\`amines sobre la freq\"u\`encia natural, de bigues formades per diverses l\`amines. Es van subjectar 5 bigues amb p\`erdua de l\`amines a diverses c\`arregues, i les freq\"u\`encies resultants van ser les seg\"uents (en Hz.)
\[
230.66,\;233.05,\;232.58,\;229.48,\; 232.58
\]
Trobeu un interval de confian\c{c}a del 90\% per a la desviaci\'o t\'{\i}pica de la freq\"u\`encia natural. Suposeu que la poblaci\'o segueix
una distribuci\'o normal.
\end{problema}

\vskip 0.5cm

\begin{problema}
El director de producci\'o de Finestres Nord S. A. desitja avaluar un nou m\`etode per produir finestres de doble fulla. El proc\'es actual t\'e una producci\'o mitjana de 80 unitats per hora. El director indica que no vol substituir l'actual proc\'es tret que existeixin proves contundents que el nivell mitj\`a de producci\'o \'es major amb el nou m\`etode. Obtenim una mostra de 25 hores de producci\'o amb el nou m\`etode amb una producci\'o mitjana de 83 finestres. Suposant que un nivell de significaci\'o $\alpha=0.05$ \'es una prova contundent, quina recomanaci\'o faries al director? (Suposam que la desviaci\'o t\'{\i}pica de la producci\'o amb el m\`etode nou
\'es $\sigma=8$).
\end{problema}


\end{document}


