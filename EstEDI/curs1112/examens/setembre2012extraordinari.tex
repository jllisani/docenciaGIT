\documentclass[a4paper,10pt]{article}
%\usepackage[active]{srcltx}      	%% necesario para pasar del dvi al tex%
\usepackage[spanish]{babel}
\usepackage[latin1]{inputenc}
%\usepackage{sw20res1}
\usepackage{amsmath}
\usepackage{amsfonts}
\usepackage{amssymb}
\usepackage{graphicx}
\usepackage{enumerate}
\setlength{\topmargin}{-2.5cm}	%%formato de pagina que ocupa todo
\setlength{\textwidth}{18.5cm}
\setlength{\textheight}{28cm}
\setlength{\oddsidemargin}{-1.5cm}

\pagestyle{empty}
\newcounter{prbcont}
\stepcounter{prbcont}
\setcounter{prbcont}{0}
\newtheorem{problema}[prbcont]{Problema}

\begin{document}

\begin{center}
\textbf{{\large {Examen d'Estad\'istica Enginyeria Edificaci\'o. 
\\
Setembre 2012. Extraordinari.}}}
\end{center}

\vspace{0.3cm}

\begin{problema}
La seg\"uent taula mostra les dades de consum de ciment (en tones) i de nombre d'aturats a les Illes Balears
entre els mesos de gener i desembre de l'any 2008.

\vskip 0.3 cm
\begin{tabular}{|c|cccccccccccc|}
\hline
Ciment & 88218 & 94935 & 77395 & 96706 & 76975 & 75862 & 62318 & 41726 & 50628 & 60192 & 50970 & 36850 \\
\hline
Aturats & 50518 & 48335 & 45184 & 41233 & 36439 & 36929 & 39927 & 43540 & 46807 & 56982 & 70144 & 73298 \\
\hline
\end{tabular}

\vskip 0.3 cm
\noindent
Es demana calcular, \textbf{amb dues xifres decimals de precissi\'o en els c\`alculs}:

\begin{enumerate}[a)]
\item Mediana, primer i tercer quartils i percentil 90 de la variable ``nombre d'aturats''.
\item Mitjana i desviaci\'o t\'{\i}pica de la variable ``nombre d'aturats''.
\item Mitjana i desviaci\'o t\'{\i}pica de la variable ``consum de ciment''.
\item Covari\`ancia i coeficient de correlaci\'o entre les variables ``nombre d'aturats'' i ``consum de ciment'', donant una interpretaci\'o del valor trobat.
\end{enumerate}

\end{problema}

\begin{problema} 
Rafel y Carlos juegan un partido de tenis. 
Al final del partido, Carlos coge por error una pelota de las que originalmente tra\'{\i}a Rafel.
Inicialmente Rafel tra\'{\i}a 7 pelotas marca Funlop y 3 marca Milson,
y Carlos tra\'{i}a 5 pelotas marca Funlop y 6 marca Milson.
\begin{itemize}
\item [(a)] Identifica y da nombre a los sucesos m\'as significativos del problema. 
\item [(b)] Si al finalizar el partido cogemos una pelota de las que lleva Carlos, cu\'al es la
probabilidad de que sea de marca Funlop? 
\item [(c)] Si al final del partido cogemos una pelota de Carlos y es de marca Milson, 
cu\'al es la probabilidad de que la pelota cogida por error a Rafel 
sea de esta marca?
\end{itemize}
\end{problema}



\begin{problema}
Al apostar en un juego de azar la probabilidad de ganar es igual a $\frac {7}{20}$ y la de perder,
$\frac {13}{20}.$ De un total de 19 apuestas:
\begin{itemize} 
\item[(a)] Cu\'al es la probabilidad de ganar exactamente en 5 de ellas?
\item[(b)] Cu\'al es la probabilidad de ganar m\'as de la mitad de las apuestas?
\item[(c)] Cu\'al es la probabilidad de ganar m\'as de 5 y menos de 15 apuestas?
\item[(d)] Cu\'al es el valor esperado y la varianza del n\'umero de apuestas ganadas? 
\end{itemize}
Consideremos un total de 100 apuestas. 
\begin{itemize}
\item[(e)] Si cada apuesta vale 3 euros y cobramos 5 euros en caso de ganarla, cu\'al es el valor
esperado y la varianza de la cantidad ganada tras 100 apuestas?
\item[(f)] Cu\'al es la probabilidad de perder dinero en 100 apuestas?\\ (Indicaci\'on:
Usa la distribuci\'on normal para aproximar la distribuci\'on que has utilizado en este problema.) 
\end{itemize}
\end{problema}

\begin{problema}
En una encuesta hecha a 500 constructores, 196 contestaron afirmativamente a la siguiente pregunta:
``cree usted que el precio de las viviendas es razonable?".
\begin{itemize}
\item [(a)] Halla un intervalo de confianza, con un nivel de confianza del $95\%,$ para el
porcentaje de constructores que cree que el precio de las viviendas es razonable. Sin hacer c\'alculos, razona si un intervalo de $99\%$ de nivel de confianza es mayor, igual o menor que el de $95\%.$
\item[(b)] Hace 5 a\~nos, el porcentaje de constructores que cre\'{\i}a que el precio de la vivienda era razonable era del $41.9\%.$ Con un nivel de significaci\'on de $0.05,$ los datos de la muestra permiten asegurar que este porcentaje ha disminuido?
\end{itemize}
\end{problema}





\end{document}

