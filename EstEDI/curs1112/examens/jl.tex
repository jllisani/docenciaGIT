\documentclass[a4paper,12pt]{article}
%\usepackage[active]{srcltx}      	%% necesario para pasar del dvi al tex%
\usepackage[spanish]{babel}
%\usepackage[latin1]{inputenc}
\usepackage[utf8]{inputenc}
%\usepackage{sw20res1}
\usepackage{amsmath}
\usepackage{amsfonts}
\usepackage{amssymb}
\usepackage{graphicx}
\usepackage{enumerate}
\setlength{\topmargin}{-2.5cm}	%%formato de pagina que ocupa todo
\setlength{\textwidth}{18.5cm}
\setlength{\textheight}{28cm}
\setlength{\oddsidemargin}{-1.5cm}

\pagestyle{empty}
\newcounter{prbcont}
\stepcounter{prbcont}
\setcounter{prbcont}{0}
\newtheorem{problema}[prbcont]{Problema}

\begin{document}

\noindent
{\large \bf Escola Politècnica Superior}

\noindent
{\large Grau en Enginyeria d'Edificació}

\vskip 0.3cm
\noindent
{\large \bf Assignatura: Aplicacions Estadístiques}

\hrule

\vskip 0.3cm

{ \footnotesize

\noindent
Tipus d'activitat

\begin{tabular}{|l|c|c|c|c|}
\hline
 & Exercici & Treball / Pràctica & Examen & Altres \\
\hline
Puntuable & & & X & \\ \hline
No Puntuable & & & & \\ \hline
\end{tabular}

\vskip 0.3cm

\noindent
Competències específiques que es treballen

\begin{tabular}{|l|c|}
\hline
Capacitat per a utilitzar les tècniques i mètodes probabilístics i d'anàlisi estadística & X \\
\hline
\end{tabular}

\vskip 0.3cm

\noindent
Competències genèriques que es treballen

\begin{tabular}{|l|c|}
\hline
Resolució de problemes (CI-1) & X \\ \hline
Capacitat d'anàlisi i síntesi (CI-4) & X \\ \hline
Coneixement d'informàtica relatiu a l'àmbit d'estudis (CI-2) & \\ \hline
Aptitud per a la gestió de l'informació (CI-5) & \\ \hline
Compromís ètic (CP-1) & X \\ \hline
Raonament crític (CP-2) & X \\ \hline
Aptitud per al treball en equip (CP-3) & \\ \hline
Aprenentatge autònom (CP-9) & \\ \hline
\end{tabular}

}
\vskip 0.3 cm

\noindent
\textbf{Data: 18/06/2012}

\hrule

\vskip 0.5cm

\begin{problema}
La següent taula mostra les notes dels 2 primers controls de 20 alumnes d'Estadística:

\begin{center}
\begin{tabular}{|cccccccccccccccccccc|}
\hline
 8.3 & 9.5 & 3.7 & 7.9 & 6.7 & 9 & 9.7 & 10 & 5.9 & 7.4 & 4.1 & 9.4 & 9 & 10 & 7.1 & 5.7 & 4.2 & 7 & 4.9 & 8.2 \\
\hline
6 & 5 & 0.5 & 5 & 0 & 9.2 & 2.5 & 9.7 & 7 & 7 & 3 & 6 & 9 & 9.7 & 1.5 & 3.5 & 7 & 4.2 & 4 & 8.5 \\
\hline
\end{tabular}
\end{center}

\vskip 0.3 cm
Es demana:

\begin{enumerate}[a)]
\item Utilitza el rang interquartílic per comparar la dispersió de les notes dels dos controls i comenta els resultats.
\item Dibuixa el diagrama de dispersió dels dos conjunts de notes
\item Calcula el coeficient de correlació entre els dos conjunts de notes i comenta el resultat, relacionat-lo amb el diagrama
de dispersió dibuixat a l'apartat anterior.
\end{enumerate}
\end{problema}

\vskip 0.5cm

\begin{problema}
En una pastisseria hi ha dues vitrines on s'exposen els pastissos. En la primera d'elles hi ha 10 pastissos de poma i 7 de xocolata
i en la segona 5 de poma i 10 de xocolata. Per un error una de les dependentes passa un pastís de la primera a la segona vitrina
i a continuació arriba un client que tria un pastís a l'atzar d'aquesta segona vitrina.


\vskip 0.3 cm
Es demana:

\begin{enumerate}[a)]
\item Identifica i dóna nom als successos que es descriuen a l'enunciat del problema.
En cas que hi hagi successos que estiguin condicionats per uns altres successos, calcula les probabilitats condicionades. 
\item Quina és la probabilitat que el pastís que tria el client sigui de poma?
\item Si el pastís que tria el client és de poma, quina és la probabilitat que el pastís que va passar de la primera a la segona 
vitrina fos de xocolata?
\item Quina és la probabilitat que el pastís que canvia de vitrina sigui de poma i que el pastís que tria el client sigui de xocolata?
\end{enumerate}
\end{problema}


\vskip 0.5cm
\begin{problema}
Una fabricant de vidres sap que, per terme mitjà, els vidres que fabrica tenen 10 imperfeccions per metre quadrat.

\vskip 0.3 cm
Es demana:

\begin{enumerate}[a)]
\item Quina és la probabilitat de trobar més de 20 imperfeccions en un vidre d'un metre quadrat?
\item Quina és la probabilitat de trobar més de 10 i menys de 15 imperfeccions en un vidre de mig metre quadrat?
\item Quina és la probabilitat de trobar menys de 40 imperfeccions en un vidre de 5 metres quadrats?
\item Si es fabriquen 10 vidres d'un metre quadrat cadascun, quina és la probabilitat que més de 7 d'aquests 
vidres presentin 20 o menys imperfeccions?
\end{enumerate}
\end{problema}


\vskip 0.5cm
\begin{problema}
El president d'un determinat país afirma que l'economia del país funciona molt bé i que no necessita ajuda
financera externa ja que l'Índex Econòmic mitjà de les seves empreses és de 10, al nivell de la mitjana mundial. 
Una agència internacional d'estudis econòmics decideix contrastar aquestes afirmacions i pren una mostra de
10 empreses, obtenint els següents valors de l'Índex Econòmic per a cada una d'elles:

\begin{center}
\begin{tabular}{cccccccccc}
9 & 8.9 & 11 & 9.5 & 10.4 & 7.8 & 9.7 & 10.1 & 8.5 & 9.9 
\end{tabular}
\end{center}

\vskip 0.3 cm
Es demana:

\begin{enumerate}[a)]
\item Calcula la mitjana i la desviació típica mostrals de l'Índex Econòmic de les empreses.
\item A partir de les dades anteriors troba un interval de confiança per a la mitjana poblacional de l'Índex Econòmic de les empreses,
amb un nivell de confiança del $95\%$.
\item Contrasta les afirmacions del president del país, indicant de manera raonada quines són les hipòtesis nul.la i alternativa,
amb un nivell de significació del $10\%$. Calcula el p-valor del contrast.
\end{enumerate}
\end{problema}


\vskip 1cm

{\footnotesize

\textbf{Formulari Estad\'{\i}stica Descriptiva}
\vskip 0.2 cm

\begin{itemize}
\item Percentil $p$ de dades agrupades en intervals:
$
\displaystyle
P_p=L_p + (L_{p+1}-L_p) \frac{N \cdot p - N_{p-1}}{n_p}
$

\item Coeficient de simetria: $g_1=\frac{m_3}{s^3}$, $s$: desviaci\'o t\'{\i}pica
\begin{itemize}
\item Dades brutes:
$
\displaystyle
m_3=\frac{(x_1-\bar{x})^3+(x_2-\bar{x})^3+\cdots+(x_N-\bar{x})^3}{N}
$
\item Dades en taula de freq\"u\`encies:
$
\displaystyle
m_3=\frac{(x_1-\bar{x})^3 n_1+(x_2-\bar{x})^3 n_2+\cdots+(x_k-\bar{x})^3 n_k}{N}
$
\end{itemize}

\item Coeficient d'apuntament: $g_2=\frac{m_4}{s^4} - 3$, $s$: desviaci\'o t\'{\i}pica
\begin{itemize}
\item Dades brutes:
$
\displaystyle
m_4=\frac{(x_1-\bar{x})^4+(x_2-\bar{x})^4+\cdots+(x_N-\bar{x})^4}{N}
$
\item Dades en taula de freq\"u\`encies:
$
\displaystyle
m_4=\frac{(x_1-\bar{x})^4 n_1+(x_2-\bar{x})^4 n_2+\cdots+(x_k-\bar{x})^4 n_k}{N}
$
\end{itemize}

\item Recta de regressi\'o: $\hat{Y}=aX+b$, $\qquad 
a=\frac{\mathrm{Cov(X, Y)}}{\mathrm{Var(X)}} \qquad b=\bar{y}-a \bar{x}
$

\item Coeficient de conting\`encia:
$
\displaystyle
0\leq C \leq \sqrt{1-\dfrac{1}{\min(k,l)}},\quad C=\sqrt{\dfrac {\chi^2}{\chi^2+N}},\quad \chi^2=\sum_i \sum_j \dfrac {(n_{ij}-e_{ij})^2}{e_{ij}},\quad e_{ij}=\dfrac {n_{i*}n_{*j}}{N}
$


\end{itemize}

\textbf{Formulari Estad\'{\i}stica Inferencial}
\vskip 0.2 cm

\textbf{Variables aleat\`ories usuals}
\vskip 0.1 cm

\begin{tabular}{|c|cl|c|c|l|}
V.A. (X) & $f_X(x)$ & & $E(X)$ & $Var(X)$ & Altres propietats \\
\hline
Binomial $B(n, p)$ & $\binom{n}{x} p^x (1-p)^{n-x}$ & si $x\in \Omega_X$ &
 $np$ & $np(1-p)$ & \\
$\Omega_X=\{ 0, 1, \cdots, n \}$ & $0$ & si $x \notin \Omega_X$ & & & \\ \hline
Poisson $Po(\lambda)$ & $\frac{\lambda^x}{x!} e^{-\lambda}$ & si $x\in \Omega_X$ & 
 $\lambda$ & $\lambda$ & $B(n, p) \approx Po(np)$ \\
$\Omega_X=\{ 0, 1, \cdots \}$ & $0$ & si $x \notin \Omega_X$ & & & ($n$ gran, $p$ petit)\\ \hline
Uniforme ${\cal U}(a, b)$ & $\frac{1}{b-a}$ & si $x \in [a, b]$ & 
$\frac{b+a}{2}$ & $\frac{(b-a)^2}{12}$ & 
$F_X(x)=\begin{cases} 
\frac{x-a}{b-a} & x \in [a, b] \\
0 & x < a \\
1 & x > b
\end{cases}$ \\
$\Omega_X=[a, b]$ & 0 & si $x \notin [a, b]$ &  & & \\ \hline
Gaussiana $X(\mu, \sigma^2)$ & & & $\mu$ & $\sigma^2$ & $Z\sim N(0, 1)$ normal est\'andar \\
$\Omega_X=\mathbb{R}$ & & & &  & $F_Z(-z)=1-F_Z(z)$ \\
 & & & &  & $F_X(x)=F_Z(\frac{x-\mu}{\sigma})$ \\ 
 & & & &  & $B(n, p) \approx N(np, np(1-p))$ \\ 
 & & & &  & ($n$ gran) \\
 & & & &  & $Po(\lambda) \approx N(\lambda, \lambda)$ \\ 
 & & & &  & ($\lambda$ gran) \\ \hline
\end{tabular}

\vskip 0.2 cm

\textbf{Estad\'\i stics m\'es usuals}
\vskip 0.2 cm

\begin{tabular}{c|c|c|cl}
Par\`ametre  & Esperan\c{c}a & Vari\`ancia & Distribuci\'o  & \\
mostral &  &  & de probabilitat & \\
(estad\'\i stic) & & & & \\ 
\hline
$\bar{X}$ & $E(\bar{X})=\mu$ & $\mathrm{Var}(\bar{X})=\frac{\sigma^2}{n}$ & 
$\bar{X} \sim N(\mu, \frac{\sigma^2}{n})$ & poblaci\'o normal, $\sigma$ conegut \\
& & & $\frac{\bar{X}-\mu}{\hat{s}_X / \sqrt{n}} \sim t_{n-1}$ & 
poblaci\'o normal, $\sigma$ desconegut, $n \leq 30$ \\
& & & 
$\bar{X} \sim N(\mu, \frac{\hat{s}_X^2}{n})$ & 
$\sigma$ desconegut, $n > 30$ \\
& & & & \\
$\hat{s}_X^2$ & $E(\hat{s}_X^2)=\sigma^2$ & $\mathrm{Var}(\hat{s}_X^2)=\frac{2\sigma^4}{n-1}$ & 
$\frac{n-1}{\sigma^2}\hat{s}_X^2 \sim \chi^2_{n-1}$ & poblaci\'o normal \\
& & & & \\
$\hat{p}_X$ & $E(\hat{p}_X)=p$ & $\mathrm{Var}(\hat{p}_X)=\frac{p(1-p)}{n}$ &
$\hat{p}_X \sim N(p, \frac{p(1-p)}{n})$ & $n > 30$ \\
 & & & $\hat{p}_X \sim t_{n-1}$ & poblaci\'o normal, $n \leq 30$ 
\end{tabular}

\vskip 0.7 cm
\textbf{Intervals de confian\c{c}a m\'es usuals}
\vskip 0.2 cm

\begin{tabular}{l|ll}
Par\`ametre mostral & Interval de confian\c{c}a & \\
\hline
& & \\
Mitjana & $\displaystyle \bar{X} \pm z_{\alpha/2} \frac{\sigma}{\sqrt{n}}$ & poblaci\'o normal,
$\sigma$ conegut \\
& & \\
 & $\displaystyle \bar{X} \pm t_{n-1, \alpha/2} \frac{\hat{s}_X}{\sqrt{n}}$ & poblaci\'o normal,
$\sigma$ desconegut  \\
& & i $n \leq 30$\\
& & \\
& $\displaystyle \bar{X} \pm z_{\alpha/2} \frac{\hat{s}_X}{\sqrt{n}}$ & si $n > 30$ \\
& & \\
& & \\
Vari\`ancia & $\displaystyle \left[ \frac{n-1}{\chi^2_{n-1, 1-\alpha/2}} \hat{s}_X^2,
 \frac{n-1}{\chi^2_{n-1, \alpha/2}} \hat{s}_X^2 \right]$ & si la poblaci\'o segueix una llei normal  \\
& & \\
& & \\
Proporci\'o & $\displaystyle \hat{p}_X \pm z_{\alpha/2} \sqrt{\frac{\hat{p}_X (1-\hat{p}_X)}{n}}$ & si $n > 30$ \\
& & 
\end{tabular}

}

\end{document}

