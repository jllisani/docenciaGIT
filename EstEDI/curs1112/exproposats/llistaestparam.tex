\documentclass[11pt]{article}
%\usepackage[active]{srcltx}      	%% necesario para pasar del dvi al tex

\usepackage[utf8]{inputenc}		%%para utilizar tildes en el texto
\usepackage[spanish]{babel}		%%corta las palabras segun el castellano, pero pone comas en los puntos
							%% decimales
\usepackage{amsmath}			%% AMS-LaTeX	
\usepackage{amsfonts}
\usepackage{amssymb}			%%simbolos del AMS-LaTeX

\setlength{\topmargin}{-2cm}
\setlength{\textwidth}{16cm}
\setlength{\textheight}{24cm}
\setlength{\oddsidemargin}{0cm}


\newcounter{prbcont}
\stepcounter{prbcont}
\setcounter{prbcont}{0}
\newtheorem{problema}[prbcont]{Problema}
\newtheorem{ejemplo}[prbcont]{Ejemplo}
\newcount\problemes


\newcommand\probl{\advance\problemes by 1 \vskip 2ex\noindent{\bf\the\problemes) }}

\def\probl{\advance\problemes by 1
\vskip 1mm\noindent{\bf \the\problemes) }}
\newcounter{pepe}

\newcommand{\pr}[1]{P(#1)}

%\newcounter{problema}
%\newcommand{\prb}{\addtocounter{problema}{1}
%\noindent\vskip 2mm {\textbf{\theproblemes  }}
\newcommand{\sol}[1]{{\textbf{\footnotetext[\the\problemes]{Sol.: #1} }}}


\begin{document}


\begin{center}
\textbf{{\large {Aplicacions Estad\'{i}stiques. }\\}}
\vspace{0.5cm}
\textbf{Enginyeria Edificaci\'o. Estimaci\'o de par\`ametres i contrast d'hip\`otesis.}
\end{center}


\begin{problema}
Calculeu una estimaci\'o puntual de $\mu$ i l'error est\`andard estimat a partir d'una mostra
aleat\`oria de la qual se sap:
\[ n = 81;\quad \sum_{i=1}^n x_i = 1752;\quad \sum_{i=1}^n (x_i - \bar{x})^2 = 235 \]
\end{problema}

\begin{problema}
S'ha enregistrat durant 100 serveis ordinaris el temps que tarda un autob\'us en realitzar
el seu trajecte habitual. S'ha trobat que $\bar{x} = 92$ minuts i $\hat{S}_{X} = 7$ minuts. Calculeu:
\begin{itemize}
\item [(a)] Una estimaci\'o puntual de la mitjana del temps que tarda l'autob\'us en cobrir el
seu trajecte.
\item [(b)] L'error est\`andard de l'estimaci\'o realitzada.
\end{itemize}
\end{problema}

\begin{problema}
Els errors aleatoris que es produeixen en les pesades que es realitzen amb una determinada
balan\c{c}a es distribueixen segons una llei normal de mitjana 0 i desviaci\'o t\`{\i}pica
$0.5$ decigrams.
\begin{itemize}
\item [(a)] Calculeu l'error m\`axim, per defecte i per exc\'es, que es pot produir en una pesada,
amb probabilitat $0.99.$ %[1.29 dg]
\item [(b)] Si es fan 10 pesades d'un mateix objecte i es pren com a pes la mitjana de les 10
pesades, calculeu l'error m\`axim d'aquest pes final, amb probabilitat 0.99. %[0.41 dg]
\item [(c)] Calculeu el nombre m\'{\i}nim $n$ de pesades que cal realitzar d'un mateix objecte per
tal que, si es pren com a pes la mitjana de les $n$ pesades, l'error m\`axim d'aquest
pes final sigui inferior a $0.1$ decigram amb probabilitat $0.99.$ %[n = 166]
\end{itemize}
\end{problema}

\begin{problema}
Es treuen mostres aleat\`ories simples de $n = 9$ unitats cadascuna d'una poblaci\'o que
segueix una llei $N(\mu; \sigma_X^2).$
\begin{itemize}
\item [(a)] Quina probabilitat hi ha que, de 9 mostres consecutives, n'hi hagi com a m\'{\i}nim
7 que tinguin una mitjana mostral superior a $\mu$ ? %[0.0898]
\item [(b)] Quina probabilitat hi ha que dues mostres consecutives tinguin una mitjana mostral
superior al 3r quartil $Q_3$ de la poblaci\'o? %[4.6·10−4]
\end{itemize}
\end{problema}

\begin{problema}
Les puntuacions en un test que amida la variable creativitat segueixen, en la poblaci\'o general d'adolescents, una distribuci\'o normal de mitjana $11,5.$ En un centre escolar que ha implantat un programa d'estimulaci\'o de la creativitat una mostra de 30 alumnes ha proporcionat les seg\"uents puntuacions: 
\[11, 9, 12, 17, 8, 11, 9, 4, 5, 9, 14, 9, 17, 24, 19, 10, 17, 17, 8 , 23, 8, 6, 14, 16, 6, 7, 15, 20, 14, 15.\] 
A un nivell de confian\c{c}a del 95\% Pot afirmar-se que el programa \'es efectiu? 
\end{problema}



\begin{problema}
Per quant s'ha de multiplicar la mida $n$ d'una mostra si volem reduir a la quarta part
l'error est\`andard de l'estimaci\'o puntual de la mitjana $\mu$ d'una poblaci\'o? %[16]
\end{problema}

\begin{problema}
En un proc\'es d'envasat autom\`atic d'ampolles d'aigua mineral, el contingut net $X$ de les
ampolles segueix una llei normal de desviaci\'o t\'{\i}pica $\sigma_X = 2 cl.$ Per tal de controlar
el proc\'es, s'ha seleccionat una mostra de 5 ampolles i s'ha mesurat el seu contingut
net d'aigua mineral (en cl). Els resultats obtinguts s\'on: 1002, 1000, 1002, 999 i 1001.
Determineu un interval de confian\c{c}a del 99\% de la mitjana $\mu_X$ del contingut net de les
ampolles. %[(998.50,1003.10)]
\end{problema}

\begin{problema}
Un investigador estima sempre el valor de la mitjana $\mu$ d'una poblaci\'o fent servir intervals
de confian\c{c}a del 90\%. Despr\'es de 400 estimacions, quin \'es el nombre aproximat
d'intervals de confian\c{c}a que contindran el verdader valor de $\mu$? %[360]
\end{problema}

\begin{problema}
Una poblaci\'o t\'e una mitjana $\mu$ desconeguda i una desviaci\'o est\`andard igual a 5. Trobeu
la mida $n$ que ha de tenir una mostra per tal de tenir un 95\% de confian\c{c}a que
l'estimador $\bar{X}$ calculat sobre una mostra de mida $n$ est\`a dins l'interval $(\mu-1.5,\mu+1.5).$ %[43]
\end{problema}

\begin{problema}
Es vol estimar - a un nivell de confian\c{c}a del 0.99 - l'esperan\c{c}a d'una distribuci\'o normal
$N(\mu; \sigma_X^2)$ de vari\`ancia coneguda. Calculeu quina ha de ser la mida $n$ de la mostra per
que l'interval d'estimaci\'o tingui una longitud igual a $2\delta.$ (doneu el resultat en termes de $\sigma_X$ i $\delta$)%[(2.576/)2]
\end{problema}

\begin{problema}
A partir d'una mostra de mida $n = 12$ s'ha calculat un interval de confian\c{c}a del 95\%
del valor de $\mu$ i s'ha obtingut l'interval $(18.6,26.2).$ Quins s\'on els valors de $\bar{x}$ i de $\hat{s}_X$? %[22.4;5.98]
\end{problema}

\begin{problema}
Es vol estimar la mitjana $\mu$ del pes dels fulls de paper que es produeixen en una cadena
de producci\'o. S'escullen a l'atzar 22 d'aquests fulls i s'obt\'e una mitjana mostral de
$\bar{x} = 2.4 dg.$ Es demana:
\begin{itemize}
\item [(a)] Si la desviaci\'o est\`andard del pes d'una fulla de paper \'es de $0.2$ dg, trobeu un
interval de confian\c{c}a del 95\% del valor de $\mu.$ %[(2.32,2.48)]
\item [(b)] Si la desviaci\'o est\`andard \'es desconeguda per\`o la desviaci\'o est\`andard 
de la nostra mostra \'es igual a 0.2 dg, determineu l'interval de confian\c{c}a del 95\%
del valor de $\mu.$. %[(2.31,2.49)]
\end{itemize}
\end{problema}

\begin{problema}
Una empresa ha fet un test sobre 50 dels seus empleats que treballen en terminals i va
trobar que 22 fan servir l'ordinador m\'es de 6 hores al dia.
\begin{itemize}
\item [(a)] Trobeu un interval de confian\c{c}a al 95\% per a estimar la proporci\'o d'empleats que
fan servir l'ordinador m\'es de 6 hores al dia.
\item [(b)] Quina hauria de ser la mida de la mostra per a assegurar que l'amplada de l'interval fos m\'es petita que 0.1?
\end{itemize}
\end{problema}

\begin{problema}
Hom vol estimar la mitjana $\mu$ del temps d'espera dels clients en un caixer d'un gran
supermercat a una determinada hora punta. Se sap d'altres vegades que la vari\`ancia
del temps d'espera \'es aproximadament de $8.0 min^2.$ Quants clients caldr\'{\i}a controlar
si hom vol, amb un 90\% de confian\c{c}a, que el vertader valor $\mu$ difereixi com a m\`axim
en 1 min de la mitjana de la nostra mostra? %[n > 21]
\end{problema}

\begin{problema}
En una mostra aleat\`oria de 400 persones adultes d'una poblaci\'o, 260 varen afirmar que
votarien el partit A en les properes eleccions municipals. Trobeu l'interval de confian\c{c}a
del 95\% per a la vertadera proporci\'o de votants del partit A. %[(60.3%,69.7%)]
\end{problema}


\begin{problema}
Una empresa de programaci\'o ha mesurat el nombre de l\'{\i}nies de codi per programador
i dia d'una mostra de 30 programes, obtenint-se $\bar{x} = 75$ i $\hat{s}_X^2 = 90.$ Si es suposa
que $X=$``nombre de l\'{\i}nies per programador i dia'' t\'e una distribuci\'o normal, trobeu
interval de confian\c{c}a al 95\% per a $\mu$.
\end{problema}





\begin{problema}
Un proc\'es industrial fabrica peces les longituds de les quals es distribueixen segons una
llei normal $\mathcal{N}(\mu = 190 mm; \sigma = 10 mm).$ Es vol realitzar un contrast de la hip\'otesi $H_0 : \mu = 190$ vers la hip\'otesi contr\`aria a partir d'una mostra de mida $n = 5,$ a un nivell de significaci\'o $\alpha = 0.05.$ Es demana:
\begin{itemize}
\item [(a)] Indiqueu l'estad\'{\i}stic a utilitzar per a realitzar el contrast i la distribuci\'o d'aquest estad\'{\i}stic.
\item [(b)] Si la mostra obtinguda ha resultat ser igual a $187, 212, 195, 208$ i $200,$ es demana
quina decisi\'o cal prendre i el p-valor d'aquest contrast. %[p = 0.02]
\item [(c)] Suposeu ara que la vari\`ancia poblacional es desconeguda. Realitzeu novament
el contrast sobre la mostra anterior i indiqueu la decisi\'o adoptada. %[1.837 < t0.025( = 4) = 2.776]
\end{itemize}
\end{problema}

\begin{problema}
El temps de vida d'uns components electr\`onics es distribueix segons una normal.
Es vol contrastar la hip\`otesi $H_0 : \mu = 300 h.$ vers la hip\`otesi alternativa
$H_1 : \mu \neq 300 h,$ amb un nivell de significaci\'o $\alpha = 0.05.$ El contrast es vol realitzar a partir d'una mostra de mida $n = 100.$
Suposam $sigma=20$.
\begin{itemize}
\item [(a)] Indiqueu l'estad\'{\i}stic a utilitzar per realitzar el contrast i la distribuci\'o d'aquest estad\'{\i}stic.
\item [(b)] La vida mitjana d'una MAS de $100$ components electr\`onics ha resultat ser igual
a 250 h. Indiqueu quina \'es la decisi\'o que cal prendre i el p--valor d'aquest
contrast. %[H0; p= 0.096]
\end{itemize}
\end{problema}

\begin{problema}
La duraci\'o mitjana d'unes certes peces en les que intervenen diferents materials \'es
igual a 1800 h. Variant un d'aquests materials, la vida mitjana d'una mostra de 10
peces ha resultat ser igual a 2000 h amb una desviaci\'o t\'{\i}pica mostral $\hat{S}_X = 150 h.$
Creieu que el canvi del material ha incrementat significativament la vida mitjana de
les peces? %[S´ı, ja que 4 > t0.05( = 9) = 1.833]
\end{problema}

\begin{problema}
Una ag\`encia de viatges afirma que el temps per realitzar una determinada ruta \'es, en
mitjana, igual a 15 h amb $sigma = 2$ h. Per a contrastar la hip\'otesi $H_0 : \mu = 15 h.$ vers
la hip\'otesi $H_1 : \mu < 15 h.,$ es controla 25 vegades el temps de durada d'aquesta ruta.
\begin{itemize}
\item [(a)] Indiqueu l'estad\'{\i}stic a utilitzar per realitzar el contrast i la distribuci\'o d'aquest estad\'{\i}stic.
\item [(b)] Si despr\'es de realitzar 25 vegades la ruta esmentada, la mitjana del temps resulta
ser igual a $13.8 h,$ indiqueu la decisi\'o a prendre si $\alpha=0.05$ i el $p$--valor d'aquest contrast.
%[H1, ja que 3 < z0.01 = 2.33]
\end{itemize}
\end{problema}

\begin{problema}
Se sabe que la desviaci\'on t\'{\i}pica de las notas de cierto examen de Matem\'aticas es $2,4.$ Para una muestra de 36 estudiantes se obtuvo una nota media de $5,6.$ ¿Sirven estos datos para confirmar la hip\'otesis de que la nota media del examen fue de 6, con un nivel de confianza del 95\%?
\end{problema}

\begin{problema}
Les puntuacions en un test de raonament abstracte segueixen una distribuci\'o normal de mitjana $35$ i vari\`ancia $60.$ Per a avaluar un programa de millora de les capacitats intel.lectuals, a $101$ individus que estan realitzant aquest programa se'ls passa el test, obtenint-se una mitjana de 50 punts i una vari\`ancia de 80. Pot assegurar-se, a un nivell de confian\c{c}a del 90\%, que el programa incrementa les difer\`encies individuals en aquesta variable? 
\end{problema}


\begin{problema}
Un soci\'ologo ha pronosticado, que en una determinada ciudad, el nivel de abstenci\'on en las pr\'oximas elecciones ser\'a del 40\% como m\'{\i}nimo. Se elige al azar una muestra aleatoria de 200 individuos, con derecho a voto, 75 de los cuales estar\'{\i}an dispuestos a votar. Determinar con un nivel de significaci\'on del 1\%, si se puede admitir el pron\'ostico.
\end{problema}

\begin{problema}
Un informe indica que el precio medio del billete de avi\'on entre Canarias y Madrid es, como m\'aximo, de 120 euros con una desviaci\'on t\'{\i}pica de 40 euros. Se toma una muestra de 100 viajeros y se obtiene que la media de los precios de sus billetes es de 128 euros. ¿Se puede aceptar, con un nivel de significaci\'on igual a $0,1$ la afirmaci\'on de partida?
\end{problema}

\begin{problema}
Una marca de nueces afirma que, como m\'aximo, el 6\% de las nueces est\'an vac\'{\i}as. Se eligieron 300 nueces al azar y se detectaron 21 vac\'{\i}as.
\begin{itemize}
\item [(a)] Con un nivel de significaci\'on del 1\%, ?` se puede aceptar la afirmaci\'on de la marca ?
\item [(b)] Si se mantiene el porcentaje muestral de nueces que est\'an vac\'{\i}as y $1-\alpha = 0.95,$ ?` qu\'e tama\~no muestral se necesitar\'{\i}a para estimar la proporci\'on de nueces con un error menor del 1\% por ciento?
\end{itemize}
\end{problema}

\begin{problema}
Un banc vol analitzar si les comissions que cobra als seus clients per operacions en el mercat borsari difereixen significativament de les quals cobra la compet\`encia, la mitjana de la qual \'es de 12 euros mensuals amb una desviaci\'o est\`andard de $4,3$ euros. Aquest banc pren una mostra de 64 operacions bors\`aries i observa que la comissi\'o terme mitj\`a \'es de $13,6$ euros. Contrastar, al nivell de significaci\'o del 5\%, que aquest banc no difereix significativament en el cobrament de les comissions per operacions en la Borsa pel que fa a la compet\`encia 
\end{problema}


\begin{problema}
La duraci\'on de la bombillas de 100 W que fabrica una empresa sigue una distribuci\'on normal con una desviaci\'on t\'{\i}pica de 120 horas de duraci\'on. Su vida media est\'a garantizada durante un m\'{\i}nimo de 800 horas. Se escoge al azar una muestra de 50 bombillas de un lote y, despu\'es de comprobarlas, se obtiene una vida media de 750 horas. Con un nivel de significaci\'on de $0,01$ ?` habr\'{\i}a que rechazar el lote por no cumplir la garant\'{\i}a?
\end{problema}

\begin{problema}
Un fabricante de l\'amparas el\'ectricas est\'a ensayando un nuevo m\'etodo de producci\'on que se considerar\'a aceptable si las l\'amparas obtenidas por este m\'etodo dan lugar a una poblaci\'on normal de duraci\'on media 2400 horas, con una desviaci\'on t\'{\i}pica igual a 300. Se toma una muestra de 100 l\'amparas producidas por este m\'etodo y esta muestra tiene una duraci\'on media de 2320 horas. ?` Se puede aceptar la hip\'otesis de validez del nuevo proceso de fabricaci\'on con un riesgo igual o menor al 5\%?
\end{problema}

\begin{problema}
El control de calidad una f\'abrica de pilas y bater\'{\i}as sospecha que hubo defectos en la producci\'on de un modelo de bater\'{\i}a para tel\'efonos m\'oviles, bajando su tiempo de duraci\'on. Hasta ahora el tiempo de duraci\'on en conversaci\'on segu\'{\i}a una distribuci\'on normal con media 300 minutos y desviaci\'on t\'{\i}pica 30 minutos. Sin embargo, en la inspecci\'on del \'ultimo lote producido, antes de enviarlo al mercado, se obtuvo que de una muestra de 60 bater\'{\i}as el tiempo medio de duraci\'on en conversaci\'on fue de 290 minutos. Suponiendo que ese tiempo sigue siendo Normal con la misma desviaci\'on t\'{\i}pica, ?` se puede concluir que las sospechas del control de calidad son ciertas a un nivel de significaci\'on del 2\%?
\end{problema}

\begin{problema}
La directora del departament de personal d'una important corporaci\'o est\`a reclutant un gran nombre d'empleats per a un lloc en l'estranger. Durant el proc\'es de selecci\'o, l'administraci\'o li pregunta com van les coses, i ella respon que creu que la puntuaci\'o terme mitj\`a en la prova d'aptituds ser\'a d'aproximadament 90 punts. Quan l'administraci\'o revisa 19 dels resultats de la prova compilats, troba que la puntuaci\'o mitja \'es 83,24 i la desviaci\'o est\`andard d'aquesta puntuaci\'o \'es 11. Si l'administraci\'o desitja provar la hip\`otesi al nivell de significaci\'o del 10\%, Quin \'es el valor de l'estad\'{\i}stic de contrast i la seva p-valor?
\end{problema}

\begin{problema}
Un portal e-business sap que el 60\% de tots els seus visitants a la web estan interessats a adquirir els seus productes per\`o s\'on poc inclinats al comer\c{c} electr\`onic i no realitzen finalment la compra via Internet. No obstant aix\`o, en l'adre\c{c}a del portal es pensa que en l'\'ultim any, el percentatge de gent que est\`a disposada a comprar per Internet ha augmentat i aix\`o s'ha de reflectir en els seus resultats empresarials. Contrastar al nivell de significaci\'o del 2\% si en l'\'ultim any s'ha redu\"{\i}t el percentatge de gent que no est\`a disposada a comprar per Internet, si per a aix\`o es va prendre una mostra de 500 visitants per a con\`eixer la seva opini\'o i es va observar que el 55\% no estava disposat a realitzar compres via on-line. 
Suposau que la distribuci\'o del nombre de visitants del portal que no estan disposats a comprar via internet es va a aproximar a una normal.
\end{problema}


%\begin{problema}
%El Dpto. de M\`arqueting d'una empresa europea vol analitzar l'efic\`acia de la seva for\c{c}a de vendes. Per a aix\`o va prendre una mostra de 150 comercials repartits per les seves v\`aries delegacions a Europa i es va obtenir en euros el que cada comercial ha facturat en els \'ultims sis mesos. S'ha comprovat que, fins a ara, que el volum facturat per la for\c{c}a de vendes fins a ara seguia una distribuci\'o aproximadament normal de mitjana 165.000 euros i desviaci\'o t\'{\i}pica de 45.000 euros. Realitzar un contrast d'hip\`otesi bilateral sobre la mitjana de la poblaci\'o per a un nivell de significaci\'o $\alpha=0,05.$ Realitzar un contrast similar a l'anterior suposant que aquesta vegada no coneixem la desviaci\'o t\'{\i}pica? 
%\end{problema}


\begin{problema}
Se cree que el nivel medio de protombina en una poblaci\'on normal es de 20 mg/100 ml de plasma con una desviaci\'on t\'{\i}pica de 4 miligramos/100 ml. Para comprobarlo, se toma una muestra de 40 individuos en los que la media es de $18.5$ mg/100 ml. ?` Se puede aceptar la hip\'otesis, con un nivel de significaci\'on del 5\%? 
\end{problema}


\end{document}
