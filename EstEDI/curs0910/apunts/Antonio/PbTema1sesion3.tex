\documentclass[compress,red,mathsans,10pt]{beamer}
\usepackage{beamerthemesplit}
\usepackage{amssymb}
\usepackage{multirow}
%\usetheme{Antibes}
\usepackage{pgf,pgfarrows,pgfnodes}

\setbeamercolor{uppercol}{fg=white,bg=purple}%
\setbeamercolor{lowercol}{fg=black,bg=pink}%
\setbeamerfont{child}{size=\scriptsize}%
\setbeamerfont{normal}{size=\normalsize}%

\usecolortheme{lily}
\begin{document}
\title{Aplicacions Estad\'{\i}stiques}
\subtitle{Enginyeria Edificaci\'o 2009/10.}  
\author{Antonio E. Teruel}
\date{}

\frame{\titlepage} 
 
\frame{\frametitle{Exercici 1} 
En una enquesta sobre inmigraci\'o s'han obtingut les seg\"uents dades sobre la nacionalitat de 1000 persones:
\begin{center}
\begin{tabular}{c|c}
Nacionalitat & Quantitat \\ \hline
Col\'ombia & 350 \\
Equador & 250 \\
Per\'u & 120 \\
Argentina & 100 \\
Romania & 80 \\
Marroc & 70 \\
Senegal & 30
\end{tabular}
\end{center}
\begin{enumerate}[a)]
\item <1-> Quins estad\'{\i}stics de tend\`encia central es poden calcular per aquesta distribuci\'o? Calculau-los.
\item[] <2-> \textbf{Resp.} Moda. Moda=Col\'ombia.
\vspace{4cm}
\end{enumerate}
}

\frame{\frametitle{Exercici 2}
En una enquesta entre els estudiants de la UIB s'han obtingut les seg\"uents dades sobre la seva edat:
\vspace{1.30cm}
\usebeamerfont{child}
\begin{center}
\begin{tabular}{|c|c|} \hline
Edat & Quantitat \\ \hline
18 & 120 \\ \hline
19 & 150 \\ \hline
20 & 90 \\ \hline
21 & 70 \\ \hline
22 & 65 \\ \hline
23 & 50 \\ \hline
24 & 30 \\ \hline
25 & 20 \\ \hline
26 & 10 \\ \hline
27 & 7 \\ \hline
28 & 8 \\ \hline
29 & 2 \\ \hline
30 & 1 \\ \hline
34 & 1 \\ \hline
35 & 1 \\ \hline
40 & 1 \\ \hline
\end{tabular}
\end{center}
\usebeamerfont{normal}
}

\frame{\frametitle{Exercici 2 (cont.)}
\begin{itemize}
\item [(a)] Calcula la taula de freq\"u\`encies.
\item[]<2->\textbf{Resp.}
\item[]
\item[]
\end{itemize}
\only<1>{
\usebeamerfont{child}
\begin{center}
\begin{tabular}{|c|c|} \hline
Edat & Quantitat \\ \hline
18 & 120 \\ \hline
19 & 150 \\ \hline
20 & 90 \\ \hline
21 & 70 \\ \hline
22 & 65 \\ \hline
23 & 50 \\ \hline
24 & 30 \\ \hline
25 & 20 \\ \hline
26 & 10 \\ \hline
27 & 7 \\ \hline
28 & 8 \\ \hline
29 & 2 \\ \hline
30 & 1 \\ \hline
34 & 1 \\ \hline
35 & 1 \\ \hline
40 & 1 \\ \hline
\end{tabular}
\end{center}
\usebeamerfont{normal}}
\only<2->{
\usebeamerfont{child}
\begin{center}
\begin{tabular}{|c|c|c|c|c|c|c|} \hline
$x_i$ & $n_i$ & $N_i$ & $f_i$ &$F_i$ & $p_i$ &$ P_i$ \\ \hline
18&120  &120	&0,1917	&0,1917	&19,1693 &19,1693  \\ \hline
19&150	&270	&0,2396	&0,4313	&23,9617 &43,1310 \\ \hline
20&90	&360	&0,1438	&0,5751	&14,3770 &57,5080 \\ \hline
21&70	&430	&0,1118	&0,6869	&11,1821 &68,6901 \\ \hline
22&65	&495	&0,1038	&0,7907	&10,3834 &79,0735 \\ \hline
23&50	&545	&0,0799	&0,8706	&7,9872	 &87,0607 \\ \hline
24&30	&575	&0,0479	&0,9185	&4,7923	 &91,8530 \\ \hline
25&20	&595	&0,0319	&0,9505	&3,1949	 &95,0479 \\ \hline
26&10	&605	&0,0160	&0,9665	&1,5974	 &96,6454 \\ \hline
27&7	&612	&0,0112	&0,9776	&1,1182	 &97,7636 \\ \hline
28&8	&620	&0,0128	&0,9904	&1,2780	 &99,0415 \\ \hline
29&2	&622	&0,0032	&0,9936	&0,3195	 &99,3610 \\ \hline
30&1	&623	&0,0016	&0,9952	&0,1597	 &99,5208 \\ \hline
34&1	&624	&0,0016	&0,9968	&0,1597	 &99,6805 \\ \hline
35&1	&625	&0,0016	&0,9984	&0,1597	 &99,8403 \\ \hline
40&1	&626	&0,0016	&1,0000	&0,1597	 &100,0000 \\ \hline
\end{tabular}
\end{center}
\usebeamerfont{normal}}
}


\frame{\frametitle{Exercici 2 (cont.)}
\begin{itemize}
\item [(a)] Calcula la moda
\item[]<3-> \textbf{Resp.} moda=$19$ 
\item[]
\item[]
\end{itemize}
\usebeamerfont{child}
\begin{center}
\begin{tabular}{|c|c|c|c|c|c|c|} \hline
$x_i$ & $n_i$ & $N_i$ & $f_i$ &$F_i$ & $p_i$ &$ P_i$ \\ \hline
18&120  &120	&0,1917	&0,1917	&19,1693 &19,1693  \\ \hline
19&150	&270	&0,2396	&0,4313	&23,9617 &43,1310 \\ \hline
20&90	&360	&0,1438	&0,5751	&14,3770 &57,5080 \\ \hline
21&70	&430	&0,1118	&0,6869	&11,1821 &68,6901 \\ \hline
22&65	&495	&0,1038	&0,7907	&10,3834 &79,0735 \\ \hline
23&50	&545	&0,0799	&0,8706	&7,9872	 &87,0607 \\ \hline
24&30	&575	&0,0479	&0,9185	&4,7923	 &91,8530 \\ \hline
25&20	&595	&0,0319	&0,9505	&3,1949	 &95,0479 \\ \hline
26&10	&605	&0,0160	&0,9665	&1,5974	 &96,6454 \\ \hline
27&7	&612	&0,0112	&0,9776	&1,1182	 &97,7636 \\ \hline
28&8	&620	&0,0128	&0,9904	&1,2780	 &99,0415 \\ \hline
29&2	&622	&0,0032	&0,9936	&0,3195	 &99,3610 \\ \hline
30&1	&623	&0,0016	&0,9952	&0,1597	 &99,5208 \\ \hline
34&1	&624	&0,0016	&0,9968	&0,1597	 &99,6805 \\ \hline
35&1	&625	&0,0016	&0,9984	&0,1597	 &99,8403 \\ \hline
40&1	&626	&0,0016	&1,0000	&0,1597	 &100,0000 \\ \hline
\end{tabular}
\end{center}
\usebeamerfont{normal}
\begin{picture}(0,0)
\only<2->{\put(80,133){\color{red}\circle{15}\color{black}}}
\end{picture}
}

\frame{\frametitle{Exercici 2 (cont.)}
\begin{itemize}
\item [(a)] Calcula la mediana
\item[]<3-> \textbf{Resp.} $n=626,$ mediana$=\dfrac{x_{313}+x_{314}}{2}=\dfrac{20+20}{2}=20$  
\item[]
\item[]
\end{itemize}
\usebeamerfont{child}
\begin{center}
\begin{tabular}{|c|c|c|c|c|c|c|} \hline
$x_i$ & $n_i$ & $N_i$ & $f_i$ &$F_i$ & $p_i$ &$ P_i$ \\ \hline
18&120  &120	&0,1917	&0,1917	&19,1693 &19,1693  \\ \hline
19&150	&270	&0,2396	&0,4313	&23,9617 &43,1310 \\ \hline
20&90	&360	&0,1438	&0,5751	&14,3770 &57,5080 \\ \hline
21&70	&430	&0,1118	&0,6869	&11,1821 &68,6901 \\ \hline
22&65	&495	&0,1038	&0,7907	&10,3834 &79,0735 \\ \hline
23&50	&545	&0,0799	&0,8706	&7,9872	 &87,0607 \\ \hline
24&30	&575	&0,0479	&0,9185	&4,7923	 &91,8530 \\ \hline
25&20	&595	&0,0319	&0,9505	&3,1949	 &95,0479 \\ \hline
26&10	&605	&0,0160	&0,9665	&1,5974	 &96,6454 \\ \hline
27&7	&612	&0,0112	&0,9776	&1,1182	 &97,7636 \\ \hline
28&8	&620	&0,0128	&0,9904	&1,2780	 &99,0415 \\ \hline
29&2	&622	&0,0032	&0,9936	&0,3195	 &99,3610 \\ \hline
30&1	&623	&0,0016	&0,9952	&0,1597	 &99,5208 \\ \hline
34&1	&624	&0,0016	&0,9968	&0,1597	 &99,6805 \\ \hline
35&1	&625	&0,0016	&0,9984	&0,1597	 &99,8403 \\ \hline
40&1	&626	&0,0016	&1,0000	&0,1597	 &100,0000 \\ \hline
\end{tabular}
\end{center}
\usebeamerfont{normal}
\begin{picture}(0,0)
\only<2->{\put(163,135){\color{red}\oval(30,20)\color{black}}}
\end{picture}
}

\frame{\frametitle{Exercici 2 (cont.)}
\begin{itemize}
\item [(a)] Calcula els quartils $Q_1$ i $Q_3.$
\item[]<3-> \textbf{Resp.} $k=\dfrac{25*626}{100}=156.5 \Rightarrow Q_1=P_{25}=d_{157}=19$
\item[]<5-> $k=\dfrac{75*626}{100}=469.5 \Rightarrow Q_3=P_{75}=d_{470}=22$
\item[]
\end{itemize}
\usebeamerfont{child}
\begin{center}
\begin{tabular}{|c|c|c|c|c|c|c|} \hline
$x_i$ & $n_i$ & $N_i$ & $f_i$ &$F_i$ & $p_i$ &$ P_i$ \\ \hline
18&120  &120	&0,1917	&0,1917	&19,1693 &19,1693  \\ \hline
19&150	&270	&0,2396	&0,4313	&23,9617 &43,1310 \\ \hline
20&90	&360	&0,1438	&0,5751	&14,3770 &57,5080 \\ \hline
21&70	&430	&0,1118	&0,6869	&11,1821 &68,6901 \\ \hline
22&65	&495	&0,1038	&0,7907	&10,3834 &79,0735 \\ \hline
23&50	&545	&0,0799	&0,8706	&7,9872	 &87,0607 \\ \hline
24&30	&575	&0,0479	&0,9185	&4,7923	 &91,8530 \\ \hline
25&20	&595	&0,0319	&0,9505	&3,1949	 &95,0479 \\ \hline
26&10	&605	&0,0160	&0,9665	&1,5974	 &96,6454 \\ \hline
27&7	&612	&0,0112	&0,9776	&1,1182	 &97,7636 \\ \hline
28&8	&620	&0,0128	&0,9904	&1,2780	 &99,0415 \\ \hline
29&2	&622	&0,0032	&0,9936	&0,3195	 &99,3610 \\ \hline
30&1	&623	&0,0016	&0,9952	&0,1597	 &99,5208 \\ \hline
34&1	&624	&0,0016	&0,9968	&0,1597	 &99,6805 \\ \hline
35&1	&625	&0,0016	&0,9984	&0,1597	 &99,8403 \\ \hline
40&1	&626	&0,0016	&1,0000	&0,1597	 &100,0000 \\ \hline
\end{tabular}
\end{center}
\usebeamerfont{normal}
\begin{picture}(0,0)
\only<2-3>{\put(240,138){\color{red}\oval(30,12)\color{black}}}
\only<4->{\put(240,125){\color{red}\oval(30,37)\color{black}}}
\end{picture}
}

\frame{\frametitle{Exercici 2 (cont.)}
\begin{itemize}
\item [(a)] Calcula els percentils $P_{30}$ i $P_{60}.$
\item[]<3-> \textbf{Resp.} $k=\dfrac{30*626}{100}=187.80.5 \Rightarrow P_{30}=d_{188}=19$
\item[]<5-> $k=\dfrac{60*626}{100}=375.6\Rightarrow P_{60}=d_{376}=21$
\item[]
\end{itemize}
\usebeamerfont{child}
\begin{center}
\begin{tabular}{|c|c|c|c|c|c|c|} \hline
$x_i$ & $n_i$ & $N_i$ & $f_i$ &$F_i$ & $p_i$ &$ P_i$ \\ \hline
18&120  &120	&0,1917	&0,1917	&19,1693 &19,1693  \\ \hline
19&150	&270	&0,2396	&0,4313	&23,9617 &43,1310 \\ \hline
20&90	&360	&0,1438	&0,5751	&14,3770 &57,5080 \\ \hline
21&70	&430	&0,1118	&0,6869	&11,1821 &68,6901 \\ \hline
22&65	&495	&0,1038	&0,7907	&10,3834 &79,0735 \\ \hline
23&50	&545	&0,0799	&0,8706	&7,9872	 &87,0607 \\ \hline
24&30	&575	&0,0479	&0,9185	&4,7923	 &91,8530 \\ \hline
25&20	&595	&0,0319	&0,9505	&3,1949	 &95,0479 \\ \hline
26&10	&605	&0,0160	&0,9665	&1,5974	 &96,6454 \\ \hline
27&7	&612	&0,0112	&0,9776	&1,1182	 &97,7636 \\ \hline
28&8	&620	&0,0128	&0,9904	&1,2780	 &99,0415 \\ \hline
29&2	&622	&0,0032	&0,9936	&0,3195	 &99,3610 \\ \hline
30&1	&623	&0,0016	&0,9952	&0,1597	 &99,5208 \\ \hline
34&1	&624	&0,0016	&0,9968	&0,1597	 &99,6805 \\ \hline
35&1	&625	&0,0016	&0,9984	&0,1597	 &99,8403 \\ \hline
40&1	&626	&0,0016	&1,0000	&0,1597	 &100,0000 \\ \hline
\end{tabular}
\end{center}
\usebeamerfont{normal}
\begin{picture}(0,0)
\only<2-3>{\put(240,138){\color{red}\oval(30,12)\color{black}}}
\only<4->{\put(240,130){\color{red}\oval(30,29)\color{black}}}
\end{picture}
}


\frame{\frametitle{Exercici 2 (cont.)}
\begin{itemize}
\item [(a)] Calcula la mitjana.
\item[]<2-> \textbf{Resp.} $\bar{x}=\dfrac{18*120+19*150+\ldots+40*1}{626}=20.69$
\item[]
\item[]
\end{itemize}
\usebeamerfont{child}
\begin{center}
\begin{tabular}{|c|c|c|c|c|c|c|} \hline
$x_i$ & $n_i$ & $N_i$ & $f_i$ &$F_i$ & $p_i$ &$ P_i$ \\ \hline
18&120  &120	&0,1917	&0,1917	&19,1693 &19,1693  \\ \hline
19&150	&270	&0,2396	&0,4313	&23,9617 &43,1310 \\ \hline
20&90	&360	&0,1438	&0,5751	&14,3770 &57,5080 \\ \hline
21&70	&430	&0,1118	&0,6869	&11,1821 &68,6901 \\ \hline
22&65	&495	&0,1038	&0,7907	&10,3834 &79,0735 \\ \hline
23&50	&545	&0,0799	&0,8706	&7,9872	 &87,0607 \\ \hline
24&30	&575	&0,0479	&0,9185	&4,7923	 &91,8530 \\ \hline
25&20	&595	&0,0319	&0,9505	&3,1949	 &95,0479 \\ \hline
26&10	&605	&0,0160	&0,9665	&1,5974	 &96,6454 \\ \hline
27&7	&612	&0,0112	&0,9776	&1,1182	 &97,7636 \\ \hline
28&8	&620	&0,0128	&0,9904	&1,2780	 &99,0415 \\ \hline
29&2	&622	&0,0032	&0,9936	&0,3195	 &99,3610 \\ \hline
30&1	&623	&0,0016	&0,9952	&0,1597	 &99,5208 \\ \hline
34&1	&624	&0,0016	&0,9968	&0,1597	 &99,6805 \\ \hline
35&1	&625	&0,0016	&0,9984	&0,1597	 &99,8403 \\ \hline
40&1	&626	&0,0016	&1,0000	&0,1597	 &100,0000 \\ \hline
\end{tabular}
\end{center}
\usebeamerfont{normal}
}


\frame{\frametitle{Exercici 2 (cont.)}
\begin{itemize}
\item [(b)] Repetiu l'apartat anterior per\'o amb les dades agrupades en els seg\"uents intervals: ``Menors de $21$'', 
$[21, 23)$, $[23, 25)$, $[25, 27)$, ``Majors de $27$''.
\item[] <2->\textbf{Resp.}
\item[] <3->Moda 
\item[] <4->Menor de 21 anys
\end{itemize}
\only<1>{
\usebeamerfont{child}
\begin{center}
\begin{tabular}{|c|c|} \hline
Edat & Quantitat \\ \hline
18 & 120 \\ \hline
19 & 150 \\ \hline
20 & 90 \\ \hline
21 & 70 \\ \hline
22 & 65 \\ \hline
23 & 50 \\ \hline
24 & 30 \\ \hline
25 & 20 \\ \hline
26 & 10 \\ \hline
27 & 7 \\ \hline
28 & 8 \\ \hline
29 & 2 \\ \hline
30 & 1 \\ \hline
34 & 1 \\ \hline
35 & 1 \\ \hline
40 & 1 \\ \hline
\end{tabular}
\end{center}
\usebeamerfont{normal}}
\only<2->{\vspace{1.5cm}
\usebeamerfont{child}
\begin{center}
\begin{tabular}{|c|c|c|c|c|c|c|c|} \hline
$[x_i,x_i+1]$&	$m_i$&	$n_i$	&$N_i$	&$f_i$	&$F_i$	&$p_i$	&$P_i$ \\ \hline
[18,21) &19	&360	&360	&0,5751	&0,5751	&57,5080&	57,5080 \\ \hline
[21,23)	&22	&135	&495	&0,2157	&0,7907	&21,5655&	79,0735 \\ \hline
[23,25)	&24	&80	&575	&0,1278	&0,9185	&12,7796&	91,8530 \\ \hline
[25,27)	&26	&30	&605	&0,0479	&0,9665	&4,7923	&96,6454 \\ \hline
[27,40)	&33	&21	&626	&0,0335	&1,0000	&3,3546	&100,0000 \\ \hline
\end{tabular}
\end{center}
\usebeamerfont{normal}}
}


\frame{\frametitle{Exercici 2 (cont.)}
\begin{itemize}
\item [(b)] Repetiu l'apartat anterior per\'o amb les dades agrupades en els seg\"uents intervals: ``Menors de $21$'', 
$[21, 23)$, $[23, 25)$, $[25, 27)$, ``Majors de $27$''.
\item[] <1->\textbf{Resp.}
\item[] <1->Mediana=$P_{50}$ 
\item[] <2-> $[x_k,x_k+1] \Rightarrow P_p=x_k+\dfrac{\frac{p*n}{100}-N_{k-1}}{n_k}(x_{k+1}-x_k)$
\item[] <3-> $[18,21] \Rightarrow P_{50}=18+\dfrac{0.5*626-0}{360}(21-18)=20,60833$
\item[]
\item[]
\end{itemize}
\usebeamerfont{child}
\begin{center}
\begin{tabular}{|c|c|c|c|c|c|c|c|} \hline
$[x_i,x_i+1]$&	$m_i$&	$n_i$	&$N_i$	&$f_i$	&$F_i$	&$p_i$	&$P_i$ \\ \hline
[18,21) &19	&360	&360	&0,5751	&0,5751	&57,5080&	57,5080 \\ \hline
[21,23)	&22	&135	&495	&0,2157	&0,7907	&21,5655&	79,0735 \\ \hline
[23,25)	&24	&80	&575	&0,1278	&0,9185	&12,7796&	91,8530 \\ \hline
[25,27)	&26	&30	&605	&0,0479	&0,9665	&4,7923	&96,6454 \\ \hline
[27,40)	&33	&21	&626	&0,0335	&1,0000	&3,3546	&100,0000 \\ \hline
\end{tabular}
\end{center}
\usebeamerfont{normal}
}

\frame{\frametitle{Exercici 2 (cont.)}
\begin{itemize}
\item [(b)] Repetiu l'apartat anterior per\'o amb les dades agrupades en els seg\"uents intervals: ``Menors de $21$'', 
$[21, 23)$, $[23, 25)$, $[25, 27)$, ``Majors de $27$''.
\item[] <1->\textbf{Resp.}
\item[] <1->Quartils $Q_1=P_{25}$ i $Q_3=P_{75}$ 
\item[] <2-> $[x_k,x_k+1] \Rightarrow P_p=x_k+\dfrac{\frac{p*n}{100}-N_{k-1}}{n_k}(x_{k+1}-x_k)$
\item[] <3-> $[18,21] \Rightarrow Q_1=18+\dfrac{0.25*626-0}{360}(21-18)=19,3041$
\item[] <4-> $[21,23] \Rightarrow Q_3=21+\dfrac{0.75*626-360}{360}(23-21)=21,6083$
\item[]
\end{itemize}
\usebeamerfont{child}
\begin{center}
\begin{tabular}{|c|c|c|c|c|c|c|c|} \hline
$[x_i,x_i+1]$&	$m_i$&	$n_i$	&$N_i$	&$f_i$	&$F_i$	&$p_i$	&$P_i$ \\ \hline
[18,21) &19	&360	&360	&0,5751	&0,5751	&57,5080&	57,5080 \\ \hline
[21,23)	&22	&135	&495	&0,2157	&0,7907	&21,5655&	79,0735 \\ \hline
[23,25)	&24	&80	&575	&0,1278	&0,9185	&12,7796&	91,8530 \\ \hline
[25,27)	&26	&30	&605	&0,0479	&0,9665	&4,7923	&96,6454 \\ \hline
[27,40)	&33	&21	&626	&0,0335	&1,0000	&3,3546	&100,0000 \\ \hline
\end{tabular}
\end{center}
\usebeamerfont{normal}
}

\frame{\frametitle{Exercici 2 (cont.)}
\begin{itemize}
\item [(b)] Repetiu l'apartat anterior per\'o amb les dades agrupades en els seg\"uents intervals: ``Menors de $21$'', 
$[21, 23)$, $[23, 25)$, $[25, 27)$, ``Majors de $27$''.
\item[] <1->\textbf{Resp.}
\item[] <1-> Percentils $P_{30}$ i $Q_3=P_{60}$ 
\item[] <2-> $[x_k,x_k+1] \Rightarrow P_p=x_k+\dfrac{\frac{p*n}{100}-N_{k-1}}{n_k}(x_{k+1}-x_k)$
\item[] <3-> $[18,21] \Rightarrow P_{30}=18+\dfrac{0.30*626-0}{360}(21-18)=19,565$
\item[] <4-> $[21,23] \Rightarrow P_{60}=21+\dfrac{0.60*626-360}{360}(23-21)=21,0866$
\item[]
\end{itemize}
\usebeamerfont{child}
\begin{center}
\begin{tabular}{|c|c|c|c|c|c|c|c|} \hline
$[x_i,x_i+1]$&	$m_i$&	$n_i$	&$N_i$	&$f_i$	&$F_i$	&$p_i$	&$P_i$ \\ \hline
[18,21) &19	&360	&360	&0,5751	&0,5751	&57,5080&	57,5080 \\ \hline
[21,23)	&22	&135	&495	&0,2157	&0,7907	&21,5655&	79,0735 \\ \hline
[23,25)	&24	&80	&575	&0,1278	&0,9185	&12,7796&	91,8530 \\ \hline
[25,27)	&26	&30	&605	&0,0479	&0,9665	&4,7923	&96,6454 \\ \hline
[27,40)	&33	&21	&626	&0,0335	&1,0000	&3,3546	&100,0000 \\ \hline
\end{tabular}
\end{center}
\usebeamerfont{normal}
}

\frame{\frametitle{Exercici 2 (cont.)}
\begin{itemize}
\item [(b)] Repetiu l'apartat anterior per\'o amb les dades agrupades en els seg\"uents intervals: ``Menors de $21$'', 
$[21, 23)$, $[23, 25)$, $[25, 27)$, ``Majors de $27$''.
\item[] <1->\textbf{Resp.}
\item[] <1-> Mitjana  
\item[] <2-> $\bar{x}=\dfrac{m_1 n_1+m_2n_2+\ldots m_kn_k}{n}$
\item[] <3-> $\bar{x}=\dfrac{19*360+22*135+24*80+26*30+33*21}{626}=21,091$
\item[] 
\item[]
\end{itemize}
\usebeamerfont{child}
\begin{center}
\begin{tabular}{|c|c|c|c|c|c|c|c|} \hline
$[x_i,x_i+1]$&	$m_i$&	$n_i$	&$N_i$	&$f_i$	&$F_i$	&$p_i$	&$P_i$ \\ \hline
[18,21) &19	&360	&360	&0,5751	&0,5751	&57,5080&	57,5080 \\ \hline
[21,23)	&22	&135	&495	&0,2157	&0,7907	&21,5655&	79,0735 \\ \hline
[23,25)	&24	&80	&575	&0,1278	&0,9185	&12,7796&	91,8530 \\ \hline
[25,27)	&26	&30	&605	&0,0479	&0,9665	&4,7923	&96,6454 \\ \hline
[27,40)	&33	&21	&626	&0,0335	&1,0000	&3,3546	&100,0000 \\ \hline
\end{tabular}
\end{center}
\usebeamerfont{normal}
}


\end{document}