\documentclass[10pt]{report}
%\overfullrule=10pt
\usepackage[latin1]{inputenc}   % Permet usar tots els accents i car?�cters llatins de forma directa.
\usepackage[spanish]{babel}
   
\spanishdecimal{.}
%\usepackage{multimedia}linkcolor
\usepackage{latexsym}
\usepackage[pdftex,%
            colorlinks=true,
            linkcolor=cyan,
            citecolor=cyan,
            bookmarksnumbered=true,
           bookmarksopen=false,            
           pdftitle={Notas de Clase de Estadistica para Informatica de Gestion},    % title
           pdfauthor={Ricardo Alberich},     % author
            pdfsubject={UIB. 2009-2010. Informatica de Gestion}
            ]{hyperref}
\usepackage[all]{hypcap}
\usepackage{theorem}
\usepackage{enumerate}
\usepackage{amsfonts, amscd, amsmath, amssymb}
\usepackage[pdftex]{graphicx} 
\usepackage{amsmath}
\usepackage{epstopdf}
\usepackage{eurosym}
\usepackage{latexsym,theorem}
\usepackage{epsdice}


%
%
%%\usepackage{eurosans}
%%\spanishdeactivate{."'~<>}%
%%\addto\shorthandsspanish{\spanishdeactivate{'"~<>}}
%% \usepackage{pictex}
%%%%
%
%%\addto\shorthandsspanish{\spanishdeactivate{/.~<>}}
%%%%\addto\shorthandsspanish{\spanishdeactivate{.~<>@}}
%%
%%%
%%%\bbl@deactivate
%%%\usepackage[pdftex]{graphicx}
%%\usepackage[pdftex]{graphicx}
%%%\usepackage[dvips]{graphicx}
%\usepackage{latexsym,enumerate,hyperref,theorem}
%%  % \usepackage{theorem,latexsym,makeidx,enumerate}
%%  % \input 8bitdefs
%%\DeclareGraphicsExtensions{.eps,.jpg,.png,.pdf,.ps}
%
%
%%
%% Declaracions pels dibuixos i gr\`afics
%%
%% Depenent de si es compila a dvi o a pdf la cosa canvia
%% Primerament definir un nou if
%pppppp
%\newif\ifpdf
%\ifx\pdftexversion\undefined
%    \pdffalse    % no s'utilitza PDFLaTeX
%\else
%    \pdfoutput=1 % s'est\`a utilitzant PDFLaTeX
%    \pdftrue
%\fi
%
%% Incorporar el package que conv\'e
%
%%\usepackage{graphicx}
%
%\ifpdf
%    % pdf - ps
%
%    \usepackage[pdftex]{graphicx}
%    \usepackage{epstopdf}
%    \usepackage{epsdice}
%%\DeclareGraphicsExtensions{eps}
%   % \DeclareGraphicsExtensions{pstex,eps,pdf}
%   % \DeclareGraphicsRule{.eps}{eps}{*}{`epstopdf #1}
%   % \DeclareGraphicsRule{.pstex}{pdf}{*}{epstopdf #1}
%   % \DeclareGraphicsRule{.pstex}{pdf}{.pstex}{`epstopdf  #1}
%   % \DeclareGraphicsExtensions{.pstex}
%
%\else
%    % dvi
%
%    \usepackage[dvips]{graphicx}
%    \usepackage{epsfig}
%    \usepackage{epsdice}
%
%   % \DeclareGraphicsRule{*}{eps}{*}{}
%   % \DeclareGraphicsExtensions{.eps,.pstex}
%\fi % ifpdf
%pppppp
%\usepackage{color}

%\usepackage {amssymb}
%\usepackage {amsmath}

%
% incorporar packages per fer gr\`afiques i esquemes
%
%\usepackage[all]{xy}
%\usepackage{epic}
%\usepackage{eepic}

%
% Directoris on es trobaran les figures
%\graphicspath{{.}}
%
%\graphicspath{{./figures/}}
%%%%%%%%%%%%%%%%%
 \newtheorem{definition}{Definici\'on}
\newtheorem{theorem}[definition]{Teorema}
\newtheorem{proposition}[definition]{Proposici\'on}
\newtheorem{corollary}[definition]{Corolario}
\newtheorem{lemma}[definition]{Lema}
\newtheorem{example}[definition]{Ejemplo}
\newtheorem{Rem}[definition]{Nota:}
  \newcommand{\va}{variable aleatoria }
    \def\N{I\!\!N}
\def\R{I\!\!R}
\def\Z{Z\!\!\!Z}
\def\Q{O\!\!\!\!Q}
\def\C{I\!\!\!\!C}

% \setlength{\textwidth}{17cm} \setlength{\textheight}{24cm}
%  \setlength{\oddsidemargin}{-0.3cm}
%  \setlength{\evensidemargin}{1cm}
% \addtolength{\headheight}{\baselineskip}
% \addtolength{\topmargin}{-3cm}
\makeindex

%\includeonly{cap1descrip}
%\includeonly{cap1descrip,cap2probabilidad,cap3va,cap4distnot}
%\includeonly{cap3va,cap4distnot}
%\includeonly{cap2probabilidad}
%\includeonly{cap2probabilidad,cap3va,cap4distnot,cap5vavect}
%\includeonly{cap1descrip,cap2probabilidad,cap3va,cap4distnot,cap5vavect2008,
%cap6muestreo2008,cap7inferencia}
%\includeonly{cap5vavect2008,cap6muestreo2008}

%\includeonly{cap6muestreo}
%\includeonly{cap7inferencia}
%\includeonly{cap8contrateshipotesis}


%%%%Definici�n de las cabeceras y los pies de p�gina
\usepackage{fancyhdr}
\lhead{}
\chead{}
\rhead{}
\lfoot{Campus Extens}
\cfoot{}
\rfoot{ \thepage}
\renewcommand{\headheight}{15pt}
\renewcommand{\headrulewidth}{0pt}
\renewcommand{\footrulewidth}{0pt}
\pagestyle{fancy}


%%%%%%%%%%%%Para los dibujos de portada y fondo.
\usepackage{fancybox}
%\fancyput(2cm,-19cm){\includegraphics[height=24,width=17cm]{fons.pdf}}
\fancyput(-1cm,-22cm){\includegraphics[height=23cm]{fons.pdf}}



\begin{document}
%\renewcommand\textspanish{}
%%%\font\fiverm=cmr5 \font\thinlinefont=cmr5
%%%\input prepictex.tex
%%%\input pictex.tex
%%%\input postpictex.tex
%%%\newcommand{\ZZ}{{{\rm Z}\kern-.28em{\rm Z}}}
\newcommand{\RR}{\mbox{I\kern-.2em\hbox{R}}}
\font\fiverm=cmr5 \font\thinlinefont=cmr5
%%%\input prepictex.tex
%%%\input pictex.tex
%%%\input postpictex.tex
\newcommand{\ZZ}{{{\rm Z}\kern-.28em{\rm Z}}}
\documentclass{article}
\usepackage[catalan]{babel}
\usepackage[latin1]{inputenc}   % Permet usar tots els accents i car�ters llatins de forma directa.
\usepackage{enumerate}
\usepackage{amsfonts, amscd, amsmath, amssymb}
\usepackage[pdftex]{graphicx}
\usepackage{longtable}
\usepackage{url}

\setlength{\textwidth}{16cm}
\setlength{\textheight}{24.5cm}
\setlength{\oddsidemargin}{-0.3cm}
\setlength{\evensidemargin}{0.25cm} \addtolength{\headheight}{\baselineskip}
\addtolength{\topmargin}{-3cm}

\newcommand\Z{\mathbb{Z}}
\newcommand\R{\mathbb{R}}
\newcommand\N{\mathbb{N}}
\newcommand\Q{\mathbb{Q}}
\newcommand\K{\Bbbk}
\newcommand\C{\mathbb{C}}

\newcounter{exctr}
\newenvironment{exemple}
{ \stepcounter{exctr} 
\hspace{0.2cm} 
\textit{Exemple  \arabic{exctr}: }
\it
\begin{quotation}
}{\end{quotation}}

\pagestyle{empty}
\begin{document}

\textbf{\Large Pr�ctica PDS amb Scilab}

\vskip 0.3 cm
L'objectiu de la pr�ctica es utilitzar Scilab per a fer un processament b�sic d'un senyal
de veu, seguint l'esquema del document ``Exemples b�sics de Processament Digital de Senyal amb Scilab''.


Per a aix� cada alumne haur� de fer el seg�ent:

\begin{enumerate}
\item Instal.lar Scilab (\url{http://www.scilab.org/}).
\item Triar un senyal de veu o m�sica en format .wav. En podeu trobar per Internet
(\url{http://www.thefreesite.com/Free_Sounds/Free_WAVs/}
i \url{http://www.moviewavs.com/}) o els podeu crear amb
la gravadora de sons de Windows (al men� Todos los programas/Accesorios/Entretenimiento).�

�s important que cada alumne treballi amb un senyal diferent. Per aix� es crear� un forum a
Campus Extens on cada alumne anunciar� a la resta amb quin senyal est� treballant.
Abans de descarregar-vos un senyal de la web consultau el f�rum per comprovar que el
senyal no l'utilitza un altre company. A continuaci� anunciau al f�rum quin �s el vostre senyal.

\item Si el senyal t� m�s de $50000$ mostres selecionau un boc� de tamany m�xim $50000$.

\item Damunt el boc� seleccionat s'han de fer les seg�ents operacions:


\begin{enumerate}
\item Representau el senyal.
\item Dibuixau el seu espectre.
\item Digau quin �s la seva freq��ncia de mostreig.
\item Reproduiu el senyal amb freq��ncia de mostreig doble de l'original i guardau el resultat.
\item Reproduiu el senyal amb freq��ncia de mostreig meitat de l'original i guardau el resultat.
\item A cada alumne se li ha assignat una freq��ncia (cont�nua) $F_1$ amb la qual fer els seg�ents tests.
Podeu trobar l'assignaci� a Campus Extens.
Per a una banda de freq��ncies (cont�nues) entre $F_1$ i $F_1+1000$:
\begin{enumerate}
\item eliminau la banda de freq��ncies, reprodu�u el nou senyal i guardau el resultat;
\item amplificau la banda de freq��ncies (factor 2), reprodu�u el nou senyal i guardau el resultat;
\item assignau un valor constant (de la vostra elecci�) a la banda de freq��ncies, reprodu�u el nou senyal i guardau el resultat.
\end{enumerate}
\item Submostrejau el senyal original, reprodu�u-lo i guardau el resultat. Observau la DFT del senyal
i comentau si s'ha produ�t aliasing.
\item Repetiu l'apartat anterior despr�s d'aplicar un filtre anti-aliasing al senyal original.
\end{enumerate}

\item Cada alumne haur� d'escriure un breu informe on es mostrin totes les operacions 
efectuades damunt el senyal. Aix� mateix haur� de guardar els sons resultants en fitxers .wav.
L'informe i els fitxers es comprimiran en format .zip i es penjaran de Campus Extens per a
la seva avaluaci�. Teniu fins el dia de l'examen de juny (28 de juny) per presentar el treball, el qual
enguany �s voluntari i comptar� com una activitat d'avaluaci� m�s.



\end{enumerate}
 



\end{document}
%\include{cap2probabilidad}
%\include{cap3va}
%\include{cap4distnot}
%\include{cap5vavect2008}
%\include{cap6muestreo2008}
%\include{cap7inferencia}
%\include{cap8contrateshipotesis}
%\include{cap8contrateshipotesisdosparametros}

\end{document}
