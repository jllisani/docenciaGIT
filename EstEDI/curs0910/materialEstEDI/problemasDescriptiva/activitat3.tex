\documentclass[a4paper,12pt]{report}
\usepackage[latin1]{inputenc}   % Permet usar tots els accents i car�ters llatins de forma directa.
\usepackage[spanish]{babel}
%\decimalspanish{.}
\usepackage{latexsym}
\usepackage{hyperref}
\usepackage{theorem}
\usepackage{enumerate}
\usepackage{amsfonts, amscd, amsmath, amssymb}
\usepackage[pdftex]{graphicx}
\usepackage{epstopdf}
\usepackage{epsdice}

\setlength{\textheight}{23cm}
\addtolength{\topmargin}{-1.5cm}

\pagestyle{empty}
\begin{document}
\begin{center}
\textbf{\large Estad�stica Aplicada}

\textbf{\large Seguretat i Ci�ncies Policials}

\vskip 0.5 cm
\textbf{\large Activitat dirigida 3}
\end{center}

\vskip 1cm


\begin{enumerate}
\item Volem estudiar el grau de depend�ncia entre les variables ``Frac�s escolar'' i ``Situaci� familiar''
a partir de les seg�ents dades (fict�cies):

\begin{center}
\begin{tabular}{c|c|c|}
Frac�s $\backslash$ Fam�lia & Pares separats & Pares no separats \\ \hline
5 o m�s suspensos & 73 & 38 \\ \hline
4 o menys suspensos & 87 & 132 \\ \hline  
\end{tabular}
\end{center}

Calcula el coeficient de conting�ncia de les variables i raona si podem considerar les variables
independents o no.


\vskip 1cm

\item Considera les seg�ents dades de 10 alumnes d'Estad�stica

\begin{center}
\begin{tabular}{c|c}
Nota primer parcial & Nota segon parcial \\ \hline
8 & 5 \\
5 & 6 \\
7 & 5 \\
6 & 8 \\
9 & 6 \\
3 & 4 \\
2 & 3 \\
4 & 3 \\
6 & 7 \\
7 & 8 
\end{tabular}
\end{center}

\begin{enumerate}[a)]
\item Calcula el coeficient de correlaci� entre les variables a partir de les dades brutes.
Fes primer els c�lculs amb la calculadora i despr�s amb la fulla de c�lcul. 
\item Calcula el coeficient de correlaci� entre les variables a partir de la taula de freq��ncies.
Fes primer els c�lculs amb la calculadora i despr�s amb la fulla de c�lcul. 
\item Quin �s el grau de correlaci� lineal entre les variables? Dibuixa (amb l'ordinador) el diagrama
de dispersi�.
\item Calcula la recta de regressi� lineal i dibuixa-la amb l'ordinador. Prediu la nota del
segon parcial d'un alumne que hagi tret un $5,5$ al primer. 
\end{enumerate}


\end{enumerate}

\end{document}