\documentclass{article}
\usepackage[catalan]{babel}
\usepackage[latin1]{inputenc}   % Permet usar tots els accents i car�ters llatins de forma directa.
\usepackage{enumerate}
\usepackage{amsfonts, amscd, amsmath, amssymb}
\usepackage{graphicx}

\setlength{\textwidth}{16.5cm}
\setlength{\textheight}{24cm}
\setlength{\oddsidemargin}{-0.3cm}
\setlength{\evensidemargin}{1cm} \addtolength{\headheight}{\baselineskip}
\addtolength{\topmargin}{-3cm}

\def\N{I\!\!N}
\def\R{I\!\!R}
\def\Z{Z\!\!\!Z}
\def\Q{O\!\!\!\!Q}
\def\C{I\!\!\!\!C}

\begin{document}

%\pagestyle{empty}
\font\sc=cmcsc10
\parskip=1ex
\newcount\problemes

\newcommand\probl{\advance\problemes by 1 \vskip 2ex\noindent{\bf
\the\problemes) }}

\def\probl{\advance\problemes by 1
\vskip 1mm\noindent{\bf \the\problemes) }}
\newcounter{pepe}

\newcommand{\pr}[1]{P(#1)}

%\newcounter{problema}
%\newcommand{\prb}{\addtocounter{problema}{1}
%\noindent\vskip 2mm {\textbf{\theproblemes  }}
\newcommand{\sol}[1]{{\textbf{\footnotetext[\the\problemes]{Sol.: #1} }}}


\begin{centerline}
{\textbf{\textsc{PROBLEMES ESTAD\'ISTICA ENGINYERIA}}}
\end{centerline}

\problemes=0
\begin{centerline}
{\bf DISTRIBUCIONS DE PROBABILITAT CONT�NUES NOTABLES}
\end{centerline}


\probl Consideremos una v.a. $X$ que sigue una distribuci\'{o}n    $\mathcal{U}(0,100)$. 
Calcular $P(2 < X \leq 3)$, $P(X < 3X)$, $P(X+1 < 3)$,        $P(X^{2} < 50)$
    \sol{  $P(2 <X \leq 3) = \frac{1}{100}$, $P(X < 3X) = 1$,    $P(X+1 < 3) = \frac{1}{50}$, $P(X^{2} < 50) = \frac{\sqrt{50}}{100}$}
    
\probl    Consideremos una v.a. $X$ que sigue una distribuci\'{o}n    $\mathcal{U}(-5,5)$. Calcular:  $P(2 < X \leq 3)$, $P(X < 3X)$, $P(X+1 < 3)$,    $P(X^{2} < 5)$
 \sol{ $P(2 < X \leq 3) = \frac{1}{10}$, $P(X < 3X) = \frac{1}{2}$,    $P(X+1 < 3) = \frac{7}{10}$, $P(X^{2} < 5) = \frac{\sqrt{5}}{5}$}
 
\probl    Dibujar y hallar el \'area encerrada bajo la curva normal est\'andar en    cada uno de los casos siguientes:    \begin{itemize}        \item [a)] Entre $z = 0$ y $z = 1.2$.        \item [b)] Entre $z = -0.68$ y $z = 0$.        \item [c)] Entre $z = -0.46$ y $z = 2.21$.        \item [d)] Entre $z = 0.81$ y $z = 1.94$.        \item [e)] A la izquierda de $z = -0.6$.        \item [f)] A la derecha de  $z = -1.28$.        \item [g)] A la derecha de  $z = 2.05$ o a la izquierda de        $z = -1.44$.        \item [h)] El valor $z$ tal que el \'{a}rea a su izquierda es    $0.4960$.        \item [i)] El valor $z$ tal que el \'{a}rea a su derecha es    $0.9678$.        \item [j)] El valor $z > 0$ tal que el \'{a}rea comprendida    entre $-z$ y $z$ es $0.6318$.    \end{itemize}    \sol{ a) 0.3849, b) 0.2517, c) 0.6636, d) 0.1828, e) 0.2743, f)    0.8997, g) 0.0951, h) -0.01, i) -1.85, j) 0.9}

\probl    Dada una distribuci\'on normal con $\mu = 30$ y $\sigma = 6$,    encontrar:    \begin{itemize}        \item [a)] El \'area de la curva normal a la derecha de $x = 17$.        \item [b)] El \'area de la curva normal a la izquierda de    $x = 22$.        \item [c)] El \'area de la curva normal entre $x = 32$ y    $x = 41$.        \item [d)] El valor de $x$ que tiene el 80.23\% del \'area de la    curva normal a la izquierda.        \item [e)] El valor $\delta$ tal que $P(\mu - \delta \leq X \leq        \mu + \delta) = 0.75$.    \end{itemize}    \sol{ a) 0.9850, b) 0.0918, c) 0.3371, d) 35.1, e) $\delta = 6.9$}

\probl    Dada la v.a. $X$ distribuida normalmente con media 18 y desviaci\'on    t\'{\i}pica 2.5, encontrar:    \begin{itemize}        \item [a)] $P(X \leq 15)$.        \item [b)] El valor de $k$ tal que $P(X < k) = 0.2236$.        \item [c)] El valor de $k$ tal que $P(X > k) = 0.1814$.        \item [d)] $P(17 < X < 21)$.    \end{itemize}    \sol{ a) 0.1151, b) 16.1, c) 20.275, d) 0.5403}


\probl    {\bf Cuartiles:}    \newline    Dada una v.a. continua $X$, llamaremos cuartil de orden    q $(0 \leq q \leq 1)$ a cualquier valor, $x_{q} \in \Re$ tal que    $$P(X\leq x_{q})= q$$    Sea $X$ una v.a. tal que $X \sim \mathcal{U}(0,100)$. Calcular sus    cuartiles $0.25$, $0.5$, $0.75$. Dar una f\'ormula general para el    cuartil $q$ de una v.a. $X \sim \mathcal{U}(a,b)$.    \sol{ $x_{0.25} = 25$, $x_{0.5} = 50$, $x_{0.75} = 75$, $x_{q} =    (b-a)q +a$ con $0 \leq q \leq 1$}

\probl    A los cuartiles $0.25$, $0.5$ y $0.75$ se les denomina primer, segundo,    y tercer cuartil respectivamente. Al cuartil $0.5$ tambi\'en se le    denomina mediana. Los percentiles son los cuartiles formados por    cent\'esimas partes de la unidad.    \begin{itemize}        \item [a)] Calcular el primer cuartil, el tercel cuartil y la    mediana para una v.a. $X \sim \mathcal{N}(0,1)$.        \item [b)] Calcular el percentil $0.25$ y el percentil $0.96$ para    una v.a. $Y$  con distribuci\'{o}n normal de media $1$ y varianza    $4$.    \end{itemize}    \sol{ a) $x_{0.25} = -0.67$, $x_{0.75} = 0.67$, $x_{0.5} = 0$, b)    $y_{0.25} = -0.34$, $y_{0.96} = 4.6$}

\probl   Un ejecutivo de televisi\'on est\'a estudiando propuestas para nuevas    series. A su juicio, la probabilidad de que una serie tenga audiencia    mayor que $17.35$ es  $0.25$; adem\'as la probabilidad de que la serie    tenga audiencia mayor que $19.2$ es $0.15$. Si la incertidumbre de este    ejecutivo puede representarse mediante una v.a. normal, ?`cu\'al es la    media y la desviaci\'on t\'{\i}pica de esta distribuci\'on?    \sol{ $\mu = 14$, $\sigma = 5$}


\probl    Supongamos que el n\'umero de horas que un auxiliar administrativo    necesita para aprender el nuevo programa de facturaci\'{o}n es una    v.a. $X$ con distribuci\'{o}n normal. Si el 84.13\% de los    auxiliares emplean m\'as de tres horas y s\'olo el 22.96\% m\'as de    nueve, ?`cu\'anto valen $\mu_{X}$ y $\sigma_{X}^2$?    \sol{ $\mu_{X} = 6.4483$, $\sigma_{X}^2 = 11.8906$}

\probl    Si $X$ es una v.a. con distribuci\'on normal est\'andar y se define    $Y = 2X - 1$, calcular la probabilidad de que $Y$ no se aparte de su    media m\'as de una desviaci\'on t\'{\i}pica.    \sol{ 0.6826}

\probl    Si un conjunto de calificaciones de un examen de estad\'{\i}stica se    aproxima a la distribuci\'on normal con media 74 y desviaci\'on    t\'{\i}pica 7.9, encontrar:    \begin{itemize}        \item [a)] La calificaci\'on m\'as baja de apto si el 10\% de los    estudiantes de nota m\'as baja se les declararon no aptos.        \item [b)] El notable de puntuaci\'on m\'as alta si el 5\% de los    estudiantes tiene un sobresaliente.        \item [c)] El notable m\'as bajo si al $5\%$ de los estudiantes se les    di\'o sobresaliente y al siguiente $25\%$ se le di\'o notable.    \end{itemize}    
\sol{ a) 63.888, b) 86.956,    (tomando $0.95 \approx 0.9495 = F_{z}(1.64)$),    c) 78.108}

\probl    La vida promedio de un cierto tipo de motor peque\~no es de 10 a\~nos    con una desviaci\'on t\'{\i}pica de 2 a\~nos. El fabricante repone sin    cargo todos los motores que fallen dentro del periodo de garant\'{\i}a.    Si est\'a dispuesto a reponer s\'olo  el 3\% de los motores que fallan,    ?`cu\'{a}l debe ser la duraci\'{o}n de la garant\'{\i}a que otorgue?    (Suponer que las vidas de los motores siguen una distribuci\'on    normal.)    \sol{ 6.24}

\probl    La compra media que realiza un cliente en un determinado comercio, es de    82 euros, siendo la desviaci\'on est\'andar 5 euros. Todos los clientes    que compran entre 88 y 94 euros son clientes clasificados como    preferentes. Si las compras est\'an distribuidas aproximadamente como    una normal y 8 clientes son preferentes, ?`cu\'antos clientes tiene    este comercio?    \sol{ 75}


\probl  La variable aleatoria $X$ sigue una ley  $\displaystyle N(\mu, \sigma^2). $ Sabemos
que   $ \mu = 5  \sigma,
 $ y  que $ \pr{X < 6} = 0.8413.$

\begin{enumerate}[a)]
\item Determinar la esperanza y la varianza de $X$.
\item ?`Cu�l es la funci�n de distribuci�n de $ Y = 3
- X^2  $ y su esperanza? \sol{a)$\mathbf{(5,\quad 1)}$ ; b) $\mathbf{E(Y) = -23}$}
\end{enumerate}


\probl  El percentil $90$ de una variable aleatoria $X$ es el valor $x_{90}$ para el que $
\displaystyle F_{X}(x_{90}) = \pr{X \leq x_{90}} = 0.9. $ De manera similar, el percentil
$50$ es el valor $x_{50}$ que satisface $ \pr{X \leq x_{50}} = 0.5 $ que recibe el nombre
de  {\bf mediana} (poblacional). Determinar estos dos valores para  una variable aleatoria
exponencial de valor medio $10$. \sol{$\mathbf{(23.0585,\quad 6.93147)}$}


\probl  Una centralita recibe llamadas telef�nicas con un ritmo medio de $\mu$ llamadas por
minuto. Calcular la probabilidad de que el intervalo entre dos llamadas consecutivas supere
al intervalo  medio en m�s de dos desviaciones t�picas \sol{$\mathbf{0.05}$}.



\end{document}

