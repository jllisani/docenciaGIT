% problemas del tema 5

\begin{center}
    {\Large PROBLEMAS DE ESTADISTICA ECONOMICA}  \\
    \vspace{0.3cm}
    {\Large CONTRASTE DE HIPOTESIS}
\end{center}

%%%FATLTAN SOLUCIONES
% \setcounter{problema}{103}

\begin{prob}%        10.1 pag 285 J.Amon
    Siendo $\overline{x} = 63.5$ la media de una muestra aleatoria simple de
    tama\~{n}o 36 extra\'{\i}da de una poblaci\'{o}n normal con
    $\sigma^2 = 144$, poner a prueba, con un nivel de significaci\'{o}n
    $\alpha = 0.05$, la hip\'{o}tesis nula $\mu = 60$ y decir si se rechaza
    en favor de la alternativa $\mu < 60$.
    \sol{ No se rechaza}
\end{prob}

\begin{prob}%         10.2 pag 285 J.Amon
    Siendo $\overline{x}=72.5$ la media de una muestra aleatoria simple de
    tama\~{n}o 100 extra\'{\i}da de una poblaci\'on normal con
    $\sigma^2 = 900$, poner a prueba, con un nivel de significaci\'{o}n
    $\alpha = 0.10$, la hip\'{o}tesis nula $\mu = 70$ y decir si se rechaza
    en favor de las hip\'{o}tesis alternativas $\mu \not = 70$, $ \mu > 70$,
    y $\mu < 70$. Hallar el p-valor del contraste, para la hip\'otesis
    alternativa $ \mu > 70$.
   \sol{ No se rechaza en ninguno de los tres casos, 20.33\%}
\end{prob}

\begin{prob}%        10.9 pag 285 J.Amon
    En un contraste bilateral, con $\alpha = 0.01$, ?`para qu\'{e} valores
    de $\overline{X}$ re\-cha\-zar\'{\i}amos la hip\'{o}tesis nula
    $H_{0}: \mu = 70$, a partir de una muestra aleatoria simple de
    tama\~{n}o 64 extra\'{\i}da de una poblaci\'{o}n normal con
    $\sigma^2 = 256$?
    \sol{ Para $\overline{X} < 64.85$ o $\overline{X} > 75.15$}
\end{prob}

\begin{prob}%        11.2 pag 324 J.Amon
    Con los datos del ejercicio anterior, ?`son compatibles con el
    resultado obtenido los siguientes contrastes?:
    \begin{itemize}
    \item [a)] $\left\{
    \begin{array}{ll}
        H_{0}: \mu = 44000    \\
        H_{1}: \mu > 44000
    \end{array}
    \right.$

    \item [b)] $\left\{
    \begin{array}{ll}
        H_{0}: \mu = 46250     \\
        H_{1}: \mu > 46250
    \end{array}
    \right.$
    \end{itemize}
    \sol{ S\'{\i}}
\end{prob}

\begin{prob}%       11.6 pag 325 J.Amon
    El peso medio de los paquetes de caf\'{e} puestos a la venta por
    la casa comercial CAFEINASA es supuestamente de 1 kg. Para comprobar
    esta suposici\'{o}n, elegimos una muestra aleatoria simple de 100
    paquetes y encontramos que su peso medio es de 0.978 kg y su
    desviaci\'{o}n t\'{\i}pica $\hat{s} = 0.10$ kg. Siendo $\alpha = 0.05$,
    ?`es compatible este resultado con la hip\'{o}tesis nula $H_{0}:
    \mu = 1$ frente a $H_{1}: \mu \not = 1$? ?`Lo es frente a $H_{1}:
    \mu > 1$?
    \sol{ Con la primera, no; con la segunda, s\'{\i}}
\end{prob}

\begin{prob}%       11.10 pag 325 J.Amon
    El fabricante de la marca de tornillos FDE afirma que el di\'{a}metro
    medio de sus tornillos vale 20 mm. Para comprobar dicha afirmaci\'{o}n,
    extraemos aleatoria e independientemente 16 tornillos, y vemos que
    la media de sus di\'{a}metros  es 22 mm y la desviaci\'{o}n t\'{\i}pica
    4 mm. ?`Podemos aceptar la pretensi\'{o}n del fabricante, suponiendo
    $\alpha = 0.05$ y siendo el contraste bilateral? Hallar el p-valor
    del contraste.
    \sol{ S\'{\i}, 6.7\%}
\end{prob}

\begin{prob}%       11.23 pag 328 J.Amon
    Una m\'{a}quina produce cierto tipo de piezas mec\'{a}nicas. El
    tiempo en producirlas se distribuye normalmente con varianza
    desconocida $\sigma^2$. Elegida una muestra aleatoria simple de
    21 de dichas piezas ($x_{1}, x_{2}, \ldots, x_{21}$), se obtiene que
    $\overline{x} = 30$ y $\sum_{i=1}^{21} x_{i}^2 = 19100$. Comprobar
    si es compatible la hip\'{o}tesis nula $H_{0}: \sigma^2 = 22$ frente
    a $H_{1}: \sigma^2 \not = 22$, para $\alpha = 0.1$, y construir un
    intervalo de confianza del $(1 - \alpha)100\%$ para el verdadero
    valor de $\sigma^2$.
    \sol{ No es compatible, (6.37, 18.35)}
\end{prob}

\begin{prob}%       11.26 pag 328 J.Amon
    A partir de las puntuaciones 15, 22, 20, 21, 19 y 23, construir el
    in\-ter\-va\-lo de confianza de $\sigma^2$ y decir si es compatible con
    estos resultados la hip\'{o}tesis $H_{0}: \sigma = 2$, siendo
    $\alpha = 0.01$. Decir si se utiliza alguna hip\'{o}tesis adicional.
    \sol{ Suponemos que la poblaci\'on es normal. La hip\'otesis es
    compatible, (2.395, 97.09)}
\end{prob}

\begin{prob}%        11.29 pag 329 J.Amon
    Sabiendo que con $\hat{p} = 0.52$ ha sido rechazada la hip\'otesis
    nula $H_{0}: p = 0.50$, al nivel de significaci\'{o}n $\alpha = 0.05$,
    ?`cu\'{a}l ha tenido que ser el tama\~{n}o m\'{\i}nimo de la muestra
    mediante la cual fue rechazada $H_{0}$
    \begin{itemize}
    \item [a)] frente a $H_{1}: p \not = 0.5$?

    \item [b)] frente a $H_{1}: p > 0.5$?
    \end{itemize}
    \sol{ a) 2401, b) 1692}
\end{prob}

\begin{prob}%
    Un fabricante de productos farmac\'{e}uticos tiene que mantener un
    est\'andar de impurezas en el proceso de producci\'{o}n de sus
    p\'{\i}ldoras. Hasta ahora el n\'{u}mero medio poblacional de
    impurezas es correcto, pero est\'{a} preocupado porque en algunas
    partidas las impurezas se salen del rango admitido, de forma que
    provocan devoluciones y reclamaciones por da\~{n}os a la salud. El
    gabinete de control de calidad afirma que si la distribuci\'{o}n de
    las impurezas es normal y el proceso de producci\'{o}n mantiene una
    varianza inferior a 1 no tendr\'{\i}a que existir ning\'{u}n problema
    pues las p\'{\i}ldoras tendr\'{\i}an una concentraci\'{o}n aceptable.
    Preocupado por esta tema, la direcci\'{o}n encarga una prueba externa
    en la que se toma una muestra aleatoria de 100 de las partidas,
    obteni\'{e}ndose $\hat{s}^2 = 1.1$. Tomando $\alpha = 0.25$, ?`puede
    aceptar el director de la prueba externa que el proceso de producci\'on
    cumple la recomendaci\'on del gabinete de control?
    \sol{ No}
\end{prob}

\begin{prob}%       similar to 3 pag 299 Newbold
    Una empresa que vende vacaciones est\'{a} preocupada por su est\'{a}ndar
    de calidad y quiere compararlo con el medio europeo. El est\'{a}ndar
    medio europeo dice que una empresa de este sector tiene una calidad
    aceptable si tiene un n\'{u}mero de quejas que no excede del 3\%. Se sabe
    que la poblaci\'on es normal y la varianza de las quejas es $0.16$.
    Examinando 64 ventas escogidas al azar entre las del \'{u}ltimo mes
    se encuentra que el porcentaje de quejas es 3.07\%.
    \begin{itemize}
    \item [a)] Contrastar, al nivel de significaci\'{o}n del 5\%, la
    hip\'otesis nula de que la media poblacional del porcentaje de quejas
    es del 3\% frente a la alternativa de que es superior al 3\%.

    \item [b)] Hallar el $p$-valor del contraste.

    \item [c)] Supongamos que la hip\'{o}tesis alternativa fuese
    bilateral en lugar de unilateral (con hip\'otesis nula $H_{0}:
    \mu = 3$). Deducir, sin hacer ning\'un c\'{a}lculo, si el $p$-valor
    del contraste ser\'{\i}a ma\-yor, menor o igual que el del apartado
    anterior. Construir un gr\'{a}fico para ilustrar el razonamiento.

    \item [d)] En el contexto de este problema, explicar por qu\'e una
    hip\'{o}tesis alternativa unilateral es m\'{a}s apropiada que una
    bilateral.

    \item [e)] Hallar la probabilidad de que en un contraste unilateral
    al nivel del 5\% se rechace la hip\'{o}tesis nula cuando el
    verdadero porcentaje de quejas es del 3.10\%.
    \end{itemize}
    \sol{ a) Es compatible, b) 8.08\%, c) mayor, e) 0.6554}
\end{prob}

\begin{prob}%          similar to 10.6 pag 285 Amon
    Para una poblaci\'on normal $\mathcal{N}(\mu, 20^2)$, sabiendo que
    la probabilidad de rechazar la hip\'otesis nula $H_{0}: \mu = 80$
    %siendo falsa,
    en favor de la alternativa $H_{1}: \mu = 84.66$ es
    0.6293 y que el tama\~no de la muestra es 100, calcular la
    probabilidad de rechazar $H_{0}$ siendo verdadera. (Suponer contraste
    unilateral.)
    \sol{ 0.0228}
\end{prob}

\begin{prob}%          10.9 pag 285 Amon
    En un contraste bilateral, con $\alpha = 0.01$, ?`para qu\'e
    valores de $\overline{X}$ rechazar\'{\i}amos la hip\'otesis nula
    $H_{0}: \mu = 70$, a partir de una muestra aleatoria simple de
    tama\~no 64 extra\'{\i}da de una poblaci\'on normal
    $\mathcal{N}(\mu, 16^2)$?
    \sol{ $\overline{X} \leq 64.85$, $\overline{X} \geq 75.15$}
\end{prob}

\begin{prob}%          10.10 pag 286 Amon
    Calcular $\beta$, al contrastar la hip\'otesis nula
    $H_{0}: \mu = 35$ frente a la alternativa $H_{1}: \mu = 39$, como
    medias de una poblaci\'on normal $\mathcal{N}(\mu, 120)$,
    valiendonos de una muestra aleatoria simple de tama\~no 30,
    extra\'{\i}da de dicha poblaci\'on. El contraste es unilateral
    y $\alpha = 0.10$.
    \sol{ 0.2358}
\end{prob}

\begin{prob}%          similar to 10.26 + 10.27 pag 287 Amon
    Lanzamos una moneda al aire 100 veces consecutivas,
    \begin{itemize}
        \item  [a)] ?`Podemos aceptar la hip\'otesis nula
    $H_{0}: P(\text{cara}) = 1/2$ frente a la alternativa
    $H_{1}: P(\text{cara}) \neq 1/2$ si se han obtenido 40 caras?
    (Tomar $\alpha = 0.10$.)

        \item  [b)] ?`Con qu\'e n\'umero de caras rechazar\'{\i}amos
    $H_{0}: P(\text{cara}) = 1/2$, siendo $\alpha = 0.01$?
    \end{itemize}
    \sol{ a) No, b) con 63 o m\'as, 37 o menos}
\end{prob}

\begin{prob}%
    A partir de una muestra aleatoria se contrasta:

    $\left\{
    \begin{array}{l}
    H_{0}: \mu = \mu_{0}    \\
    H_{1}: \mu > \mu_{0}
    \end{array}
    \right.$
    \newline
    y se acepta la hip\'otesis nula al nivel de significaci\'{o}n del 5\%.
    \begin{itemize}
    \item [a)] ?`Implica esto necesariamente que $\mu_{0}$ est\'{a}
    contenido en el intervalo de confianza del 95\% para $\mu$?

    \item [b)] Si la media muestral observada es mayor que $\mu_{0}$,
    ?`implica necesariamente que $\mu_{0}$ est\'{a} contenido en el
    intervalo de confianza del 90\% para $\mu$?
    \end{itemize}
    \sol{ a) No, b) s\'{\i}}
\end{prob}

\begin{prob}%
    Una compa\~n\'{\i}a que se dedica a la venta de franquicias afirma
    que, por t\'{e}rmino medio, los delegados obtienen un redimiendo del
    10\% en sus inversiones iniciales. Una muestra aleatoria de diez de
    estas franquicias presentaron los siguientes rendimientos el primer
    a\~{n}o de operaci\'{o}n:
    $$6.1 \quad 9.2 \quad 11.5 \quad 8.6 \quad 12.1 \quad 3.9 \quad
        8.4 \quad 10.1 \quad 9.4 \quad 8.9$$
    Asumiendo que los rendimientos poblacionales tienen distribuci\'on
    normal, contrastar la afirmaci\'{o}n de la compa\~{n}\'{\i}a tomando
    $\alpha = 0.05$.
    \sol{ Se acepta la afirmaci\'on}
\end{prob}

\begin{prob}%
    Una distribuidora de bebidas refrescantes cre\'{\i}a que una buena
    fotograf\'{\i}a de tama\~{n}o real de un conocido actor
    incrementar\'{\i}a las ventas de un producto en los supermercados en una
    media de 50 cajas semanales. Para una muestra de 20 supermercados, el
    incremento medio fue de 41.3 cajas con una desviaci\'{o}n t\'{\i}pica de
    12.2 cajas. Contrastar al nivel de significaci\'{o}n $\alpha = 0.05$, la
    hip\'otesis nula de que la media poblacional del incremento en las ventas
    es al menos 50 cajas, indicando cualquier supuesto que se haga. Calcular
    el $p$-valor del contraste e interpretarlo.
    \sol{ Se rechaza la hip\'otesis nula. 0.005}
\end{prob}

\begin{prob}%            10.22 pag 287 Amon
    Dos investigadores, Alberto y Timoteo, desean comprobar la
    hip\'{o}tesis nula $H_{0}: \mu = 50$, acerca de la misma
    poblaci\'on normal con $\sigma = 20$ e imponiendo un mismo nivel
    de significaci\'on $\alpha = 0.05$ (contraste bilateral). Alberto
    extrae una muestra aleatoria simple de tama\~{n}o 100 y Timoteo
    de tama\~{n}o 64.
    \begin{itemize}
        \item  [a)] Siendo realmente verdadera la hip\'otesis nula,
        ?`para qui\'en de los dos es mayor la probabilidad de cometer
        el error tipo I? ?`Cu\'anto valen ambas?

        \item  [b)] Siendo realmente verdadera la hip\'otesis
        alternativa $H_{1}: \mu = 52$, ?`para qui\'en de los dos es
    mayor la probabilidad de cometer el error tipo II? ?`Cu\'anto
    valen ambas? (Suponer contraste bilateral.)
    \end{itemize}
    \sol{ a) es igual para ambos; 0.05, b) Alberto: 0.8300; Timoteo:
    0.8741}
\end{prob}

\begin{prob}%        54 pag 331 Newbold
    Una cadena de restaurantes de comida r\'apida contrasta cada
    d\'{\i}a que el peso medio de sus hamburguesas es al menos de 320
    gr. La hip\'otesis alternativa es que el peso medio es menor que
    320 gr., indicando que se necesiata un proceso nuevo de
    producci\'on. Se puede asumir que los pesos de las hamburguesas
    siguen una distribuci\'on normal, con una desviaci\'on t\'{i}pica
    de 30 gr. La regla de decisi\'on adoptada es rechazar la
    hip\'otesis nula si el peso medio muestral es menor de 308 gr.
    \begin{itemize}
        \item  [a)] Si se seleccionan muestras aleatorias de $n = 36$
        hamburguesas, ?`cu\'al es la probabilidad de cometer un error
        tipo I usando esta regla de decisi\'on?

        \item  [b)] Si se seleccionan muestras aleatorias de $n = 9$
        hamburguesas, ?`cu\'al es la probabilidad de cometer un error
        tipo I usando esta regla de decisi\'on? Explicar por qu\'e el
        resultado es distinto al del apartado anterior.

        \item  [c)] Suponer que el verdadero peso medio es de 310 gr.
    Si se seleccionan muestras aleatorias de $n = 36$
        hamburguesas, ?`cu\'al es la probabilidad de cometer un error
        tipo II usando esta regla de decisi\'on?
    \end{itemize}
    \sol{ a) 0.0082, b) 0.1151, c) 0.6554}
\end{prob}

\begin{prob}%  Newbold, pag.318, 28
    Un funcionario que trabaja en el departamento de colocaci\'on de
    una universidad, quiere determinar si los hombres y las mujeres
    licenciados en Administraci\'on de Empresas reciben, en promedio,
    diferentes ofertas de salario en su primer trabajo despu\'es de
    licenciarse. El funcionario seleccion\'o aleatoriamente ocho
    pares de licenciados en esa disciplina de manera que las
    calificaciones, intereses e historial de los integrantes de cada
    pareja fuesen lo m\'as parecidos posible. La mayor diferencia fue
    que un miembro de cada pareja era hombre y el otro mujer. La tabla
    adjunta ofrece la mayor oferta salarial (en millones de pesetas)
    que recibi\'o cada miembro de la muestra al terminar su carrera.
    \begin{center}
        \begin{tabular}{l|cc}
        \hline
            pareja & \multicolumn{2}{c}{mayor oferta salarial}  \\
            \hline
                   & HOMBRE     & MUJER       \\
            \hline
            1      & 2.620      & 2.260       \\
            2      & 2.470      & 2.360       \\
            3      & 2.840      & 2.930       \\
        4      & 2.170      & 2.230       \\
        5      & 2.860      & 2.620       \\
        6      & 2.930      & 2.590       \\
        7      & 2.830      & 2.850       \\
        8      & 2.430      & 2.130       \\
            \hline
        \end{tabular}
    \end{center}
    Asumiendo que las distribuciones son normales,
    \begin{itemize}
        \item  [a)] Para $\alpha$ = 5\%, contrastar la hip\'otesis nula de
    que las medias poblacionales son iguales frente a la alternativa
    que la verdadera media es mayor para los hombres que para las
    mujeres.

        \item  [b)] Para $\alpha$ = 10\%, contrastar frente a una
        alternativa bilateral, la hip\'otesis nula de que la media
        para los hombres difiere en 300000 ptas. de la media para las
    mujeres.
    \end{itemize}
    \sol{ a) el salario medio de los hombres es superior al salario
    medio de las mujeres, b) los salarios medios no difieren en 300000
    ptas.}
\end{prob}

\begin{prob}%  BIO
    Con el objeto de comparar dos m\'etodos de aprendizaje de la
    lectura, se toman 6 parejas de hermanos gemelos, aplic\'andose un
    m\'etodo a cada hermano y evaluando los resultados mediante un test.
    Dichos resultados fueron:
    $$
    \begin{tabular}{|l|cccccc|}
    \hline
    m\'etodo 1 & 112 & 133 & 152 & 107 & 120 & 124   \\
    \hline
    m\'etodo 2 & 100 & 130 & 150 & 110 & 112 & 110   \\
    \hline
    \end{tabular}
    $$
    ?`Puede afirmarse, con un nivel de signaificaci\'on 0.1, que los
    m\'etodos no son igualmente eficaces?
    \sol{ Si.}
\end{prob}

\begin{prob}% BIO
    Dos m\'aquinas dosificadoras, exactamente iguales, se utilizan una
    para empaquetar garbanzos y otra para empaquetar jud\'{\i}as.  Con
    el fin de contrastar que est\'an ajustadas al mismo peso, se
    pesaron el \'ultimo paquete de garbanzos y jud\'{\i}as producidas
    cada d\'{\i}a durante una semana, obteni\'endose los siguientes
    resultados (en gramos):
    $$
    \begin{tabular}{|l|ccccc|}
    \hline
                & lunes & martes & miercoles & jueves & viernes   \\
    \hline
    garbanzos   & 980   & 1000   & 990       & 990    & 1000      \\
    \hline
    jud\'{\i}as & 1000  & 1010   & 990       & 1000   & 990       \\
    \hline
    \end{tabular}
    $$
    Suponiendo que las dos distribuciones de peso son normales y con
    la misma varianza, ?`podemos aceptar, con un nivel de
    significaci\'on de 0.05, que las dos m\'aquinas est\'an
    ajustadas al mismo peso?
    \sol{ Si}
\end{prob}

\begin{prob}%   Newb 9.9  pag. 317
    De una muestra aleatoria de 203 anuncios publicados en revistas
    brit\'anicas, 52 eran humor\'{\i}sticos. De una muestra aleatoria
    independiente de 270 anuncios publicados en revistas americanas,
    56 eran humor\'{\i}sticos. Contrastar, frente a una alternativa
    bilateral, la hip\'otesis nula de que la proporci\'on de anuncios
    c\'omicos de las revistas brit\'anicas y americanas son iguales.
    Calcular el p-valor.
    \sol{ 21.12\%}
\end{prob}

\begin{prob}% red book 5.8 pag. 114
    Se quiere averiguar si existen diferencias significativas entre
    dos plataformas digitales, al considerar los ingresos por cliente
    cuando utilizan la modalidad de pago por visi\'on. Para tal
    prop\'osito se elegieron al azar 20 clientes de cada plataforma y
    se anot\'o el gasto por cliente y mes en dicha modalidad,
    onteni\'endose para la plataforma 1 un gasto medio mensual por cliente
    de 4500 ptas. y una desviaci\'on t\'{\i}pica muestral de 1000 ptas.
    y para la plataforma 2 un gasto medio mensual por cliente de 4600
    ptas. y una desviaci\'on t\'{\i}pica muestral de 1500 ptas. A un
    nivel de significaci\'on $\alpha = 0.05$, ?`existen diferencias
    significativas entre las dos plataformas en cuanto a la modalidad
    analizada?
    \sol{ No existen diferencias significativas.}
\end{prob}

\begin{prob}%  red book 5.10 pag. 118
    Se quiere averiguar si ha habido una reducci\'on significativa en
    el porcentaje de votantes a un determinado partido pol\'{\i}tico,
    el el \'ultimo a\~{n}o. Para ello se eligieron al azar 100
    personas y se las pregunt\'o si votar\'{\i}an al partido en
    cuesti\'on, obteni\'endose un porcentaje de respuestas
    afirmativas del 39\%. Si el porcentaje de votantes a favor era
    del 42\% hace un a\~{n}o, cuando se pregunt\'o a 150 personas,
    contrastar a nivel $\alpha = 0.05$ si la reducci\'on habida ha
    sido significativa. Calcular el p-valor del contraste.
    \sol{ 31.92\%}
\end{prob}

\begin{prob}%varios  6.9 pag. 138
    En un centro de ense\~{n}anza de idiomas se escogieron dos grupos
    de 25 alumnos para comprobar la distinta eficacia de dos
    m\'etodos de aprendizaje. El primer grupo estudi\'o con el
    m\'etodo tradicional y el segundo con un m\'etodo audiovisual
    nuevo. Los alumnos del primer grupo obtuvieron una calificaci\'on
    media de 7.2 con una desviaci\'on t\'{\i}pica de 2.4, y los del
    segundo, 5.8 con una desviaci\'on t\'{\i}pica de 1.5. Para
    $\alpha = 0.01$, ?`existen diferencias significativas entre ambos
    m\'etodos? (Suponer poblaciones normales con distinta varianza.)
    \sol{ No existen diferencias.}
\end{prob}

\begin{prob}% NOVO 7.7 pag. 287
    Una muestra de 200 bombillas de la marca A dio una vida media de
    funcionamiento de 2280 horas con una desviaci\'on t\'{\i}pica de
    80 horas. Otra muestra de 180 bombillas de la marca B dio vida
    media 2320 horas con desviaci\'on t\'{\i}pica 100 horas. ?`Se
    puede afirmar al nivel 0.01, que es mayor la vida media para la
    marca B?
    \sol{ Si.}
\end{prob}

\begin{prob}%canavos 9.45 pag. 361
    Un inversionista desea comparar los riesgos asociados con dos
    mercados A y B. El riesgo de un mercado dado se mide por la
    variaci\'on de los cambios diarios de precios. El inversionista
    piensa que el riesgo asociado con el mercado B es mayor que el
    del mercado A. Se obtienen muestras aleatorias de 21 cambios de
    precios diarios para el mercado A y de 16 para el mercado B. Se
    obtienen los siguientes resultados:
    $$
    \begin{tabular}{|l|cc|}
    \hline
               & $\overline{X}$ & $\hat{s}_{X}$   \\
    \hline
    mercado A  & 0.3          & 0.25          \\
    \hline
    mercado B  & 0.4          & 0.45          \\
    \hline
    \end{tabular}
    $$
    \begin{itemize}
        \item  [a)] Si se supone que las muestras provienen de dos
        poblaciones normales e independientes, a un nivel de
        $\alpha = 0.05$, ?`encuentra apoyo la creencia del
        inversionista?

        \item  [b)] Si la varianza muestral de A es la dada, ?`cu\'al
        es el m\'aximo valor de la varianza muestral de B con base en
        $m = 16$ que no llevar\'a al rechazo de la hip\'otesis nula
        del apartado a?
    \end{itemize}
    \sol{ a) No, b) 0.1072}
\end{prob}

\begin{prob}%canavos 9.49 pag. 361
    Un economista al servicio de una agencia estatal desea determinar
    si la frecuencia de desempleo en dos grandes \'areas urbanas del
    estado son diferentes. Toma dos muestras aleatorias de 500
    personas de cada ciudad y encuentra 35 personas desempleadas en
    un \'area y 25 en la otra. Bajo las suposiciones adecuadas y con
    un nivel de $\alpha = 0.05$, ?`existe alguna raz\'on para creer
    que las frecuencias de desempleo en las dos \'areas son
    diferentes? ?`Cu\'al es el valor de p?
    \sol{ No, 0.1836}
\end{prob}

\begin{prob}%
    El Rector de una Universidad opina que el 60\% de los estudiantes
    con\-si\-de\-ra muy \'{u}tiles los cursos que realizan, el 20\% algo
    \'{u}tiles y el 20\% nada \'{u}tiles. Se toma una muestra aleatoria de
    100 estudiantes, y se les pregunta sobre la utilidad de los cursos. Hay
    68 que consideran que los cursos son muy \'{u}tiles, 18 que son poco
    \'{u}tiles y 14 que no son nada \'{u}tiles. Tomando $\alpha = 0.05$,
    contrastar la hip\'otesis nula de que los resultados obtenidos se
    corresponden con la opini\'{o}n personal del Rector.
    \sol{ La opini\'on del Rector es correcta}
\end{prob}

\begin{prob}%
    A una muestra aleatoria de 502 personas se les pregunt\'{o} la
    im\-por\-tan\-cia que daban al precio a la hora de elegir un hospital.
    Se les pidi\'{o} que valoraran entre: ``ninguna importancia'', ``alguna
    importancia'' y ``mucha importancia''. El n\'{u}mero respectivo de
    respuestas en cada tipo fueron 169, 136 y 197. Contrastar la
    hip\'{o}tesis nula de que la probabilidad de que una persona elegido al
    azar conteste cualquiera de las tres respuestas es la misma. (Tomar
    $\alpha = 0.01$.)
    \sol{ Una respuesta al azar no es equiprobable}
\end{prob}

\begin{prob}%
    Durante cien semanas se ha venido observando el n\'{u}mero de veces a la
    semana que se ha averiado una m\'{a}quina, present\'{a}ndose los
    resultados de la siguiente tabla:
    \begin{center}
        \begin{tabular}{|l|cccccc|}
            \hline
            n\'{u}mero de aver\'{\i}as & 0  & 1  & 2  & 3  & 4
            & 5 o m\'{a}s    \\
            \hline
            n\'{u}mero de semanas      & 10 & 24 & 32 & 23 & 6
           & 5               \\
            \hline
            \end{tabular}
    \end{center}
    La experiencia indica que el n\'umero medio de aver\'{\i}as por
    semana es de 2.5. Contrastar la hip\'{o}tesis nula de que la
    distribuci\'{o}n de aver\'{\i}as es una Poisson de par\'ametro
    $\lambda = 2.5$. (Tomar $\alpha = 0.10$.)
    \sol{ No es una Poisson}
\end{prob}

\begin{prob}%
    Para los datos del problema anterior, contrastar la hip\'{o}tesis
    nula de que la distribuci\'{o}n de aver\'{\i}as es una Poisson (de
    par\'ametro desconocido). (Tomar $\alpha = 0.10$.)
    \sol{ Es una Poisson}
\end{prob}

\begin{prob}
    Se lanzan 5 monedas simultaneamente 1000 veces y en cada
    lanzamiento se observa el n\'umero de caras, obteniendose los
    siguientes resultados:
    \begin{center}
        \begin{tabular}{|c|c|}
            \hline
            n\'umero de caras  & $o_{i}$           \\
            \hline
        0                  & \quad  38 \quad   \\
        1                  & \quad 144 \quad   \\
        2                  & \quad 342 \quad   \\
        3                  & \quad 287 \quad   \\
        4                  & \quad 164 \quad   \\
        5                  & \quad  25 \quad   \\
            \hline
        \end{tabular}
    \end{center}
    Para $\alpha = 0.05$, ?`son compatibles estos datos con la
    hip\'otesis de que se extrajeron de una poblaci\'on que sigue una
    distribuci\'on binomial? (Suponer que las monedas est\'an
    balanceadas.)
    \sol{ S\'{\i}}
\end{prob}

\begin{prob}
    La informaci\'on que la marca de pilas AAA aporta en el exterior
    de su envoltorio dice: ``la duraci\'on media de las pilas AAA
    sigue una distribuci\'on normal de media 3.5 horas y desviaci\'on
    t\'{\i}pica 0.7 horas siempre que se conserven en sitio fresco y
    seco''. La inspecci\'on de AENOR toma una muestra aleatoria de
    pilas obteni\'endose los siguientes resultados:
    \begin{center}
        \begin{tabular}{|c|c|}
            \hline
            duraci\'on (horas)      & $o_{i}$   \\
            \hline
        menor que 1.95          & \quad 2  \quad            \\
        1.95 - 2.45             & \quad 1  \quad            \\
        2.45 - 2.95             & \quad 4  \quad            \\
        2.95 - 3.45             & \quad 15 \quad            \\
        3.45 - 3.95             & \quad 10 \quad            \\
        3.95 - 4.45             & \quad 5  \quad            \\
            mayor que 4.45          & \quad 3  \quad            \\
            \hline
        \end{tabular}
    \end{center}
    ?`A la vista de estos datos podemos afirmar que la informaci\'on del
    fabricante es cierta al nivel de significaci\'on de 5\%?
    \sol{ La informaci\'on del fabricante es correcta}
\end{prob}

\begin{prob}
    Durante la segunda guerra mundial se dividi\'o el mapa de Londres
    en cuadr\'{\i}culas de 0.25 $km^2$ y se cont\'o el n\'umero de
    bombas ca\'{\i}das en cada cuadr\'{\i}cula durante un bombardeo
    alem\'an. Los resultados fueron:
    \begin{center}
        \begin{tabular}{|c|c|}
            \hline
            n\'umero de impactos     & $o_{i}$          \\
            \hline
        0                        & \quad 229 \quad  \\
        1                        & \quad 211 \quad  \\
        2                        & \quad  93 \quad  \\
        3                        & \quad  35 \quad  \\
        4                        & \quad   7 \quad  \\
        5 o m\'as                & \quad   1 \quad  \\
            \hline
        \end{tabular}
    \end{center}
    Si realmente los bombardeos no segu\'{\i}an un plan prefijado, la
    distribuci\'on del n\'umero de bombas tendr\'{\i}a que ser una
    Poisson $\mathcal{P}(\lambda)$. Contrastar esta hip\'otesis al
    nivel de significaci\'on $\alpha =$ 0.5\%.
    \sol{ El bombardeo es aleatorio}
\end{prob}

\begin{prob}
    Una empresa hace un estudio sobre la prevenci\'on de riesgos
    laborales. Para ello considera 50 empresas y examina el n\'umero de
    lesiones por millar de horas trabajadas, obteniendo la siguiente
    tabla:
    \begin{center}
        \begin{tabular}{|c|c|}
            \hline
            lesiones por millar de horas trabajadas      & n\'umero de
                empresas   \\
            \hline
        1.45 - 1.75             & \quad 3  \quad           \\
        1.75 - 2.05             & \quad 12  \quad          \\
        2.05 - 2.35             & \quad 14  \quad          \\
        2.35 - 2.65             & \quad 9 \quad            \\
        2.65 - 2.95             & \quad 8 \quad            \\
        2.95 - 3.25             & \quad 4  \quad           \\
            \hline
        \end{tabular}
    \end{center}
    \begin{itemize}
        \item  [a)] Obtener una estimaci\'on del n\'umero medio de lesiones
        por millar de horas trabajadas. Razonar la respuesta.

        \item  [b)] Para un nivel de significaci\'on del 5\%, ?`podemos
        afirmar que la distribuci\'on de frecuencias sigue una
        distribuci\'on normal con desviaci\'on t\'{\i}pica 0.43?
    \end{itemize}
    \sol{ a) 2.314; b) Si.}
\end{prob}

\begin{prob}
    La pintura para autopista se surte en dos colores: amarillo y
    blanco. Se llev\'o a cabo un estudio sobre el tiempo de secado de
    ambas pinturas. Para ello se tomaron datos sobre una misma autopista
    bajo las mismas condiciones, obteni\'endose los siguientes tiempos de
    secado (en minutos).
    \begin{center}
        \begin{tabular}{|l|cccccccc|}
            \hline
            amarillo & 126 & 124 & 116 & 125 & 109 & 130 & 125 & 117   \\
            \hline
            blanco   & 130 & 142 & 133 & 132 & 150 & 120 & 130 & 117   \\
            \hline
        \end{tabular}
    \end{center}
    Suponiendo que los tiempos de secado se distribuyen seg\'un una
    normal,
    \begin{itemize}
        \item  [a)] Encontrar un intervalo de confianza del 90\% para la
        media del tiempo de secado de la pintura amarilla.

        \item  [b)] Encontrar un intervalo de confianza del 99\% para la
        desviaci\'on t\'{\i}pica del tiempo de secado de la pintura blanca.

        \item  [c)] ?`Existe alguna evidencia que indique que la
        pintura amarilla seca m\'as rapidamente que la blanca?
        \begin{itemize}
            \item  [i)] Para un nivel de significaci\'on del 1\%,
            dise\~{n}ar razonadamente un contraste de hip\'otesis y
            explicar el criterio de decisi\'on utilizado
            para verificar las hip\'otesis establecidas.

            \item  [ii)] ?`Qu\'e es el p-valor? Obtener el p-valor del
            contraste anterior e interpretarlo.
        \end{itemize}
    \end{itemize}
    \sol{ a) (116.8, 126.2); b) (6.28, 28.47).}
\end{prob}

\begin{prob}
    Una empresa de refrescos contrasta que el contenido medio de
    refresco en las latas es al menos de $33$cl. La hip\'{o}tesis
    alternativa es que el contenido medio es menor de $33$cl. Se puede
    asumir que el contenido de refresco en las latas sigue una
    distribuci\'{o}n normal con una desviaci\'{o}n t\'{i}pica de $2$cl.
    La regla de decisi\'{o}n adoptada es rechazar la hip\'{o}tesis nula
    si la capacidad media muestral es menor de $30$cl.
    \begin{itemize}
        \item  [(a)] Si se seleccionan muestras aleatorias de $n=100$
    latas ?`cu\'{a}l es la probabilidad de cometer un error de
    tipo I usando esta regla de decisi\'{o}n?

        \item  [(b)] Suponer que el verdadero valor para el contenido
    medio de refresco es $32$cl. Si se seleccionan muestras
    aleatorias de $n=100$ latas, ?`cu\'{a}l es la probabilidad de
    cometer un error de tipo II usando esta regla de decisi\'{o}n?
    \end{itemize}
\end{prob}

\begin{prob}
    Un cliente de correos presenta una queja respecto del tiempo de
    espera para la compra de sellos. Para ello lleva a cabo una
    encuesta a 45 clientes y encuentra que el tiempo de espera medio
    es de $5.1$ minutos con una desviaci\'{o}n t\'{\i}pica de 0.6
    minutos. Suponiendo que el tiempo de espera sigue una
    distribuci\'{o}n normal
    \begin{itemize}
        \item  [(a)] buscar el intervalo de confianza del 95\% para
    el tiempo medio de espera.

        \item  [(b)] Si queremos que el intervalo de confianza al
    95\% tenga una amplitud m\'{a}xima a cada lado de la media
    muestral de 1 minuto ?`cu\'{a}l deber\'{\i}a ser el m\'{\i}nimo
    tama\~{n}o de la muestra?
    \end{itemize}
    El director de correos contesta al cliente afirmando que el tiempo
    de espera es de 4 minutos.
    \begin{itemize}
        \item  [(c)] Dise\~{n}ar un contraste de hip\'{o}tesis para
    rechazar la afirmaci\'{o}n del director tomando $\alpha=0.1$.
    Explica razonadamente las hip\'{o}tesis que establezcas.

        \item  [(d)] Obtener el $p$--valor del contraste e interpretarlo.
    \end{itemize}
\end{prob}

\begin{prob}
    Debido a su dilatada experiencia, la Compa\~{n}\'{\i}a Long Life ha
    determinado que la duraci\'on de la bateria de su tel\'efono m\'ovil
    modelo 8800 es de 3.5 dias (suponer distribuci\'on normal).
    La OCU no est\'a de acuerdo con esta afirmaci\'on y lleva a cabo
    un estudio obteniendo los siguientes tiempos (en dias):
    $$2.6 \quad 1.8 \quad 5.5 \quad 3.4 \quad 1.7 \quad 2.9
    \quad 1.3 \quad 2.0 \quad 3.8 \quad 3.3$$
    \begin{itemize}
        \item  [a)]  ?`Qu\'e estimador utilizas para estimar la media
        poblacional? ?`por qu\'e? ?`Qu\'e tipo de distribuci\'on
        sigue el estimador? ?`por qu\'e?

    \item  [b)] Obtener una estimaci\'on puntual de la duraci\'on media
    de la bateria del m\'ovil. Construir un intervalo de confianza del
    80\% para la media poblacional.

        \item  [c)] Calcular la probabilidad de que la media muestral
        difiera de la media poblacional en menos de 0.5 dias.

        \item  [d)] Para $\alpha = 0.1$, dise\~{n}ar un contraste de
        hip\'otesis para rechazar la afirmaci\'on de la Compa\~{n}\'{\i}a
    Long Life. Explicar razonadamente las hip\'otesis que establezcas y
    tu conclusi\'on del contraste.

    \item  [e)] Define el p-valor y calc\'ulalo.
    \end{itemize}
\end{prob}

\begin{prob}
    Un nuevo programa de dieta se anuncia diciendo que los
    participantes perder\'an en promedio como m\'{\i}nimo 8 kilos
    durante los tres primeros meses. Una muestra aleatoria de 60
    participantes di\'o como resultado $\bar{x} = 7$
    kilos con $\hat{s}_{x} = 3.2$ kilos
    \begin{itemize}
    \item  [a)] Define hip\'otesis nula e hip\'otesis alternativa.
    Establece razonadamente la hip\'otesis nula y la hip\'otesis
        alternativa de este problema.

        \item  [b)] Para un nivel de significaci\'on de $\alpha =
        0.05$, ?`cu\'al es tu conclusi\'on sobre el anuncio? Razona
        tu respuesta.

    \item  [c)] Define el p-valor y calc\'ulalo.
    \end{itemize}
\end{prob}
