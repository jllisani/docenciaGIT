\documentclass[12pt]{article}
\usepackage[catalan]{babel}
\usepackage[latin1]{inputenc}   % Permet usar tots els accents i car�ters llatins de forma directa.
\usepackage{enumerate}
\usepackage{amsfonts, amscd, amsmath, amssymb}
\usepackage{graphicx}

\setlength{\textwidth}{16.5cm}
\setlength{\textheight}{24cm}
\setlength{\oddsidemargin}{-0.3cm}
\setlength{\evensidemargin}{1cm} \addtolength{\headheight}{\baselineskip}
\addtolength{\topmargin}{-3cm}

\def\N{I\!\!N}
\def\R{I\!\!R}
\def\Z{Z\!\!\!Z}
\def\Q{O\!\!\!\!Q}
\def\C{I\!\!\!\!C}

\begin{document}

%\pagestyle{empty}
\font\sc=cmcsc10
\parskip=1ex
\newcount\problemes

\newcommand\probl{\advance\problemes by 1 \vskip 2ex\noindent{\bf
\the\problemes) }}

\def\probl{\advance\problemes by 1
\vskip 1mm\noindent{\bf \the\problemes) }}
\newcounter{pepe}

\newcommand{\pr}[1]{P(#1)}

%\newcounter{problema}
%\newcommand{\prb}{\addtocounter{problema}{1}
%\noindent\vskip 2mm {\textbf{\theproblemes  }}
\newcommand{\sol}[1]{{\textbf{\footnotetext[\the\problemes]{Sol.: #1} }}}


\begin{centerline}
{\textbf{\textsc{PROBLEMES ESTAD\'ISTICA ENGINYERIA}}}
\end{centerline}

\problemes=0
\begin{centerline}
{\bf DISTRIBUCIONS DE PROBABILITAT DISCRETES NOTABLES}
\end{centerline}

\probl    Un estudiante contesta a una prueba con 20 cuestiones. Cada una de ellas    admite diez respuestas posibles, de las cuales s\'olo dos son verdaderas.    El cuestionario se cumplimenta eligiendo s\'olo una de las diez    opciones.    ?`Cu\'al es la probabilidad de responder mal a las 20 preguntas? ?`Cu\'al    es la probabilidad de responder bien a 5? ?`Cu\'al es la probabilidad de    responder bien a m\'as de 5? ?`Cu\'al es el valor esperado y la varianza    de las respuestas correctas?    \sol{ $X =$ ``n\'{u}mero de respuestas correctas'', 
$P_{X}(0) =    0.0115$, $P_{X}(5) = 0.1746$, % aproximando por una $\mathcal{P}(5)$    $P(X > 5) = 0.1958$,    % si aproximamos por una $\mathcal{P}(5)$ entonces $P(X > 5) \approx 0.384$,    
$E(X) = 4$, $Var(X) = 3.2$}

\probl    Un peque\~{n}o hotel rural dispone de 10 habitaciones. El    departamento de reservas hace 12 reservas al d\'{\i}a    porque sabe que la probabilidad de que cada una de ellas falle es    del 25\%. Las reservas se realizan de forma independiente.    \begin{itemize}        \item  [a)] Hallar la probabilidad de que en un d\'{\i}a        elegido al azar no tenga suficientes habitaciones para atender        todas las reservas. (Indica claramente la variable que utilizas    y el tipo de distribuci\'on que sigue la variable.)    \item  [b)] Hallar la probabilidad de que en un d\'{\i}a        elegido al azar no llene todas las habitaciones.        \item  [c)] Hallar la probabilidad de que en un fin de semana largo    (3 dias) no le falten habitaciones ning\'un dia.    \end{itemize}    \sol{ a) 0.0317, b) 0.6093, c) 0.9079}

\probl    Por una larga experiencia se ha estimado que el promedio de errores    tipogr\'aficos al componer un libro es de 2 por cada 20 p\'aginas.    \begin{itemize}        \item [a)] Hallar la probabilidad de que en un libro de 100    p\'aginas existan  a lo sumo 10 erratas.        \item [b)] Hallar la probabilidad de que en un libro de 50    p\'aginas existan m\'as de 15 erratas.        \item [c)] Hallar la probabilidad de que el n\'umero de erratas    de una publicaci\'on de 10 p\'aginas sea mayor o igual que 2 y    menor o igual que 4.    \end{itemize}    \sol{ a) 0.5830, b) 0.0001, c) 0.2605}

\probl    En promedio llegan 2.4 clientes por minuto al mostrador de una    compa\~n\'{\i}a a\'erea durante el per\'{\i}odo de m\'axima    actividad. Asumir que el n\'umero de llegadas es Poisson.    \begin{itemize}    \item [a)] ?`Cu\'al es la probabilidad de que no llegue nadie    en un minuto?    \item [b)] ?`Cu\'al es la probabilidad de que se produzcan m\'as    de tres llegadas en un minuto?    \end {itemize}    \sol{ a) 0.0907, b) 0.2213}

\probl  Una cadena de producci�n da salida a $10000$ unidades diarias,  el n�mero medio de
unidades incorrectas es $200$. Una vez al d�a, se inspecciona un lote de $100$ unidades.
Determinar la probabilidad de que el lote contenga m�s de $3$ unidades incorrectas
\begin{enumerate}[a)]
\item Utilizando la distribuci�n binomial.
\item Utilizando la aproximaci�n de Poisson.
\sol{a) $\mathbf{(0.1410)}$ ; b)  $\mathbf{(0.1429)}$}
\end{enumerate}

\probl  Los taxis llegan aleatoriamente (seg�n un proceso Poisson) a la terminal de un
aeropuerto con un ritmo medio de un taxi cada $3$ minutos. ?`Cu�l es la probabilidad de que
el �ltimo pasajero de una cola de $4$ tenga que esperar un taxi m�s de un cuarto de hora?
\sol{$\mathbf{(0.265)}$}

\probl  En una planta de fabricaci�n de circuitos integrados, la proporci�n de circuitos
defectuosos es $p$. Supongamos que la incidencia de circuitos defectuosos es completamente
aleatoria.
\begin{enumerate}[a)]
\item  Determinar la distribuci�n del n�mero $X$ de circuitos
aceptables producidos antes del primer circuito defectuoso.
\item  ?`Cu�l es la longitud media de una cadena de producci�n exitosa? si
$p = 0.05.$\sol{b) $\mathbf{(19)}$}
\end{enumerate}

\probl Un servidor de mensajer�a esta en funcionamiento. Los clientes acceden a �l de forma
independiente. La probabilidad de que el servidor caiga cuando accede el cliente es $p$.
Calcular la distribuci�n de probabilidad del n�mero de clientes a los que se dar� servicio
antes de que el servidor caiga.

\begin{enumerate}[a)]
\item Calcular el valor esperado y la varianza de esta variable.
\item  ?`Cu�l es la probabilidad de que se de servicio a m�s de $1000$ clientes sin que se
caiga el servidor?
\end{enumerate}
\sol{a) $\mathbf{(\frac{q}{p};\, \frac{q}p^2)}$ b) $\mathbf{(q^{1001})}$}

\probl Un sistema inform�tico  dispone de un sistema de seguridad compuesto por tres claves
de $3$ d�gitos (del $0$ al $9$) cada una. Para entrar en el sistema hay que averiguar la
primera clave, luego la segunda y por �ltimo la tercera. Un pirata inform�tico intenta
entrar ilegalmente en el sistema, para ello va introduciendo al azar distintas claves de
forma independiente, olvidando las que ha introducido antes. Calcular el valor esperado y
la varianza del n�mero de intentos antes de romper el sistema.

Comparar el resultado anterior cuando se ataca el sistema de forma similar pero cuando el
sistema de seguridad s�lo consta de una clave de $9$ d�gitos. ?`Cu�l es el sistema m�s
seguro, desde el punto de vista del n�mero de intentos necesarios para violarlo?


\probl La probabilidad de que una cajero autom�tico se estropee en una operaci�n es $p=0.001$.
Responder a las siguientes cuestiones:
\begin{enumerate}[a)]
\item ?`Cu�l es el n�mero esperado de operaciones correctas hasta el primer error?
\item ?`Cu�l es la varianza del n�mero de operaciones correctas hasta el primer error?
\item ?`Cu�l es la probabilidad de que el n�mero de operaciones consecutivas correctas 
sea mayor o igual que 100?
\end{enumerate}
\sol{ a) 999, b) 999000, c) 0.9048}

\end{document}

