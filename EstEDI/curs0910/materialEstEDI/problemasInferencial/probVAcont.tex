\documentclass[12pt]{article}
\usepackage[catalan]{babel}
\usepackage[latin1]{inputenc}   % Permet usar tots els accents i car�ters llatins de forma directa.
\usepackage{enumerate}
\usepackage{amsfonts, amscd, amsmath, amssymb}
\usepackage{graphicx}

\setlength{\textwidth}{16.5cm}
\setlength{\textheight}{24cm}
\setlength{\oddsidemargin}{-0.3cm}
\setlength{\evensidemargin}{1cm} \addtolength{\headheight}{\baselineskip}
\addtolength{\topmargin}{-3cm}

\def\N{I\!\!N}
\def\R{I\!\!R}
\def\Z{Z\!\!\!Z}
\def\Q{O\!\!\!\!Q}
\def\C{I\!\!\!\!C}

\begin{document}

%\pagestyle{empty}
\font\sc=cmcsc10
\parskip=1ex
\newcount\problemes

\newcommand\probl{\advance\problemes by 1 \vskip 2ex\noindent{\bf
\the\problemes) }}

\def\probl{\advance\problemes by 1
\vskip 1mm\noindent{\bf \the\problemes) }}
\newcounter{pepe}

\newcommand{\pr}[1]{P(#1)}

%\newcounter{problema}
%\newcommand{\prb}{\addtocounter{problema}{1}
%\noindent\vskip 2mm {\textbf{\theproblemes  }}
\newcommand{\sol}[1]{{\textbf{\footnotetext[\the\problemes]{Sol.: #1} }}}


\begin{centerline}
{\textbf{\textsc{PROBLEMES ESTAD\'ISTICA ENGINYERIA}}}
\end{centerline}

\problemes=0
\begin{centerline}
{\bf VARIABLES ALEAT�RIES CONT�NUES}
\end{centerline}

\probl  Sea  $X$ una variable aleatoria continua con funci�n de densidad $f(x)$ dada por:
$$f(x) = \begin{cases}
 k \cdot (1+x^2) & \text{si } x \in (0,3)\\ 0 & \text{si } x \not \in (0,3)
 \end{cases}$$
\begin{enumerate}[a)]
\item Calcular la constante $k$  y la funci�n de distribuci�n de $X$.
\item Calcular la probabilidad de que $X$ est� comprendida entre $1$ y $2$
\item Calcular la probabilidad de  que $X $ sea menor que $1.$
\item Sabiendo que $X$ es mayor que $1$, calcular la probabilidad de
que sea menor que $2$. \sol{a)$\mathbf{k=1/12}$; b) $\mathbf{5/18}$; c)$\mathbf{1/9}$; e)
$\mathbf{5/16}$}
\end{enumerate}

\probl  La funci�n de densidad de una variable aleatoria continua es: 
$$f(x) = \begin{cases} a
\cdot x^2 + b & \text{si } x \in (0,3)\\
 0 & \text{si } x \not \in (0,3) 
\end{cases}$$.

 Determinar $ a$ y $b$ , sabiendo que
$P(1 < X \leq 2) = 2/3$.  \sol{$\mathbf{a=-1/2, b=11/6}$}


% \probl  Sea $X$ una variable aleatoria
% continua con densidad:
%
%$$f(x) = \begin{cases}1-|x| & \text{si } |x| \leq 1\\ 0 & \text{en caso
%contrario}\end{cases}$$
%\begin{enumerate}[a)]
%\item  Encontrar  la funci�n de distribuci�n de $ X.$
%\item Calcular $ \pr{X \geq 0}$  y $\pr{|X| < 1/2}.$  \sol{b) $\mathbf{1/2,
%3/4}$}
%\end{enumerate}




\probl  Consideremos $ f:{\R}\to {\R}  $ dada por

$$f(x) = \begin{cases} 0 & \text{si } x \leq 0\\ a (1+x) & \text{si } 0 < x \leq
1\\ 2/3 & \text{si } 1 < x \leq 2\\ 0 & \text{si } x > 2\end{cases}$$
\begin{enumerate}[a)]
\item  Determinar el valor de $a$ para que $f$  sea una densidad.
\item  En este caso, si $X$ es una variable aleatoria
 continua con densidad $f$, calcular $ \pr{1/2 < X\leq 3/2}$.
 \item Calcular, para el valor de $a$ encontrado $E(X)$ y $Var(X)$.
 \sol{a)$\mathbf{a=2/9}$ ; b) $\mathbf{19/36}$; c) $E(X)=\frac{32}{27}$,
    $Var(X)=\frac{409}{1458}$.}
\end{enumerate}


%\probl  Conocida  la funci�n de distribuci�n de una variable aleatoria continua $X$, hallar
%la funci�n de densidad de $\displaystyle Y = X^2$ y de $ Z = e^X.$

\probl  El precio por estacionar un veh�culo en un aparcamiento es de $0.75$ euros por  la
primera hora o fracci�n, y de $0.60$ euros a partir de la segunda hora o fracci�n. Supongamos
que el tiempo, en horas, que un veh�culo cualquiera permanece en el aparcamiento se
modeliza seg�n la siguiente funci�n de densidad

$$f_{X}(x) = \begin{cases}e^{-x} & \text{si } x \geq 0\\ 0 & \text{si } x <
0\end{cases}$$

Calcular  el ingreso medio por veh�culo. \sol{$\mathbf{(1.1)}$}

\probl Sea $X$ una variable aleatoria con funci�n de densidad:

$$f_{X}(x) = \begin{cases}2 x & \text{si }x \in (0,1)\\ 0 & \text{si } x \not \in
(0,1)\end{cases}$$
Determinar $E(\sqrt{X})$.
\sol{$\mathbf{(4/5)}$}

\probl  Consideremos una variable aleatoria $X$ con funci�n de densidad $f_X$ dada por:
$$f_X(x) = \left\{\begin{array}{ll}
1/2 & \mbox{si }  0<x<2\\ 0 & \mbox{en otro caso}\end{array}\right.$$
 determinar
$\mathrm{E}( Y)$, donde $Y= \ln X.$ \sol{$\mathbf{(-0.3069)}$}



\end{document}

