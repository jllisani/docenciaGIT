\documentclass[12pt]{article}
\usepackage[latin1]{inputenc}   % Permet usar tots els accents i car�ters llatins de forma directa.
\usepackage{graphicx}
\usepackage{enumerate}
\setlength{\textwidth}{16.5cm}
\setlength{\textheight}{24cm}
\setlength{\oddsidemargin}{-0.3cm}
\setlength{\evensidemargin}{1cm} \addtolength{\headheight}{\baselineskip}
\addtolength{\topmargin}{-3cm}

\def\N{I\!\!N}
\def\R{I\!\!R}
\def\Z{Z\!\!\!Z}
\def\Q{O\!\!\!\!Q}
\def\C{I\!\!\!\!C}

\begin{document}

%\pagestyle{empty}
\font\sc=cmcsc10
\parskip=1ex
\newcount\problemes

\newcommand\probl{\advance\problemes by 1 \vskip 2ex\noindent{\bf
\the\problemes) }}

\def\probl{\advance\problemes by 1
\vskip 1mm\noindent{\bf \the\problemes) }}
\newcounter{pepe}

\newcommand{\pr}[1]{P(#1)}

%\newcounter{problema}
%\newcommand{\prb}{\addtocounter{problema}{1}
%\noindent\vskip 2mm {\textbf{\theproblemes  }}
\newcommand{\sol}[1]{{\textbf{\footnotetext[\the\problemes]{Sol.: #1} }}}


\begin{centerline}
{\textbf{\textsc{PROBLEMES ESTAD\'ISTICA ENGINYERIA}}}
\end{centerline}

\problemes=0
\begin{centerline}
{\bf VARIABLE ALEAT�RIA DISCRETA}
\end{centerline}

\textbf{En els problemes seg\"{u}ents,(1 a 9)  determinau la
funci\'{o} de probabilitat i la de distribuci\'{o} de totes les variables
aleat\`{o}ries que apareixen. }

\probl  Considerem l'experiment consistent en llan\c{c}ar
simult\`{a}niament dos daus; repetim l'ex\-pe\-ri\-ment 2 vegades.
Sigui $X$ la variable aleat\`{o}ria que d\'{o}na el nombre de llan\c{c}aments
en qu\`{e} els dos daus han mostrat un nombre parell. Sigui $Y$ la
variable aleat\`{o}ria que d\'{o}na el nombre de llan\c{c}aments en qu\`{e} la
suma dels dos daus ha donat un nombre parell.

\probl  Suposem que tenim un estoc de 10 peces, de les quals sabem
que n'hi ha 8 del tipus I i 2 del tipus II; n'agafam dues a
l'atzar. Sigui $X$ la variable aleat\`{o}ria que d\'{o}na el nombre de
peces del tipus I que hem agafat.

\probl  Suposem que un alumne realitza el tipus d'examen seg\"{u}ent:
El professor li va formulant preguntes fins que l'alumne en falla
una (no vos demaneu com se l'avalua, ni jo ho s\'{e}). La probabilitat
que l'alumne encerti una resposta qualsevol \'{e}s 0.9 (examen f\`{a}cil).
Sigui $X$ la variable aleat\`{o}ria que d\'{o}na el nombre de preguntes
formulades a l'alumne. Quin \'{e}s el nombre m\'{e}s probable de preguntes
formulades?

\probl   Considerem dos canons que van disparant alternativament
cap a un mateix objectiu. El primer can\'{o} t\'{e} una probabilitat
d'encert igual a 0.3 i el segon igual a 0.7. El primer can\'{o}
comen\c{c}a la s\`{e}rie de llan\c{c}aments i no s'aturen fins que un dels dos
encerta el blanc. Siguin $X$ la variable aleat\`{o}ria que d\'{o}na el
nombre de projectils llan\c{c}ats pel primer can\'{o} i $Y$ el nombre de
llan\c{c}aments fets pel segon can\'{o}.

\probl  El mateix problema anterior, calculau la probabilitat de
la variable aleat\`{o}ria $X$ que d\'{o}na el nombre de projectils
llan\c{c}ats pel primer can\'{o} condicionat a que guanya i $Y$ \'{e}s la
variable aleat\`{o}ria que d\'{o}na el nombre de projectils llan\c{c}ats pel
segon can\'{o} condicionat a que guanya.

\probl  Suposem que se fa una tirada de 100.000 exemplars d'un
determinat llibre. La probabilitat d'una enquadernaci\'{o} incorrecta
\'{e}s 0.0001. Quina \'{e}s la probabilitat que hi hagi 5 llibres de la
tirada mal enquadernats?

\probl  Dos companys d'estudis se troben en un conegut pub i
decideixen jugar a dards d'una manera especial: llan\c{c}aran
consecutivament un dard perhom fins que un dels dos encerti el
triple 20. El que llan\c{c}a en primer lloc t\'{e} una probabilitat 0.7
d'encertar-lo i el que ho fa en segon lloc, una probabilitat 0.8.
Sigui $X$ la variable aleat\`{o}ria que d\'{o}na el nombre total de
llan\c{c}aments de dards fets pels dos companys.

\probl  Un examen tipus test consta de 5 preguntes amb 3 possibles
opcions cadascuna, de les quals nom\'{e}s una \'{e}s la correcta. Un
alumne contesta a l'atzar les 5 q\"{u}estions. Sigui $X$ la variable
aleat\`{o}ria que d\'{o}na el nombre de punts obtinguts per l'alumne:
\begin{enumerate}[a)]
\item Si les respostes err\`{o}nies no resten punts.
\item Si cada resposta err\`{o}nia resta 1 punt.
\end{enumerate}

\probl  Un coche tiene que  pasar por cuatro sem�foros. En cada uno de ellos el coche tiene
la misma probabilidad de seguir su marcha que de detenerse. Sea $X$ la variable aleatoria
que cuenta n�mero de sem�foros que pasa el coche sin detenerse.

\probl Calcular la esperanza y la varianza de todas las
variables que aparecen en los problemas anteriores.


\probl Un individuo quiere invertir un capital de medio mill�n de euros en un negocio que
tiene una rentabilidad del $50\%$, pero con el riesgo de perder toda la inversi�n. Su
asesor financiero le informa que este negocio tiene una probabilidad de ser rentable del
$0.8$ ?`Cu�l es el beneficio esperado? \sol{$\mathbf{(100000)}$}

\probl Un juego se dice justo si la ganancia esperada de cada jugador es $0$. Dos jugadores
A y  B tiran un dado por turnos, y gana el primero que obtiene un $5$. Cada jugador apuesta
una cantidad $ c_j \ (j=1,2),  $  y el total se lo queda el ganador. Si suponemos que
comienza  a jugar A ?`qu� relaci�n tienen que verificar $c_1$ y $c_2$ para que el juego sea
justo? \sol{$6c_2=5c_1$}



\probl Se venden 5000 billetes de loter�a a 1 euro. cada uno, para  un sorteo con un
premio de  3000 euros ?`Cu�l es la ganancia (p�rdida) esperada de una persona que compra
tres billetes? \sol{$\mathbf{(-1.20\quad \textrm{euros})}$}



\probl  Dos personas juegan a cara o cruz, y han decidido continuar la partida hasta que se
obtengan como m�nimo $3$ caras  y $3$ cruces. Hallar la probabilidad de que el juego no se
acabe en $10$ tiradas o menos y el n�mero esperado de tiradas.
\sol{$\mathbf{\frac{7}{64}=0.109375}$; $\mathbf{E(X)=\frac{63}{8}}$.}

%%%\probl  Calcular la esperanza y la varianza del n�mero de puntos obtenidos en el
%%%lanzamiento de un dado. \sol{$\mathbf{(7/2,\quad 35/12)}$}



\probl Sea $X$ la variable que nos da la puntuaci�n obtenida al lanzar un dado. Calcular la
distribuci�n de las variables $Y=X^2$, $Z=X^2-6X+6$. Calcular las esperanzas y las
varianzas de las variables $Y$ y $Z$.

%\probl Un sistema de transmisi�n emite los d�gitos -1, 0, 1. Cuando se transmite el s�mbolo
%{\it i}, se recibe el s�mbolo {\it j} con las probabilidades siguientes: $ \pr{r_1 / t_1} =
%1, \ \pr{r_{-1} / t_{-1}} = 1, \ \pr{r_1 / t_0} = 0.1, \ \pr{r_{-1} / t_0} = 0.1, \ \pr{r_0
%/ t_0} = 0.8. $ Se dice que en este caso se ha producido una distorsi�n $ \displaystyle
%(i-j)^2. $ ?`Cu�l es el valor medio de la distorsi�n? \sol{$\mathbf{(1/15)}$}


\probl   Un contratista estima la probabilidad del n\'{u}mero de d\'{\i}as    necesarios para concluir un proyecto como indica la tabla siguiente:    \begin{center}    \begin{tabular}{l|lllll}        Tiempo (en d\'{\i}as) & 1    & 2    & 3    & 4    & 5    \\        \hline        Probabilidad          & 0.05 & 0.20 & 0.35 & 0.30 & 0.10    \end{tabular}    \end{center}    \begin{itemize}    \item [a)] ?`Cu\'{a}l es la probabilidad de que un proyecto elegido    aleatoriamente necesite de tres d\'{\i}as para su conclusi\'{o}n?    \item [b)] Hallar el tiempo esperado necesario para acabar un    proyecto.    \item [c)] Hallar la desviaci\'{o}n t\'{\i}pica del tiempo necesario    para terminar un proyecto.    \item [d)] El coste del proyecto se divide en dos partes: un coste    fijo de dos mil euros, m\'{a}s 200 euros por cada    d\'{\i}a de duraci\'{o}n del proyecto. Hallar la media y la    desviaci\'{o}n t\'{\i}pica del coste total del proyecto.    \end{itemize}    \sol{ a) 0.35, b) 3.2, c) 1.0296, d) 2640; 205.912}


\probl    Una tienda vende paquetes de caramelos. El n\'umero de    caramelos por paquete var\'{\i}a tal como indica la tabla adjunta.    $$    \begin{tabular}{|l|ccccccc|}    \hline    caramelos    & 97   & 98   & 99   & 100  & 101  &102   &103   \\    \hline    probabilidad & 0.05 & 0.14 & 0.21 & 0.29 & 0.20 & 0.09 & 0.02 \\    \hline    \end{tabular}    $$    \newline    a) ?`Cu\'al es la probabilidad de que un paquete elegido al azar    tenga 101 o m\'as caramelos?    \newline    b) Halla el n\'umero esperado de caramelos por paquete y la    desviaci\'on t\'{\i}pica.    \newline    c) El coste en la elaboraci\'on de un paquete de caramelos viene dada    por una cantidad fija de 2.00 euros m\'as 0.05 euros por cada caramelo.    Cada paquete de caramelos cuesta 10.00 euros (independientemente del    n\'umero de caramelos que contiene). Halla la media y la desviaci\'on    t\'{\i}pica del beneficio por paquete.    \sol{ a) 0.31, b) E(X) = 99.8, d.t.(X) = 1.386, c) E(B) = 3.01,    d.t.(B) = 0.069.}

\end{document}
