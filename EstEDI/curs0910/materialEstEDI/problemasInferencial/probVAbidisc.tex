\documentclass[12pt]{article}
\usepackage[catalan]{babel}
\usepackage[latin1]{inputenc}   % Permet usar tots els accents i car�ters llatins de forma directa.
\usepackage{enumerate}
\usepackage{amsfonts, amscd, amsmath, amssymb}
\usepackage{graphicx}

\setlength{\textwidth}{16.5cm}
\setlength{\textheight}{24cm}
\setlength{\oddsidemargin}{-0.3cm}
\setlength{\evensidemargin}{1cm} \addtolength{\headheight}{\baselineskip}
\addtolength{\topmargin}{-3cm}

\def\N{I\!\!N}
\def\R{I\!\!R}
\def\Z{Z\!\!\!Z}
\def\Q{O\!\!\!\!Q}
\def\C{I\!\!\!\!C}

\begin{document}

%\pagestyle{empty}
\font\sc=cmcsc10
\parskip=1ex
\newcount\problemes

\newcommand\probl{\advance\problemes by 1 \vskip 2ex\noindent{\bf
\the\problemes) }}

\def\probl{\advance\problemes by 1
\vskip 1mm\noindent{\bf \the\problemes) }}
\newcounter{pepe}

\newcommand{\pr}[1]{P(#1)}

%\newcounter{problema}
%\newcommand{\prb}{\addtocounter{problema}{1}
%\noindent\vskip 2mm {\textbf{\theproblemes  }}
\newcommand{\sol}[1]{{\textbf{\footnotetext[\the\problemes]{Sol.: #1} }}}


\begin{centerline}
{\textbf{\textsc{PROBLEMES ESTAD\'ISTICA ENGINYERIA}}}
\end{centerline}

\problemes=0
\begin{centerline}
{\bf VARIABLES ALEAT�RIES VECTORIALS DISCRETES}
\end{centerline}


\probl
Los estudiantes de una universidad se clasifican de acuerdo a sus a�os en la universidad
($X$) y el n�mero de visitas a un museo el �ltimo a�o($Y=0$ si no  hizo ninguna visita
$Y=1$ si hizo una visita, $Y=2$ si hizo m�s de una visita). En la tabla siguiente aparecen
la probabilidades conjuntas que se estimaron para estas dos variables:

    $$\begin{tabular}{c|cccc}
        \hline\\
        N�m. de Visitas (Y) & &N�m. de a�os (X) & & \\
        \hline
          & 1& 2 & 3  & 4 \\
         \hline\\
         0 & 0.07 & 0.05 & 0.03 & 0.02\\
         1 & 0.13 & 0.11 & 0.17 & 0.15\\
         2 & 0.04 & 0.04 & 0.09 & 0.10\\
         \hline
        \end{tabular}
        $$
        \begin{enumerate}[a)]
            \item Hallar la probabilidad de que un estudiante elegido
            aleatoriamente no haya visitado ning�n museo el �ltimo a�o.
            \item Hallar las medias de las variables aleatoria $X$ e $Y$.
            \item Hallar e interpretar la covarianza y la correlaci�n
            entre las variables aleatorias $X$ e $Y$.
            \end{enumerate}
        \sol{a) 0.17; b) $E(X)= 2.59$, $E(Y)=1,10$; c) $Cov(X,Y)=0.191$,
        $r_{XY}$=0.259291}
       

\probl
    Un vendedor de libros  de texto realiza llamadas  a los despachos de
    lo profesores, y tiene la impresi�n que �stos suelen ausentarse
    m�s de los despachos los viernes  que cualquier otro d�a
    laborable. Un repaso a las llamadas, de las cuales un quinto se
    realizan los viernes, indica que  para el 16\% de las llamadas
    realizadas en viernes, el profesor no estaba en su despacho, mientras
    que esto ocurre s�lo  para el 12\% de llamadas que se realizan en
    cualquier otro d�a laborable. Definamos las variables aleatorias
    siguientes:
    $$X=\left\{\begin{array}{ll}
    1 & \mbox{si la llamada es realizada el viernes}\\
    0 & \mbox{en cualquier otro caso}
    \end{array}\right.
    $$
    $$Y=\left\{\begin{array}{ll}
    1 & \mbox{si  el profesor est� en el despacho}\\
    0 & \mbox{en cualquier otro caso}
    \end{array}\right.
    $$
    \begin{enumerate}[a)]
    \item Hallar la funci�n de probabilidad conjunta de $X$ e $Y$.
    \item Hallar la funci�n de probabilidad condicional de $Y$, dado
    que $X=0$.
    \item Hallar las funciones de probabilidad marginal de $X$ e $Y$.
    \item Hallar e interpretar la covarianza de $X$ e $Y$.
    \end{enumerate}
    \sol{\begin{tabular}{l|ll|l|}
     & & $X$ & \\
    \hline
    $Y$ & 0 & 1 & $P_{Y}(y)$\\
    \hline
    0 & 0.096 & 0.032 & 0.128\\
    1 & 0.704 & 0.168&  0.872\\
    \hline
    $P_{X}(x)$ & 0.8 & 0.2 & 1
    \end{tabular};
    b) \begin{tabular}{l|ll}
    $Y$ & 0 & 1 \\
    \hline
     $P_{Y/X}(y|0)$&0.12& 0.88
    \end{tabular}
    d) $Cov(X,Y)=-0.0064$}
 


\probl
 Se lanzan al aire dos dados de diferente color, uno es blanco y el otro rojo.
     Sea $X$ la variable aleatoria "n�mero de puntos obtenidos con el dado blanco, e $Y$
     la variable aleatoria "n�mero m�s grande de puntos obtenido entre los dos dados".
\begin{enumerate}[a)]
\item Determinar la funci�n de probabilidad conjunta.
\item Obtener las funciones de probabilidad marginales.
\item ?`Son independientes? (\textbf{No})
\end{enumerate}



\probl
  Si $X_1 \mbox{ y } X_2 $ son dos  variables aleatorias con distribuci�n Poisson,
   independientes y con medias respectivas $\alpha \mbox{ y } \beta$, probar que $Y = X_1 +
X_2 $ tambi�n una variable aleatoria Poisson (con media $\> \alpha + \beta$).




\end{document}