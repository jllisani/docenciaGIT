\documentclass{article}
\usepackage[catalan]{babel}
\usepackage[latin1]{inputenc}   % Permet usar tots els accents i car�ters llatins de forma directa.
\usepackage{enumerate}
\usepackage{amsfonts, amscd, amsmath, amssymb}
\usepackage{graphicx}

\setlength{\textwidth}{16.5cm}
\setlength{\textheight}{24cm}
\setlength{\oddsidemargin}{-0.3cm}
\setlength{\evensidemargin}{1cm} \addtolength{\headheight}{\baselineskip}
\addtolength{\topmargin}{-3cm}

\def\N{I\!\!N}
\def\R{I\!\!R}
\def\Z{Z\!\!\!Z}
\def\Q{O\!\!\!\!Q}
\def\C{I\!\!\!\!C}

\begin{document}

%\pagestyle{empty}
\font\sc=cmcsc10
\parskip=1ex
\newcount\problemes

\newcommand\probl{\advance\problemes by 1 \vskip 2ex\noindent{\bf
\the\problemes) }}

\def\probl{\advance\problemes by 1
\vskip 1mm\noindent{\bf \the\problemes) }}
\newcounter{pepe}

\newcommand{\pr}[1]{P(#1)}

%\newcounter{problema}
%\newcommand{\prb}{\addtocounter{problema}{1}
%\noindent\vskip 2mm {\textbf{\theproblemes  }}
\newcommand{\sol}[1]{{\textbf{\footnotetext[\the\problemes]{Sol.: #1} }}}


\begin{centerline}
{\textbf{\textsc{PROBLEMES ESTAD\'ISTICA ENGINYERIA}}}
\end{centerline}

\problemes=0
\begin{centerline}
{\bf VARIABLES ALEAT�RIES VECTORIALS CONT�NUES}
\end{centerline}

\probl
Las variables aleatorias continuas $X$ e $Y$ tienen por funci�n de densidad

$$
f(x,y)=\left\{\begin{array}{ll} k(3x^2+2y) & \mbox{si } 0\leq x \leq 1 \mbox{ y }  0\leq y
\leq 1\\ 0 & \mbox{en cualquier otro caso}
\end{array}\right.
$$
Se pide:
\begin{enumerate}[a)]
    \item El valor de la constante $k$.
    \item Las funciones de densidad marginales.
    \item Las probabilidades $P(X\leq 0.5)$ y $P(Y\leq 0.3)$
    \item Las medias y las varianzas  de $X$ y de $Y$.
    \item La covarianza de $X$ e $Y$.
    \item La matriz de varianzas-covarianzas y la de correlaciones.
    \end{enumerate}
    \sol{a) $k=\frac{1}{2}$;
     b) $f_{X}(x)=\left\{\begin{array}{ll}
    \frac{1}{2} (3x^2+1) & \mbox{si } 0\leq x\leq 1\\
    0 & \mbox{en el resto de casos}\end{array}\right.$;
    $f_{Y}(y)=\left\{\begin{array}{ll}
    \frac{1}{2} (1+2y) & \mbox{si } 0\leq y\leq 1\\
    0 & \mbox{en el resto de casos}\end{array}\right.$;
    c) $P(X\leq 0.5)=0.312$ y $P(Y\leq 0.3)=0.195$;
    d) $E(X)=\frac{5}{8}$; $E(Y)=\frac{7}{12}$; $Var(X)=\frac{73}{960}$;
    $Var(Y)=\frac{11}{144}$.
    e) $Cov(X,Y)=\frac{-1}{96}$; f)$\left(\begin{array}{rr}
     \frac{73}{960} & \frac{-1}{96}\\
      \frac{-1}{96} & \frac{11}{144}\end{array}\right)$}



\probl
    Los gastos $X$  e ingresos $Y$ de una familia se consideran como una
    variable bidimensional con funci�n de densidad dada por:


$$
f(x,y)=\left\{\begin{array}{ll} k (x+y)  & \mbox{si } 0\leq x \leq 100 \mbox{ y }  0\leq y
\leq 100\\ 0 & \mbox{en cualquier otro caso}
\end{array}\right.
$$
Se pide:
\begin{enumerate}[a)]
    \item El valor de la constante $k$ para que $f(x,y)$ sea densidad.
    \item La probabilidad $P(0\leq X\leq 60, 0\leq Y\leq 50)$
    \item Las funciones de densidad marginales.
    \item Los gastos e ingresos medios.
    \item La covarianza de $X$ e $Y$.
    \item La matriz de varianzas-covarianzas.
    \item \textbf{Opcional.} La densidad (condicionada) de los gastos
    de las familias con ingresos $Y=50$. La esperanza de los gastos
    condicionados a que los ingresos valen $Y=50$.
    \end{enumerate}
    \sol{a) $k=\frac{1}{100^3}$; b) 0.165;
    c) $f_{X}(x)=\left\{\begin{array}{ll}
    \frac{x+50}{100^2} & \mbox{ si } 0<x<100\\
    0 & \mbox{ en el resto de casos}
    \end{array}\right.$, la de $Y$ es similar;
    d)$E(X)=E(Y)=\frac{175}{3}$;
    e) $Cov(X,Y)=\frac{-625}{9}$;
    f) $\left(\begin{array}{cc}
      \frac{6875}{9} & \frac{-625}{9}\\
      \frac{-625}{9} &  \frac{6875}{9} \end{array}\right)$;
    g) $f_{X/Y}(x|50)=
    \left\{ \begin{array}{ll}
\frac{1}{100^2} (x+50) & \mbox{si } 0\leq x \leq 100 \\ 0 & \mbox{en cualquier otro caso}
\end{array}\right.$;
 $E(X/Y=50)=\frac{175}{3}$}



\probl
 Dos amigos desayunan cada ma�ana en una cafeter�a entre la 8 y las 8:30 de la ma�ana. La
distribuci�n  conjunta de sus tiempos de llegada es uniforme en dicho intervalo, es decir
(y para simplificar tomando el tiempo en minutos):
$$f(x,y)=\left\{\begin{array}{ll}
k  & \mbox{ si } 0\leq x \leq 30 \mbox{ y } 0\leq y \leq 30\\ 0 & \mbox{ en el resto de
casos}
\end{array}\right. .$$
Si los amigos esperan un m�ximo de 10 minutos, calcular la probabilidad de que se
encuentren (sugerencia: resolverlo gr�ficamente). Calcular las distribuciones marginales.
\sol{(Para la soluci�n completa v�ase Daniel Pe�a ``\textit{Estad�stica Modelos y M�todos.
Vol 1. 2 Ed p�g.160})  \  $\frac{5}{9}$; las marginales son uniformes en el intervalo
$(0,30)$ }


\probl
La variable $X$ representa la proporci�n de errores tipo A en ciertos documentos y la
variable $Y$ la proporci�n de errores de tipo B. Se verifica que $X+Y\leq 1$ ( es decir
puede haber m�s tipos de errores posibles) y la densidad conjunta de ambas variables es
$$f(x,y)=\left\{\begin{array}{ll}
k  & \mbox{ si } 0\leq x \leq 1\mbox{; } 0\leq y\leq 1\mbox{ y } x+y\leq 1\\ 0 & \mbox{ en
el resto de casos}
\end{array}\right. .$$
\begin{enumerate}[a)]
\item Calcular el valor de la constante $k$.
\item \textbf{Opcional.} Calcular la densidad condicional de $X$ a $Y=y_{0}$  con
$0<y_{0}<1$.
\item \textbf{Opcional.}  Calcular la esperanza de la variable condicionada del
apartado anterior.
\item Calcular el vector de medias y  la matriz de correlaciones de $(X,Y)$
\end{enumerate}
\sol{a) $k=2$; b) Dado $0<y_{0}<1$ la densidad condicional de $X$ a $Y=y_{0}$ es
$f_{X|Y}(x|y_{0})= \left\{\begin{array}{ll} \frac{1}{(1-y)} & \mbox{si } 0<x<1 \mbox{y }
x<1-y_{0}\\ 0 & \mbox{en el resto de casos}
\end{array}\right.$; c) $\frac{1}{3}$}


\probl
  La funci�n de densidad conjunta de dos variables aleatorias  continuas
es:
$$f (x,y) = \begin{cases}k(x+xy) & \text{si } (x,y) \in (0,1)^2\\ 0 & \text{en otro caso}\end{cases}$$

\begin{enumerate}[a)]
\item Determinar $\> k.$ 
\item Encontrar las funciones de densidad marginales.
\item ?`Son independientes? 
\end{enumerate}
\sol{a) $\mathbf{4/3}$, c) S�}

\probl
 Las variables aleatorias $X_1 \mbox{ y  } X_2$ son independientes y ambas tienen la misma densidad
$$f(x) = \begin{cases}1 & \text{si } 0 \leq x \leq 1\\ 0 & \text{en caso contrario}\end{cases}$$
\begin{enumerate}[a)]
\item Determinar la densidad de $Y = X_1 + X_2.$
\item Determinar la densidad de $Z = X_1 - X_2.$
\item Calcular la esperanza y varianza de $Y$ y de $Z$.
\end{enumerate}


\probl
 Un proveedor de servicios inform�ticos tiene  una cantidad $X$ de cientos de unidades
de un cierto producto al principio de cada mes. Durante el mes se venden $Y$ cientos de
unidades del producto. Supongamos que  $X$
 e $Y$ tienen una densidad conjunta dada por

$$f(x,y) = \begin{cases}2/9 & \text{si }  0 < y < x < 3\\ 0 & \text{en caso
contrario}\end{cases}$$
\begin{enumerate}[a)]
\item Comprobar que {\it f} es una densidad.
\item Determinar $F_{X,Y}.$
\item \textbf{Opcional.}  Calcular la probabilidad de que al final de mes se hayan vendido como m�nimo la mitad de las unidades
que hab�a inicialmente. 
\item \textbf{Opcional.}  Si se han vendido 100 unidades, ?`cu�l es la probabilidad de que hubieran, como m�nimo 200
a principio de mes? 
\end{enumerate}
\sol{c) $\mathbf{1/2}$, d) $\mathbf{1/2}$ }

\probl
 \textbf{Opcional.}  Sean $X$ e $Y$ dos variables aleatorias conjuntamente absolutamente continuas.
Supongamos que

$$ f_{X}(x) = \begin{cases}4x^3 & \text{si } 0 < x < 1\\ 0 & \text{en caso
contrario}\end{cases}$$

y que

$$ f_{Y}(y|x) = \begin{cases}{2y \over x^2} & \text{si } 0 < y < x\\ 0 &
\text{en caso contrario}\end{cases}$$
\begin{enumerate}[a)]
\item  Determinar $f_{X,Y}$.
\item Obtener la distribuci�n de $Y.$
\item  Calcular $f_{X}(x|y).$
\end{enumerate}


\probl
  Sea $W = X + Y + Z$, donde  $X$, $Y$ y $Z$ son variables aleatorias con media
0 y varianza 1,
\begin{enumerate}[a)]
\item Sabiendo que  $ Cov(X,Y) = 1/4, Cov(X,Z) = 0, Cov(Y,Z) = -1/4.$
calcular la esperanza y la varianza de $W$. 
\item Sabiendo  que $X, Y$ y $Z$ 
son incorreladas calcular la esperanza y la varianza de $W$. 
\item Calcular la esperanza y la varianza de $W$ si  $ Cov(X,Y) = 1/4, Cov(X,Z) = 1/4,
 Cov(Y,Z) = 1/4.$ 
\item Sabiendo  que $X, Y$ y $Z$ son independientes calcular $Var(W)$ y $E(W)$.
\end{enumerate}
\sol{a) $\mathbf{0; 3}$, b) $\mathbf{0; 3}$, c) $\mathbf{0; 15/4 }$ }

\end{document}