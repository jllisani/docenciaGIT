\documentclass{beamer}
\usepackage[spanish]{babel}
\usepackage[utf8]{inputenc}
\usepackage{beamerthemeshadow}

\title{LaTeX - Beamer}
\subtitle{Introduccion a Beamer}
\author{
	Juan Dapena Paz
}
\date{$19$ de Enero de $2009$}

\begin{document}
\begin{frame}
	\titlepage
\end{frame}

\begin{frame}
	\frametitle{Índice}
	\tableofcontents
\end{frame}

\section{Instalación en Debian}
	\begin{frame}
		\frametitle{Que nos aporta esta instalación}
			\begin{itemize}
				\item Instalación de Beamer
				\item La instalación ocupa sobre 500 MB
				\item Soporte para temas
				\item Soporte para Español
			\end{itemize}
	\end{frame}
	
	\begin{frame}[fragile]{Proceso de instalación}
		\begin{block}{Desde superusuario...}
		\begin{verbatim}
			su
		\end{verbatim}
		\end{block}
		\begin{block}{Instalación de todo lo necesario para Beamer}
		\begin{verbatim}
			apt-get install latex-beamer
		\end{verbatim}
		\end{block}
		\begin{block}{Para soporte de idioma}
		\begin{verbatim}
			apt-get install texlive-latex-extra \
			texlive-latex-recommended
		\end{verbatim}
		\end{block}
	\end{frame}

\section{Elementos de un documento}
\subsection{Cabecera del documento}
	\begin{frame}[fragile]
		\frametitle{Cabecera del documento}
		Para el caso de este documento
		\begin{block}{El tipo de documento}
			\begin{verbatim}
				\documentclass{beamer}
			\end{verbatim}
		\end{block}
		\begin{block}{El lenguaje}
			\begin{verbatim}	
				\usepackage[spanish]{babel}
			\end{verbatim}
		\end{block}
		\begin{block}{El tipo de codificación}
			\begin{verbatim}
				\usepackage[utf8]{inputenc}
			\end{verbatim}
		\end{block}
		\begin{block}{El tema de las transparencias (opcional)}
			\begin{verbatim}
				\usepackage{beamerthemeshadow}
			\end{verbatim}
		\end{block}
	\end{frame}
	
\subsection{Cuerpo del documento}
	\begin{frame}[fragile]
		\frametitle{Creación de una transparencia}
		\begin{block}{El contenido de la transparencia irá dentro}
			\begin{verbatim}
				\begin{frame}
				...
				\\end{frame}
			\end{verbatim}
			o
			\begin{verbatim}
				\frame{
				...
				}
			\end{verbatim}
			\alert{En el end, sobra la primera barra}
		\end{block}
	\end{frame}
	
	\begin{frame}[fragile]
		\frametitle{Definición de una tabla de contenidos}
		\begin{block}{Código}
			\begin{verbatim}
				\begin{frame}
				   \title{Índice}
				   \tableofcontents
				\\end{frame}
			\end{verbatim}
		\end{block}
	\end{frame}
	
	\begin{frame}[fragile]
		\frametitle{Definición de una lista}
		\begin{block}{Aparecen todos los elementos}
			\begin{verbatim}
				\begin{itemize}
					\item loquesea
					\item loquesea 2
					\item loquesea 3
				\\end{itemize}
			\end{verbatim}
		\end{block}
		\begin{block}{Aparecen gradualmente los elementos}
			\begin{verbatim}
				\begin{itemize}
					\item<1-> loquesea
					\item<2-> loquesea 2
					\item<3-> loquesea 3
				\\end{itemize}
			\end{verbatim}
		\end{block}
	\end{frame}
	
	\begin{frame}[fragile]
		\frametitle{Introducción de texto plano}
		\begin{block}{Código}
			\begin{verbatim}
				begin{verbatim}
					...
				end{verbatim}
			\end{verbatim}
			\alert{No tiene \ delante, porque si no el compilador creería que estoy realmente metiendo una sentencia verbatim}
		\end{block}
	\end{frame}
	
	\begin{frame}[fragile]
		\frametitle{Introducción de bloques}
		\begin{block}{Código}
			\begin{verbatim}
				\begin{block}{titulo del bloque}
				...
				\\end{block}
			\end{verbatim}
		\end{block}
	\end{frame}
	
\subsection{Cosas a recordar}
	\begin{frame}
		\frametitle{Cosas a recordar}
		\begin{itemize}
			\item Los comandos empiezan por \
			\item Las marcas
			\begin{itemize}
				\item Puede ser begin ... end
				\item Puede ser etiqueta ... 
			\end{itemize}
			\item El verbatim siempre debe llevar [fragile] de argumento en el frame
			\item Los itemize, enumerate y type\_list son equivalentes
			\item Los elementos tipo itemize son anidables (como esta lista)
		\end{itemize}
	\end{frame}
	
\section{Anexo}
	\begin{frame}
		\frametitle{Links}
		\begin{block}{Fuente de este documento}
			http://blzarovich.miguelpuig.com/ejemplo-transpas-latex.tex
		\end{block}
		\begin{block}{Mi blog}
			Aquí podrás encontrar más detalles de Beamer en futuros posts
			http://blzarovich.miguelpuig.com/
		\end{block}
	\end{frame}
	
\end{document}
