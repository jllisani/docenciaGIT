\documentclass[11pt]{article}

\usepackage[active]{srcltx}
\usepackage[latin1]{inputenc}	%%para utilizar tildes en el texto
%%\usepackage[spanish]{babel}	%%corta las palabras segun el castellano
\usepackage{epsfig}
\usepackage{enumerate}


\setlength{\topmargin}{-2cm}	%%formato de pagina que ocupa todo
\setlength{\textwidth}{16cm}
\setlength{\textheight}{24cm}
\setlength{\oddsidemargin}{0cm}


\usepackage{amsmath}
\usepackage{amsfonts}
\usepackage{amssymb}

\newcounter{prbcont}
\stepcounter{prbcont}
\setcounter{prbcont}{0}
\newtheorem{problema}[prbcont]{Problema}

\begin{document}
\begin{center}
\textbf{{\large{Aplicaciones Estad\'{\i}stiques.  09/10}}}

\vspace{0.5cm}

\textbf{Pr\`atica Tema I: Distribuci\`o del consum de ciment}
\end{center} 

L'enunciat d'aquesta pr\`actica \'es variable i dep\`en dels valors que us han estat assignats personalment, en particular de l'\textbf{any-inici} i de l'\textbf{any-fi}. En primer lloc, dirigiu-vos a la plana web de l'Institut Balear de Estad\'{\i}stica 
\begin{center}
http://www.caib.es/ibae/ibae.htm
\end{center}
En Estad\'{\i}stiques Sectorials entreu en l'apartat d'Economia, i dintre d'aquest, en el de Conjuntura Econ\`omica. En l'apartat de Construcci\'o seleccioneu les dades sobre Consum de ciment. Les dades que es mostren s\'on els corresponents al consum de ciment en tones per als \'ultims mesos estudiats. Prement en la pestanya (baix esquerra) etiquetada com s\`erie hist\`orica, s'obt\'e el consum des de gener de 1987 fins a juliol del 2006.

\begin{enumerate}[a)]
\item  Seleccionau les dades corresponents al per\'{\i}ode, des de gener d'\textbf{any-inici} fins a desembre de l'\textbf{any-fi}, ambd\'os inclosos.
\begin{enumerate}[1-]
\item Calculau el rang i el rang interquart\'{\i}lic. Dibuixau el diagrama de capsa, marcant els valors at\'{\i}pics, si n'hi ha. 
\end{enumerate}
\item  Siguin $T_m$ i $T_M$ els valors menor i major de consum en el per\'{\i}ode seleccionat. Dividiu l'interval $[T_m,T_M]$ en 10 subintervals d'igual longitud i constru\"{\i}u a partir de les dades seleccionades a l'apartat (a) una taula de freq\"u\`encies per a aquests intervals.
\begin{enumerate}[1-]
\item Representau mitjan\c{c}ant un diagrama de barres la freq\"u\`encia absoluta i mitjan\c{c}ant un diagrama de 
tarta el percentatge acumulat.
\item  Calculau la moda, la mitjana, la mediana, el primer i tercer quartils i el percentil
90.
\item  Calculau el ratio de variaci\'o i el rang interquart\'{\i}lic. 
\item  Calculau la vari\`ancia i la desviaci\'o t\'{\i}pica.
\end{enumerate}
\end{enumerate}
\end{document} 
