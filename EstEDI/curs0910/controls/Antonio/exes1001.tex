%% This document created by Scientific Word (R) Version 3.0
%%
%% Examen extraordinario para un chico que no pudo asistir al oficial
%%
%%%%%
\documentclass[a4paper,10pt]{article}
%\usepackage[active]{srcltx}      	%% necesario para pasar del dvi al tex%
\usepackage[spanish]{babel}
\usepackage[latin1]{inputenc}
%\usepackage{sw20res1}
\usepackage{amsmath}
\usepackage{amsfonts}
\usepackage{amssymb}
\usepackage{graphicx}
\setlength{\topmargin}{-3cm}	%%formato de pagina que ocupa todo
\setlength{\textwidth}{18cm}
\setlength{\textheight}{27cm}
\setlength{\oddsidemargin}{-1cm}

\pagestyle{empty}
\newcounter{prbcont}
\stepcounter{prbcont}
\setcounter{prbcont}{0}
\newtheorem{problema}[prbcont]{Problema}

\begin{document}

\begin{center}
\textbf{Enginyeria d'Edificaci\'o}
\textbf{Aplicacions Estad\'{\i}stiques: 05/03/10}
\vspace{0.3cm}
\end{center}

\begin{problema}
Una empresa constructora internacional cuenta con 120 empleados en Baleares. Se realiza un estudio sobre los salarios mensuales (en cientos de euro) de estos empleados, ofreciendo el siguiente resultado
\begin{center}
\begin{tabular}{|c|c|}\hline
Salario & $n^\circ$ empleados \\ \hline
$[3,6)$ & 22 \\ \hline
$[6,9)$ & 54 \\ \hline
$[9,12)$ & 20 \\ \hline
$[12,18)$ & 15 \\ \hline
$[18,20)$ & 9 \\ \hline
\end{tabular}
\end{center}
\begin{itemize}
\item [(a)] Identifica la poblaci\'on y la muestra sobre la que se realiza el estudio.
\item [(b)] Identifica la variable del estudio. Clasificala.
\item [(c)] Calcula la tabla de frecu\'encias a partir de los datos anteriores
\item [(d)] Dibuja el histograma de la frecuencia absoluta
\item [(e)] Calcular la m\'edia aritm\'etica de los salarios de la empresa.
\item [(f)] Qu\'e porcentage de trabajadores tiene un sueldo superior a 1200 euros. 
\item [(g)] Calcular el rango intercualt\'{\i}lico.
\item [(h)] Se conocen los siguientes datos correspondientes a los salarios de los empleados en la comunidad de Madrid $\bar{y}=861$ euros i $Var_{Y}=4.5133$ euros. Si dos trabajadores uno balear y otro de madrid cobran lo mismo $1000$ euros, en t\'erminos relativos cu\'al de los dos est\'a peor pagado?
\end{itemize}
\end{problema}

\begin{problema}
Un curso de formaci\'on on-line cuenta con diez alumnos matriculados. El responsable del curso controla diariamente el número de veces que cada alumno se conecta al campues virtual. Un día cualquiera obtuvo los siguientes registros:
\[3,4,1,2,0,4,3,2,2,0\]
\begin{itemize}
\item [(a)] Identifica la variable y la poblaci\'on de estudio.
\item [(b)] Calcula la tabla de fecuencias.
\item [(c)] Dibuja el diagrama de barras de la frecuencia absoluta.
\item [(d)] Calcula el coeficiente de simetr\'ia y de curtosis (apuntamiento)
\end{itemize}
\end{problema}

\begin{problema}
Una empresa de fabricaci\'on de productos met\'alicos (pernos, tornillos, etc.) dispone de datos hist\'oricos correspondientes a beneficios (millones de euros) de explotaci\'on (X) e importe neto (millones de euros) por ventas (Y)
\begin{center}
\begin{tabular}{|c|c|c|}\hline
A\~no   & beneficio & Ventas \\ \hline
1997	&	3   & 7	\\ \hline
1998	&	5   & 7 \\ \hline
1999	&	3   & 5 \\ \hline
2000	&	2   & 4 \\ \hline
2001	&	2   & 5 \\ \hline
\end{tabular}
\end{center}
\begin{itemize}
\item [(a)] Calcular las medias aritm\'eticas de ambas variables y sus varianzas.
\item [(b)] Calcular la covarianza y el coeficiente de correlaci\'on. Interpretar los resultados obtenidos.
\end{itemize}
\end{problema}

\end{document}


