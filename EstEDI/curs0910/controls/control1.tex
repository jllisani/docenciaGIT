\documentclass[12pt]{article}
\usepackage{enumerate}
\usepackage[active]{srcltx}      	%% necesario para pasar del dvi al tex

\usepackage[latin1]{inputenc}		%%para utilizar tildes en el texto
%%\usepackage[spanish]{babel}		%%corta las palabras segun el castellano, pero pone comas en los puntos
							%% decimales
\usepackage{amsmath}			%% AMS-LaTeX	
\usepackage{amsfonts}
\usepackage{amssymb}			%%simbolos del AMS-LaTeX

\setlength{\topmargin}{-2cm}
\setlength{\textwidth}{16cm}
\setlength{\textheight}{24cm}
\setlength{\oddsidemargin}{0cm}


\newcounter{prbcont}
\stepcounter{prbcont}
\setcounter{prbcont}{0}
\newtheorem{problema}[prbcont]{Problema}
\newtheorem{ejemplo}[prbcont]{Ejemplo}


\pagestyle{empty}
\begin{document}

\begin{center}
\textbf{{\large {Aplicacions Estad\'{i}stiques \\
Enginyeria Edificaci\'o\\
Control 12/03/2010}\\}}
\vspace{1cm}
\end{center}


\noindent
Considerau les seg�ents dades corresponents a les temperatures m�ximes i m�nimes a 
Palma de Mallorca en els darrers 20 dies (font www.diariodemallorca.es):

\begin{center}
\begin{tabular}{c|c|c}
Dia & Max. (${ }^o C$) & Min. (${ }^o C$) \\ \hline
10/03 & 7 & 3 \\
09/03 & 9 & 1 \\
08/03 & 10 & 6 \\
07/03 & 13 & 4 \\
06/03 & 12 & 8 \\
05/03 & 13 & 11 \\
04/03 & 14 & 11 \\
03/03 & 14 & 7 \\
02/03 & 16 & 11 \\
01/03 & 18 & 11 \\
28/02 & 18 & 11 \\
27/02 & 18 & 12 \\
26/02 & 10 & 6 \\
25/02 & 18 & 11 \\
24/02 & 18 & 11 \\
23/02 & 17 & 11 \\
22/02 & 17 & 11 \\
21/02 & 17 & 7 \\
20/02 & 17 & 7 \\
19/02 & 13 & 11
\end{tabular}
\end{center}

\begin{enumerate}[a)]
\item Representau les temperatures m�ximes en un diagrama de capsa, indicant 
tots els valors num�rics rellevants i quins s�n, si n'hi ha, els valors at�pics i extrems.
\item Agrupau les temperatures m�ximes en intervals d'amplada 2, comen�ant per l'interval $[7, 9)$,
i calculau:
\begin{enumerate}[1)]
\item Taula de freq��ncies (absolutes, relatives, acumulades i percentatges).
\item Moda, mitjana, mediana i percentil $60\%$. 
\item Dibuixau l'histograma de freq��ncies absolutes.
\end{enumerate}
\item Calculau la covari�ncia i el coeficient de correlaci� entre les temperatures m�ximes i m�nimes,
a partir de les dades brutes (sense agrupar en intervals). Dibuixau el diagrama de dispersi� i
interpretau els resultats.
\end{enumerate}



\vskip 1cm
\noindent
\textbf{Nota:} feu els c�lculs amb una precisi� de 2 decimals.


\end{document}

