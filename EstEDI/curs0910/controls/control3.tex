\documentclass[a4paper,10pt]{article}
%\usepackage[active]{srcltx}      	%% necesario para pasar del dvi al tex%
\usepackage[spanish]{babel}
\usepackage[latin1]{inputenc}
%\usepackage{sw20res1}
\usepackage{amsmath}
\usepackage{amsfonts}
\usepackage{amssymb}
\usepackage{graphicx}
\usepackage{enumerate}
\setlength{\topmargin}{-2.5cm}	%%formato de pagina que ocupa todo
\setlength{\textwidth}{18.5cm}
\setlength{\textheight}{28cm}
\setlength{\oddsidemargin}{-1.5cm}

\pagestyle{empty}
\newcounter{prbcont}
\stepcounter{prbcont}
\setcounter{prbcont}{0}
\newtheorem{problema}[prbcont]{Problema}

\begin{document}

\begin{center}
\textbf{{\large {Control d'Estad\'istica Enginyeria Edificaci�. 
\\
Abril 2010}}}
\end{center}

\vspace{0.3cm}


\begin{problema}
Un promotor immobiliari sap que nom�s un $5\%$ de la gent que crida per interesar-se
per un pis l'acaba comprant. Durant la darrera setmana ha rebut $20$ cridades de persones 
interessades en una promoci� de pisos.
\begin{enumerate}[a)]
\item Quina �s la probabilitat que m�s de 2 d'aquestes persones acabin comprant un pis?
\item Quina �s la probabilitat que cap d'aquestes persones acabi comprant un pis?
\item Quin �s el nombre esperat de persones que comprar� un pis?
\item Repetiu els c�lculs anteriors per al cas que s'hagin rebut $100$ cridades.
\end{enumerate}
\end{problema}


\vskip 1cm

\begin{problema}
Un professor d'Estad�stica analitza les notes dels seus alumnes 
i observa que un $10\%$ t� nota superior a $9$ i un $40\%$ nota inferior a $4$.
Si suposam que la distribuci� de les notes �s Gaussiana, responeu a les
seg�ents q�estions:
\begin{enumerate}[a)]
\item Calculau la mitjana i la desviaci� t�pica de la distribuci�.
\item A partir de quina nota hauria d'aprovar els seus alumnes per tenir un $70\%$ d'aprovats?
\end{enumerate}
\end{problema}


\vskip 3cm
\textbf{Nota: Utilitzau una notaci� clara i precisa i definiu els successos i/o variables aleat�ries rellevants dels problemes} 

\vskip 3cm
\textbf{Variables aleat�ries usuals}
\vskip 0.2 cm

\begin{tabular}{|c|cl|c|c|l|}
V.A. (X) & $f_X(x)$ & & $E(X)$ & $Var(X)$ & Altres propietats \\
\hline
Binomial $B(n, p)$ & $\binom{n}{x} p^x (1-p)^{n-x}$ & si $x\in \Omega_X$ &
 $np$ & $np(1-p)$ & \\
$\Omega_X=\{ 0, 1, \cdots, n \}$ & $0$ & si $x \notin \Omega_X$ & & & \\ \hline
Poisson $Po(\lambda)$ & $\frac{\lambda^x}{x!} e^{-\lambda}$ & si $x\in \Omega_X$ & 
 $\lambda$ & $\lambda$ & \\
$\Omega_X=\{ 0, 1, \cdots \}$ & $0$ & si $x \notin \Omega_X$ & & & \\ \hline
Uniforme ${\cal U}(a, b)$ & $\frac{1}{b-a}$ & si $x \in [a, b]$ & 
$\frac{b+a}{2}$ & $\frac{(b-a)^2}{12}$ & 
$F_X(x)=\begin{cases} 
\frac{x-a}{b-a} & x \in [a, b] \\
0 & x < a \\
1 & x > b
\end{cases}$ \\
$\Omega_X=[a, b]$ & 0 & si $x \notin [a, b]$ &  & & \\ \hline
Gaussiana $X(\mu, \sigma^2)$ & & & $\mu$ & $\sigma^2$ & $Z\sim N(0, 1)$ normal est\'andar \\
$\Omega_X=\mathbb{R}$ & & & &  & $F_Z(-z)=1-F_Z(z)$ \\
 & & & &  & $F_X(x)=F_Z(\frac{x-\mu}{\sigma})$ \\ \hline
\end{tabular}

\end{document}

