\documentclass[12pt]{article}
\usepackage{enumerate}
\usepackage[active]{srcltx}      	%% necesario para pasar del dvi al tex

\usepackage[latin1]{inputenc}		%%para utilizar tildes en el texto
%%\usepackage[spanish]{babel}		%%corta las palabras segun el castellano, pero pone comas en los puntos
							%% decimales
\usepackage{amsmath}			%% AMS-LaTeX	
\usepackage{amsfonts}
\usepackage{amssymb}			%%simbolos del AMS-LaTeX

\setlength{\topmargin}{-2cm}
\setlength{\textwidth}{16cm}
\setlength{\textheight}{24cm}
\setlength{\oddsidemargin}{0cm}


\newcounter{prbcont}
\stepcounter{prbcont}
\setcounter{prbcont}{0}
\newtheorem{problema}[prbcont]{Problema}
\newtheorem{ejemplo}[prbcont]{Ejemplo}


\pagestyle{empty}
\begin{document}

\begin{center}
\textbf{{\large {Aplicacions Estad\'{i}stiques \\
Enginyeria Edificaci\'o\\
Control 26/03/2010}\\}}
\vspace{1cm}
\end{center}

\noindent
\textbf{Problema 1}. Una persona treu 5 bolles, amb reposici�, d'una urna que cont�
7 bolles blanques i 3 negres.
\begin{enumerate}[a)]
\item Quina �s la probabilitat de treure 3 blanques?
\item Quina �s la probabilitat de treure 3 blanques i que alguna d'elles surti en la primera o la segona extracci�?
%\item Si ha tret 3 blanques, quina �s la probabilidad que alguna d'elles surti en la primera o la segona extracci�?
\end{enumerate}

\vskip 1.5 cm
\noindent
\textbf{Problema 2}. En Toni, en Pep i na Maria es reuneixen per resoldre problemes d'estad�stica.
Toni resol el $40\%$ del total dels problemes, Pep el $30\%$ i Maria el $30\%$ restant.
Toni s'equivoca en un $2\%$ dels problemes que resol, Pep en el $6\%$ y Maria en l'$1\%$. 
Agafam a l'atzar un dels problemes resolts.
\begin{enumerate}[a)]
\item Quina �s la probabilitat que estigui ben resolt?
\item Si el problema est� mal resolt, quina �s la probabilitat
que l'hagi resolt en Pep?
\item Quina �s la probabilitat que estigui ben resolt i que l'hagi resolt en Toni?
\end{enumerate}


\end{document}

