\documentclass[11pt]{article}
\usepackage[active]{srcltx}      	%% necesario para pasar del dvi al tex

\usepackage[latin1]{inputenc}		%%para utilizar tildes en el texto
%%\usepackage[spanish]{babel}		%%corta las palabras segun el castellano, pero pone comas en los puntos
							%% decimales
\usepackage{amsmath}			%% AMS-LaTeX	
\usepackage{amsfonts}
\usepackage{amssymb}			%%simbolos del AMS-LaTeX

\setlength{\topmargin}{-2cm}
\setlength{\textwidth}{16cm}
\setlength{\textheight}{24cm}
\setlength{\oddsidemargin}{0cm}


\newcounter{prbcont}
\stepcounter{prbcont}
\setcounter{prbcont}{0}
\newtheorem{problema}[prbcont]{Problema}
\newtheorem{ejemplo}[prbcont]{Ejemplo}
\begin{document}

\begin{center}
\textbf{{\large {Aplicacions Estad\'{i}stiques. 2009/10}\\}}
\vspace{0.5cm}
\textbf{Enginyeria Edificaci\'o.}
\end{center}



\begin{problema}
Una cadena de franqu\'{\i}cies de restaurants de 1991 a 1995 va tenir 4, 8, 16, 26 i 82 restaurants respectivament. Un model estad\'{\i}stic estima que el nombre de restaurants entre 1991 i 1997 \'es 4, 8, 16, 33, 140, 280, i 586. Representau les dades en un diagrama de barres combinat.
\end{problema}


\begin{problema}
Determinau la mitjana (aritm\`etica), la mediana i la moda del seg\"uent conjunt de dades: 
\begin{center}
\begin{tabular}{rrrrrrrrrr}
3,&5,&7,&8,&8,&8,&10,&11,&12,&12\\ 13,&14,&14,&15,&16,&18,&19,&21,&23,&25
\end{tabular}
\end{center}
\end{problema}

\begin{problema}
Estimar la mitjana, la mediana i la moda de les seg\"uents dades cont�nues agrupades en intervals: 

\begin{center}
\begin{tabular}{l|c}
Interval de classe & Freq\"u\`encia \\ \hline de 0 a 9 & 50 \\ de 10 a 19 & 150\\ de 20 a 29 &
100\\ de 30 a 40 & 50\\ \hline
\end{tabular}
\end{center}
\end{problema}

\begin{problema}
\'Idem que en l'anterior per a la taula: 


\begin{center}
\begin{tabular}{l|c}
Interval de classe & Freq\"u\`encia\\ \hline $\left[ -0.5,9.5 \right)$ & 50 \\ $\left[
9.5,19.5 \right)$ & 150\\ $\left[ 19.5,29.5 \right)$ & 100\\ $\left[ 29.5,40.5 \right)$ &
50\\ \hline
\end{tabular}
\end{center}
\end{problema}


\begin{problema}
S\'on diferents els resultats dels dos exercicis anteriors, per qu\`e?
\end{problema}


\begin{problema}
Una editorial t\'e $4000$ t\'{\i}tols en cat\`aleg. Podem classificar els diferents llibres en novel�les, biografies i altres tipus de llibres de venda m\'es preferent en llibreries. De les seg\"uents dades de nombre de c\`opies venudes, estimau les vendes mitjanes per t\'{\i}tol. 
\begin{center} 
\begin{tabular}{l|c} 
{ {Interval d'unitats venudes}} & Freq\"u\`encia \\ \hline 
0-999 & 500 \\ 1000-4999 & 800 \\ 
5000-24999 & 700 
\\ 25000- 49999 & 1500 
\\ 50000 o m\'es & 500\\ \hline 
\end{tabular} 
\end{center} 
\end{problema}

\begin{problema}
Amb les dades de l'exercici anterior. Si el t\'{\i}tol de majors vendes va vendre 1000000 c\`opies l'any en qu� es van recollir les dades quina \'es la desviaci\'o est\`andard estimada per a les vendes per t\'{\i}tol? 
\end{problema}


\begin{problema}A 50 aspirants a un determinat lloc de treball se'ls va sotmetre a una prova. Les puntuacions obtingudes van �sser: 
\begin{center} 
\begin{tabular}{cccccccccc}
4 & 4 & 2 & 10 & 1 & 9 & 5 & 3 & 4 & 5 \\ 
6 & 6 & 7 & 6 & 8 & 7 & 6 & 8 & 7 & 6 \\ 
5 & 4 & 4 & 4 & 5 & 6 & 6 & 7 & 5 & 6 \\ 
6 & 7 & 5 & 6 & 6 & 7 & 5 & 6 & 4 & 3 \\ 
2 & 6 & 6 & 7 & 7 & 8 & 8 & 9 & 8 & 7 
\end{tabular} 
\end{center} 
\begin{itemize} 
\item [(a)] {Construiu la taula de freq\"u\`encies i la representaci\'o gr\`afica corresponent.} 
\item [(b)] {Trobau la puntuaci\'o que seleccioni al 20\% dels millors candidats.} 
\end{itemize} 
\end{problema}

\begin{problema}
En la poblaci\'o d'estudiants de la facultat, es van seleccionar una mostra de 20 alumnes i es van obtenir les seg\"uents talles en cm.: 
\[ 
\begin{tabular}{llllllllll} 
162 & 168 & 174 & 168 & 166 & 170 & 168 & 166 & 170 & 172 \\ 
188 & 182 & 178 & 180 & 176 & 168 & 164 & 166 & 164 & 172 
\end{tabular} 
\]
Es demana: 
\begin{itemize} 
\item [(a)] {Descripci\'o num\`erica i representaci\'o gr\`afica.} 
\item [(b)] {Mitjana, mediana i moda.} 
\end{itemize} 
\end{problema}

\begin{problema}
Agrupant les dades de l'exercici anterior en intervals d'amplitud 10 cm., es demana: 
\begin{itemize} 
\item [(a)] {Descripci\'o num\`erica i representaci\'o gr\`afica.} 
\item [(b)] {Mitjana, mediana i moda.} 
\item [(c)] {Analitzau els c\`alculs fets comparant-los amb l'exercici anterior.} \end{itemize} 
\end{problema}

\begin{problema}
Les tres factories d'una ind\'ustria han produ\"{\i}t en l'\'ultim any el seg\"uent nombre de motocicletes per trimestre: 
$$ \vbox{\halign{\offinterlineskip\strut\hskip0.25cm # \hskip0.25cm& \hskip0.25cm\hfill # \hskip0.25cm&\hskip0.25cm\hfill # \hskip0.25cm& \hskip0.25cm\hfill # \hskip0.25cm\cr & factoria 1 & factoria 2 & factoria 3 \cr \hline 
$1^{\underline{o}}$. trimestre & 600 & 650 & 550 \cr $2^{\underline{o}}$. trimestre & 750 & 1200 & 900 \cr  $3^{\underline{o}}$. trimestre & 850 &1250 & 1050 \cr $4^{\underline{o}}$. trimestre & 400 & 800 & 650 \cr }} $$ Obteniu: 
\begin{itemize} 
\item [(a)] {Producci\'o mitjana trimestral de cada factoria i de tota la ind\' ustria.} 
\item [(b)] {Producci\'o mitjana di\`aria de cada factoria i de tota la ind\'ustria tenint en compte que durant el primer trimestre van haver 68 dies laborables, el segon, 78, el tercer, 54 i el quart, 74.} 
\end{itemize} 
\end{problema}

\begin{problema}
Una empresa ha pagat per un cert article: 225, 250 , 300 i 200 euros respectivament. Determinau el preu mitj� en les seg\"uents hip\`otesis: 
\begin{itemize} 
\item [(a)] {Compra per valor de 38250, 47500, 49500 i 42000 euros respectivament.} 
\item [(b)] {Compra cada vegada un mateix import global.} 
\item [(c)] {Compra 174, 186, 192 i 214 unitats respectivament.} 
\end{itemize} 
\end{problema}

\begin{problema}
Sobre una mostra de 56 botigues distintes, es van obtenir els seg\"uents preus de venda d'un determinat article: 
$$ \vbox{\halign{\offinterlineskip\strut\hskip0.25cm # \hskip0.25cm& \hskip0.25cm # \hskip0.25cm&\hskip0.25cm # \hskip0.25cm& \hskip0.25cm # \hskip0.25cm&\hskip0.25cm # \hskip0.25cm& \hskip0.25cm # \hskip0.25cm&\hskip0.25cm # \hskip0.25cm\cr 3260 & 3510 & 3410 & 3180 & 3300 & 3540 & 3320 \cr 3450 & 3840 & 3760 & 3340 & 3260 & 3720 & 3430 \cr 3320 & 3460 & 3600 & 3700 & 3670 & 3610 & 3910 \cr 3610 & 3610 & 3620 & 3150 & 3520 & 3430 & 3330 \cr 3370 & 3620 & 3750 & 3220 & 3400 & 3520 & 3360 \cr 3300 & 3340 & 3410 & 3600 & 3320 & 3670 & 3420 \cr 3320 & 3290 & 3550 & 3750 & 3710 & 3530 & 3500 \cr 3290 & 3410 & 3100 & 3860 & 3560 & 3440 & 3620 \cr }} $$ 
Es demana: 
\begin{itemize} 
\item [(a)] {Agrupau la informaci\'o en sis intervals d'igual amplitud i fer la representaci\'o gr\`afica corresponent} \item [(b)] {Mitjana i desviaci\'o t\'{\i}pica} 
\item [(c)] {Desviaci\'o mitjana respecte de la mitjana i la mediana } 
\end{itemize} 
\end{problema}

\begin{problema} 
La seg\"uent distribuci\'o correspon al capital pagat per les 420 empreses de la cons\-truc\-ci\'o amb domicili fiscal en una regi\'o determinada: 
$$
\vbox{\halign{\offinterlineskip\strut\hskip0.25cm \hfill # \hskip0.25cm& \hskip0.25cm \hfill # \hskip0.25cm\cr Capital (milions d'euros.) & Nombre d'empreses\cr \hline
 menys de 5 &12\cr de 5 a 13 &66\cr de 13 a 20 &212\cr de 20 a 30 &84\cr de 30 a 50 &30\cr de 50 a 100 &14\cr m\'es de 100 &2\cr}} $$ 
\begin{itemize} 
\item [(a)] {Fent servir com a marques de classe del primer i \'ultim interval 4 i 165 respectivament, trobau la mitjana aritm\`etica i la desviaci\'o t\'{\i}pica} 
\item [(b)] {Calculau la moda i la mediana} 
\item [(c)] {Estudiau gr\`aficament la seva simetria} 
\end{itemize} 
\end{problema}

\begin{problema}
La distribuci\'o dels ingressos a Espanya en l'inici i final del segon pla de desenvolupament (1967-70) era: $$\vbox{\halign{\offinterlineskip\strut\hskip0.25cm \hfill # \hskip0.25cm& \hskip0.25cm \hfill # \hskip0.25cm&\hskip0.25cm \hfill # \hskip0.25cm\cr Ingressos mitjans(en euros)& \% Llars 1967& \% Llars 1970\cr \hline
fins a a 60 & 20.02 & 13.87\cr de 60 a 120 & 48.46 & 39.20\cr de 120 a 180 & 17.27 & 24.31\cr de 180 a 240 & 6.48 & 11.44\cr de 240 a 500 & 5.14 & 8.54\cr de 500 a 1000 & 1.46 & 1.42\cr de 1000 a 2000 & 0.88 & 0.80\cr de 2000 a 5000 & 0.21 & 0.30\cr m\'es de 5000 & 0.08 & 0.12\cr }}$$ 
Utilitzant el $Q_{1}$ (percentil 25) i $Q_{3}$ (percentil 75) com a llindars de pobresa i riquesa entre els quals es troba la classe mitjana de la poblaci\'o i usau-lo en la classificaci\'o seg\"uent: 
$$ \vbox{\halign{\strut\offinterlineskip\strut #&\hskip1cm#&#&#\cr \vbox{\halign{\strut #\cr Interval al que  pertanyen\cr}}& classe&&\cr \noalign{\hrule} fins a $Q_{1}$\hfill & baixa&&\cr \vbox{\halign{\strut \offinterlineskip#\cr de $Q_{1}$ a $M_{i}$ (Mediana)\hfill \cr de $M_{i}$ a $Q_{3}$\hfill\cr}}& $ \left. \vbox{\halign{\strut \offinterlineskip#\cr mitjana baixa\hfill\cr mitjana alta\hfill\cr}} \right\} $&&\cr m\'es de $Q_{3}$\hfill & alta\hfill&&\cr}} $$ 
Discutiu la veracitat de les seg\"uents conclusions relatives al segon pla de desenvolupament 
\begin{itemize} 
\item [(a)] {La difer\`encia entre la classe baixa i alta va augmentar.} 
\item [(b)] {El recorregut (rang) entre les classes mitja baixa i mitjana alta tamb\'e va augmentar, �ssent menor l'increment en el primer cas que en el segon.} 
\end{itemize} 
\end{problema}


\begin{problema}
La segu\"ent taula mostra la distribuci\'o de les c�rregues m\`aximes que suporten els fils produ\"{\i}ts en una certa f\`abrica: 
$$\vbox{\halign{\offinterlineskip\strut \vrule \hfill#\hfill\vrule&\hfill#\hfill\vrule \cr \noalign{\hrule} C\`arrega m\`axima(T)&Nombre de fils\cr \noalign{\hrule} 9.25-9.75&2\cr 9.75-10.25&5\cr 10.25-10.75&12\cr 10.75-11.25&17\cr 11.25-11.75&14\cr 11.75-12.25&6\cr 12.25-12.75&3\cr 12.75-13.25&1\cr \noalign{\hrule} }}$$ 
Trobau la mitjana i la vari\`ancia. Donau un interval on estan almenys el 90\% de les dades. 
\end{problema}

\begin{problema}
Les qualificacions finals de 20 estudiants d'estad\'{\i}stica s\'on: 
$$ 
\begin{array}{ll}
&59 , 60, 62, 68, 68, 71, 73, 73, 75, 75,\\ 
& 76, 79, 82, 84, 85, 88, 88, 90, 93, 93 . 
\end{array} 
$$ 
Feu la distribuci\'o de freq��ncies, i els histogrames de freq\"u\`encies relatives i relatives acumulades en tants per cent. 
\end{problema}


\begin{problema}
La seg\"uent taula mostra els preus per persona i nit en hotels i pensions de l'\`area metropolitana d'una ciutat espanyola en euros: 
\[
\begin{tabular}{ccccccccccccc} 
65 & 38 & 54 & 28 & 25 & 32 & 84 & 47 & 45 & 33 \\ 
70 & 37 & 64 & 26 & 40 & 45 & 34 & 47 & 61 & 66 \\ 
43 & 46 & 62 & 56 & 47 & 52 & 28 & 28 & 26 & 32 \\ 
94 & 40 & 57 & 36 & 30 & 54 & 60 & 24 & 24 & 24 \\ 
24 & 25 & 50 & 65 & 35 & 60 & 32 & 32 & 26 & 25 \\ 
33 & 100 
\end{tabular} 
\]
\begin{itemize} 
\item [(a)] Calculau la distribuci\'o de freq\"u\`encies (agrupant de forma oportuna) dels preus. 
\item [(b)] Dibuixau l'histograma de freq\"u\`encies absolutes i absolutes acumulades i els seus pol\'{\i]}gons associats. 
\item [(c)] Dibuixau l'histograma de freq\"u\`encies relatives i relatives acumulades i els seus pol\'{\i}gons associats. 
\item [(d)] Dibuixau el diagrama de capsa associat a les dades. 
\item [(e)] Dibuixau el diagrama de past�s dels preus. 
\item [(f)] Comentau totes les gr\`afiques. 
\end{itemize} 
\end{problema}


\begin{problema}
Suposem que sis venedors necessiten vendre un total de 50 aspiradores en un mes. El senyor A ven 7 durant els primers 4 dies; el senyor B ven 10 durant els seg\"uents 5 dies; el senyor C ven 12 durant els seg\"uents 5 dies; el senyor D ven 10 durant els seg\"uents 4 dies; el senyor I ven 6 durant els seg\"uents 3 dies i el senyor F ven 5 durant els seg\"uents 3 dies. Trobau el terme mitj\`a d'aspiradores venudes per dia. 
\end{problema} 


\begin{problema}
Considerem la seg\"uent variable discreta que ens d\'ona el nombre de vegades que la gent s'examina per a aprovar l'examen de conduir. Els resultats, donada una mostra de 30 aspirants, s\'on: 
$$ \vbox{\halign{\offinterlineskip\strut\hskip0.05cm #\hskip0.05cm& \hskip0.05cm #\hskip0.05cm&\hskip0.05cm #\hskip0.05cm& \hskip0.05cm #\hskip0.05cm&\hskip0.05cm #\hskip0.05cm& \hskip0.05cm #\hskip0.05cm&\hskip0.05cm #\hskip0.05cm& \hskip0.05cm #\hskip0.05cm&\hskip0.05cm #\hskip0.05cm& \hskip0.05cm #\hskip0.05cm\cr 4, & 3, & 1, & 1 , & 1, & 1, & 2, & 2, & 2, & 3, \cr 2, & 1, & 2, & 2, & 1, & 2, & 1, & 1, & 1, & 2, \cr 2 , & 3, & 1, & 3, & 3, & 3, & 2, & 1, & 4, & 1, \cr}} $$ 
Trobau $\overline{X}$ i $s^2_X$. 
\end{problema}

\begin{problema}
Els pesos en Kg. de 120 llagostes comprades en una peixateria van anar: 
\[
\begin{tabular}{ccccccccccc} 
0.66 & 0.47 & 0.58 & 0.68 & 0.56 & 0.52 & 0.54 & 0.59 & 0.56 &  0.72 & 0.63\\ 
0.59 & 0.56 & 0.56 & 0.49 & 0.63 & 0.53 & 0.56 & 0.55 & 0.50 &  0.75 & 0.56 \\ 
0.59 & 0.66 & 0.61 & 0.56 & 0.52 & 0.48 & 0.56 & 0.68 & 0.77 &  0.59 & 0.53 \\ 
0.56 & 0.65 & 0.51 & 0.59 & 0.49 & 0.62 & 0.54 & 0.56 & 0.56 &  0.61 & 0.50 \\ 
0.61 & 0.45 & 0.65 & 0.55 & 0.54 & 0.61 & 0.64 & 0.56 & 0.71 &  0.59 & 0.56 \\ 
0.59 & 0.64 & 0.49 & 0.56 & 0.48 & 0.64 & 0.56 & 0.62 & 0.54 &  0.53 & 0.55 \\ 
0.56 & 0.63 & 0.56 & 0.52 & 0.66 & 0.68 & 0.62 & 0.56 & 0.59 &  0.54 & 0.50 \\ 
0.56 & 0.62 & 0.49 & 0.56 & 0.64 & 0.60 & 0.53 & 0.55 & 0.64 &  0.59 & 0.60 \\ 
0.52 & 0.56 & 0.66 & 0.54 & 0.68 & 0.59 & 0.56 & 0.48 & 0.54 &  0.56 & 0.67 \\ 
0.63 & 0.46 & 0.48 & 0.68 & 0.61 & 0.56 & 0.54 & 0.49 & 0.65 &  0.56 & 0.61 \\ 
0.45 & 0.73 & 0.60 & 0.68 & 0.65 & 0.56 & 0.54 & 0.55 & 0.60 & 0.60 
\end{tabular} 
\]
\begin{itemize} 
\item [(b)] Calculau la distribuci\'o de freq\"u\`encies (agrupant de forma oportuna) dels pesos. 
\item [(c)] Dibuixau l'histograma de freq\"u\`encies absolutes i absolutes acumulades i els seus pol\'{\i}gons associats. 
\item [(d)] Dibuixau l'histograma de freq\"u\`encies relatives i relatives acumulades i els seus pol\'igons associats. \item [(i)] Dibuixau el diagrama de capsa associat a les dades. 
\item [(f)] Dibuixau el diagrama de past\'{\i}s dels pesos. 
\item [(g)] Comentau tots els gr\`afics. 
\end{itemize} 
\end{problema}


\begin{problema}
Els seg\"uents pesos en Kg. corresponen a llagostes comprades en la mateixa peixateria per\`o en un mes distint: 
\[
\begin{tabular}{cccccccccccc} 
& 0.76 & 0.81 & 0.72 & 0.80 & 0.57 & 0.52 & 0.67 &  0.59 & 0.67 & 0.85 & 1.10 \\ 
& 0.60 & 0.82 & 1.19 & 0.61 & 0.77 & 0.83 & 1.15 &  0.56 & 0.75 & 0.96 & 0.57 \\ 
& 0.95 & 0.81 & 0.97 & 0.64 & 0.62 & 0.86 & 0.70 &  0.79 & 1.00 & 0.70 & 1.06 \\ 
& 0.79 & 0.67 & 0.95 & 0.81 & 0.53 & 0.92 & 0.73 &  0.64 & 0.65 & 0.71 & 0.68 \\ 
& 0.92 & 0.56 & 0.76 & 1.04 & 0.61 & 0.62 & 0.93 &  0.81 & 0.87 & 0.76 & 0.77 \\ 
& 0.75 & 0.89 & 0.53 & 0.82 & 0.95 & 0.88 & 0.65 &  0.85 & 0.76 & 0.85 & 0.64 \\ 
& 0.84 & 0.74 & 0.76 & 0.90 & 0.96 & 0.94 & 1.10 &  0.69 & 0.62 & 0.58 & 0.52 \\ 
& 0.57 & 0.88 & 0.69 & 0.79 & 0.66 & 0.92 & 0.93 &  0.74 & 1.17 & 0.67 & 0.61 \\ 
& 0.81 & 0.87 & 1.15 & 0.66 & 0.87 & 0.87 & 0.68 &  0.49 & 0.89 & 1.21 & 0.92 \\ 
& 0.72 & 0.48 & 1.03 & 1.05 & 0.70 & 0.58 & 0.70 &  1.04 & 0.76 & 0.65 & 0.68 \\ 
& 0.52 & 0.79 & 1.03 & 0.77 & 0.99 & 1.24 & 0.59 &  0.91 & 0.66 & 0.71 
\end{tabular} 
\]
Realitzau un estudi comparatiu dels dos grups de llagostes mitjan\c{c}ant gr\`afiques. 
\end{problema}

\begin{centerline}
{{\bf ESTAD\'ISTICA DESCRIPTIVA BIVARIANT}}
\end{centerline}


\begin{problema}
{Les puntuacions obtingudes per 26 concursants a un lloc de treball en les proves de processador de textos (PT) i full de c\`alcul (C) han estat, en aquest ordre: 
$$ 
\begin{array}{l} 
(1,2)\ (1,3)\ (2,1)\ (2,3)\ (2,2)\ (2,1)\ (1,2)\ (2,1)\ (1,3)\ (3,2)\ (2,2)\ (2,3)\ (1,3)\ \\ (3,1)\ (3,2)\ (1,1)\ (3,2)\ (2,1) \ (3,3)\ (1,1)\ (2,1)\ (1,3)\ (1,2)\ (2,2)\ (2,1)\ (1,3) 
\end{array} 
$$ 
Calculau la taula de conting\`encia $n_{i,j}$.} 
\end{problema}

\begin{problema}
{Un servei regular de transports a llarga dist\`ancia disposa del seg\"uent model relatiu a les variables: 
$$ 
\begin{array}{l} 
X=\mbox{retard en hores sobre l'hora d'arribada prevista,}\\ 
I=\mbox{velocitat modal en el recorregut.} 
\end{array} 
$$ 
$$
\vbox{\halign{\offinterlineskip\strut\vrule\ $#$ &\vrule \ $#$ &\vrule\ $#$ &\vrule\ $#$ \vrule\cr\noalign{\hrule}X\backslash I & 40-50 & 50-60 & 60-80\cr\noalign{\hrule}0-1&0&0&0,32\cr\noalign{\hrule} 1-2&0&0,13&0,08\cr\noalign{\hrule}2-3&0,16&0,10&0\cr\noalign{\hrule} 3-4&0,15&0,06&0\cr\noalign{\hrule}}}
$$ 
Estudiau la independ\`encia entre $X$ i $I$ calculant el coeficient de conting\`encies de Pearson $CP$.} 
\end{problema}

\begin{problema}
En un proc\'es de manufacturaci\'on d'un article de vestir s'han controlat dues caracter\'{\i}stiques: temps emprat i perfeccionament en l'acabat, tenint la seg\"uent distribuci\'o de freq\"u\`encies conjunta sobre una mostra de 120 unitats: 
$$
\vbox{\halign{\offinterlineskip\strut\vrule\ $#$ &\vrule \ $#$ &\vrule\ $#$ &\vrule\ $#$ &\vrule\ $#$ \vrule \cr\noalign{\hrule}\hbox{Errors trobats}\backslash\hbox{minuts emprats}&3&4&5&6\cr\noalign{\hrule}0&2&5&10&12\cr\noalign{\hrule} 1&6&10&28&8\cr\noalign{\hrule}2&15&12&6&6\cr\noalign{\hrule}}}
$$ 
Es demana: 
\begin{itemize} 
\item [(a)] {Distribucions marginals.} 
\item [(b)] Mitjana aritm\`etica, moda i desviaci\'o t\'{\i}pica de les distribucions marginals.
\end{itemize}  
\end{problema}


\begin{problema}
{Les 130 ag\`encies d'una entitat banc\`aria presentaven, en l'exercici 1984, les observacions seg\"uents: 
$$ 
\begin{array}{l} 
X=\mbox{tipus de compte (corrent, a termini fix,...)/total de comptes,}\\ 
I=\mbox{saldo mitj� dels comptes a 31-XII (en centenars d'euros.).} 
\end{array} 
$$ 

$$\vbox{\halign{\offinterlineskip\strut\vrule\ $#$ &\vrule \ $#$ &\vrule\ $#$ &\vrule\ $#$ \vrule\cr\noalign{\hrule}I\backslash X & \hbox{menys de 0,1}&\hbox{de 0,1 a 0,3}& \hbox{m�s de 0,3}\cr\noalign{\hrule}\hbox{menys de 20}&48&0&0\cr\noalign{\hrule}\hbox{de 20 a 50}&21&11&0\cr\noalign{\hrule}\hbox{de 50 a 100}&14&8&2\cr\noalign{\hrule}\hbox{de 100 a 250}&7&5&1\cr\noalign{\hrule}\hbox{m\'es de 250}&6&6&1\cr\noalign{\hrule}}}
$$ 
Es demana: 
\begin{itemize} 
\item [(a)] {Distribucions marginals.} 
\item [(b)] {Mitjana de $I$ i tercer cuartil de $X$.} 
\item [(c)] {Distribuci\'o de les ag�ncies segons $I$,quan el r\`atio $X$ esta entre $0,1$ i $0,3$.} 
\end{itemize} } 
\end{problema}

\begin{problema}
{D'una mostra de 24 llocs de venda en un mercat de provisions es va recollir informaci\'o sobre $X:$ nombre de balances i $I:$ nombre de dependents. 
$$\vbox{\halign{\offinterlineskip\strut\vrule\ $#$ &\vrule \ $#$ &\vrule\ $#$ &\vrule\ $#$ &\vrule\ $#$ \vrule\cr\noalign{\hrule}X\backslash I & 1&2&3&4\cr\noalign{\hrule}1&1&2&0&0\cr\noalign{\hrule} 2&1&2&3&1\cr\noalign{\hrule}3&0&1&2&6\cr\noalign{\hrule} 4&0&0&2&3\cr\noalign{\hrule}}}
$$ 
Calculau la covari�ncia entre aquestes dues caracter\'{\i}stiques.} 
\end{problema}

\begin{problema}
{La seg\"uent distribuci\'o correspon als controls als que han estat sotmeses 42 peces per dues seccions de l'equip de control de qualitat: 
$$
\vbox{\halign{\offinterlineskip\strut\vrule\ $#$ &\vrule \ $#$ &\vrule\ $#$ &\vrule\ $#$ &\vrule\ $#$ \vrule\cr\noalign{\hrule} \hbox{Controls secci\'o 2}\backslash \hbox{Controls secci\'o 1} & 0&1&2&3\cr\noalign{\hrule}0&0&3&6&6\cr\noalign{\hrule} 1&2&4&3&4\cr\noalign{\hrule}2&6&2&0&0\cr\noalign{\hrule} 3&3&2&1&0\cr\noalign{\hrule}}}
$$ 
Es demana: 
\begin{itemize} 
\item [(a)] {Distribuci\'o dels controls efectuats per la secci\'o 2 , mitjana, moda i mediana.} 
\item [(b)] {Coeficient de correlaci\'o lineal entre aquestes variables.} 
\end{itemize} } 
\end{problema}

\begin{problema}
{Les sis cooperatives agr\`aries d'una comarca presentaven les seg\"uents xifres corresponents a les variables: 
$$ 
\begin{array}{l} 
X=\mbox{estoc mitj� diari en naus d'emmagatzematge (milers d'euros.)}\\ 
I= \mbox{xifra comercialitzada di\`ariament (en milers d'euros.)}\\ 
Z=\mbox{empleats fixos en plantilla.}\\ 
V= \mbox{empleats eventuals.} 
\end{array} 
$$ 
$$
\vbox{\halign{\offinterlineskip\strut\vrule\ $#$ &\vrule \ $#$ &\vrule\ $#$ &\vrule\ $#$ &\vrule\ $#$ \vrule\cr\noalign{\hrule} \hbox{Cooperativa}&X&I&Z&V\cr\noalign{\hrule}A&26&146&6&8 \cr\noalign{\hrule} B&33&167&8&6\cr\noalign{\hrule}C&12&92&6&8\cr\noalign{\hrule} D&18&125&8&6\cr\noalign{\hrule}I&18&118&10&4 \cr\noalign{\hrule}F&25&132&10&4\cr\noalign{\hrule}}}
$$ 
Calculau el coeficient de correlaci\'o lineal entre les variables $(X,I)$, $(I,Z)$ i $(Z,V)$.} 
\end{problema}

\begin{problema}
{Es va demanar a dos usuaris de detergents que classifiquessin 6 detergents d'acord amb les seves prefer\`encies. Els resultats van �sser: 
$$
\vbox{\halign{\offinterlineskip\strut\vrule\ $#$ &\vrule \ $#$ &\vrule\ $#$ \vrule\cr\noalign{\hrule} \hbox{Detergent}&\hbox{Usu. A}&\hbox{Usu. B}\cr\noalign{\hrule} A&2&3\cr\noalign{\hrule}B&4&2\cr\noalign{\hrule} C&5&4\cr\noalign{\hrule}D&1&1\cr\noalign{\hrule} I&6&6\cr\noalign{\hrule}F&3&5\cr\noalign{\hrule}}}
$$ 
Calculau el coeficient de correlaci\'o lineal, i interpretau el resultat } 
\end{problema}

\begin{problema}
{Les seg\"uents dades corresponen a les qualificacions atorgades a 18 empleats des\-pr\'es d'uns cursets d'especialitzaci\'o realitzats per una ag\`encia de vendes: 
$$
\vbox{\halign{\offinterlineskip\strut\vrule\ # &\vrule\ $#$ &\vrule\ $#$ &\vrule\ $#$ &\vrule\ $#$ &\vrule\ $#$ &\vrule\ $#$ &\vrule\ $#$ &\vrule\ $#$ &\vrule\ $#$ &\vrule\ $#$ &\vrule\ $#$ &\vrule\ $#$ &\vrule\ $#$ &\vrule\ $#$ &\vrule\ $#$ &\vrule\ $#$ &\vrule\ $#$ &\vrule\ $#$ \vrule\cr\noalign{\hrule} empleat&1&2&3&4&5&6&7&8&9&10&11&12&13&14&15&16&17&18 \cr\noalign{\hrule}\noalign{\hrule}persuasi\'o&0&0&1&2&1&2&0&1&2&1&1&1&2&1&0&1&1&0 \cr\noalign{\hrule}retentiva&1&0&1&0&1&0&2&1&0&1&2&1&1&2&0&1&1&2 \cr\noalign{\hrule}prud\`encia&1&1&1&2&2&1&0&1&2&0&1&1&0&0&0&1&1&2 \cr\noalign{\hrule}}}
$$ 
Es demana: 
\begin{itemize} 
\item [(a)] {Distribuci\'o de les puntuacions de retentiva i prud\`encia.} 
\item [(b)] {Distribuci\'o de les puntuacions en persuasi\'o i prud\`encia per a aquells que no han obtingut un zero en retentiva.} 
\item [(c)] {Distribuci\'o de les puntuacions en persuasi\'o d'aquells que han tret un 1 en les proves de retentiva i prud\`encia.} 
\end{itemize} } 
\end{problema}

\begin{problema}
{Una cartera de valors pot estar composta per dos tipus d'accions amb una ren\-di\-bi\-li\-tat donada per la seg\"uent distribuci\'o conjunta, segons una escala subjectiva que els concedim ``a priori'': 
$$
\vbox{\halign{\offinterlineskip\strut\vrule\ $#$ &\vrule \ $#$ &\vrule\ $#$ \vrule\cr\noalign{\hrule} \hbox{Rendibilitat $X_2$}\backslash\hbox{Rendibilitat $X_1$}&\hbox{De 5 a 10}&\hbox{De 10 a 15}\cr\noalign{\hrule} \hbox{De 0 a 5}&0,06&0,02\cr\noalign{\hrule}\hbox{De 5 a 10}&0,14&0,38\cr\noalign{\hrule} \hbox{De 10 a 15}&0,30&0,10\cr\noalign{\hrule}}}
$$ 
Es demana: 
\begin{itemize} 
\item [(a)] {Quina de les dues accions presenta major rendibilitat mitjana esperada?} 
\item [(b)] {Quina de les dues accions presenta menys vari\`ancia en la seva rendibilitat?} 
\end{itemize}}
\end{problema}
\end{document}



