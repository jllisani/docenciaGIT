\documentclass{article}
\usepackage[catalan]{babel}
\usepackage[latin1]{inputenc}   % Permet usar tots els accents i car�ters llatins de forma directa.
\usepackage{enumerate}
\usepackage{amsfonts, amscd, amsmath, amssymb}
\usepackage{eepic}
\usepackage{graphicx}

%NOTA: si usam eepic hem de compilar a .dvi o .ps (NO PDF)

\setlength{\textwidth}{16cm}
\setlength{\textheight}{24cm}
\setlength{\oddsidemargin}{-0.3cm}
\setlength{\evensidemargin}{0.25cm} \addtolength{\headheight}{\baselineskip}
\addtolength{\topmargin}{-3cm}

\newcommand\Z{\mathbb{Z}}
\newcommand\R{\mathbb{R}}
\newcommand\N{\mathbb{N}}
\newcommand\Q{\mathbb{Q}}
\newcommand\K{\Bbbk}
\newcommand\C{\mathbb{C}}

%\pagestyle{empty}
\begin{document}


\begin{center}
\textbf{\Large Tema 1. An�lisi explorat�ria de dades
\newline
Exercicis proposats}
\end{center}

\vskip 0.3 cm





\section*{Sessi� 1: dades i variables}


\noindent
\textbf{Exercici 1} 

Identificau la poblaci\'o i la mostra estudiats 
en els seg�ents casos:
\begin{enumerate}[a)]
\item En un estudi sobre el consum de drogues en un institut es fa una enquesta  
al $30\%$ per cent dels alumnes de $2^\text{on}$ de 
batxillerat.
\item En un estudi sobre el consum de drogues entre els joves de
Ciutadella (menors de 35 anys) s'entrevista al $10\%$ dels clients 
dels principals locals de copes.
\item En un estudi a nivell nacional sobre la influ�ncia de l'alcohol 
en els accidents de tr�fic es fan 10.000 controls d'alcohol�mia
en diferents carreteres del pais.
\end{enumerate}

%\vskip 0.2 cm
%\noindent
%\textbf{Ejercicio 2} 
%
%Decidir si para estudiar los siguientes casos se utilizan
%herramientas de estad\'istica descriptiva o inferencial:
%\begin{enumerate}[a)]
%\item Un profesor de universidad debe proporcionar a su jefe de Departamento
%un informe sobre el n\'umero de alumnos matriculados y sus calificaciones 
%en el periodo 2005-2007. Para ello utilizar\'a estad\'istica ............... .
%\item Una empresa desea conocer los h\'abitos de trabajo de sus trabajadores.
%Para ello les hace rellenar una encuesta sobre sus horas de llegada y salida,
%tiempo dedicado a responder el tel\'efono o el correo electr\'onico, 
%tiempo dedicado a reuniones de trabajo con los jefes u otros compa\~neros, etc.
%Los datos obtenidos se organizar\'an y analizar\'an usando estad\'istica ............. .
%\item La Direcci\'on General de Tr\'afico desea evaluar la eficiencia a nivel
%nacional de
%la \'ultima campa\~na de prevenci\'on de accidentes a partir de los 
%datos en una serie de municipios. Para ello utilizar\'a
%herramientas de la estad\'istica ................ . 
%\end{enumerate}

\vskip 0.2 cm
\noindent
\textbf{Exercici 2} 

Identificau al menys tres variables que puguin apar�ixer
en els seg�ents estudis estad\'istics:
\begin{enumerate}[a)]
\item Consum de drogues en una ciutat.
\item Satisfacci\'o laboral dels empleats d'una empresa.
\item Notes obtengudes pels alumnes d'una assignatura. 
\item Seguretat dels edificis d'un municipi.
\end{enumerate}

\vskip 0.2 cm
\noindent
\textbf{Exercici 3} 

Classificau les seg�ents variables segons el seu tipus, dimensi� i nivell temporal:
\begin{enumerate}[a)]
\item Nombre de persones que han sofert un accident de tr�fic en els darrers 5 anys.
\item Nivell professional d'un militar (por exemple: soldat, sergent, etc.).
\item Nombre de gols aconseguits per un jugador de futbol al llarg de la temporada 2008-09. 
\item Religi� d'una persona (por exemple: cat�lic, musulm�, budista, etc.).
\item Pes i al�ada de les participants en una desfilada de moda.
\item Quantitat de diners gastada per una Administraci� en obres p�bliques durant el darrer any. 
\end{enumerate}


%\section{Soluciones de los ejercicios}
%
%\noindent
%\textbf{Ejercicio 1}
%\begin{enumerate}[a)]
%\item Poblaci\'on: el total de los alumnos del instituto. 
%Muestra: el $30\%$ por ciento de los alumnos de $2^\text{o}$ de bachillerato
%\item Poblaci\'on: los habitantes de Ciutadella menores de 35 a\~nos.
%Muestra: el $10\%$ de los clientes de los principales locales de copas.
%\item Poblaci\'on: todos los conductores del pa\'is. Muestra: 10.000 conductores.
%\end{enumerate}
%
%\vskip 0.2 cm
%\noindent
%\textbf{Ejercicio 2}\begin{enumerate}[a)]
%\item Descriptiva.
%\item Descriptiva.
%\item Inferencial.
%\end{enumerate}
%
%\vskip 0.2 cm
%\noindent
%\textbf{Ejercicio 3}
%\begin{enumerate}[a)]
%\item Sexo, edad, nivel de estudios, nivel de ingresos, etc.
%\item Sexo, edad, antig\"uedad en la empresa, categor\'ia profesional, salario, etc.
%\item Tiempo dedicado al estudio, compaginaci\'on de estudios
%y trabajo, n\'umero de veces que se ha cursado la asignatura, 
%nota en los examenes de acceso a la universidad, etc.
%\end{enumerate}
%
%\vskip 0.2 cm
%\noindent
%\textbf{Ejercicio 4}
%\begin{enumerate}[a)]
%\item Cuantitativa discreta, unidimensional, temporal.
%\item Ordinal, unidimensional, atemporal.
%\item Cuantitativa discreta, unidimensional, temporal.
%\item Nominal, unidimensional, atemporal.
%\item Cuantitativa continua, multidimensional (bidimensional), atemporal.
%\item Cuantitativa continua, unidimensional, temporal.
%\end{enumerate}
%
%


\end{document}

