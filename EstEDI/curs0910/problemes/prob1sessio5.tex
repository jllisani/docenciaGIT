\documentclass{article}
\usepackage[catalan]{babel}
\usepackage[latin1]{inputenc}   % Permet usar tots els accents i car�ters llatins de forma directa.
\usepackage{enumerate}
\usepackage{amsfonts, amscd, amsmath, amssymb}
\usepackage{eepic}
\usepackage{graphicx}

%NOTA: si usam eepic hem de compilar a .dvi o .ps (NO PDF)

\setlength{\textwidth}{16cm}
\setlength{\textheight}{24cm}
\setlength{\oddsidemargin}{-0.3cm}
\setlength{\evensidemargin}{0.25cm} \addtolength{\headheight}{\baselineskip}
\addtolength{\topmargin}{-3cm}

\newcommand\Z{\mathbb{Z}}
\newcommand\R{\mathbb{R}}
\newcommand\N{\mathbb{N}}
\newcommand\Q{\mathbb{Q}}
\newcommand\K{\Bbbk}
\newcommand\C{\mathbb{C}}

%\pagestyle{empty}
\begin{document}


\begin{center}
\textbf{\Large Tema 1. An�lisi explorat�ria de dades
\newline
Exercicis proposats}
\end{center}

\vskip 0.3 cm





\section*{Sessi� 5: mesures de simetria i apuntament}



\vskip 0.2 cm
\noindent
\textbf{Exercici 1} 

En una enquesta entre els estudiants de la UIB s'han obtingut les seg�ents dades sobre la seva edat:

\begin{center}
\begin{tabular}{c|c}
Edat & Quantitat \\ \hline
18 & 120 \\
19 & 150 \\
20 & 90 \\
21 & 70 \\
22 & 65 \\
23 & 50 \\
24 & 30 \\
25 & 20 \\
26 & 10 \\
27 & 7 \\
28 & 8 \\
29 & 2 \\
30 & 1 \\
34 & 1 \\
35 & 1 \\
40 & 1
\end{tabular}
\end{center}

Estudiau la simetria i l'apuntament d'aquesta distribuci� de valors.


\end{document}

