\documentclass[a4paper,10pt]{article}
%\usepackage[active]{srcltx}      	%% necesario para pasar del dvi al tex%
\usepackage[spanish]{babel}
\usepackage[applemac]{inputenc}
%\usepackage[utf8]{inputenc}
%\usepackage[latin1]{inputenc}
%\usepackage{sw20res1}
\usepackage{amsmath}
\usepackage{amsfonts}
\usepackage{amssymb}
\usepackage{graphicx}
\usepackage{enumerate}
\setlength{\topmargin}{-2.5cm}	%%formato de pagina que ocupa todo
\setlength{\textwidth}{18.5cm}
\setlength{\textheight}{28cm}
\setlength{\oddsidemargin}{-1.5cm}

\pagestyle{empty}
\newcounter{prbcont}
\stepcounter{prbcont}
\setcounter{prbcont}{0}
\newtheorem{problema}[prbcont]{Problema}

\begin{document}

\noindent
{\large \bf Escola Polit�cnica Superior}

\noindent
{\large Grau en Enginyeria d'Edificaci�}

\vskip 0.3cm
\noindent
{\large \bf Assignatura: Aplicacions Estad�stiques}

\hrule

\vskip 0.3cm

\noindent
Tipus d'activitat

\begin{tabular}{|l|c|c|c|c|}
\hline
 & Exercici & Treball / Pr�ctica & Examen & Altres \\
\hline
Puntuable & & & X & \\ \hline
No Puntuable & & & & \\ \hline
\end{tabular}

\vskip 0.3cm

\noindent
Compet�ncies espec�fiques que es treballen

\begin{tabular}{|l|c|}
\hline
Capacitat per a utilitzar les t�cniques i m�todes probabil�stics i d'an�lisi estad�stica & X \\
\hline
\end{tabular}

\vskip 0.3cm

\noindent
Compet�ncies gen�riques que es treballen

\begin{tabular}{|l|c|}
\hline
Resoluci� de problemes (CI-1) & X \\ \hline
Capacitat d'an�lisi i s�ntesi (CI-4) & X \\ \hline
Coneixement d'inform�tica relatiu a l'�mbit d'estudis (CI-2) & \\ \hline
Aptitud per a la gesti� de l'informaci� (CI-5) & \\ \hline
Comprom�s �tic (CP-1) & X \\ \hline
Raonament cr�tic (CP-2) & X \\ \hline
Aptitud per al treball en equip (CP-3) & \\ \hline
Aprenentatge aut�nom (CP-9) & \\ \hline
\end{tabular}


\vskip 0.3 cm

\noindent
\textbf{Data: 19/03/2013}

\hrule

\vspace{0.3 cm}

\begin{problema}

\noindent
Es pren una mostra de 11 elements d'una poblaci� i per a cada un d'ells es recopilen dues dades estad�stiques $X$ i $Y$.
Les dades brutes es mostren en la taula seg�ent:

\begin{center}
\begin{tabular}{c|ccccccccccc}
X & 5 & 6 & 7 & 8 & 9 & 10 & 11 & 12 & 13 & 14 & 15\\ \hline
Y & 27 & 18 & 11 & 6 & 3 & 2 & 3 & 6 & 11 & 18 &27 
\end{tabular}
\end{center}

\noindent
Es demana:
\begin{enumerate}[a)]
\item Mitjana i desviaci� t�pica de la variable $X$.
\item Mitjana i desviaci� t�pica de la variable $Y$.
\item Mediana i quartils primer i tercer de la variable $Y$.
\item Diagrama de dispersi�.
\item Covari�ncia i coeficient de correlaci� lineal. Comentau el resultat.
\item Taula de conting�ncia.
\item Coeficient de conting�ncia. Comentau el resultat en relaci� amb
el valor de la correlaci� lineal.
\end{enumerate}


\end{problema}




\end{document}

