\documentclass[11pt]{article}

\usepackage[active]{srcltx}
\usepackage[latin1]{inputenc}	%%para utilizar tildes en el texto
\usepackage[catalan]{babel}	%%corta las palabras segun el castellano
\usepackage{epsfig}
\usepackage{enumerate}
\usepackage{url}
\usepackage{hyperref}


\setlength{\topmargin}{-2cm}	%%formato de pagina que ocupa todo
\setlength{\textwidth}{16cm}
\setlength{\textheight}{24cm}
\setlength{\oddsidemargin}{0cm}


\usepackage{amsmath}
\usepackage{amsfonts}
\usepackage{amssymb}

\newcounter{prbcont}
\stepcounter{prbcont}
\setcounter{prbcont}{0}
\newtheorem{problema}[prbcont]{Problema}

\begin{document}
\noindent
{\large \bf Escola Polit\`ecnica Superior}

\noindent
{\large Grau en Enginyeria d'Edificaci\'o}

\vskip 0.3cm
\noindent
{\large \bf Assignatura: Aplicacions Estad\'\i stiques}

\hrule

\vskip 0.3cm

\noindent
Tipus d'activitat

\begin{tabular}{|l|c|c|c|c|}
\hline
 & Exercici & Treball / Pr\`actica & Examen & Altres \\
\hline
Puntuable & & X &  & \\ \hline
No Puntuable & & & & \\ \hline
\end{tabular}

\vskip 0.3cm

\noindent
Compet\`encies espec\'\i fiques que es treballen

\begin{tabular}{|l|c|}
\hline
Capacitat per a utilitzar les t\`ecniques i m\`etodes probabil\'\i stics i d'an\`alisi estad\'\i stica & X \\
\hline
\end{tabular}

\vskip 0.3cm

\noindent
Compet\`encies gen\`eriques que es treballen

\begin{tabular}{|l|c|}
\hline
Resoluci\'o de problemes (CI-1) & X \\ \hline
Capacitat d'an\`alisi i s\'\i ntesi (CI-4) & X \\ \hline
Coneixement d'inform\`atica relatiu a l'\`ambit d'estudis (CI-2) & X \\ \hline
Aptitud per a la gesti\'o de l'informaci\'o (CI-5) & X\\ \hline
Comprom\'\i s \`etic (CP-1) & X \\ \hline
Raonament cr\'\i tic (CP-2) & X \\ \hline
Aptitud per al treball en equip (CP-3) &  \\ \hline
Aprenentatge aut\`onom (CP-9) & X\\ \hline
\end{tabular}


\vskip 0.3 cm

\noindent
\textbf{Data entrega: 07/04/2013}

\hrule

\vspace{0.3cm}
\noindent
\textbf{Pr\`actica Tema I}

\vspace{0.3cm}


\begin{quotation}
\textit{L'enunciat d'aquesta pr\`actica \'es variable i dep\`en dels valors que us han estat assignats personalment, en particular de l'\textbf{any-inici} i
de l'\textbf{any-fi}. Podeu consultar la vostra assignaci\'o en el document \textit{Assignaci\'o Pr\`actiques} que trobareu a Campus Extens.}
\end{quotation}

En aquesta pr\`actica analitzarem el nombre d'habitatges per habitant, visats durant el per\'{\i}ode compr\`es entre any--inici any--fi per a cadascun dels municipis de Mallorca. Les dades necess\`aries per a aquest estudi han d'obtenir-se de l'IBESTAT mitjan\c{c}ant la seva adre\c{c}a 
\begin{center}
\url{http://ibestat.caib.es/ibestat/page?lang=ca}
\end{center}
A continuaci\'o descrivim el procediment per descarregar-se les dades de la plana web.
\begin{itemize}
\item [i)] En la barra de menu trobareu la opci\'o \textit{Estad\'{\i}stiques}. Entrau en l'apartat d'\textit{Economia}, i dins d'aquest, en el de \textit{Construcci\'o i habitatge}. 
Anau a la secci\'o \textit{Visats, llic\`encies i certificacions d'obra} i triau dins de \textit{Projectes visats} l'item
\textit{Nombre, pressupost i metres dels projectes visats per any i mes, illa i municipi i tipus d'\'us}. Accedireu a una plana web amb diferents finestres que us permetran seleccionar les dades a baixar.
\item [ii)] En la finestra titulada \textbf{per\'{\i}ode} s'ha de seleccionar el per\'{\i}ode de temps que se us ha assignat: des del mes de gener (M01) de l'\textbf{any--inici} a desembre (M12) de l'\textbf{any--fi}, ambd\'os inclosos. Tant mateix, a la finestra titulada \textbf{Illa i municipi} tothom ha de seleccionar els 53 municipis de Mallorca (del codi \textit{07001 Alar\'o} fins al \textit{07901 Ariany}) . A la finestra \textbf{Tipus d'\'us} s'ha de seleccionar l'\'us Residencial i a la finestra \textbf{Dades} el Nombre total d'habitages.
\item [iii)] Ara construirem la taula amb les dades seleccionades. Trieu \'unicament la variable \textbf{Illa i municipi} per a la finestra \textbf{Variable en files} i passeu la resta a la finestra \textbf{Variables en columnes}. Finalment col.loqueu la variable \textbf{Per\'{\i}ode} com la primera a la finestra \textbf{Variables en columnes}.
\item [iv)] Pitjau el bot\'o \textbf{Consultar la selecci\'o}. Obtindrem una taula amb la informaci\'o requerida. En la primera columna apareixen els diferents municipis de Mallorca. En les seg\"uents columnes el nombre d'habitatges visats com a residencials depenent del mes i any. Aqu\'i podeu descarregar-vos la taula en diferents formats, trieu el format Excel.
\item [v)] Obriu la taula amb el full de c\`alcul. Per obtenir el nombre d'habitatges residencials visats per municipi durant el per\'{\i}ode assignat \'unicament us resta sumar els valors de les diferents files. 
\item [vi)] Ara obtindrem el nombre d'habitants per municipi. A la barra de menu trobareu la opci\'o \textit{Estad\'{\i}stiques}. Entrau en l'apartat \textit{Poblaci\'o}, i dins d'aquest, en el de \textit{Padr\'o (xifres oficials de poblaci\'o)}. En l'item de \textit{Dades anuals} seleccioneu l'\textbf{any de fi}. Anau a la taula \textit{Poblaci\'o per illa, municipi, sexe i grup d'edat}. Trieu el format Excel i descarregar-vos la taula. A la primera columna de la taula hi ha la poblaci\'o per municipi. Per obtenir el nombre d'habitatges residencials visats per habitant, heu de dividir el nombre d'habitages per el nombre d'habitants. 
\end{itemize}

A partir de l'anterior informaci\'o es demana:

\begin{enumerate}[1-]
\item  Calculau, a partir de les dades brutes, la mediana, mitjana i moda, la desviaci\'o t\'ipica i els coeficients de simetria i curtosi.

\item Calculau el rang i el rang interquart\'{\i}lic. Dibuixau el diagrama de capsa, marcant els valors at\'{\i}pics, si n'hi ha. 
Si no podeu dibuixar el diagrama de capsa amb l'ordinador, al manco heu de donar la seg\"uent informaci\'o: mediana; primer i tercer quartils; l\'\i mits
superior i inferior entre valors t\'\i pics i at\'\i pics; l\'\i mits superior i inferior entre valors at\'\i pics i extrems; llista de valors
at\'\i pics; llista de valors extrems.


\item  Agrupau les dades en un m\`axim de 10 subintervals d'igual amplitud (excepte els intervals inicial i final que poden tenir diferent amplitud)
i constru\"{\i}u una taula de freq\"u\`encies per a aquests intervals.
\begin{enumerate}[a)]
\item Representau mitjan\c{c}ant un diagrama de barres la freq\"u\`encia absoluta i mitjan\c{c}ant un diagrama de 
tarta el percentatge.
\item  Calculau la moda, la mitjana, la mediana, el primer i tercer quartils i el percentil
90.
\item  Calculau el ratio de variaci\'o i el rang interquart\'{\i}lic. 
\item  Calculau la vari\`ancia i la desviaci\'o t\'{\i}pica.
\end{enumerate}
\end{enumerate}
\end{document} 