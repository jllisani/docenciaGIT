\documentclass[11pt]{article}

\usepackage[active]{srcltx}
\usepackage[latin1]{inputenc}	%%para utilizar tildes en el texto
\usepackage[catalan]{babel}	%%corta las palabras segun el castellano
\usepackage{epsfig}
\usepackage{enumerate}
\usepackage{url}
\usepackage{hyperref}


\setlength{\topmargin}{-2cm}	%%formato de pagina que ocupa todo
\setlength{\textwidth}{16cm}
\setlength{\textheight}{24cm}
\setlength{\oddsidemargin}{0cm}


\usepackage{amsmath}
\usepackage{amsfonts}
\usepackage{amssymb}

\newcounter{prbcont}
\stepcounter{prbcont}
\setcounter{prbcont}{0}
\newtheorem{problema}[prbcont]{Problema}

\begin{document}
\noindent
{\large \bf Escola Polit\`ecnica Superior}

\noindent
{\large Grau en Enginyeria d'Edificaci\'o}

\vskip 0.3cm
\noindent
{\large \bf Assignatura: Aplicacions Estad\'\i stiques}

\hrule

\vskip 0.3cm

\noindent
Tipus d'activitat

\begin{tabular}{|l|c|c|c|c|}
\hline
 & Exercici & Treball / Pr\`actica & Examen & Altres \\
\hline
Puntuable & & X &  & \\ \hline
No Puntuable & & & & \\ \hline
\end{tabular}

\vskip 0.3cm

\noindent
Compet\`encies espec\'\i fiques que es treballen

\begin{tabular}{|l|c|}
\hline
Capacitat per a utilitzar les t\`ecniques i m\`etodes probabil\'\i stics i d'an\`alisi estad\'\i stica & X \\
\hline
\end{tabular}

\vskip 0.3cm

\noindent
Compet\`encies gen\`eriques que es treballen

\begin{tabular}{|l|c|}
\hline
Resoluci\'o de problemes (CI-1) & X \\ \hline
Capacitat d'an\`alisi i s\'\i ntesi (CI-4) & X \\ \hline
Coneixement d'inform\`atica relatiu a l'\`ambit d'estudis (CI-2) & X \\ \hline
Aptitud per a la gesti\'o de l'informaci\'o (CI-5) & X\\ \hline
Comprom\'\i s \`etic (CP-1) & X \\ \hline
Raonament cr\'\i tic (CP-2) & X \\ \hline
Aptitud per al treball en equip (CP-3) &  \\ \hline
Aprenentatge aut\`onom (CP-9) & X\\ \hline
\end{tabular}


\vskip 0.3 cm

\noindent
\textbf{Data entrega: 07/04/2013}

\hrule

\vspace{0.3cm}
\noindent
\textbf{Pr\`actica Tema II}

\vspace{0.3cm}


\begin{quotation}
\textit{L'enunciat d'aquesta pr\`actica \'es variable i dep\`en dels valors que us han estat assignats personalment, en particular de l'\textbf{any-inici} i
de l'\textbf{any-fi}. Podeu consultar la vostra assignaci\'o en el document \textit{Assignaci\'o Pr\`actiques} que trobareu a Campus Extens.}
\end{quotation}

En aquesta pr\`actica analitzarem les variables \textit{nombre d'habitatges visats per habitant} i \textit{nombre de vehicles matriculats per habitant} durant el per\'{\i}ode compr\`es entre any--inici any--fi per a cadascun dels municipis de Mallorca. Les dades necess\`aries per a aquest estudi han d'obtenir-se de l'IBESTAT mitjan\c{c}ant la seva adre\c{c}a 
\begin{center}
\url{http://ibestat.caib.es/ibestat/page?lang=ca}
\end{center}
Les dades de \textit{nombre d'habitatges visats per habitant} s\'on les que s'han utilitzat en la Pr\`actica 1. 
A continuaci\'o descrivim el procediment per descarregar-se les dades de \textit{nombre de vehicles matriculats} de la plana web (passos del (i) 
al (v)). Les dades sobre el nombre d'habitants s'obtenen igual que en la Pr\`actica Tema I (veure pas (vi) de la Pr\`actica Tema I) 
\begin{itemize}
\item [i)] En la barra de menu trobareu la opci\'o \textit{Estad\'{\i}stiques}. Entrau en l'apartat d'\textit{Economia}, i dins d'aquest, en el de \textit{Sector serveis}. 
Anau a la secci\'o \textit{Transport} i, dins \textit{Matriculaci\'o de vehicles} triau 
\textit{Illa-municipi, mes-any i tipus de vehicle}. 
Accedireu a una plana web amb diferents finestres que us permetran seleccionar les dades a baixar.
\item [ii)] En la finestra titulada \textbf{per\'{\i}ode} s'ha de seleccionar el per\'{\i}ode de temps que se us ha assignat: des del mes de gener (M01) al mes de desembre (M12) dels anys \textbf{any--inici}  a \textbf{any--fi}, ambd\'os inclosos. Tant mateix, a la finestra titulada \textbf{Illa i municipi} tothom ha de seleccionar els 53 municipis de Mallorca (del codi \textit{07001 Alar\'o} fins al \textit{07901 Ariany}) . A la finestra \textbf{Tipus de vehicle} s'ha de seleccionar Turismes.
\item [iii)] Ara construirem la taula amb les dades seleccionades. Trieu \'unicament la variable \textbf{Illa i municipi} per a la finestra \textbf{Variable en files} i passeu la resta a la finestra \textbf{Variables en columnes}. Finalment col.loqueu la variable \textbf{Per\'{\i}ode} com la primera a la finestra \textbf{Variables en columnes}.
\item [iv)] Pitjau el bot\'o \textbf{Consulta la selecci\'o}. Obtindrem una taula amb la informaci\'o requerida. En la primera columna apareixen els diferents municipis de Mallorca. En les seg\"uents columnes el nombre de turismes matriculats depenent del mes i any. Aqu\'i podeu descarregar-vos la taula en diferents formats, trieu el format Excel.
\item [v)] Obriu la taula amb el full de c\`alcul. Per obtenir el nombre vehicles matriculats per municipi durant el per\'{\i}ode assignat \'unicament us resta sumar els valors de les diferents files. 
\item [vi)] Per obtenir el \textit{nombre de vehicles matriculats per habitant}, heu de dividir el nombre de vehicles matriculats per el nombre d'habitants.
\end{itemize}

A partir de les dades d'aquest parell de variables es demana:

\begin{enumerate}[a)]
\item Dibuixau el diagrama de dispersi\'o.
\item Calculau el coeficient de correlaci\'o. Interpretau el resultat.
\item Si el coeficient de correlaci\'o \'es major que $0.5,$ calculau i dibuixau la recta de regressi\'o. Interpretau el resultat.
\item Si el coeficient de correlaci\'o \'es menor que $0.5,$ calculau el coeficient de conting\`encia i determineu si existeix depend\`encia entre les variables.
\end{enumerate}
\end{document} 