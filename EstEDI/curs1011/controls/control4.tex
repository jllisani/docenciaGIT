\documentclass[a4paper,10pt]{article}
%\usepackage[active]{srcltx}      	%% necesario para pasar del dvi al tex%
\usepackage[spanish]{babel}
\usepackage[latin1]{inputenc}
%\usepackage{sw20res1}
\usepackage{amsmath}
\usepackage{amsfonts}
\usepackage{amssymb}
\usepackage{graphicx}
\usepackage{enumerate}
\setlength{\topmargin}{-2.5cm}	%%formato de pagina que ocupa todo
\setlength{\textwidth}{18.5cm}
\setlength{\textheight}{28cm}
\setlength{\oddsidemargin}{-1.5cm}

\pagestyle{empty}
\newcounter{prbcont}
\stepcounter{prbcont}
\setcounter{prbcont}{0}
\newtheorem{problema}[prbcont]{Problema}

\begin{document}

\noindent
{\large \bf Escola Polit�cnica Superior}

\noindent
{\large Grau en Enginyeria d'Edificaci�}

\vskip 0.3cm
\noindent
{\large \bf Assignatura: Aplicacions Estad�stiques}

\hrule

\vskip 0.3cm

\noindent
Tipus d'activitat

\begin{tabular}{|l|c|c|c|c|}
\hline
 & Exercici & Treball / Pr�ctica & Examen & Altres \\
\hline
Puntuable & & & X & \\ \hline
No Puntuable & & & & \\ \hline
\end{tabular}

\vskip 0.3cm

\noindent
Compet�ncies espec�fiques que es treballen

\begin{tabular}{|l|c|}
\hline
Capacitat per a utilitzar les t�cniques i m�todes probabil�stics i d'an�lisi estad�stica & X \\
\hline
\end{tabular}

\vskip 0.3cm

\noindent
Compet�ncies gen�riques que es treballen

\begin{tabular}{|l|c|}
\hline
Resoluci� de problemes (CI-1) & X \\ \hline
Capacitat d'an�lisi i s�ntesi (CI-4) & X \\ \hline
Coneixement d'inform�tica relatiu a l'�mbit d'estudis (CI-2) & \\ \hline
Aptitud per a la gesti� de l'informaci� (CI-5) & \\ \hline
Comprom�s �tic (CP-1) & X \\ \hline
Raonament cr�tic (CP-2) & X \\ \hline
Aptitud per al treball en equip (CP-3) & \\ \hline
Aprenentatge aut�nom (CP-9) & \\ \hline
\end{tabular}


\vskip 0.3 cm

\noindent
\textbf{Data: 23/05/2011}

\hrule

\vspace{0.3 cm}

\begin{problema}
Suposem que un $60\%$ dels universitaris opinen que s�n millors les pizzes de salami que les
de roquefort.
\begin{enumerate}[a)]
\item Quina �s la probabilitat que m�s del $70\%$ dels components d'una mostra de 200 universitaris
siguin d'aquesta opini�?
\item Quina �s la probabilitat que menys del $50\%$ dels components d'una mostra de 100 universitaris
siguin d'aquesta opini�?
\item Si la proporci� d'universitaris que opinen que les pizzes de salami s�n millors que les
de roquefort �s un valor desconegut $p$, quin �s el tamany m�nim de la mostra que ens 
`assegura' (amb probabilitat superior al $95\%$) que l'error com�s en estimar $p$ a
partir de la proporci� mostral �s inferior a $0,01$? (Suposau que la mostra est� formada per m�s de
30 persones).
\end{enumerate}

\end{problema}


\vskip 1cm

\begin{problema}
Un conductor fa habitualment el trajecte entre Bunyola i la UIB i assegura
que tarda una mitjana de 9 minuts en fer el trajecte. Per comprovar si
es cumpleix l'afirmaci� d'aquest conductor es pren una mostra dels seus 
7 darrers trajectes i s'obtenen els 
seg�ents temps (en minuts):

\[
10.5 \quad 7.3 \quad 15.1 \quad 8.9 \quad 9.6 \quad 11.7 \quad 12.5
\]
Suposant que el temps que tarda el conductor en fer el trajecte segueix una distribuci� normal
feu un contrast d'hip�tesis per confirmar o rebutjar l'afirmaci� del conductor amb un nivell de significaci� del $10\%$.
Justificau les hip�tesis utilitzades i calculau el p-valor del contrast.



\end{problema}

\newpage
\vskip 0.5 cm
\textbf{Estad\'\i stics m\'es usuals}
\vskip 0.2 cm

\begin{tabular}{c|c|c|cl}
Par\`ametre  & Esperan\c{c}a & Vari\`ancia & Distribuci\'o  & \\
mostral &  &  & de probabilitat & \\
(estad\'\i stic) & & & & \\ 
\hline
$\bar{X}$ & $E(\bar{X})=\mu$ & $\mathrm{Var}(\bar{X})=\frac{\sigma^2}{n}$ & 
$\bar{X} \sim N(\mu, \frac{\sigma^2}{n})$ & poblaci\'o normal, $\sigma$ conegut \\
& & & $\frac{\bar{X}-\mu}{\hat{s}_X / \sqrt{n}} \sim t_{n-1}$ & 
poblaci\'o normal, $\sigma$ desconegut, $n \leq 30$ \\
& & & 
$\bar{X} \sim N(\mu, \frac{\hat{s}_X^2}{n})$ & 
$\sigma$ desconegut, $n > 30$ \\
& & & & \\
$\hat{s}_X^2$ & $E(\hat{s}_X^2)=\sigma^2$ & $\mathrm{Var}(\hat{s}_X^2)=\frac{2\sigma^4}{n-1}$ & 
$\frac{n-1}{\sigma^2}\hat{s}_X^2 \sim \chi^2_{n-1}$ & poblaci\'o normal \\
& & & & \\
$\hat{p}_X$ & $E(\hat{p}_X)=p$ & $\mathrm{Var}(\hat{p}_X)=\frac{p(1-p)}{n}$ &
$\hat{p}_X \sim N(p, \frac{p(1-p)}{n})$ & $n > 30$ \\
 & & & $\hat{p}_X \sim t_{n-1}$ & poblaci\'o normal, $n \leq 30$ 
\end{tabular}

\vskip 0.7 cm
\textbf{Intervals de confian\c{c}a m\'es usuals}
\vskip 0.2 cm

\begin{tabular}{l|ll}
Par\`ametre mostral & Interval de confian\c{c}a & \\
\hline
& & \\
Mitjana & $\displaystyle \bar{X} \pm z_{\alpha/2} \frac{\sigma}{\sqrt{n}}$ & poblaci\'o normal,
$\sigma$ conegut \\
& & \\
 & $\displaystyle \bar{X} \pm t_{n-1, \alpha/2} \frac{\hat{s}_X}{\sqrt{n}}$ & poblaci\'o normal,
$\sigma$ desconegut  \\
& & i $n \leq 30$\\
& & \\
& $\displaystyle \bar{X} \pm z_{\alpha/2} \frac{\hat{s}_X}{\sqrt{n}}$ & si $n > 30$ \\
& & \\
& & \\
Vari\`ancia & $\displaystyle \left[ \frac{n-1}{\chi^2_{n-1, 1-\alpha/2}} \hat{s}_X^2,
 \frac{n-1}{\chi^2_{n-1, \alpha/2}} \hat{s}_X^2 \right]$ & si la poblaci\'o segueix una llei normal  \\
& & \\
& & \\
Proporci\'o & $\displaystyle \hat{p}_X \pm z_{\alpha/2} \sqrt{\frac{\hat{p}_X (1-\hat{p}_X)}{n}}$ & si $n > 30$ \\
& & 
\end{tabular}

\end{document}

