\documentclass[a4paper,10pt]{article}
%\usepackage[active]{srcltx}      	%% necesario para pasar del dvi al tex%
\usepackage[spanish]{babel}
\usepackage[latin1]{inputenc}
%\usepackage{sw20res1}
\usepackage{amsmath}
\usepackage{amsfonts}
\usepackage{amssymb}
\usepackage{graphicx}
\usepackage{enumerate}
\setlength{\topmargin}{-2.5cm}	%%formato de pagina que ocupa todo
\setlength{\textwidth}{18.5cm}
\setlength{\textheight}{28cm}
\setlength{\oddsidemargin}{-1.5cm}

\pagestyle{empty}
\newcounter{prbcont}
\stepcounter{prbcont}
\setcounter{prbcont}{0}
\newtheorem{problema}[prbcont]{Problema}

\begin{document}


\noindent
\textbf{Soluci� Control 3 Aplicacions Estad�stiques. Data: 20/04/2011}

\hrule

\vspace{0.3 cm}

\begin{problema}
En una universitat s'ha observat que el 60\% dels estudiants que es
matriculen ho fan en una carrera de Ci\`encies, mentre que l'altre 40\%
ho fan en carreres d'Humanitats. Si un determinat dia es realitzen
20 matr\'{\i}cules, calcular la probabilitat que:
\begin{itemize}
\item [a)] hi hagi igual nombre de matr\'{\i}cules en Ci\`encies i en Humanitats;
\item [b)] el nombre de matr\'{\i}cules en Ci\`encies sigui menor que en Humanitats;
\item [c)] hi hagi almenys 8 matr\'{\i}cules en Ci\`encies;
\item [d)] no hi hagi m\'es de 12 matr\'{\i}cules en Ci\`encies.
\item [e)] Si les cinc primeres matr\'{\i}cules s\'on d'Humanitats, calcular la
probabilitat que en total hi hagi igual nombre de matr\'{\i}cules en Ci\`encies i en Humanitats.
\end{itemize}
\end{problema}

\vskip 0.5 cm
\noindent
\textbf{Soluci�}

\noindent
Podem definir dues variables aleat�ries que ens ajudaran a resoldre el problema:

\noindent
$X$: nombre de matr�cules en Ci�ncies (del total de 20 matr�cules)

\noindent
$Y$: nombre de matr�cules en Humanitats (del total de 20 matr�cules)

\vskip 0.3 cm
\noindent
A partir de les dades de l'enunciat dedu�m que:

\noindent
$X \sim B(20, 0.6)$

\noindent
$Y \sim B(20, 0.4)$

\vskip 0.3 cm
\noindent
Es tracta per tant de dues variables de tipus \textbf{binomial}. Com que volem
resoldre el problema amb l'ajuda de les taules i en aquestes nom�s estan tabulats
els valors per a $n=20$ i $p=0.4$, plantejarem les preguntes del problema en termes
de la variable $Y$.

\vskip 0.3cm
\noindent
a) $P(X=10)=P(Y=10)=P(Y \leq 10)-P(Y\leq 9)=\text{(taules)}=0.8725-0.7553=0.1172$

\vskip 0.3cm
\noindent
b) $P(X<10)=P(Y>10)=1-P(Y \leq 10)=\text{(taules)}=1-0.8725=0.1275$

\vskip 0.3cm
\noindent
c) $P(X\geq 8)=P(Y\leq 12)=\text{(taules)}=0.9790$

\vskip 0.3cm
\noindent
d) $P(X \leq 12)=P(Y \geq 8)=1-P(Y<8)=1-P(Y\leq 7)=\text{(taules)}=1-0.4159=0.5841$

\vskip 0.3cm
\noindent
e) Sabem que ja hi ha 5 matriculats d'Humanitats (del total de 20 persones matriculades).
Perqu� hi hagi el mateix nombre total de matriculats de Ci�ncies i d'Humanitats (�s a dir, 10 
de cada tipus), el nombre de matriculats d'Humanitats entre les 15 persones que queden 
per matricular-se ha d'�sser igual a 5. 

\vskip 0.3 cm
\noindent
Definim una nova variable:

\noindent
$Y'$: nombre de matr�cules en Humanitats (del total de 15 matr�cules que queden per fer-se)

\vskip 0.3 cm
\noindent
Tenim que $Y' \sim B(15, 0.4)$. 

\vskip 0.3 cm
\noindent
Per tant:

\noindent
$P(Y'=5)=P(Y' \leq 5) - P(Y' \leq 4)=\text{(taules)}=0.4032-0.2173=0.1859$


\newpage
\begin{problema}
L'empresa \textsc{EMPIPATSA} vol comenar a produir bosses de pipes de pes nominal 100g. La normativa vigent exigeix que el pes del producte envasat no
pot ser inferior al 95\% del pes nominal. L'empresa considera que el pes del producte
envasat seguir\`a una llei normal de par\`ametres 98g i desviaci\'o t\'{\i}pica 1g. Es demana:
\begin{itemize}
\item [a)] Demostreu que la probabilitat que una bossa no compleixi la normativa \'es
igual a $0.0013.$
\item [b)] Calculeu la probabilitat que el pes d'una bossa que compleixi la normativa sigui m\'es petit que el pes nominal. 
\item [c)] Calculeu la probabilitat que una caixa de 20 bosses contingui exactament
3 bosses que no compleixen la normativa.
\end{itemize}
\end{problema}


\vskip 0.5 cm
\noindent
\textbf{Soluci�}

\noindent
Definim la variable aleat�ria seg�ent:

\noindent
$X$: pes (en grams) d'una bossa de pipes

\vskip 0.3 cm
\noindent
A partir de les dades de l'enunciat tenim que $X \sim N(98, 1^2)$.

\vskip 0.3 cm
\noindent
a) $P(X < 95)=\text{(�s v.a. cont�nua)}=P(X \leq 95)=F_X(95)=F_Z(\frac{95-98}{1})=F_Z(-3)=1-F_Z(3)=\text{(taules)}=1-0.9987=0.0013$

\vskip 0.3 cm
\noindent
b) $P(X < 100 |_{X \geq 95})=\frac{P(95 \leq X < 100)}{P(X \geq 95)}$

\vskip 0.3 cm
\noindent
$P(95 \leq X < 100)=\text{(�s v.a. cont�nua)}=F_X(100)-F_X(95)=F_Z(\frac{100-98}{1})-F_Z(\frac{95-98}{1})=F_Z(2)-F_Z(-3)=
\text{(taules i apartat a)}=0.9772-0.0013=0.9759$

\vskip 0.3 cm
\noindent
$P(X \geq 95)=1-P(X < 95)=\text{(apartat a)}=1-0.0013=0.9987$

\vskip 0.3 cm
\noindent
Per tant: $P(X < 100 |_{X \geq 95})=\frac{0.9759}{0.9987}=0.9771$

\vskip 0.3 cm
\noindent
c) Definim una nova variable:

\noindent
$Y$: nombre de bosses que no cumpleixen la normativa (en una capsa de 20 bosses)

\vskip 0.3 cm
\noindent
A partir de les dades de l'enunciat i de l'apartat a) dedu�m: $Y \sim B(20, 0.0013)$

\vskip 0.3 cm
\noindent
$P(Y=3)=\binom{20}{3} \cdot 0.0013^3 \cdot (1-0.0013)^{17} = 2.45 \cdot 10^{-6}$


\end{document}

