%% This document created by Scientific Word (R) Version 3.0
%%
%% Examen extraordinario para un chico que no pudo asistir al oficial
%%
%%%%%
\documentclass[a4paper,10pt]{article}
%\usepackage[active]{srcltx}      	%% necesario para pasar del dvi al tex%
\usepackage[spanish]{babel}
\usepackage[latin1]{inputenc}
%\usepackage{sw20res1}
\usepackage{amsmath}
\usepackage{amsfonts}
\usepackage{amssymb}
\usepackage{graphicx}
\setlength{\topmargin}{-3cm}	%%formato de pagina que ocupa todo
\setlength{\textwidth}{18cm}
\setlength{\textheight}{27cm}
\setlength{\oddsidemargin}{-1cm}

\pagestyle{empty}
\newcounter{prbcont}
\stepcounter{prbcont}
\setcounter{prbcont}{0}
\newtheorem{problema}[prbcont]{Problema}

\begin{document}
\noindent\textbf{\large{Escola Polit\`ecnica Superior}}\\
\noindent{\large{Grau en Enginyeria d'Edificaci\'o}}\\

\noindent\textbf{\underline{\large{Assignatura: Aplicacions Estad\'{i}stiques}\hspace{10cm}} }\\ 
\noindent\begin{small}\textbf{Tipus d'activitat:}\end{small}
\begin{center}
\begin{tabular}{|l|c|c|c|c|}\hline
 		&Exercici & Treball/Pr\`actica & Examen & Altres \\ \hline
Puntuable       & 	  & 			 & \textbf{X}  	  & \\ \hline
No Puntuable    &  	  &			 & 	  & \\ \hline
\end{tabular}
\end{center}
\noindent\begin{small}\textbf{Compet\`encies espec\'{\i}fiques que es treballen:}\end{small}\\
\begin{tabular}{|l|c|}\hline
Capacitat per a utilitzar les t\`ecniques i m\`etodes probabil\'{\i}stics i d'an\`alisi estad\'{\i}stica & \textbf{X} \\ \hline
\end{tabular}\vspace{0.25cm}

\noindent\begin{small}\textbf{Compet\`encies gen\`eriques que es treballen:}\end{small}\\
\begin{tabular}{|l|c|}\hline
Resoluci\'o de problemes (CI-1) & \textbf{X}\\ \hline
Capacitat d'an\`alisi i s\'{\i}ntesi (CI-4) & \textbf{X}  \\ \hline
Comprom\'{\i}s \`etic (CP-1)$\quad$ & \textbf{X} \\ \hline
\end{tabular}\vspace{0.25cm}

\noindent\begin{small}\textbf{Data: 20/04/2011 }\end{small}\\
\noindent\underline{\hspace{18cm}}


\begin{problema}
En una universitat s'ha observat que el 60\% dels estudiants que es
matriculen ho fan en una carrera de Ci\`encies, mentre que l'altre 40\%
ho fan en carreres d'Humanitats. Si un determinat dia es realitzen
20 matr\'{\i}cules, calcular la probabilitat que:
\begin{itemize}
\item [a)] hi hagi igual nombre de matr\'{\i}cules en Ci\`encies i en Humanitats;
\item [b)] el nombre de matr\'{\i}cules en Ci\`encies sigui menor que en Humanitats;
\item [c)] hi hagi almenys 8 matr\'{\i}cules en Ci\`encies;
\item [d)] no hi hagi m\'es de 12 matr\'{\i}cules en Ci\`encies.
\item [e)] Si les cinc primeres matr\'{\i}cules s\'on d'Humanitats, calcular la
probabilitat que en total hi hagi igual nombre de matr\'{\i}cules en Ci\`encies i en Humanitats.
\end{itemize}
\end{problema}

\begin{problema}
L'empresa \textsc{EMPIPATSA} vol comenar a produir bosses de pipes de pes nominal 100g. La normativa vigent exigeix que el pes del producte envasat no
pot ser inferior al 95\% del pes nominal. L'empresa considera que el pes del producte
envasat seguir\`a una llei normal de par\`ametres 98g i desviaci\'o t\'{\i}pica 1g. Es demana:
\begin{itemize}
\item [a)] Demostreu que la probabilitat que una bossa no compleixi la normativa \'es
igual a $0.0013.$
\item [b)] Calculeu la probabilitat que el pes d'una bossa que compleixi la normativa sigui m\'es petit que el pes nominal. 
\item [c)] Calculeu la probabilitat que una caixa de 20 bosses contingui exactament
3 bosses que no compleixen la normativa.
\end{itemize}
\end{problema}

\end{document}


