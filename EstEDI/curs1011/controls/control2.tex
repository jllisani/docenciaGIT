\documentclass[a4paper,10pt]{article}
%\usepackage[active]{srcltx}      	%% necesario para pasar del dvi al tex%
\usepackage[spanish]{babel}
\usepackage[latin1]{inputenc}
%\usepackage{sw20res1}
\usepackage{amsmath}
\usepackage{amsfonts}
\usepackage{amssymb}
\usepackage{graphicx}
\usepackage{enumerate}
\setlength{\topmargin}{-2.5cm}	%%formato de pagina que ocupa todo
\setlength{\textwidth}{18.5cm}
\setlength{\textheight}{28cm}
\setlength{\oddsidemargin}{-1.5cm}

\pagestyle{empty}
\newcounter{prbcont}
\stepcounter{prbcont}
\setcounter{prbcont}{0}
\newtheorem{problema}[prbcont]{Problema}

\begin{document}

\noindent
{\large \bf Escola Polit�cnica Superior}

\noindent
{\large Grau en Enginyeria d'Edificaci�}

\vskip 0.3cm
\noindent
{\large \bf Assignatura: Aplicacions Estad�stiques}

\hrule

\vskip 0.3cm

\noindent
Tipus d'activitat

\begin{tabular}{|l|c|c|c|c|}
\hline
 & Exercici & Treball / Pr�ctica & Examen & Altres \\
\hline
Puntuable & & & X & \\ \hline
No Puntuable & & & & \\ \hline
\end{tabular}

\vskip 0.3cm

\noindent
Compet�ncies espec�fiques que es treballen

\begin{tabular}{|l|c|}
\hline
Capacitat per a utilitzar les t�cniques i m�todes probabil�stics i d'an�lisi estad�stica & X \\
\hline
\end{tabular}

\vskip 0.3cm

\noindent
Compet�ncies gen�riques que es treballen

\begin{tabular}{|l|c|}
\hline
Resoluci� de problemes (CI-1) & X \\ \hline
Capacitat d'an�lisi i s�ntesi (CI-4) & X \\ \hline
Coneixement d'inform�tica relatiu a l'�mbit d'estudis (CI-2) & \\ \hline
Aptitud per a la gesti� de l'informaci� (CI-5) & \\ \hline
Comprom�s �tic (CP-1) & X \\ \hline
Raonament cr�tic (CP-2) & X \\ \hline
Aptitud per al treball en equip (CP-3) & \\ \hline
Aprenentatge aut�nom (CP-9) & \\ \hline
\end{tabular}


\vskip 0.3 cm

\noindent
\textbf{Data: 25/03/2011}

\hrule

\vspace{0.3 cm}

\begin{problema}
\noindent
Una persona treu 5 bolles, amb reposici�, d'una urna que cont�
7 bolles blanques i 3 negres.
\begin{enumerate}[a)]
\item Quina �s la probabilitat de treure 3 blanques?
\item Quina �s la probabilitat de treure 3 blanques i que alguna d'elles surti en la primera o la segona extracci�?
%\item Si ha tret 3 blanques, quina �s la probabilidad que alguna d'elles surti en la primera o la segona extracci�?
\end{enumerate}
\end{problema}

\vskip 0.5 cm
\begin{problema}
\noindent
En Toni, en Pep i na Maria es reuneixen per resoldre problemes d'estad�stica.
Toni resol el $40\%$ del total dels problemes, Pep el $30\%$ i Maria el $30\%$ restant.
Toni s'equivoca en un $2\%$ dels problemes que resol, Pep en el $6\%$ y Maria en l'$1\%$. 
Agafam a l'atzar un dels problemes resolts.
\begin{enumerate}[a)]
\item Quina �s la probabilitat que estigui ben resolt?
\item Si el problema est� mal resolt, quina �s la probabilitat
que l'hagi resolt en Pep?
\item Quina �s la probabilitat que estigui ben resolt i que l'hagi resolt en Toni?
\end{enumerate}
\end{problema}

\end{document}

