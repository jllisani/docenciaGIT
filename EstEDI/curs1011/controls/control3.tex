\documentclass[a4paper,10pt]{article}
%\usepackage[active]{srcltx}      	%% necesario para pasar del dvi al tex%
\usepackage[spanish]{babel}
\usepackage[latin1]{inputenc}
%\usepackage{sw20res1}
\usepackage{amsmath}
\usepackage{amsfonts}
\usepackage{amssymb}
\usepackage{graphicx}
\usepackage{enumerate}
\setlength{\topmargin}{-2.5cm}	%%formato de pagina que ocupa todo
\setlength{\textwidth}{18.5cm}
\setlength{\textheight}{28cm}
\setlength{\oddsidemargin}{-1.5cm}

\pagestyle{empty}
\newcounter{prbcont}
\stepcounter{prbcont}
\setcounter{prbcont}{0}
\newtheorem{problema}[prbcont]{Problema}

\begin{document}

\noindent
{\large \bf Escola Polit�cnica Superior}

\noindent
{\large Grau en Enginyeria d'Edificaci�}

\vskip 0.3cm
\noindent
{\large \bf Assignatura: Aplicacions Estad�stiques}

\hrule

\vskip 0.3cm

\noindent
Tipus d'activitat

\begin{tabular}{|l|c|c|c|c|}
\hline
 & Exercici & Treball / Pr�ctica & Examen & Altres \\
\hline
Puntuable & & & X & \\ \hline
No Puntuable & & & & \\ \hline
\end{tabular}

\vskip 0.3cm

\noindent
Compet�ncies espec�fiques que es treballen

\begin{tabular}{|l|c|}
\hline
Capacitat per a utilitzar les t�cniques i m�todes probabil�stics i d'an�lisi estad�stica & X \\
\hline
\end{tabular}

\vskip 0.3cm

\noindent
Compet�ncies gen�riques que es treballen

\begin{tabular}{|l|c|}
\hline
Resoluci� de problemes (CI-1) & X \\ \hline
Capacitat d'an�lisi i s�ntesi (CI-4) & X \\ \hline
Coneixement d'inform�tica relatiu a l'�mbit d'estudis (CI-2) & \\ \hline
Aptitud per a la gesti� de l'informaci� (CI-5) & \\ \hline
Comprom�s �tic (CP-1) & X \\ \hline
Raonament cr�tic (CP-2) & X \\ \hline
Aptitud per al treball en equip (CP-3) & \\ \hline
Aprenentatge aut�nom (CP-9) & \\ \hline
\end{tabular}


\vskip 0.3 cm

\noindent
\textbf{Data: 20/04/2011}

\hrule

\vspace{0.3 cm}

\begin{problema}
En una universitat s'ha observat que el 60\% dels estudiants que es
matriculen ho fan en una carrera de Ci\`encies, mentre que l'altre 40\%
ho fan en carreres d'Humanitats. Si un determinat dia es realitzen
20 matr\'{\i}cules, calcular la probabilitat que:
\begin{itemize}
\item [a)] hi hagi igual nombre de matr\'{\i}cules en Ci\`encies i en Humanitats;
\item [b)] el nombre de matr\'{\i}cules en Ci\`encies sigui menor que en Humanitats;
\item [c)] hi hagi almenys 8 matr\'{\i}cules en Ci\`encies;
\item [d)] no hi hagi m\'es de 12 matr\'{\i}cules en Ci\`encies.
\item [e)] Si les cinc primeres matr\'{\i}cules s\'on d'Humanitats, calcular la
probabilitat que en total hi hagi igual nombre de matr\'{\i}cules en Ci\`encies i en Humanitats.
\end{itemize}
\end{problema}

\begin{problema}
L'empresa \textsc{EMPIPATSA} vol comenar a produir bosses de pipes de pes nominal 100g. La normativa vigent exigeix que el pes del producte envasat no
pot ser inferior al 95\% del pes nominal. L'empresa considera que el pes del producte
envasat seguir\`a una llei normal de par\`ametres 98g i desviaci\'o t\'{\i}pica 1g. Es demana:
\begin{itemize}
\item [a)] Demostreu que la probabilitat que una bossa no compleixi la normativa \'es
igual a $0.0013.$
\item [b)] Calculeu la probabilitat que el pes d'una bossa que compleixi la normativa sigui m\'es petit que el pes nominal. 
\item [c)] Calculeu la probabilitat que una caixa de 20 bosses contingui exactament
3 bosses que no compleixen la normativa.
\end{itemize}
\end{problema}

\newpage
\textbf{Variables aleat�ries usuals}
\vskip 0.2 cm

\begin{tabular}{|c|cl|c|c|l|}
V.A. (X) & $f_X(x)$ & & $E(X)$ & $Var(X)$ & Altres propietats \\
\hline
Binomial $B(n, p)$ & $\binom{n}{x} p^x (1-p)^{n-x}$ & si $x\in \Omega_X$ &
 $np$ & $np(1-p)$ & \\
$\Omega_X=\{ 0, 1, \cdots, n \}$ & $0$ & si $x \notin \Omega_X$ & & & \\ \hline
Poisson $Po(\lambda)$ & $\frac{\lambda^x}{x!} e^{-\lambda}$ & si $x\in \Omega_X$ & 
 $\lambda$ & $\lambda$ & $B(n, p) \approx Po(np)$ \\
$\Omega_X=\{ 0, 1, \cdots \}$ & $0$ & si $x \notin \Omega_X$ & & & ($n$ gran, $p$ petit)\\ \hline
Uniforme ${\cal U}(a, b)$ & $\frac{1}{b-a}$ & si $x \in [a, b]$ & 
$\frac{b+a}{2}$ & $\frac{(b-a)^2}{12}$ & 
$F_X(x)=\begin{cases} 
\frac{x-a}{b-a} & x \in [a, b] \\
0 & x < a \\
1 & x > b
\end{cases}$ \\
$\Omega_X=[a, b]$ & 0 & si $x \notin [a, b]$ &  & & \\ \hline
Gaussiana $X(\mu, \sigma^2)$ & & & $\mu$ & $\sigma^2$ & $Z\sim N(0, 1)$ normal est\'andar \\
$\Omega_X=\mathbb{R}$ & & & &  & $F_Z(-z)=1-F_Z(z)$ \\
 & & & &  & $F_X(x)=F_Z(\frac{x-\mu}{\sigma})$ \\ 
 & & & &  & $B(n, p) \approx N(np, np(1-p))$ \\ 
 & & & &  & ($n$ gran) \\
 & & & &  & $Po(\lambda) \approx N(\lambda, \lambda)$ \\ 
 & & & &  & ($\lambda$ gran) \\ \hline
\end{tabular}
\end{document}

