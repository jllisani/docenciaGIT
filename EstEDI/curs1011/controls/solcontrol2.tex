\documentclass[a4paper,10pt]{article}
%\usepackage[active]{srcltx}      	%% necesario para pasar del dvi al tex%
\usepackage[spanish]{babel}
\usepackage[latin1]{inputenc}
%\usepackage{sw20res1}
\usepackage{amsmath}
\usepackage{amsfonts}
\usepackage{amssymb}
\usepackage{graphicx}
\usepackage{enumerate}
\setlength{\topmargin}{-2.5cm}	%%formato de pagina que ocupa todo
\setlength{\textwidth}{18.5cm}
\setlength{\textheight}{28cm}
\setlength{\oddsidemargin}{-1.5cm}

\pagestyle{empty}
\newcounter{prbcont}
\stepcounter{prbcont}
\setcounter{prbcont}{0}
\newtheorem{problema}[prbcont]{Problema}

\begin{document}


\noindent
\textbf{Soluci� Control 2 Aplicacions Estad�stiques. Data: 25/03/2011}

\hrule

\vspace{0.3 cm}

\begin{problema}
\noindent
Una persona treu 5 bolles, amb reposici�, d'una urna que cont�
7 bolles blanques i 3 negres.
\begin{enumerate}[a)]
\item Quina �s la probabilitat de treure 3 blanques?
\item Quina �s la probabilitat de treure 3 blanques i que alguna d'elles surti en la primera o la segona extracci�?
%\item Si ha tret 3 blanques, quina �s la probabilidad que alguna d'elles surti en la primera o la segona extracci�?
\end{enumerate}
\end{problema}


\vskip 0.5 cm
\noindent
\textbf{Soluci�}

\noindent
Hi ha dues maneres de resoldre el problema, en ambd�s casos 
definim els seg�ents successos:

\noindent
B=``treure bolla blanca'' 
\newline
N=``treure bolla negra''
\newline
A=``treure 3 blanques''
\newline
C=``treure bolla blanca en primera o segona extracci�''

\vskip 0.5 cm
\noindent
\textbf{M�tode 1}

\noindent
Com les extraccions s�n amb reposici�, en cada extracci� tenim que: 
$P(B)=CF/CP=7/10$ i $P(N)=CF/CP=3/10$.

\noindent
a)

\begin{eqnarray*}
P(A) & = & P( BBBNN \cup BBNBN \cup BNBBN \cup NBBBN \cup BBNNB \cup \\ 
     &   &   BNBNB \cup NBBNB \cup BNNBB \cup NBNBB \cup NNBBB) = \\
     & = & (\text{disjunts})= \\
     & = & P( BBBNN ) + P( BBNBN ) + P( BNBBN ) + P(NBBBN) + P( BBNNB ) + \\ 
     &   & + P( BNBNB ) + P( NBBNB ) + P(BNNBB) + P( NBNBB ) + P(NNBBB) = \\
     & = & (\text{en cada sumand, successos independents}) = \\
     & = & P(B)P(B)P(B)P(N)P(N) + P(B)P(B)P(N)P(B)P(N) + \cdots + P(N)P(N)P(B)P(B)P(B) =\\
     & = & 10 \cdot (7/10)^3 \cdot (3/10)^2 = 0.3087
\end{eqnarray*}

\noindent
b)

\begin{eqnarray*}
P(A) & = & P( BBBNN \cup BBNBN \cup BNBBN \cup NBBBN \cup BBNNB \cup \\ 
     &   &   BNBNB \cup NBBNB \cup BNNBB \cup NBNBB) = \\
     & = & (\text{disjunts})= \\
     & = & P( BBBNN ) + P( BBNBN ) + P( BNBBN ) + P(NBBBN) + P( BBNNB ) + \\ 
     &   & + P( BNBNB ) + P( NBBNB ) + P(BNNBB) + P( NBNBB ) = \\
     & = & (\text{en cada sumand, successos independents}) = \\
     & = & P(B)P(B)P(B)P(N)P(N) + P(B)P(B)P(N)P(B)P(N) + \cdots + P(N)P(B)P(N)P(B)P(B) =\\
     & = & 9 \cdot (7/10)^3 \cdot (3/10)^2 = 0.27783
\end{eqnarray*}

\vskip 0.5 cm
\noindent
\textbf{M�tode 2}

\vskip 0.3cm
\noindent
CP=$VR_{10}^5 = 10^5$

\noindent
a)

\noindent
\begin{eqnarray*}
CF_A & = & \{\text{maneres de col.locar les 3 blanques}\} \cdot \\
     &   & \{\text{maneres de col.locar les 2 negres}\} \cdot \\
     &   & \{\text{maneres de col.locar les negres entre les blanques}\} =\\
     & = & VR_7^3 \cdot VR_3^2 \cdot PR_5^{32} = 7^3 \cdot 3^2 \cdot 10
\end{eqnarray*}

\noindent
$P(A)=\frac{CF_A}{CP}=\frac{7^3 \cdot 3^2 \cdot 10}{10^5}=0.3087$

\vskip 0.3 cm
\noindent
b)

\noindent
Definim

\noindent
D=``no treure blanca ni en primera ni en segona extraccions''=``treure NNBBB''

\noindent
$CF_{A \cap C}=CF_A - CF_D = 7^3 \cdot 3^2 \cdot 10 - 7^3 \cdot 3^2 = 7^3 \cdot 3^2 \cdot 9 $

\noindent
$P(A\cap C)=\frac{CF_{A\cap C}}{CP}=\frac{7^3 \cdot 3^2 \cdot 9}{10^5}=0.27783$

\newpage
\begin{problema}
\noindent
En Toni, en Pep i na Maria es reuneixen per resoldre problemes d'estad�stica.
Toni resol el $40\%$ del total dels problemes, Pep el $30\%$ i Maria el $30\%$ restant.
Toni s'equivoca en un $2\%$ dels problemes que resol, Pep en el $6\%$ y Maria en l'$1\%$. 
Agafam a l'atzar un dels problemes resolts.
\begin{enumerate}[a)]
\item Quina �s la probabilitat que estigui ben resolt?
\item Si el problema est� mal resolt, quina �s la probabilitat
que l'hagi resolt en Pep?
\item Quina �s la probabilitat que estigui ben resolt i que l'hagi resolt en Toni?
\end{enumerate}
\end{problema}


\vskip 0.5 cm
\noindent
\textbf{Soluci�}

\noindent
Definim els seg�ents successos:

\noindent
A=``problema fet per en Toni'' 
\newline
B=``problema fet per en Pep''
\newline
C=``problema fet per na Maria''
\newline
D=``problema mal resolt''

\vskip 0.3 cm
\noindent
De l'enunciat tenim:

\begin{eqnarray*}
P(A) & = & 0.4 \\
P(B) & = & 0.3 \\
P(C) & = & 0.3 \\
P(D|A) & = & 0.02 \\
P(D|B) & = & 0.06 \\
P(D|M) & = & 0.01 
\end{eqnarray*}

\vskip 0.3 cm
\noindent
a)

\noindent
L'enunciat ens demana calcular $P(\bar{D})=1-P(D)$.

\noindent
Els successos $A$, $B$ i $C$ formen un \textbf{sistema complet de successos}, ja que:
\begin{enumerate}[i)]
\item $P(A)+P(B)+P(C)=1$ (aix� equival a dir que $A \cup B \cup C=\Omega$)
\item $A \cap B = \emptyset $, $A \cap C = \emptyset $, $B \cap C = \emptyset $ (disjunts 2 a dos)
\end{enumerate}

\noindent
De manera que podem utilitzar la f�rmula de la probabilitat total:
\[
P(D)=P(D|A) \cdot P(A) + P(D|B) \cdot P(B) + P(D|C) \cdot P(C) = 
0.02 \cdot 0.4 + 0.06 \cdot 0.3 + 0.01 \cdot 0.3 = 0.029
\]

\noindent
Finalment: 

\noindent
$P(\bar{D})=1-0.029=0.971$

\vskip 0.3 cm
\noindent
b)

\noindent
Ens demanen:

\noindent
$P(B|D)=(\text{teorema de Bayes})=\frac{P(D|B) \cdot P(B)}{P(D)}=\frac{0.06 \cdot 0.3}{0.029}=0.6207$

\vskip 0.3 cm
\noindent
c)

\noindent
Ens demanen:

\noindent
$P(\bar{D} \cap A)=P(\bar{D} | A) \cdot P(A)$

\noindent
De l'enunciat sabem que $P(D | A)=0.02$, per tant $P(\bar{D} | A)=0.98$.

\noindent
Finalment:

\noindent
$P(\bar{D} \cap A)=0.98 \cdot 0.4 =0.392$

\end{document}

