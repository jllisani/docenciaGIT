\documentclass{article}
\usepackage[catalan]{babel}
\usepackage[latin1]{inputenc}   % Permet usar tots els accents i car�ters llatins de forma directa.
\usepackage{enumerate}
\usepackage{amsfonts, amscd, amsmath, amssymb}
\usepackage{eepic}
\usepackage{graphicx}

%NOTA: si usam eepic hem de compilar a .dvi o .ps (NO PDF)

\setlength{\textwidth}{16cm}
\setlength{\textheight}{24cm}
\setlength{\oddsidemargin}{-0.3cm}
\setlength{\evensidemargin}{0.25cm} \addtolength{\headheight}{\baselineskip}
\addtolength{\topmargin}{-3cm}

\newcommand\Z{\mathbb{Z}}
\newcommand\R{\mathbb{R}}
\newcommand\N{\mathbb{N}}
\newcommand\Q{\mathbb{Q}}
\newcommand\K{\Bbbk}
\newcommand\C{\mathbb{C}}

%\pagestyle{empty}
\begin{document}


\begin{center}
\textbf{\Large Tema 1. An�lisi explorat�ria de dades
\newline
Exercicis proposats}
\end{center}

\vskip 0.3 cm





\section*{Sessi� 2: representaci� de dades estad�stiques}


\noindent
\textbf{Exercici 1} 

En una enquesta sobre inmigraci� s'han obtingut les seg�ents dades sobre la nacionalitat de 1000 persones:

\begin{center}
\begin{tabular}{c|c}
Nacionalitat & Quantitat \\ \hline
Col�mbia & 350 \\
Equador & 250 \\
Per� & 120 \\
Argentina & 100 \\
Romania & 80 \\
Marroc & 70 \\
Senegal & 30
\end{tabular}
\end{center}

\begin{enumerate}[a)]
\item Representau les dades mitjan�ant una taula de freq��ncies. �s possible calcular freq��ncies acumulades?
\item Utilitzau un diagrama de barres per a representar les freq��ncies absolutes.
\item Representau els percentatges amb un diagrama de tarta.
\end{enumerate}



\vskip 0.2 cm
\noindent
\textbf{Exercici 2} 

En una enquesta entre els estudiants de la UIB s'han obtingut les seg�ents dades sobre la seva edat:

\begin{center}
\begin{tabular}{c|c}
Edat & Quantitat \\ \hline
18 & 120 \\
19 & 150 \\
20 & 90 \\
21 & 70 \\
22 & 65 \\
23 & 50 \\
24 & 30 \\
25 & 20 \\
26 & 10 \\
27 & 7 \\
28 & 8 \\
29 & 2 \\
30 & 1 \\
34 & 1 \\
35 & 1 \\
40 & 1
\end{tabular}
\end{center}

\begin{enumerate}[a)]
\item Representau les dades mitjan�ant una taula de freq��ncies. 
\item Repetiu l'apartat anterior per� amb les dades agrupades en els seg�ents intervals: ``Menors de $21$'', 
$[21, 23)$, $[23, 25)$, $[25, 27)$, ``Majors o iguals de $27$''.
\item Representau amb un histograma les freq��ncies relatives de la taula de l'apartat anterior.
\item Obteniu el pol�gon de freq��ncies a partir de l'histograma anterior.
\end{enumerate}


\end{document}

