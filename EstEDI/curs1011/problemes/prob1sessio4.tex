\documentclass{article}
\usepackage[catalan]{babel}
\usepackage[latin1]{inputenc}   % Permet usar tots els accents i car�ters llatins de forma directa.
\usepackage{enumerate}
\usepackage{amsfonts, amscd, amsmath, amssymb}
\usepackage{eepic}
\usepackage{graphicx}

%NOTA: si usam eepic hem de compilar a .dvi o .ps (NO PDF)

\setlength{\textwidth}{16cm}
\setlength{\textheight}{24cm}
\setlength{\oddsidemargin}{-0.3cm}
\setlength{\evensidemargin}{0.25cm} \addtolength{\headheight}{\baselineskip}
\addtolength{\topmargin}{-3cm}

\newcommand\Z{\mathbb{Z}}
\newcommand\R{\mathbb{R}}
\newcommand\N{\mathbb{N}}
\newcommand\Q{\mathbb{Q}}
\newcommand\K{\Bbbk}
\newcommand\C{\mathbb{C}}

%\pagestyle{empty}
\begin{document}


\begin{center}
\textbf{\Large Tema 1. An�lisi explorat�ria de dades
\newline
Exercicis proposats}
\end{center}

\vskip 0.3 cm





\section*{Sessi� 4: mesures de dispersi�}


\noindent
\textbf{Exercici 1} 

En una enquesta sobre inmigraci� s'han obtingut les seg�ents dades sobre nacionalitat:

\begin{center}
\begin{tabular}{c|c}
Nacionalitat & Quantitat \\ \hline
Rusia & 200 \\
Somalia & 10 \\
Portugal & 150 \\
Argelia & 100 \\
Bulgaria & 70 \\
Cuba & 50 \\
Xile & 30
\end{tabular}
\end{center}

Quins estad�stics de dispersi� es poden calcular per a aquesta distribuci�? Calculau-los.



\vskip 0.2 cm
\noindent
\textbf{Exercici 2} 

La seg\"uent taula mostra els preus per persona i nit en hotels i pensions de l'\`area metropolitana d'una ciutat espanyola en euros: 
\[
\begin{tabular}{ccccccccccccc} 
65 & 38 & 54 & 28 & 25 & 32 & 84 & 47 & 45 & 33 \\ 
70 & 37 & 64 & 26 & 40 & 45 & 34 & 47 & 61 & 66 \\ 
43 & 46 & 62 & 56 & 47 & 52 & 28 & 28 & 26 & 32 \\ 
94 & 40 & 57 & 36 & 30 & 54 & 60 & 24 & 24 & 24 \\ 
24 & 25 & 50 & 65 & 35 & 60 & 32 & 32 & 26 & 25 \\ 
33 & 100 
\end{tabular} 
\]

\begin{enumerate}[a)]
\item Calculau el ratio de variaci�, el rang, el rang interquart�lic, la vari�ncia i la desviaci� t�pica per a aquesta distribuci�.
\item Dibuixau el diagrama de capsa indicant (si n'hi ha) quins s�n els valors at�pics i els extrems.
\end{enumerate}


\vskip 0.2 cm
\noindent
\textbf{Exercici 3} 

Per a incentivar els treballadors d'una empresa de missatgeria 
la direcci� de l'empresa ha decidit concedir un suplement salarial a la persona 
que faci els repartiments amb major rapidesa. Els treballadors de l'empresa s'organitzen
en dos torns. En el torn del mat�, degut al tr�nsit, el temps mitj� del repartiment
�s de $30$ minuts, amb una vari�ncia de $100$, mentre que en el torn del capvespre 
la mitjana �s de $20$ minuts amb una vari�ncia de $49$. El missatger m�s velo�
del torn del mat� tarda una mitjana de $25$ minuts en fer els seus repartiments
i el del capvespre $15$ minuts. 
Utilitzau \textit{z-scores} per decidir a quin missatger aumentar el sou.


\end{document}

