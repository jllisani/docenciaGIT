\documentclass[compress,red,mathsans,10pt]{beamer}
\usepackage{beamerthemesplit}
\usepackage{amssymb}
\usepackage{multirow}
%\usetheme{Antibes}
\usepackage{pgf,pgfarrows,pgfnodes}

\setbeamercolor{uppercol}{fg=white,bg=purple}%
\setbeamercolor{lowercol}{fg=black,bg=pink}%


\usecolortheme{lily}
\begin{document}
\title{Aplicacions Estad\'{\i}stiques}
\subtitle{Enginyeria Edificaci\'o 2010/11.}  
\author{Antonio E. Teruel}
\date{}

\frame{\titlepage} 

\frame{\frametitle{Exercici 1}
Identificau la poblaci\'o i la mostra estudiats 
en els seg\"uents casos:
\begin{enumerate}[(a)]
\item <2-> En un estudi sobre el consum de drogues en un institut es fa una enquesta  
al $30\%$ per cent dels alumnes de $2^\text{on}$ de 
batxillerat.
\item <3-> \textbf{Resp.} 
\begin{tabular}{ll}
Poblaci\'o: & Alumnes de l'institut\\
Mostra: & el $30\%$ dels alumnes enquestats
\end{tabular}
\item <4-> En un estudi sobre el consum de drogues entre els joves de
Ciutadella (menors de 35 anys) s'entrevista al $10\%$ dels clients 
dels principals locals de copes.
\item <5-> \textbf{Resp.} 
\begin{tabular}{ll}
Poblaci\'o: & Els joves de Ciutadella (menors de 35 anys) \\
Mostra: & el $10\%$ dels clients dels principals locals de copes
\end{tabular}
\item <6-> En un estudi a nivell nacional sobre la influ\`encia de l'alcohol 
en els accidents de tr\`afic es fan 10.000 controls d'alcohol\'emia
en diferents carreteres del pais.
\item <7-> \textbf{Resp.} 
\begin{tabular}{ll}
Poblaci\'o: & Tots els conductors del pais \\
Mostra: & Els 10.000 conductors controlats
\end{tabular}
\end{enumerate}
}

\frame{\frametitle{Exercici 2}
Identificau al menys tres variables que puguin apar\'eixer
en els seg\"uents estudis estad\'istics:
\begin{enumerate}[(a)]
\item <2->Consum de drogues en una ciutat.
\item <3->\textbf{Resp.} $X=$ sust\`ancia consumida, $Y=$ Frecuencia de consumo (mensual), $Z=$ Edat del consumidor, $U=$ Ingressos del consumidor, etc ...
\item <4->Satisfacci\'o laboral dels empleats d'una empresa.
\item <5->\textbf{Resp.} $X=$ Sexe, $Y=$ Ingressos, $Z=$ Nombre de jefes, $U=$ Nombre de subordinats, etc ...
\item <6->Notes obtengudes pels alumnes d'una assignatura. 
\item <7->\textbf{Resp.} $X=$ Estudis previs, $Y=$ hores de dedicaci\'o, $Z=$ Calificaci\'o del professor, etc ...
\item <8->Seguretat dels edificis d'un municipi.
\item <9->\textbf{Resp.} $X=$ Anys d'antig\"uetat, $Y=$ Valor catastral, $Z=$ Destinaci\'o de l'edifici, etc ...
\end{enumerate}
}

\frame{\frametitle{Exercici 3}
Classificau les seg\"uents variables segons el seu tipus, dimensi\'o i nivell temporal:
\begin{enumerate}[a)]
\item <2->Nombre de persones que han sofert un accident de tr\`afic en els darrers 5 anys.
\item <3->\textbf{Resp.} \\
Tipus= Quantitativa,discreta.\\
Dimensi\'o=Unidimensional\\
Temporalitat=Temporal
\item <4->Nivell professional d'un militar (por exemple: soldat, sergent, etc.).
\item <5->\textbf{Resp.} \\
Tipus=Qualitativa,ordinal\\
Dimensi\'o=Unidimensional\\
Temporalitat=Atemporal\\
\item <6->Nombre de gols aconseguits per un jugador de futbol al llarg de la temporada 2008-09. 
\item <7->\textbf{Resp.}\\
Tipus=Quantitativa,discreta\\
Dimensi\'o=Unidimensional\\
Temporalitat=Temporal
\end{enumerate}
}

\frame{\frametitle{Exercici 3 (cont)}
\begin{enumerate}[(d)]
\item <2->Religi\'o d'una persona (por exemple: cat\'olic, musulm\`a, budista, etc.).
\item <3->\textbf{Resp.}\\
Tipus=Qualitativa,nominal\\
Dimensi\'o=Unidimensional\\
Temporalitat=Atemporal
\item <4->Pes i al\c{c}ada de les participants en una desfilada de moda.
\item <5->\textbf{Resp.}\\
Tipus=Quantitativa,continua\\
Dimensi\'o=Multidimensional\\
Temporalitat=Atemporal
\item <6->Quantitat de diners gastada per una Administraci\'o en obres p\'ubliques durant el darrer any
\item <7->\textbf{Resp.}\\
Tipus=Quantitativa,continua\\
Dimensi\'o=Unidimensional\\
Temporalitat=Temporal 
\end{enumerate}
}
\end{document}