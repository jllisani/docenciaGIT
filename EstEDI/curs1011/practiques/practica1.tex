\documentclass[11pt]{article}

\usepackage[active]{srcltx}
\usepackage[latin1]{inputenc}	%%para utilizar tildes en el texto
\usepackage[catalan]{babel}	%%corta las palabras segun el castellano
\usepackage{epsfig}
\usepackage{enumerate}
\usepackage{url}
\usepackage{hyperref}


\setlength{\topmargin}{-2cm}	%%formato de pagina que ocupa todo
\setlength{\textwidth}{16cm}
\setlength{\textheight}{24cm}
\setlength{\oddsidemargin}{0cm}


\usepackage{amsmath}
\usepackage{amsfonts}
\usepackage{amssymb}

\newcounter{prbcont}
\stepcounter{prbcont}
\setcounter{prbcont}{0}
\newtheorem{problema}[prbcont]{Problema}

\begin{document}

\noindent
{\large \bf Escola Polit�cnica Superior}

\noindent
{\large Grau en Enginyeria d'Edificaci�}

\vskip 0.3cm
\noindent
{\large \bf Assignatura: Aplicacions Estad�stiques}

\hrule

\vskip 0.3cm

\noindent
Tipus d'activitat

\begin{tabular}{|l|c|c|c|c|}
\hline
 & Exercici & Treball / Pr�ctica & Examen & Altres \\
\hline
Puntuable & & X &  & \\ \hline
No Puntuable & & & & \\ \hline
\end{tabular}

\vskip 0.3cm

\noindent
Compet�ncies espec�fiques que es treballen

\begin{tabular}{|l|c|}
\hline
Capacitat per a utilitzar les t�cniques i m�todes probabil�stics i d'an�lisi estad�stica & X \\
\hline
\end{tabular}

\vskip 0.3cm

\noindent
Compet�ncies gen�riques que es treballen

\begin{tabular}{|l|c|}
\hline
Resoluci� de problemes (CI-1) & X \\ \hline
Capacitat d'an�lisi i s�ntesi (CI-4) & X \\ \hline
Coneixement d'inform�tica relatiu a l'�mbit d'estudis (CI-2) & X \\ \hline
Aptitud per a la gesti� de l'informaci� (CI-5) & X\\ \hline
Comprom�s �tic (CP-1) & X \\ \hline
Raonament cr�tic (CP-2) & X \\ \hline
Aptitud per al treball en equip (CP-3) &  \\ \hline
Aprenentatge aut�nom (CP-9) & X\\ \hline
\end{tabular}


\vskip 0.3 cm

\noindent
\textbf{Data entrega: 24/04/2011}

\hrule

\vspace{0.3cm}
\noindent
\textbf{Pr\`actica Tema I}

\vspace{0.3cm}

L'enunciat d'aquesta pr\`actica \'es variable i dep\`en dels valors que us han estat assignats personalment, en particular de l'\textbf{any-inici},
de l'\textbf{any-fi} i de la \textbf{variable 1}. Podeu consultar la vostra assignaci� en el document \textit{Assignaci� Pr�ctiques}
que trobareu a Campus Extens.

En primer lloc, dirigiu-vos a la plana web de l'Institut Balear d'Estad\'{\i}stica 
\begin{center}
\url{http://ibestat.caib.es/ibestat/page?lang=ca}
\end{center}
En \textit{Estad\'{\i}stiques} entrau en l'apartat d'\textit{Economia}, i dins d'aquest, en el de \textit{Construcci� i habitatge}. 
Anau a la secci� \textit{Visats, llic�ncies i certificacions d'obres} i triau \textit{Llic�ncies d'obra}, \textit{Nombre d'edificis per tipus d'obra}.

Accedireu a una plana on es pot seleccionar un per�ode de temps i un tipus de dades a consultar.
El periode de temps va des de gener de 2000 (2000M01) a juliol de 2010 (2010M07). Cada alumne haur� de triar cada un dels mesos (de M01 a M12) 
de tots els anys compresos entre l'\textbf{any-inici} i l'\textbf{any-fi} que t� assignats, ambd�s inclosos. \underline{No s'han de triar els valors totals
per any}.

Quant al tipus de dades a consultar, cada alumne haur� de triar el tipus de dada que t� assignada (\textbf{variable 1}): \textit{Total edificis nova planta}, 
\textit{Total edificis residencials}, \textit{Edificis a rehabilitar} o \textit{Edificis a demolir}.

Una vegada seleccionats un per�ode i un tipus de dada pitjau damunt el bot� \textit{Consulta la selecci�} per obtenir una taula amb les dades.
A partir d'aquestes dades, es demana:

\begin{enumerate}[1-]
\item Calculau el rang i el rang interquart\'{\i}lic. Dibuixau el diagrama de capsa, marcant els valors at\'{\i}pics, si n'hi ha. 
Si no podeu dibuixar el diagrama de capsa amb l'ordinador, al manco heu de donar la seg�ent informaci�: mediana; primer i tercer quartils; l�mits
superior i inferior entre valors t�pics i at�pics; l�mits superior i inferior entre valors at�pics i extrems; llista de valors
at�pics; llista de valors extrems.

\item  Agrupau les dades en un m�xim de 10 subintervals d'igual amplitud (excepte els intervals inicial i final que poden tenir diferent amplitud)
i constru\"{\i}u una taula de freq\"u\`encies per a aquests intervals.
\begin{enumerate}[a)]
\item Representau mitjan\c{c}ant un diagrama de barres la freq\"u\`encia absoluta i mitjan\c{c}ant un diagrama de 
tarta el percentatge.
\item  Calculau la moda, la mitjana, la mediana, el primer i tercer quartils i el percentil
90.
\item  Calculau el ratio de variaci\'o i el rang interquart\'{\i}lic. 
\item  Calculau la vari\`ancia i la desviaci\'o t\'{\i}pica.
\end{enumerate}
\end{enumerate}
\end{document} 
