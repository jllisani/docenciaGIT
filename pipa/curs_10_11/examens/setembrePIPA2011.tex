\newcommand\Z{\mathbb{Z}}
\newcommand\R{\mathbb{R}}
\newcommand\N{\mathbb{N}}
\newcommand\Q{\mathbb{Q}}
\newcommand\K{\Bbbk}
\newcommand\C{\mathbb{C}}

\begin{document}

\begin{center}
\textsc{Examen Probabilitat i Processos Aleatoris.
Telem\`{a}tica\\
setembre 2011}
\end{center}

\vspace{1 cm}



\noindent\textbf{P2.-} Una empresa fa un \textit{backup} de les seves dades a un servidor central
cada dia al final de la jornada laboral. Suposem que hi ha una probabilitat de $0.16$ que es 
produeixi un error en la transmissi� de les dades. Anomenam $X$ a la v.a. que compta el 
nombre d'errors de transmissi� produ�ts durant una setmana laboral (5 dies, suposant 
que cada dia es produeix com a m�xim un error).
\begin{enumerate}[a)]
\item Trobau la funci� de probabilitat de $X$.
\ \hfill{\textbf{ 0.5 pt.}}
\item Anomenam $Y$ al temps setmanal, en minuts, que dedica el telem�tic
de l'empresa a comprovar l'exist�ncia d'errors i corregir-los. Si 
la distribuci� de $Y$ �s 
\[
P(Y=y|X=x)=\begin{cases} 
\frac{1}{4} & \text{si }y=5+30x \\
\frac{1}{2} & \text{si }y=10+30x \\ 
\frac{1}{4} & \text{si }y=15+30x \\ 
0 & \text{altrament} \end{cases}
\]

\noindent
trobau la probabilitat que durant una setmana qualsevol el telem�tic
de l'empresa dediqui m�s de dues hores a revisar i corregir els problemes de \textit{backup}. 
\ \hfill{\textbf{ 1.5 pt.}}
\item Si durant una setmana dedica m�s de dues hores
 a revisar i corregir els problemes de \textit{backup}, quina �s la probabilitat
que s'hagin produ�t 4 errors durant la setmana?.
\ \hfill{\textbf{ 0.5 pt.}}
\end{enumerate}




\vspace{0.75 cm}

\hrule

\vspace{0.5 cm}

\noindent Duraci\'o de l'examen 3 hores.\newline

\end{document}
