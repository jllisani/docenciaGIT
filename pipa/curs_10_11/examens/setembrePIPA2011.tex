\documentclass{report}
\usepackage[catalan]{babel}
\usepackage[latin1]{inputenc}   % Permet usar tots els accents i carï¿œters llatins de forma directa.
\usepackage{enumerate}
\usepackage{amsfonts, amscd, amsmath, amssymb}

\setlength{\textwidth}{16.5cm}
\setlength{\textheight}{27cm}
\setlength{\oddsidemargin}{-0.3cm}
\setlength{\evensidemargin}{0.25cm} \addtolength{\headheight}{\baselineskip}
\addtolength{\topmargin}{-3cm}

\newcommand\Z{\mathbb{Z}}
\newcommand\R{\mathbb{R}}
\newcommand\N{\mathbb{N}}
\newcommand\Q{\mathbb{Q}}
\newcommand\K{\Bbbk}
\newcommand\C{\mathbb{C}}

\begin{document}

\begin{center}
\textsc{Examen Probabilitat i Processos Aleatoris.
Telem\`{a}tica\\
setembre 2011}
\end{center}

\vspace{1 cm}

\noindent\textbf{P1.-}
El sistema de c�pies de seguretat d'una empresa fa una c�pia di�ria
de cada un dels dos discs durs centrals en un moment aleatori entre 
les 3h i les 5h. Les c�pies de cada disc dur es fan de manera independent.
Per un error en la configuraci�, si la difer�ncia
entre els inicis de c�pia de cada disc dur �s inferior a 2 minuts
es produeix un error en el sistema.
\begin{enumerate}[a)]
\item Definiu les variables aleat�ries associades al problema i escriviu les
seves funcions de probabilitat. \ \hfill{\textbf{ 0.5pt.}}
\item Escriviu la funci� de probabilitat conjunta i dibuixau el seu suport. 
\ \hfill{\textbf{ 0.5 pt.}}
\item Quina �s la probabilitat que es produeixi un error en el sistema
un dia qualsevol? \ \hfill{\textbf{ 1.5 pt.}}
\end{enumerate}
\vspace{0.75 cm}



\noindent\textbf{P2.-} Una empresa fa un \textit{backup} de les seves dades a un servidor central
cada dia al final de la jornada laboral. Suposem que hi ha una probabilitat de $0.16$ que es 
produeixi un error en la transmissi� de les dades. Anomenam $X$ a la v.a. que compta el 
nombre d'errors de transmissi� produ�ts durant una setmana laboral (5 dies, suposant 
que cada dia es produeix com a m�xim un error).
\begin{enumerate}[a)]
\item Trobau la funci� de probabilitat de $X$.
\ \hfill{\textbf{ 0.5 pt.}}
\item Anomenam $Y$ al temps setmanal, en minuts, que dedica el telem�tic
de l'empresa a comprovar l'exist�ncia d'errors i corregir-los. Si 
la distribuci� de $Y$ �s 
\[
P(Y=y|X=x)=\begin{cases} 
\frac{1}{4} & \text{si }y=5+30x \\
\frac{1}{2} & \text{si }y=10+30x \\ 
\frac{1}{4} & \text{si }y=15+30x \\ 
0 & \text{altrament} \end{cases}
\]

\noindent
trobau la probabilitat que durant una setmana qualsevol el telem�tic
de l'empresa dediqui m�s de dues hores a revisar i corregir els problemes de \textit{backup}. 
\ \hfill{\textbf{ 1.5 pt.}}
\item Si durant una setmana dedica m�s de dues hores
 a revisar i corregir els problemes de \textit{backup}, quina �s la probabilitat
que s'hagin produ�t 4 errors durant la setmana?.
\ \hfill{\textbf{ 0.5 pt.}}
\end{enumerate}

\vspace{0.75 cm}



\noindent\textbf{P3.-}.
El motor que permet orientar una antena parab\`olica produeix un error en l'orientaci\'o de
$\varepsilon$ graus cada vegada que s'acciona, on $\varepsilon \sim N(0, 1)$. Els errors en
l'orientaci\'o s'acumulen despr\'es de cada actuaci\'o del motor i s\'on independents entre s\'i.
\begin{enumerate}[a)]
\item Quina \'es la probabilitat que l'error d'orientaci\'o sigui superior a $5$ graus despr\'es de
$100$ actuacions del motor?\ \hfill{\textbf{ 1.25 pt.}}
\item Quan l'error acumulat (en valor absolut) \'es superior a $10$ graus l'antena s'ha de recalibrar.
Quin \'es el nombre m\`axim d'actuacions del motor que es poden fer si es vol garantir, amb una
probabilitat del $95\%$, que l'antena no necessita \'esser recalibrada?\ \hfill{\textbf{ 1.25 pt.}}
\end{enumerate}



\vspace{0.75 cm}

\noindent\textbf{P4.-} 
Sigui $X(t)$ un proc�s aleatori definit com $X(t)=t^2+3U$, on $U$ �s una variable
aleat�ria de Poisson amb par�metre $10$.
\begin{enumerate}[a)]
\item 
Calculau la mitjana i l'autocovari�ncia de $X(t)$. �s tracta d'un proc�s
estacionari?
\item Si $Y(t)$ �s un proc�s aleatori Gaussi� estacionari i amb mitjana 4,
calculau la correlaci� creuada de $X$ i $Y$ suposant que $Y(t)$ i $U$ s�n 
independents per a tot $t$.
\end{enumerate}


\ \hfill{\textbf{2.5 pt.}}


\vspace{1 cm}


\hrule

\vspace{0.5 cm}

\noindent Duraci\'o de l'examen 3 hores.\newline

\end{document}
