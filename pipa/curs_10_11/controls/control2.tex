\documentclass{report}
\usepackage[catalan]{babel}
\usepackage[latin1]{inputenc}   % Permet usar tots els accents i car�cters llatins de forma directa.
\usepackage{enumerate}
\usepackage{amsfonts, amscd, amsmath, amssymb}

\setlength{\textwidth}{16.5cm}
\setlength{\textheight}{27cm}
\setlength{\oddsidemargin}{-0.3cm}
\setlength{\evensidemargin}{0.25cm} \addtolength{\headheight}{\baselineskip}
\addtolength{\topmargin}{-3cm}

\newcommand\Z{\mathbb{Z}}
\newcommand\R{\mathbb{R}}
\newcommand\N{\mathbb{N}}
\newcommand\Q{\mathbb{Q}}
\newcommand\K{\Bbbk}
\newcommand\C{\mathbb{C}}

\begin{document}

\begin{center}
\textsc{Control 2 Probabilitat i Processos Aleatoris.
Telem\`{a}tica\\
curs 2010/11}
\end{center}

\vspace{1 cm}

\noindent
\textbf{Problema 1}
Un fabricant de panells solars vol estimar la probabilitat $p$ de fabricar un panell defectu�s.
Per a aix�, pren una mostra de $n$ panells i calcula la mitjana $M_n$ de panells defectuosos de la 
mostra. Quin �s el valor m�nim de $n$ que assegura que la difer�ncia entre el valor estimat $M_n$ i 
el valor real $p$, en valor absolut, ser� inferior a $0.01$ amb probabilitat superior al $98\%$?
Resoleu el problema amb la desigualtat de Txebytxeff i amb el Teorema del L�mit Central i comparau
els resultats.


\end{document}