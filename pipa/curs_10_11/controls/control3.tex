\documentclass{report}
\usepackage[catalan]{babel}
\usepackage[latin1]{inputenc}   % Permet usar tots els accents i car�cters llatins de forma directa.
\usepackage{enumerate}
\usepackage{amsfonts, amscd, amsmath, amssymb}

\setlength{\textwidth}{16.5cm}
\setlength{\textheight}{27cm}
\setlength{\oddsidemargin}{-0.3cm}
\setlength{\evensidemargin}{0.25cm} \addtolength{\headheight}{\baselineskip}
\addtolength{\topmargin}{-3cm}

\newcommand\Z{\mathbb{Z}}
\newcommand\R{\mathbb{R}}
\newcommand\N{\mathbb{N}}
\newcommand\Q{\mathbb{Q}}
\newcommand\K{\Bbbk}
\newcommand\C{\mathbb{C}}

\begin{document}

\begin{center}
\textsc{Control 3 Probabilitat i Processos Aleatoris.
Telem\`{a}tica\\
curs 2010/11}
\end{center}

\vspace{1 cm}

\noindent
\textbf{Problema 1}
Considerau un proc�s aleatori en temps discret definit com $S_n=X_1+X_2+\cdots+X_n$ 
(proc�s suma), on $X_i$ s�n variables i.i.d que prenen valors $0$, $1$ i $2$ amb
probabilitats respectives $\frac{1}{5}$, $\frac{2}{5}$, $\frac{2}{5}$.

\vskip 0.2 cm
\noindent
Calculau:
\begin{enumerate}[a)]
\item $P(S_3 \leq 2)$
\item $E(S_2)$
\item Comparau el resultat anterior amb $E(S_1)$. Podem afirmar que es tracta de un proc�s estacionari?
\item Calculau $C_{S}(1, 2)$
\end{enumerate} 

\end{document}