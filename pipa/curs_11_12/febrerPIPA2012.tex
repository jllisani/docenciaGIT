\documentclass{report}
\usepackage[catalan]{babel}
%\usepackage[latin1]{inputenc}   % Permet usar tots els accents i carï¿œters llatins de forma directa.
\usepackage[utf8]{inputenc}   % Permet usar tots els accents i carï¿œters llatins de forma directa.
\usepackage{enumerate}
\usepackage{amsfonts, amscd, amsmath, amssymb}

\setlength{\textwidth}{16.5cm}
\setlength{\textheight}{27cm}
\setlength{\oddsidemargin}{-0.3cm}
\setlength{\evensidemargin}{0.25cm} \addtolength{\headheight}{\baselineskip}
\addtolength{\topmargin}{-3cm}

\newcommand\Z{\mathbb{Z}}
\newcommand\R{\mathbb{R}}
\newcommand\N{\mathbb{N}}
\newcommand\Q{\mathbb{Q}}
\newcommand\K{\Bbbk}
\newcommand\C{\mathbb{C}}

\begin{document}

\begin{center}
\textsc{Examen Probabilitat i Processos Aleatoris.
Telem\`{a}tica\\
febrer 2012}
\end{center}

\vspace{0.5 cm}
\noindent\textbf{P1.-}
L'emissor d'un sistema de comunicacions emet, de manera equiprobable, quatre
tipus de s\'imbols, cada un d'ells format per un parell de valors:
$(-1/2, -1/2)$, $(-1/2, 1/2)$, $(1/2, -1/2)$, $(1/2, 1/2)$. Degut al renou en el canal de
transmissi\'o el receptor rep el parell $(\hat{U}, \hat{V})=(U+X, V+Y)$ quan
s'emet el s\'imbol $(U, V)$, on $(X, Y)$ \'es un renou additiu amb funci\'o de
densitat conjunta
\[
f_{XY}(x, y)=\begin{cases}
K x^2 e^{-y} & \text{si } -1 \leq x \leq 1 \quad \text{ i } \quad -1 \leq y \leq 1 \\ \\
0 & \text{resta}
\end{cases}
\]

La decisi\'o sobre el s\'imbol rebut es fa seguint els seg\"uents criteris:
\begin{itemize}
\item es decideix que s'havia enviat $(-1/2, -1/2)$ si $\hat{U} < 0$ i $\hat{V} < 0$
\item es decideix que s'havia enviat $(-1/2, 1/2)$ si $\hat{U} < 0$ i $\hat{V} \geq 0$
\item es decideix que s'havia enviat $(1/2, -1/2)$ si $\hat{U} \geq 0$ i $\hat{V} < 0$
\item es decideix que s'havia enviat $(1/2, 1/2)$ si $\hat{U} \geq 0$ i $\hat{V} \geq 0$
\end{itemize}

Responeu les seg\"uents q\"uestions:
\begin{enumerate}[a)]
\item Trobau el valor de la constant $K$.\ \hfill{\textbf{ 0.5 pt.}}
\item Quina \'es la probabilitat de transmetre $(-1/2, -1/2)$ i decidir $(1/2, 1/2)$?\ \hfill{\textbf{ 1 pt.}}
\item Quina \'es la probabilitat de decidir $(1/2, -1/2)$ si s'ha transm\'es $(-1/2, 1/2)$?\ \hfill{\textbf{ 1 pt.}}
\end{enumerate}


\vspace{0.5 cm}


\noindent\textbf{P2.-}
Tenim una urna amb tres bolles blanques i dues negres. Es fan tres extraccions sense reposici\'o.
Sigui $N$ la variable aleat\`oria que compta el nombre de bolles blanques extretes, i sigui
$M$ la variable que compta el nombre de bolles negres extretes {\it abans} de la primera bolla blanca.
Es demana:
\begin{enumerate}[a)]
\item Calculau la funci\'o de probabilitat conjunta de $N$ i $M$.\ \hfill{\textbf{ 0.5 pt.}}
\item Calculau $P(|M-N|\leq 2)$. \ \hfill{\textbf{ 0.5 pt.}}
\item Calculau l'esperan\c{c}a i la vari\`ancia de $N$. \ \hfill{\textbf{ 0.5 pt.}}
\item Calculau l'esperan\c{c}a i la vari\`ancia de $M$. \ \hfill{\textbf{ 0.5 pt.}}
\item Calculau la covari\`ancia i el coeficient de correlaci\'o de $N$ i $M$. \ \hfill{\textbf{ 0.5 pt.}}
\end{enumerate}


\vspace{0.5 cm}


\noindent\textbf{P3.-} Una càmera de televisió utilitzada per les retransmissions de partits de futbol
es desplaça damunt un rail. Cada vegada que s'acciona el motor que desplaça la càmera es produeix 
un error $\delta$ de desplaçament, on $\delta \sim N(0, 1)$ (mm). Els errors en
desplaçament s'acumulen despr\'es de cada actuaci\'o del motor i s\'on independents entre s\'i.
\begin{enumerate}[a)]
\item Quina \'es la probabilitat que l'error de desplaçament sigui superior a $5$mm despr\'es de
$100$ actuacions del motor?\ \hfill{\textbf{ 1.25 pt.}}
\item Quan l'error acumulat (en valor absolut) \'es superior a $10$mm la càmera s'ha de recalibrar.
Quin \'es el nombre m\`axim d'actuacions del motor que es poden fer si es vol garantir, amb una
probabilitat del $95\%$, que la càmera no necessita \'esser recalibrada?\ \hfill{\textbf{ 1.25 pt.}}
\end{enumerate}



\vspace{0.5 cm}

\noindent\textbf{P4.-} Una persona juga al següent joc: llança una moneda i si surt cara es desplaça a
la dreta 1 metre, si surt creu es desplaça a l'esquerra 1 metre. Quina és la probabilitat que després
de 15 llançaments es trobi 5 metres a la dreta de la seva posició inicial?
\noindent
Resoleu el problema en termes de processos aleatoris, justificant el tipus de procés utilitzat.
\noindent
Suposau que la moneda està equilibrada.\ \hfill{\textbf{ 1.5 pt.}}


\vspace{0.5 cm}

\noindent\textbf{P5.-} El renou d'un canal de comunicacions es pot considerar blanc Gaussià amb densitat 
espectral de potència $P_0^2$. Quina és la probabilitat que en una transmissió el renou superi el valor $P_0/2$?
\ \hfill{\textbf{ 1 pt.}}

\vspace{0.5 cm}

\hrule

\vspace{0.25 cm}

\noindent Duraci\'o de l'examen 3 hores.\newline

\end{document}
