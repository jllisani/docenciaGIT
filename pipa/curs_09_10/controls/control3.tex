\documentclass{report}
\usepackage[catalan]{babel}
\usepackage[latin1]{inputenc}   % Permet usar tots els accents i car�cters llatins de forma directa.
\usepackage{enumerate}
\usepackage{amsfonts, amscd, amsmath, amssymb}

\setlength{\textwidth}{16.5cm}
\setlength{\textheight}{27cm}
\setlength{\oddsidemargin}{-0.3cm}
\setlength{\evensidemargin}{0.25cm} \addtolength{\headheight}{\baselineskip}
\addtolength{\topmargin}{-3cm}

\newcommand\Z{\mathbb{Z}}
\newcommand\R{\mathbb{R}}
\newcommand\N{\mathbb{N}}
\newcommand\Q{\mathbb{Q}}
\newcommand\K{\Bbbk}
\newcommand\C{\mathbb{C}}

\begin{document}

\begin{center}
\textsc{Control 3 Probabilitat i Processos Aleatoris.
Telem\`{a}tica\\
curs 2009/10}
\end{center}

\vspace{1 cm}

\noindent
\textbf{Problema 1}
Sigui $S_n$ un proc�s de tipus passejada aleat�ria.
Calculau $P(S_{n+3}=0|_{S_n=1})$.



\end{document}