\documentclass{report}
\usepackage[catalan]{babel}
\usepackage[latin1]{inputenc}   % Permet usar tots els accents i car�cters llatins de forma directa.
\usepackage{enumerate}
\usepackage{amsfonts, amscd, amsmath, amssymb}

\setlength{\textwidth}{16.5cm}
\setlength{\textheight}{27cm}
\setlength{\oddsidemargin}{-0.3cm}
\setlength{\evensidemargin}{0.25cm} \addtolength{\headheight}{\baselineskip}
\addtolength{\topmargin}{-3cm}

\newcommand\Z{\mathbb{Z}}
\newcommand\R{\mathbb{R}}
\newcommand\N{\mathbb{N}}
\newcommand\Q{\mathbb{Q}}
\newcommand\K{\Bbbk}
\newcommand\C{\mathbb{C}}

\begin{document}

\begin{center}
\textsc{Control 2 Probabilitat i Processos Aleatoris.
Telem\`{a}tica\\
curs 2009/10}
\end{center}

\vspace{1 cm}

\noindent
\textbf{Problema 1}
Una tenda de llepolies ha venut en els darrers 10 dies les seg�ents quantitats di�ries de caramels de taronja:

\[
15 \quad 12 \quad 20 \quad 18 \quad 9 \quad 16 \quad 10 \quad 17 \quad 11 \quad 15
\]

\begin{enumerate}[a)]
\item Una manera d'estimar la mitjana i la vari�ncia d'una variable aleat�ria a partir de dades
emp�riques �s utilitzant les seg�ents expressions:
\[
\text{mitjana}=\bar{x}=\frac{x_1+x_2+\cdots+x_n}{n}
\qquad
\text{vari�ncia}=\bar{s}^2=\frac{n}{n-1}\left( \frac{x^2_1+x^2_2+\cdots+x^2_n}{n} - \bar{x}^2 \right)
\]
\noindent
on $x_1, x_2, \cdots , x_n$ s�n $n$ valors coneguts de la variable.

Utilitzau aquestes f�rmules i les dades de l'enunciat per a
calcular la mitjana i la vari�ncia del nombre de caramels de taronja venuts di�riament en la tenda.

\item Quin �s el nombre m�nim de caramels de taronja que ha de comanar a principi de cada mes l'encarregat
de la tenda per tenir una probabilitat superior al $90\%$ de no quedar-se sense exist�ncies? (Nota: 
considerau que un mes t� 25 dies laborables i aplicau el TLC).
\end{enumerate}




\end{document}