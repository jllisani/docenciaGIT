\documentclass{report}
\usepackage[catalan]{babel}
\usepackage[latin1]{inputenc}   % Permet usar tots els accents i car�cters llatins de forma directa.
\usepackage{enumerate}
\usepackage{amsfonts, amscd, amsmath, amssymb}

\setlength{\textwidth}{16.5cm}
\setlength{\textheight}{27cm}
\setlength{\oddsidemargin}{-0.3cm}
\setlength{\evensidemargin}{0.25cm} \addtolength{\headheight}{\baselineskip}
\addtolength{\topmargin}{-3cm}

\newcommand\Z{\mathbb{Z}}
\newcommand\R{\mathbb{R}}
\newcommand\N{\mathbb{N}}
\newcommand\Q{\mathbb{Q}}
\newcommand\K{\Bbbk}
\newcommand\C{\mathbb{C}}

\begin{document}

\begin{center}
\textsc{Control 1 Probabilitat i Processos Aleatoris.
Telem\`{a}tica\\
curs 2009/10}
\end{center}

\vspace{1 cm}

\noindent
\textbf{Problema 1}
Es demana a dues persones que facin una marca en un punt a l'atzar d'una barra de fusta d'1 metre de longitud.
Cada persona fa la seva marca de manera independent de l'altra. Quina �s la probabilitat que la dist�ncia
entre les marques sigui inferior a 10cm?
\ \hfill{\textbf{ 2.5 pt.}}

\vskip 1 cm

\noindent
\textbf{Problema 2}
Siguin $X$ i $Y$ dues v.a. discretes amb funci\'o de
probabilitat conjunta

$$f_{XY}(x,y)= \frac{6}{37} \left(\frac{1}{x} + \frac{1}{y}\right)
\mbox{ si } x=1,2,3;  y=2,3$$

\begin{enumerate}[a)]
\item Calculau la funci\'o de probabilitat de $U=X Y$. \ \hfill{\textbf{ 0.6 pt.}}
\item Calculau la funci\'o de probabilitat de $V=X+Y$. \ \hfill{\textbf{ 0.6 pt.}}
\item Trobau el vector de mitjanes de $(U,V)$. \ \hfill{\textbf{ 0.6 pt.}}
\item Trobau la covari\`ancia de $(U,V)$. \ \hfill{\textbf{ 0.7 pt.}}
\end{enumerate}

\end{document}