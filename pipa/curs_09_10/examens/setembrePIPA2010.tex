\documentclass[11pt]{report}
\usepackage[catalan]{babel}
\usepackage[latin1]{inputenc}   % Permet usar tots els accents i car�ters llatins de forma directa.
\usepackage{enumerate}
\usepackage{amsfonts, amscd, amsmath, amssymb}

\setlength{\textwidth}{16.5cm}
\setlength{\textheight}{27cm}
\setlength{\oddsidemargin}{-0.3cm}
\setlength{\evensidemargin}{0.25cm} \addtolength{\headheight}{\baselineskip}
\addtolength{\topmargin}{-3cm}

\newcommand\Z{\mathbb{Z}}
\newcommand\R{\mathbb{R}}
\newcommand\N{\mathbb{N}}
\newcommand\Q{\mathbb{Q}}
\newcommand\K{\Bbbk}
\newcommand\C{\mathbb{C}}

\begin{document}

\begin{center}
\textsc{Examen Probabilitat i Processos Aleatoris.
Telem\`{a}tica\\
setembre 2010}
\end{center}

\noindent\textbf{P1.-}
La funci\'o de densitat conjunta de dues variables aleat\`ories $X$ i $Y$ \'es:
\[
f_{XY}(x, y)=\begin{cases} k x^2 e^{-y} & \text{si } \; 1 < |x| < 2 \; \text{ i } \; 1 < y < 3 \\
\\
0 & \text{altrament}
\end{cases}
\]
\begin{enumerate}[a)]
\item Trobau el valor de la constant $k$. \ \hfill{\textbf{ 0.5 pt.}}
\item Demostrau que les variables $X$ i $Y$ s\'on independents.  \ \hfill{\textbf{ 1 pt.}}
\item Calculau $P(Y-\frac{X}{2} \geq 2)$.  \ \hfill{\textbf{ 1 pt.}}
\end{enumerate}

\vskip 0.2 cm
\textbf{Indicaci\'o:} $\int x^2 e^{-\frac{x}{2}} dx = -2 e^{-\frac{x}{2}} (x^2+4x+8)$

\vspace{0.75 cm}

\noindent\textbf{P2.-}
Siguin $X$ i $Y$ dues v.a. discretes amb funci\'o de probabilitat conjunta
\[
f_{XY}(x, y)=\alpha \sqrt{2xy}  \qquad \text{si } \; x=0, 1, 2; \; y=1, 4, 9; \qquad
\qquad f_{XY}(x, y)=0 \qquad \text{altrament}
\]
\noindent
Calculau:
\begin{enumerate}[a)]
\item El valor de la constant $\alpha$. \ \hfill{\textbf{ 0.5 pt.}}
\item La funci\'o de probabilitat de $U=\frac{X^2+1}{Y}$. \ \hfill{\textbf{ 0.5 pt.}}
\item La funci\'o de probabilitat de $V=|3Y-2X|$. \ \hfill{\textbf{ 0.5 pt.}}
\item El vector de mitjanes de $(U, V)$. \ \hfill{\textbf{ 0.5 pt.}}
\item La covari\`ancia de $U$ i $V$. \ \hfill{\textbf{ 0.5 pt.}}
\end{enumerate}

\vspace{0.75 cm}



\noindent\textbf{P3.-}.
El motor que permet orientar una antena parab\`olica produeix un error en l'orientaci\'o de
$\varepsilon$ graus cada vegada que s'acciona, on $\varepsilon \sim N(0, 1)$. Els errors en
l'orientaci\'o s'acumulen despr\'es de cada actuaci\'o del motor i s\'on independents entre s\'i.
\begin{enumerate}[a)]
\item Quina \'es la probabilitat que l'error d'orientaci\'o sigui superior a $5$ graus despr\'es de
$100$ actuacions del motor?\ \hfill{\textbf{ 1.25 pt.}}
\item Quan l'error acumulat (en valor absolut) \'es superior a $10$ graus l'antena s'ha de recalibrar.
Quin \'es el nombre m\`axim d'actuacions del motor que es poden fer si es vol garantir, amb una
probabilitat del $95\%$, que l'antena no necessita \'esser recalibrada?\ \hfill{\textbf{ 1.25 pt.}}
\end{enumerate}



\vspace{0.75 cm}

\noindent\textbf{P4.-} Consideram el proc\'es aleatori $Z(t)=t + X$, on 
$X$ \'es una v.a. uniforme en l'interval $[-0.1, 0.1]$.
\begin{enumerate}[a)]
\item Calculau la probabilitat que $Z(t)$ sigui major que 1 per a valors de $t$ entre $0$ i $1$. 
\ \hfill{\textbf{1 pt.}}
\item Calculau la mitjana i l'autocovari\`ancia del proc\'es. Es tracta d'un proc\'es 
estacionari?. \ \hfill{\textbf{0.75pt.}}
\item Consideram el proc\'es $W(t)=t^2 + Y$, on $Y$ \'es una v.a. gaussiana
de mitjana $0$ i desviaci\'o t\'\i pica $0.1$. Si $X$ i $Y$ s\'on v.a. independents,
estan incorrelats els processos?. Justificau la resposta.\ \hfill{\textbf{0.75 pt.}}
\end{enumerate}




\vspace{0.75 cm}

\hrule

\vspace{0.5 cm}

\noindent Duraci\'o de l'examen 4 hores.\newline

\end{document}
