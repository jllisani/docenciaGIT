\documentclass[11pt]{report}
\usepackage[catalan]{babel}
\usepackage[latin1]{inputenc}   % Permet usar tots els accents i car�ters llatins de forma directa.
\usepackage{enumerate}
\usepackage{amsfonts, amscd, amsmath, amssymb}

\setlength{\textwidth}{16.5cm}
\setlength{\textheight}{27cm}
\setlength{\oddsidemargin}{-0.3cm}
\setlength{\evensidemargin}{0.25cm} \addtolength{\headheight}{\baselineskip}
\addtolength{\topmargin}{-3cm}

\newcommand\Z{\mathbb{Z}}
\newcommand\R{\mathbb{R}}
\newcommand\N{\mathbb{N}}
\newcommand\Q{\mathbb{Q}}
\newcommand\K{\Bbbk}
\newcommand\C{\mathbb{C}}

\begin{document}

\begin{center}
\textsc{Examen Probabilitat i Processos Aleatoris.
Telem\`{a}tica\\
febrer 2010}
\end{center}

\vspace{1 cm}
\noindent\textbf{P1.-}
Una diana consisteix en 3 cercles conc\`entrics amb radis
respectius 1 cm, 2 cm i 3 cm. En una competici\'o de tir, la
distribuci\'o dels impactes en la diana i els voltants \'es tal que
les desviacions horitzontal i vertical respecte del centre de la
diana s\'on independents i les dues segueixen una distribuci\'o normal
N(0, 1 cm.). Determinau la proporci\'o d'impactes dins cada anell de
la diana. 


\noindent
(Indicacions: Utilitzau un canvi a coordenades polars per a simplificar els c�lculs: 
$x=r\cos\theta$, $y=r\sin\theta$, $dxdy=rdrd\theta$).
\ \hfill{\textbf{ 2.5 pt.}}


\vspace{1 cm}
\noindent\textbf{P2.-}
La seg�ent taula mostra els valors d'al�ada i pes dels jugadors de l'equip de b�squet NBA 
Los Angeles Lakers (font: www.nba.com).

\begin{center}
\begin{tabular}{cc}
\begin{tabular}{ccc}
Jugador & Al�ada (m) & Pes (Kg) \\ \hline
Artest & $2,01$ & $117,93$ \\ 
Brown & $1,93$ & $95,25$ \\
Bryant & $1,98$ & $92,99$ \\
Bynum & $2,13$ & $129,27$ \\
Farmar & $1,88$ & $81,65$ \\
Fisher & $1,85$ & $95,25$ \\
\end{tabular}
&
\begin{tabular}{ccc}
Jugador & Al�ada (m) & Pes (Kg) \\ \hline
Gasol & $2,13$ & $113,4$ \\ 
Mbenga & $2,13$ & $115,67$ \\
Morrison & $2,03$ & $92,99$ \\
Odom & $2,08$ & $104,33$ \\
Powell & $2,06$ & $108,86$ \\
Vujacic & $2,01$ & $92,99$ \\
Walton & $2,03$ & $106,59$ 
\end{tabular}
\end{tabular}
\end{center}

Calculau la millor estimaci� lineal dels valos de pes a partir dels valors d'al�ada (recta de regressi� 
lineal). 
Si un jugador med�s $2,20cm$, quin seria el seu pes d'acord amb aquesta estimaci�?
Calculau el coeficient de correlaci� entre pes i al�ada.
\ \hfill{\textbf{ 2.5 pt.}}


\vspace{1 cm}




\noindent\textbf{P3.-}.
Un equip electr�nic envia 100 bits d'informaci� a un ordinador cada 10 minuts. La probabilitat que
un bit es transmeti de manera err�nia �s de $0,001$. 
\begin{enumerate}[a)]
\item Quina �s la probabilitat que en un periode d'3h
es rebin incorrectament m�s de 2 bits? 
\item Quin �s el valor esperat de bits erronis rebuts al cap d'1 dia.
\item Quin temps ha de passar, com a m�nim, per tenir una probabilitat superior al $90\%$
d'haver rebut m�s de 5 bits erronis?
\end{enumerate}
\ \hfill{\textbf{ 2.5 pt.}}

\vspace{1 cm}

\noindent\textbf{P4.-} 
Sigui $X(t)$ un proc�s aleatori definit com $X(t)=t^2+3U$, on $U$ �s una variable
aleat�ria de Poisson amb par�metre $10$.
\begin{enumerate}[a)]
\item 
Calculau la mitjana i l'autocovari�ncia de $X(t)$. �s tracta d'un proc�s
estacionari?
\item Si $Y(t)$ �s un proc�s aleatori Gaussi� estacionari i amb mitjana 4,
calculau la correlaci� creuada de $X$ i $Y$ suposant que $Y(t)$ i $U$ s�n 
independents per a tot $t$.
\end{enumerate}


\ \hfill{\textbf{1.25 pt.}}


\vspace{1 cm}

\noindent\textbf{P5.-} 
Sigui $S_n=X_1+X_2+\cdots+X_n$ un proc�s suma on les variables aleat�ries $X_i$
s�n i.i.d. i prenen valors $1$ o $-1$ amb probabilitats respectives $\frac{1}{4}$
i $\frac{3}{4}$.
Calculau $P(S_{n+3}=0|_{S_n=1})$.
\ \hfill{\textbf{1.25 pt.}}



\vspace{0.75 cm}

\hrule

\vspace{0.5 cm}

\noindent Duraci\'o de l'examen 4 hores.\newline

\end{document}
