\documentclass{article}
\usepackage[catalan]{babel}
\usepackage[latin1]{inputenc}   % Permet usar tots els accents i car�cters llatins de forma directa.
\usepackage{enumerate}
\usepackage{amsfonts, amscd, amsmath, amssymb}
\usepackage{fancyheadings}
\usepackage{hyperref}

\setlength{\textwidth}{16cm}
\setlength{\textheight}{25cm}
\setlength{\oddsidemargin}{-0.3cm}
\setlength{\evensidemargin}{0.25cm} \addtolength{\headheight}{\baselineskip}
\addtolength{\topmargin}{-3cm}

\newcommand\Z{\mathbb{Z}}
\newcommand\R{\mathbb{R}}
\newcommand\N{\mathbb{N}}
\newcommand\Q{\mathbb{Q}}
\newcommand\K{\Bbbk}
\newcommand\C{\mathbb{C}}

\newcounter{exctr}
\setcounter{exctr}{10}
\newenvironment{exemple}
{ \stepcounter{exctr} 
\hspace{0.2cm} 
\textit{Exemple  \arabic{exctr}: }
\it
\begin{quotation}
}{\end{quotation}}

\pagestyle{fancy}
\markboth{Tema 2. Suma de variables aleat\`ories}{}
\setcounter{page}{5}
\setlength{\headrulewidth}{0pt}



\begin{document}

\noindent
\textbf{\large Teorema del l\'imit central}
\vskip 0.2cm
Sigui $S_n=X_1+X_2+\cdots+X_n$, on $X_1, X_2, \cdots, X_n$ s\'on
variables aleat\`ories i.i.d., el Teorema del L\'imit Central (TLC)
ens diu que 
\[
\lim_{n \rightarrow \infty} S_n = N(E(S_n), \mathrm{Var}(S_n))
\]

\noindent
\'es a dir, la distribuci\'o de qualsevol suma de variables aleat\`ories
independents i id\`enticament distribu\"\i des segueix una llei normal
amb mitjana i vari\`ancia iguals a les de la v.a. suma.

\vskip 0.2 cm
Si l'esperan\c{c}a i la vari\`ancia de les variables  $X_1, X_2, \cdots, X_n$
s\'on $\mu$ i $\sigma^2$, respectivament, llavors 
$S_n \longrightarrow_{\!\!\!\!\!\!\!\!\!\!\!\! n \rightarrow \infty} N(n \mu, n \sigma^2)$
%$S_n \stackrel{\rightarrow}{n \rightarrow \infty} N(n \mu, n \sigma^2)$

\vskip 0.3 cm
\noindent
Visualitzant el TLC:
\begin{itemize}
\item Llan\c{c}au un dau 20 vegades i calculau la suma dels punts. Repetiu l'experiment
50 vegades i feu una gr\`afica (histograma) del nombre de vegades que surt cada valor.
La gr\`afica mostra la distribuci\'o de $S_{20}$ i tendr\`a la forma d'una campana de Gauss. 
\item \url{http://www.rand.org/methodology/stat/applets/clt.html}
\item \url{http://www.youtube.com/watch?v=XAuMfxWg6eI}
\end{itemize}

\vskip 0.3 cm
\noindent
Conseq\"u\`encies del TLC:
\begin{itemize}
\item \textbf{Aproximaci\'o d'una binomial per una normal}: 
si $n$ \'es gran, $B(n, p) \simeq N(np, np(1-p))$
\item \textbf{Aproximaci\'o d'una Poisson per una normal}: 
si $n$ \'es gran, $\mathrm{Po}(\lambda) \simeq N(\lambda, \lambda)$
\end{itemize}

\vskip 0.2 cm
\noindent
Aproximaci\'o d'una v.a. discreta per una v.a. normal:
\vskip 0.2 cm
Si $X$ \'es un v.a. discreta que es pot aproximar per una v.a. normal $X'$ (per exemple,
quan $X$ \'es binomial o Poisson), llavors es seguiexen els seg\"uents criteris per al 
c\`alcul de valors de probabilitat:
\[
\begin{array}{l}
P(X=k)=P(k-0.5 \leq X' \leq k+0.5) \\ \\
P(a \leq X \leq b)=P(a-0.5 \leq X' \leq b+0.5)
\end{array}
\]


\vskip 0.3 cm
\begin{exemple}
(Exercici 10).
Suposem que el 10\% dels votants estan a favor d'una certa
legislaci\'o. Es fa una enquesta entre la poblaci\'o i s'obt\'e una
freq\"{u}\`encia relativa $f_n(A)$ com una estimaci\'o de la proporci\'o
anterior. Determinau, aplicant el teorema del l\'imit central,
quants de votants s'haurien d'enquestar perqu\`e  la probabilitat
que $f_n(A)$ difereixi de 0.1 menys de 0.02 sigui al menys 0.95
Qu\`e podem dir si no coneixem el valor de
la proporci\'o? 
\end{exemple}


\begin{exemple}
(Exercici 11). 
Es llan\c{c}a a l'aire un dau regular 100 vegades. Aplicau el
teorema del l\'imit central per obtenir una fita de la probabilitat
que el nombre total de punts obtinguts estigui entre 300 i 400.
\end{exemple}

\begin{exemple}
(Exercici 12).
Es sap que, en una poblaci\'o, la talla dels individus mascles
adults \'es una variable aleat\`oria $X$ amb  mitjana $\mu_x = 170$ cm
i desviaci\'o t\'{\i}pica $\sigma_x = 7 $ cm. Es tria una mostra
aleat\`oria de 140 individus. Calculau la probabilitat que la
mitjana mostral $\overline{x}$ difereixi de $\mu_x$ en menys d'1
cm.
\end{exemple}

\begin{exemple}
(Exercici 19).
Un radiofar est\`a alimentat per una bateria amb un temps de vida
\'util $T$ governat per una distribuci\'o exponencial amb una 
esperan\c{c}a d'un mes. Trobau el nombre m\'inim de bateries que
s'han de suministrar al radiofar perqu\`e sigui operatiu al menys
un any amb probabilitat $0.99$.
\end{exemple}

\begin{exemple}
(Exercici 20).
Si obtenim 447 cares en 1000 llan\c{c}aments d'una moneda suposadament
regular, hi ha algun indici per suposar que no ho \'es?
\end{exemple}


\vskip 0.3 cm
\noindent
Problemes proposats: 13, 14, 15, 16, 17, 18, 21, 22


\end{document}