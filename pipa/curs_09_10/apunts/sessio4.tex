\documentclass{article}
\usepackage[catalan]{babel}
\usepackage[latin1]{inputenc}   % Permet usar tots els accents i car�ters llatins de forma directa.
\usepackage{enumerate}
\usepackage{amsfonts, amscd, amsmath, amssymb}
\usepackage{fancyheadings}

\setlength{\textwidth}{16cm}
\setlength{\textheight}{25cm}
\setlength{\oddsidemargin}{-0.3cm}
\setlength{\evensidemargin}{0.25cm} \addtolength{\headheight}{\baselineskip}
\addtolength{\topmargin}{-3cm}

\newcommand\Z{\mathbb{Z}}
\newcommand\R{\mathbb{R}}
\newcommand\N{\mathbb{N}}
\newcommand\Q{\mathbb{Q}}
\newcommand\K{\Bbbk}
\newcommand\C{\mathbb{C}}

\newcounter{exctr}
\setcounter{exctr}{12}
\newenvironment{exemple}
{ \stepcounter{exctr} 
\hspace{0.2cm} 
\textit{Exemple  \arabic{exctr}: }
\it
\begin{quotation}
}{\end{quotation}}

\pagestyle{fancy}
\markboth{Tema 1. Variables aleat\`ories vectorials}{}
\setcounter{page}{5}
\setlength{\headrulewidth}{0pt}

\begin{document}

\noindent
\textbf{\large Transformacions de variables aleat\`ories}

\vskip 0.2 cm
\noindent
Recordatori:
\vskip 0.1 cm
donada una v.a. $X$ i una funci\'o $g:\R \rightarrow \R$, la funci\'o de probabilitat
o de densitat de $Y=g(X)$ es calculava de la seg\"uent manera:
\begin{enumerate}[1)]
\item en el cas discret: $P(Y=y)=P(g(X)=y)=P(A)$, on $A=\{x / g(x)=y\}$.
\item en el cas continu: $F_Y(y)=P(Y \leq y)=P(g(X) \leq y)=P(A)$, on $A=\{x / g(x) \leq y\}$. A continuaci\'o
derivam $F_Y(y)$ per a obtenir $f_Y(y)$. En alguns casos particulars es pot utilitzar una f\`ormula
que relaciona $f_X(x)$ i $f_Y(y)$.
\end{enumerate}

\vskip 0.2 cm
\begin{exemple}
Sigui $X \sim {\cal U}(-1, 1)$ i $Y=X^2$. Trobau $f_Y(y)$.
\end{exemple}

\vskip 1cm
\noindent
En aquest tema:
\vskip 0.1 cm
donades dues variables aleat\`ories $X$ i $Y$ distribu\"\i des conjuntament i una funci\'o $g$,
volem trobar la distribuci\'o de $g(X, Y)$.

\vskip 0.3 cm
\noindent
Podem trobar dos casos:

\vskip 0.5 cm
\noindent
\textbf{Cas 1}: $g:\R^2 \rightarrow \R$, per exemple $Z=g(X, Y)=2X+Y$, llavors:
\begin{itemize}
\item Cas discret: $P(Z=z)=P(g(X, Y)=z)=P(A)=\sum\sum_A P(X=x, Y=y)$, on $A=\{ (x, y) / g(x, y) = z\}$.
\vskip 0.4cm
\item Cas continu: $F_Z(z)=P(Z \leq z)=P(g(X, Y) \leq z)=P(A)=\iint_A f_{XY}(x, y) \, dxdy$, 
on $A=\{ (x, y) / g(x, y) \leq z\}$.
A continuaci\'o derivam $F_Z(z)$ per a obtenir $f_Z(z)$.
\end{itemize}

\vskip 1cm
\begin{exemple}
(Exercici 4). Les variables aleat\`ories $X_1 \mbox{i } X_2$ s\'on independents
i amb densitat com\'u 
\[
f(x) = \begin{cases}1 & \text{si } 0 \leq x \leq 1 \\ 
0 & \text{en cas contrari} \end{cases}
\]
\begin{enumerate}[a)]
\item Determinau la densitat de $Y = X_1 + X_2.$
\item Determinau la densitat de $Z = X_1 - X_2.$
\end{enumerate}
\end{exemple}

\vskip 0.4 cm
\begin{exemple}
Siguin $X$ i $Y$ dues v.a. amb funci\'o de probabilitat conjunta

\begin{center}
\begin{tabular}{c|c|c|c|c|}
$Y \backslash X$ & $0$ & $1$ & $2$ & $3$ \\ \hline
$-1$ & $0$ & $1/12$ & $2/12$ & $2/12$ \\ \hline
$0$ & $1/12$ & $2/12$ & $0$ & $1/12$ \\ \hline
$1$ & $1/12$ & $1/12$ & $1/12$ & $0$ \\ \hline
\end{tabular}
\end{center}

Trobau la funci\'o de probabilitat conjunta de $Z=|X-Y|$.
\end{exemple}

\newpage
\noindent
\textbf{Cas 2}: $g:\R^2 \rightarrow \R^2$, per exemple $(U, V)=g(X, Y)=(g_1(x, y), g_2(x, y))=(2X+Y, X-Y)$, llavors:
\begin{itemize}
\item Cas discret: $P(U=u, V=v)=P(g_1(X, Y)=u, g_2(x, y)=v)=P(A)=\sum\sum_A P(X=x, Y=y)$, on 
$A=\{ (x, y) / g_1(x, y) = u \quad \text{i} \quad g_2(x, y) = v \}$.
\vskip 0.4cm
\item Cas continu: $F_{UV}(u, v)=P(U \leq u, V \leq v)=P(A)=\iint_A f_{XY}(x, y) \, dxdy$, on 
$A=\{ (x, y) / g_1(x, y) \leq u \quad \text{i} \quad g_2(x, y) \leq v \}$.
A continuaci\'o derivam $F_{UV}(u, v)$ per a obtenir $f_{UV}(u, v)$.
\vskip 0.2 cm
Casos especials:
\begin{enumerate}[1)]
\item Si podem trobar dues funcions $h_1$ i $h_2$ tals que $x=h_1(u, v)$ i $y=h_2(u, v)$, llavors:
\[
f_{UV}(u, v)=f_{XY}(h_1(u, v), h_2(u, v)) \cdot |J_{h_1h_2}(u, v)| \qquad \text{on }\quad
J_{h_1h_2}(u, v)=\mathrm{det} \begin{pmatrix} 
\frac{\partial h_1(u, v)}{\partial u} & \frac{\partial h_1(u, v)}{\partial v} \\
\frac{\partial h_2(u, v)}{\partial u} & \frac{\partial h_2(u, v)}{\partial v} 
\end{pmatrix} 
\]
\noindent
per a cada conjunt de valors de $(u, v)$ on les funcions $h_1$ i $h_2$ s\'on \'uniques.
\item Si $g(X, Y)$ t\'e forma matricial: 
$\begin{pmatrix} U \\ V \end{pmatrix} =A \cdot \begin{pmatrix} X \\ Y \end{pmatrix}$ i $\mathrm{det}(A)\neq 0$,
llavors:
\[
f_{UV}(u, v)=\frac{1}{|\mathrm{det}(A)|} f_{XY} \left( A^{-1} \cdot \begin{pmatrix} u \\ v \end{pmatrix} \right)
\]
\end{enumerate}
\end{itemize}

\vskip 1cm
\begin{exemple}
Sigui $(X, Y)$ la v.a. discreta de l'exemple anterior. Calculau la funci\'o de probabilitat
conjunta de $(U, V)=(|X-Y|^2, Y^2)$
\end{exemple}

\vskip 0.4cm
\begin{exemple}
(Exercici 27). Sigui $(X, Y)$ una v.a. cont\'\i nua amb funci\'o de densitat:
\[
f_{XY}(x, y)=\begin{cases} \frac{2-x-y}{8} & -1 \leq x \leq 1 \qquad -1 \leq y \leq 1 \\ 
0 & \text{en altre cas} \end{cases}
\]
Calculau la funci\'o de densitat de $(X^2, Y^2)$.
\end{exemple}


\vskip 0.4 cm
\begin{exemple}
Sigui $(X, Y)$ una v.a. cont\'\i nua amb funci\'o de densitat:
\[
f_{XY}(x, y)=\begin{cases} 1 & 0 \leq x \leq 1 \qquad 0 \leq y \leq 1 \\ 
0 & \text{en altre cas} \end{cases}
\]
Definim una nova variable $(U, V)=(2X+Y, X-Y)$.
Calculau la funci\'o de densitat de $(U, V)$.
\end{exemple}


\vskip 1.5 cm
\noindent
Exercicis proposats: 31, 25, 29, 30

\end{document}