\documentclass{article}
\usepackage[catalan]{babel}
\usepackage[latin1]{inputenc}   % Permet usar tots els accents i car�ters llatins de forma directa.
\usepackage{enumerate}
\usepackage{amsfonts, amscd, amsmath, amssymb}
\usepackage{fancyheadings}

\setlength{\textwidth}{16cm}
\setlength{\textheight}{25cm}
\setlength{\oddsidemargin}{-0.3cm}
\setlength{\evensidemargin}{0.25cm} \addtolength{\headheight}{\baselineskip}
\addtolength{\topmargin}{-3cm}

\newcommand\Z{\mathbb{Z}}
\newcommand\R{\mathbb{R}}
\newcommand\N{\mathbb{N}}
\newcommand\Q{\mathbb{Q}}
\newcommand\K{\Bbbk}
\newcommand\C{\mathbb{C}}

\newcounter{exctr}
\setcounter{exctr}{4}
\newenvironment{exemple}
{ \stepcounter{exctr} 
\hspace{0.2cm} 
\textit{Exemple  \arabic{exctr}: }
\it
\begin{quotation}
}{\end{quotation}}

\pagestyle{fancy}
\markboth{Tema 1. Variables aleat\`ories vectorials}{}
\setcounter{page}{2}
\setlength{\headrulewidth}{0pt}

\begin{document}

\noindent
\textbf{\large Cas continu}

\vskip 0.2 cm
\noindent
Recordatori:
\vskip 0.1 cm
una \textbf{v.a. cont\'\i nua} \'es aquella que pot prendre un nombre infinit de valors. 
Per aquest motiu no es pot definir la funci\'o de probabilitat d'una v.a. cont\'\i nua, 
per\`o s\'\i $ $ la seva funci\'o de distribuci\'o i tamb\'e una \textbf{funci\'o 
de densitat}, que ens informa de com es distribueix la probabilitat entre
els valors de la variable.

\vskip 0.2 cm
\begin{exemple}
Experiment=triar a l'atzar un valor entre 0 i 1, $X$=valor triat
\end{exemple}
  

\vskip 0.3 cm
\noindent
En aquest tema:
\vskip 0.1 cm
Es diu que dues variables aleat\`ories $X$ i $Y$ s\'on \textbf{conjuntament cont\'\i nues}
si les probabilitats dels successos associats a $(X, Y)$ es poden expressar en termes de
la integral (doble) d'una \textbf{funci\'o de densitat de probabilitat} $f_{XY}(x, y)$.

\vskip 0.3 cm
\textbf{Propietats}
\begin{enumerate}
\item 
$
\iint_{\R^2} f_{XY}(x, y) \, \mathrm{dx}\mathrm{dy}=
\int_{-\infty}^{+\infty} \int_{-\infty}^{+\infty} f_{XY}(x, y) \, dxdy = 1
$
\item Funci\'o de densitat marginal de $X$:
$
f_X(x)=\int_{-\infty}^{+\infty} f_{XY}(x, y) \, dy
$
\item Funci\'o de densitat marginal de $Y$:
$
f_Y(y)=\int_{-\infty}^{+\infty} f_{XY}(x, y) \, dx
$
\item Donat un conjunt $A$: 
$
P((X, Y) \in A) = \iint_{A} f_{XY}(x, y) \, dxdy
$
\item Funci\'o de distribuci\'o conjunta:
$
F_{XY}(x, y)=P(X \leq x, Y \leq y)=\int_{-\infty}^{x} \int_{-\infty}^{y} f_{XY}(x, y) \, dxdy
$
\item 
$
f_{XY}(x, y)=\frac{\partial^2 F_{XY}(x, y)}{\partial x \partial y}
$
\item si $X$ i $Y$ s\'on v.a. independents: $f_{XY}(x, y)=f_X(x) \cdot f_Y(y)$
\end{enumerate}


\vskip 0.3 cm
\begin{exemple}
(Exercici 2ab) La funci\'o de densitat conjunta de dues variables aleat\`ories
absolutament cont\'{\i}nues \'es: 
\[
f (x,y) = \begin{cases}k(x+xy) & \text{si } (x,y) \in (0,1)^2\\
0 & \text{en tot altre cas} \end{cases}
\]
\begin{enumerate}[a)]
\item Determinau $k$. 
\item Trobau les funcions de densitat marginals.
\end{enumerate}

\end{exemple}


\vskip 0.2 cm
\begin{exemple}
(Exercici 8abc) Un prove\"{\i}dor de serveis inform\`atics t\'e una quantitat $X$ de
cents d'unitats d'un cert producte al principi de cada mes. Durant
el mes es venen $Y$ cents d'unitats del producte. Suposem que $X$
 i $Y$ tenen una densitat conjunta donada per
\[
f(x,y) = \begin{cases}2/9 & \text{si } 0 < y < x < 3\\
0 & \text{en cas contrari}\end{cases}
\]
\begin{enumerate}[a)]
\item Comprovau que {\it f} \'es una densitat.
\item Determinau $\> F_{X,Y}.$
\item Calculau la probabilitat que a final de mes s'hagi venut com a
m\'{\i}nim la meitat de les unitats que hi havia inicialment.
\end{enumerate}
\end{exemple}

\vskip 0.2 cm
\begin{exemple}
(Exercici 4a) Les variables aleat\`ories $X_1 \mbox{i } X_2$ s\'on independents
i amb densitat com\'u 
\[
f(x) = \begin{cases} 1 & \text{si } 0 \leq x \leq 1\\
0 & \text{en cas contrari}\end{cases}
\]
\begin{enumerate}[\empty]
\item Determinau la densitat de $Y = X_1 + X_2.$
\end{enumerate}
\end{exemple}

\vskip 0.5 cm
\noindent
Exercicis proposats: 4b, 5, 15, 16a, 3

\end{document}