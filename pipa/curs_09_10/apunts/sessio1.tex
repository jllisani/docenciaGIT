\documentclass{article}
\usepackage[catalan]{babel}
\usepackage[latin1]{inputenc}   % Permet usar tots els accents i car�ters llatins de forma directa.
\usepackage{enumerate}
\usepackage{amsfonts, amscd, amsmath, amssymb}

\setlength{\textwidth}{16cm}
\setlength{\textheight}{25cm}
\setlength{\oddsidemargin}{-0.3cm}
\setlength{\evensidemargin}{0.25cm} \addtolength{\headheight}{\baselineskip}
\addtolength{\topmargin}{-3cm}

\newcommand\Z{\mathbb{Z}}
\newcommand\R{\mathbb{R}}
\newcommand\N{\mathbb{N}}
\newcommand\Q{\mathbb{Q}}
\newcommand\K{\Bbbk}
\newcommand\C{\mathbb{C}}

\newcounter{exctr}
\newenvironment{exemple}
{ \stepcounter{exctr} 
\hspace{0.2cm} 
\textit{Exemple  \arabic{exctr}: }
\it
\begin{quotation}
}{\end{quotation}}


\begin{document}

\textbf{\Large Tema 1. Variables aleat\`ories vectorials}

\vskip 0.2 cm
\noindent
Recordatori:
\vskip 0.1 cm
una \textbf{variable aleat\`oria} (v.a.) \'es una funci\'o
que associa un n\'umero a cada un dels successos elementals d'un experiment 
aleatori.

\vskip 0.2 cm
\begin{exemple}
Experiment=llan\c{c}ar 2 monedes, X=nombre de cares
\end{exemple}

\vskip 0.3 cm
\noindent
En aquest tema:
\vskip 0.1 cm
estudiarem les relacions entre 2 o m\'es v.a. associades a un mateix experiment.


\vskip 0.2 cm
\begin{exemple}
Experiment=llan\c{c}ar 2 monedes, X=nombre de cares, Y=nombre de creus
\end{exemple}


\vskip  0.3 cm
\noindent
\textbf{\large Cas discret}

\noindent
Donades dues v.a. discretes $X$ i $Y$ anomenam \textbf{vector aleatori} al parell $(X, Y)$.
El conjunt de valors possibles de $(X, Y)$ es denota $\Omega_{XY}$. 
Definim la \textbf{funci\'o de probabilitat conjunta} de $(X, Y)$ com:
\[
P(X=x, Y=y)=P(X=x \cap Y=y) \qquad \forall (x, y) \in \Omega_{XY}
\]

\vskip 0.1 cm
(Notaci\'o alternativa: $f_{XY}(x, y)=P(X=x, Y=y)$)

\vskip 0.3 cm
\textbf{Propietats}
\begin{enumerate}
\item $\sum\sum_{(x, y) \in \Omega_{XY}} P(X=x, Y=y) = 1$
\item $P(X=x)=\sum_{y \in \Omega_Y} P(X=x, Y=y)$ $\qquad$ (funci\'o de probabilitat marginal de $X$)
\item $P(Y=y)=\sum_{x \in \Omega_X} P(X=x, Y=y)$ $\qquad$ (funci\'o de probabilitat marginal de $Y$)
\item si $X$ i $Y$ s\'on v.a. independents: $P(X=x, Y=y)=P(X=x) \cdot P(Y=y)$
\item Donat un conjunt $A$: $P((X, Y) \in A) = \sum\sum_{(x, y) \in A \cap \Omega_{XY}} P(X=x, Y=y)$
\end{enumerate}

\vskip 0.2 cm
\noindent
\textbf{Funci\'o de distribuci\'o conjunta}: $F_{XY}(x, y)=P(X \leq x, Y \leq y)$


\vskip 0.3 cm
\begin{exemple}
La funci\'o de probabilitat conjunta de dues v.a. discretes $X$ i $Y$ es mostra en la taula seg\"uent:
\begin{center}
\begin{tabular}{c|c|c|c|c|c|c|c|}
Y $\backslash$ X & 1 & 2 & 3 & 4 & 5 & 6 & 7 \\ \hline
1     & k/35 & 1/35 & 1/35 & 1/35 & 0 & 0 & 0 \\ \hline
2     & 1/35 & 2/35 & 2/35 & 1/35 & 1/35 & 0 & 0 \\ \hline
3     & 0 & 1/35 & 2/35 & 2/35 & 2/35 & 1/35 & 0 \\ \hline
4     & 0 & 0 & 1/35 & 2/35 & 2/35 & 1/35 & 1/35 \\ \hline
5     & 0 & 0 & 1/35 & 1/35 & 2/35 & 2/35 & 2/35 \\ \hline
\end{tabular}
\end{center}
Dibuixau el suport de la funci\'o, calculau $k$, les funcions de probabilitat marginals i
$P(3 < X+Y \leq 5)$. S\'on independents les variables $X$ i $Y$?
\end{exemple}


\vskip 0.3 cm
\begin{exemple}
(Exercici 11a) Llan\c{c}am a l'aire un dau equilibrat. Considerem dues variables
aleat\`ories $X$ i $Y$ definides com:
\[
X = \begin{cases}-1 & \text{si el resultat \'es imparell}\\ 
1 & \text{si el resultat \'es parell} \end{cases} \ \ 
Y = \begin{cases}-1 & \text{si el resultat \'es 1, 2 o 3}\\
0 & \text{si el resultat \'es 4}\\ 
1 & \text{si el resultat \'es 5 o 6}\end{cases}
\]
Trobau la llei conjunta i la funci\'o de distribuci\'o de $X$ i $Y$.
\end{exemple}


\vskip 0.6 cm
\noindent
Exercicis proposats: 1, 14a, 17a, 6

\end{document}