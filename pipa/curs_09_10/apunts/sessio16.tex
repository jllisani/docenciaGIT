\documentclass{article}
\usepackage[catalan]{babel}
\usepackage[latin1]{inputenc}   % Permet usar tots els accents i car\`acters llatins de forma directa.
\usepackage{enumerate}
\usepackage{amsfonts, amscd, amsmath, amssymb}
\usepackage{fancyheadings}

\setlength{\textwidth}{16cm}
\setlength{\textheight}{25cm}
\setlength{\oddsidemargin}{-0.3cm}
\setlength{\evensidemargin}{0.25cm} \addtolength{\headheight}{\baselineskip}
\addtolength{\topmargin}{-3cm}

\newcommand\Z{\mathbb{Z}}
\newcommand\R{\mathbb{R}}
\newcommand\N{\mathbb{N}}
\newcommand\Q{\mathbb{Q}}
\newcommand\K{\Bbbk}
\newcommand\C{\mathbb{C}}

\newcounter{exctr}
\setcounter{exctr}{11}
\newenvironment{exemple}
{ \stepcounter{exctr} 
\hspace{0.2cm} 
\textit{Exemple  \arabic{exctr}: }
\it
\begin{quotation}
}{\end{quotation}}

\pagestyle{fancy}
\markboth{Tema 3. Processos aleatoris}{}
\setcounter{page}{6}
\setlength{\headrulewidth}{0pt}



\begin{document}


\textbf{\Large Processos aleatoris estacionaris}

Molts processos tenen la propietat de no canviar el seu comportament aleatori al llarg del temps.
\'Es a dir, una observaci\'o del proc\'es en un interval $(t_0, t_k)$ mostra el mateix tipus de 
comportament aleatori que una observaci\'o en l'interval $(t_0+\nu, t_k+\nu)$. De manera que 
podem dir que, per a aquest tipus de proc\'es, les probabilitats associades als temps 
$t_1, t_2, \dots, t_k$ s\'on les mateixes que les associades als temps $t_1+\nu, t_2+\nu, \dots, t_k+\nu$.

Un proc\'es aleatori $X(t)$ (ja sigui en temps discret o continu) es diu {\bf estacionari} si:
\begin{equation}
\label{eqest}
F_{X(t_1) X(t_2) \cdots X(t_k)}(x_1, x_2, \dots, x_k)=
F_{X(t_1+\nu) X(t_2+\nu), \dots, X(t_k+\nu)}(x_1, x_2, \dots, x_k) \qquad \forall t_1, \dots, t_k, \nu
\end{equation}
\noindent
on $$F_{X(t_1) \cdots X(t_k)}(x_1, x_2, \dots, x_k)=P(X(t_1) \leq x_1, \dots, X(t_k) \leq x_k)$$ i
$$F_{X(t_1+\nu) \cdots X(t_k+\nu)}(x_1, x_2, \dots, x_k)=P(X(t_1+\nu) \leq x_1, \dots, X(t_k+\nu) \leq x_k)$$

\textbf{\large Propietats dels processos estacionaris}

\begin{enumerate}[1.]
\item Com a conseq\"u\`encia de (\ref{eqest}):
\[
F_{X(t_1)}(x_1)=F_{X(t_1+\nu)}(x_1) \qquad \iff \qquad P(X(t_1) \leq x_1)=P(X(t_1+\nu) \leq x_1)
\]
\noindent
llavors:
\begin{enumerate}[a)]
\item $m_X(t)=E(X(t))=m \qquad \forall t \quad$ (l'esperan\c{c}a \'es constant).
\item $Var(X(t))=E((X(t)-m)^2)=\sigma^2 \qquad \forall t \quad$ (la vari\`ancia \'es constant).
\end{enumerate}

\item Com a conseq\"u\`encia de (\ref{eqest}):
\[
F_{X(t_1) X(t_2)}(x_1, x_2)=F_{X(t_1 + \nu) X(t_2 + \nu)} (x_1, x_2) \qquad \forall t_1, t_2, \nu
\]
\noindent
en particular, si $\nu=-t_1$ tenim:
\[
F_{X(t_1) X(t_2)}(x_1, x_2)=F_{X(0) X(t_2-t_1)}(x_1, x_2) \qquad \forall t_1, t_2
\]
\noindent
de manera que:
\begin{enumerate}[a)]
\item $R_X(t_1, t_2)=R_X(t_2-t_1)=R_X(\tau) \qquad \forall t_1, t_2 \quad$ (on $\tau=t_2-t_1$).
\item $C_X(t_1, t_2)=C_X(t_2-t_1)=C_X(\tau) \qquad \forall t_1, t_2 \quad$ (on $\tau=t_2-t_1$).
\end{enumerate}

\end{enumerate}

\vskip 0.2 cm
Un proc\'es aleatori que verifica les propietats 
\begin{enumerate}[i)]
\item $m_X(t)=m \quad$ (constant)
\item $R_X(t_1, t_2)=R_X(t_2-t_1) \qquad$ (o b\'e $C_X(t_1, t_2)=C_X(t_2-t_1)$), $\quad \forall t_1, t_2$
\end{enumerate}
\noindent
es diu {\bf proc\'es estacionari en sentit ampli}.

\vskip 0.2 cm
\noindent
{\bf Propietat}. Si $X(t)$ \'es un proc\'es estacionari, llavors $X(t)$ \'es estacionari en sentit ampli.
La implicaci\'o contr\`aria no \'es certa en general.

\vskip 0.2 cm
\noindent
{\bf Propietat}. Si $X(t)$ \'es un proc\'es gaussi\`a, llavors: $X(t)$ estacionari $\iff$ $X(t)$ estacionari
en sentit ampli.


\vskip 1 cm
\begin{exemple}
(Febrer 2005). Definim el proc\'es aleatori en temps continu $X(t)$ com
\[
X(t)=A \cos \omega t + B \sin \omega t
\]
\noindent
on $A$ i $B$ s\'on v.a. iid amb mitjana zero. Demostrau que $X(t)$ \'es estacionari en sentit ampli.

\noindent
(Nota: $\sin(\alpha+\beta)=\sin\alpha\cos\beta+\cos\alpha\sin\beta$;
$\sin(\alpha-\beta)=\sin\alpha\cos\beta-\cos\alpha\sin\beta$;
$\cos(\alpha+\beta)=\cos\alpha\cos\beta-\sin\alpha\sin\beta$;
$\cos(\alpha-\beta)=\cos\alpha\cos\beta+\sin\alpha\sin\beta$.)
\end{exemple}

\vskip 1 cm
\begin{exemple}
(Setembre 2005). Siguin $X(t)$ i $Y(t)$ dos processos aleatoris independents i estacionaris en sentit ampli.
Es defineix el nou proc\'es $Z(t)=aX(t)+bY(t)$, on $a$ i $b$ s\'on constants.
\'Es $Z(t)$ estacionari en sentit ampli? Justificau la resposta.
\end{exemple}

\vskip 1 cm
\begin{exemple}
(Setembre 2007). Consideram el proc\'es aleatori $Z(t)=t^2 + X$, on 
$X$ \'es una v.a. uniforme en l'interval $[-0.5, 0.5]$.
\begin{enumerate}[a)]
\item Calculau la probabilitat que $Z(t)$ sigui major que 1 per a valors de $t$ positius. 
\item Calculau la mitjana i l'autocovari\`ancia del proc\'es. Es tracta d'un proc\'es 
estacionari?. 
\item Consideram el proc\'es $W(t)=3Y + t$, on $Y$ \'es una v.a. exponencial de par\`ametre $\frac{1}{2}$.
Si $X$ i $Y$ s\'on v.a. independents, estan incorrelats els processos?. 
Justificau la resposta.
\end{enumerate}
\end{exemple}



\vskip 2 cm
\textbf{\Large Mitjanes en temps i mitjanes estad\'\i stiques. Processos erg\`odics.}

L'esperan\c{c}a de la variable aleat\`oria associada a un proc\'es $X(t, \Omega)=X(t)$ en un instant de temps $t_i$
es calcula com (cas continu):
\[
E(X(t_i))=m_X(t_i)=\int_{-\infty}^{+\infty} x \, f_{X(t_i)} (x) \, dx
\]

$m_X(t)$ rep el nom de mitjana estoc\`astica, estad\'\i stica o probabil\'\i stica i no coincideix, en principi,
amb la mitjana temporal de cada una de les realitzacions del proc\'es:
\[
\bar{X}(t, \omega_i)=\langle X(t, \omega_i) \rangle = \lim_{T \rightarrow \infty} \frac{1}{T} \int_{-T/2}^{T/2} 
X(t, \omega_i) \, dt
\]

No obstant, per a un tipus molt especial de processos, aquestes mitjanes s\'on iguals. S\'on els anomenats 
{\bf processos erg\`odics}. Per a un proc\'es erg\`odic es t\'e:
\begin{itemize}
\item $\langle X(t, \omega_i) \rangle = E(X(t)) \qquad \forall \omega_i$
\item $\langle X^2(t, \omega_i) \rangle = E(X^2(t)) \qquad \forall \omega_i$
\item $\langle X(t, \omega_i) X(t-\tau, \omega_i) \rangle = 
\lim_{T \rightarrow \infty} \frac{1}{T} \int_{-T/2}^{T/2} X(t, \omega_i) X(t-\tau, \omega_i) \, dt =
E(X(t) X(t-\tau))=R_X(t, t-\tau)=R_X(\tau) \qquad \forall \omega_i$
\end{itemize}

\vskip 0.2 cm
\noindent
{\bf Propietats}.
\begin{enumerate}[1.]
\item Si un proc\'es \'es erg\`odic, llavors \'es estrictament estacionari. L'implicaci\'o contr\`aria no \'es
certa en general.
\item Si $X(t)$ \'es un proc\'es gaussi\`a, llavors: $X(t)$ ergodic $\iff$ $X(t)$ estrictament estacionari $\iff$ 
$X(t)$ estacionari.
\item En un proc\'es erg\`odic, una realitzaci\'o del proc\'es \'es representativa de tot el proc\'es. Aquesta 
propietat \'es molt \'util ja que simplifica l'estudi d'aquests processos.
\end{enumerate}






\end{document}