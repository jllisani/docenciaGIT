\documentclass[oneside,11pt]{report}
\usepackage{enumerate}
%\input 8bitdefs
\setlength{\textwidth}{17cm} \setlength{\textheight}{24cm}
 \setlength{\oddsidemargin}{-0.3cm}
 \setlength{\evensidemargin}{1cm}
\addtolength{\headheight}{\baselineskip}
\addtolength{\topmargin}{-3cm} \pagestyle{myheadings}
 %\parskip= 1 ex
% % % % % \parindent = 10pt
%%%%%CONTADORES
\newcount\secc %%CONTADOR SECCION
\secc=0
\newcount\sbsecc %%CONTADOR SUBSECCION
\sbsecc=0
\newcount\teor%%%%CONTADOR TEOREMAS
\teor=0
\newcount\prob%%%%CONTADOR PROBLEMAS
\prob=1
\newcount\defi
\defi=0
\newcount\prop
\prop=0
%%%%%%DEFINICIONES DE LAS MACROS
% \def \sp#1{\vskip #1ex} %%SALTO INDENTADO DE #1 lineas
% \def \fsp#1{\vskip #1ex \noindent} %%SALTO NO INDENTADO DE #1 lineas

%%%FORMULAS MATEMATICAS Y SIMBOLOS OPORTUNOS
% \def \va#1{\ #1_{1},\ #1_{2}, \cdots , #1_{n}} %%VARIABLE #1 DE 1 A N
% \def \fun#1#2#3{#1 : #2 \longrightarrow #3} %%FUNCION #1:#2 FLECHA #3
% \def \E{$E\ $}%% PONE E DE MODO MATEMATICO
% \def \suc#1#2{\{ #1_#2 \}}
% \def \su#1#2{#1_#2}
% \def \ep{\varepsilon}
% \def \np{\vskip 0.25 cm}
% \def \ap{\vskip 0.15 cm}
% \def \linf{\lim_{n \rightarrow \infty}}
% \def \limfun#1#2{\lim_{#1 \rightarrow #2}}

\newcommand{\pr}[1]{P( #1 )}

% \def \X{{ X}}
% \def \Y{{\cal Y}}
%\newcommand{\Z}{{\cal Z}}


%%%%CONJUNTOS N, Z, R, Rn,[0,1].vacio
\newcommand{\N}{I\!\!N}
\newcommand{\R}{I\!\!R}
\newcommand{\C}{I\!\!\!\!C}
\newcommand{\Z}{Z\!\!\!Z}
%\def\Zp{\Z^{+}}
\newcommand{\Q}{{\rm{0\!\!\!\!Q}}}
%\def \Rn{\R^{n}}
%\def \int{[0,1]}
%\def \vac{\emptyset}


%%%OTRAS MACROS


% \def \lme#1{\ \> {\hbox{\rm {#1}}}\ \> }%%PONE TEXTO EN MODO MATEMATICO
% \def \lm#1{{\hbox{\rm {#1}}}}%%PONE TEXTO EN MODO MATEMATICO

%%%%FUENTES
% \font\mg = cmr10 scaled \magstep 4 %% FONT NORMAL AUMENTO 4
% \font\g = cmr10 scaled \magstep 2 %% FONT NORMAL AUMENTO 2
% \font\cab=cmr6 %%DEFINICION DE FONT CMR6
% \font\pie=cmr5 %%DEFINICION DE FONT CMR6

%%%% COMEN�A EL DOCUMENT
%\includeonly{probtema1}
\includeonly{prob1bis,prob2,catalanprob3}
\begin{document}
\parindent = 0pt
\pagestyle{empty}

\newcounter{problema}
\newcommand{\prb}{\addtocounter{problema}{1}
\noindent\vskip 2mm {\textbf{\theproblema ) }}}
\newcommand{\sol}[1]{{\textbf{\footnotetext[\theproblema]{Sol.: #1} }}}

\setcounter{problema}{0}

\begin{center}\textsc{Processos Estoc\`astics}\end{center}

\prb Considerem el proc\'es aleatori amb temps discret $X(n)$
definit a continuaci\'o. Es llan\c{c}a una moneda a l'aire i si surt
cara $X(n)=1$, en cas contrari $X(n)=-1$, per a tot $n$.

\begin{enumerate}[a)]
\item Dibuixau alguns camins de mostra del proc\'es.
\item Calculau la funci\'o de probabilitat de $X(n)$.
\item Calculau la funci\'o de probabilitat conjunta  de $X(n)$ i
$X(n+k)$.
\item Calculau $\mu_{X}(n)$ y $C_{X}(n,m)$.
\end{enumerate}


\prb Considerem el proc\'es aleatori amb temps  discret $X(n)$
definit a continuaci\'o. Es llan\c{c}a una moneda a l'aire;  si surt
cara $X(n)=(-1)^n$ i $X(n)=(-1)^{n+1}$ si surt creu, per a tot $n$.

\begin{enumerate}[a)]
\item Dibuixau alguns camins de mostra del proc\'es.
\item Calculau la funci\'o de probabilitat de $X(n)$.
\item Calculau la funci\'o de probabilitat conjunta  de $X(n)$ i
$X(n+k)$.
\item Calculau $\mu_{X}(n)$ y $C_{X}(n,m)$.
\end{enumerate}


% \prb Considerem el proc\'es aleatori amb
%temps discret $X(n)$ definit per $X(n)=s^n$ per a tot $n\geq 0$ on
%$s$ \'es un nombre elegit a l'atzar de l'interval $(0,1)$
%
%\begin{enumerate}[a)]
%\item Dibuixau alguns camins de mostra del proc\'es.
%\item Calculau la funci\'o de distribuci\'o de $X(n)$.
%\item Calculau la funci\'o de distribuci\'o conjunta  de $X(n)$ i
%$X(n+k)$.
%\item Calculau $\mu_{X}(n)$ i $C_{X}(n,m)$.
%\end{enumerate}

\prb Sigui $g(t)$ un pols rectangular a l'interval $(0,1)$, \'es a
dir $g(t)=1$ si $t\in(0,1)$ i zero a la resta de casos. Considerem
el proc\'es aleatori definit per $X(t)=Ag(t)$ on $A=\pm 1$ amb la
mateixa probabilitat.

\begin{enumerate}[a)]
\item Calculau la funci\'o de probabilitat de $X(t)$.
\item Calculau $\mu_{X}(t)$
\item Calculau la funci\'o de probabilitat conjunta  de $X(t)$ i
$X(t+d)$, amb $d > 0$.
\item Calculau  $C_{X}(t,t+d)$ amb $d>0$.
\end{enumerate}

\prb Un proc\'es aleatori est\`a definit per l'equaci\'o $Y(t)=g(t-T)$
on $g(t)$ \'es el pols del problema anterior i $T$ \'es una v.a. amb
distribuci\'o uniforme a l'interval unitat.

\begin{enumerate}[a)]
\item Calculau la funci\'o de probabilitat de $Y(t)$.
\item Trobau la funci\'o $\mu_{Y}(t)$
\item Calculau $C_{Y}(0.5,0.75)$ y $C_Y(1.2, 1.5)$.
\end{enumerate}

\prb Sigui $Y(t)=g(t-T)$ el proc\'es del problema anterior per\`o amb
$T$ una v.a. exponencial de  par\`ametre $\lambda$.

\begin{enumerate}[a)]
\item Calculau la funci\'o de probabilitat de $Y(t)$.
\item Trobau la funci\'o $\mu_{Y}(t)$
\item Calculau $C_{Y}(0.2,1.2)$ y $C_Y(1.5, 1.5)$.
\end{enumerate}


\prb Sigui $Z(t)=A t+B$ on $A$ i $B$ s\'on v.a. independents.

\begin{enumerate}[a)]
\item Calculau la funci\'o de densitat de $Z(t)$.
\item Trobau $\mu_{Z}(t)$ i $C_{Z}(t_{1},t_{2})$.
\end{enumerate}


\prb Trobau una expressi\'o de $E((X(t_{2})-X(t_{1}))^2)$ en termes
de la funci\'o d'autocorrelaci\'o.

\prb ?`Un proc\'es ortogonal  \'es incorrelat? ?`Un proc\'es incorrelat
\'es ortogonal?

\prb Siguin $X(t)$, $Y(t)$ dos processos aleatoris conjuntament
Gaussians. Explicau quina relaci\'o hi ha, en aquest cas, entre les
condicions d'independ\`encia, incorrelaci\'o i ortogonalitat de $X(t)$
i $Y(t)$.

\prb Sigui $X(t)$ un proc\'es  estoc\`astic Gaussi\`a, amb mitjana zero
i funci\'o d'autocovari\`ancia donada per

$$C_{X}(t_{1},t_{2})=\sigma^2 e^{-|t_{1}-t_{2}|}$$

Trobau la funci\'o de densitat conjunta de $X(t)$ i $X(t+s)$.

%\newpage

\prb Sigui $S(n)$ un proc\'es suma binomial.
\begin{enumerate}[a)]
\item Demostrau que $P(S(n)=j,S(n')=i)\not= P(S(n)=j) P(S(n')=i)$.
\item Trobau $P(S(n_{2})=j|S(n_{1})=i)$ amb $n_{2}>n_{1}$.
\item Demostrau que $P(S(n_{2})=j|S(n_{1})=i,S(n_{0})=k)=
P(S(n_{2})=j|S(n_{1})=i)$ on $n_{2}>n_{1}>n_{0}$
\end{enumerate}

\prb Trobau $P(S(n)=0)$ per al proc\'es de la passejada aleat\`oria.

% \prb Consideremos los siguientes proceso de ``\textit{medias m\'oviles}'':
%
% $$Y_{n}=\frac{1}{2} (X_{n}+X_{n-1})\mbox{,  } X_{0}=0$$
%
% $$Z_{n}=\frac{2}{3} X_{n}+\frac{1}{3} X_{n-1} \mbox{,  } X_{0}=0$$
%
% \begin{enumerate}
% \item Lanzar 10 veces una monedad para obtener la realizaci\'on de un
% proceso Bernoulli $X_{n}$. Con los valores obtenidos hallar
% las realizaciones resultantes $Y_{n}$ y de
% $Z_{N}$
% \item Repetir el apartado anterior en el caso que $X_{n}=2 I_{n}-1$
% cuando $\{I_{n}\}$ es una sucesi\'on de v.a. iid $Ber(p)$.
% \item Encontrar la media, la varianza y la autocovarianza de $Y_{n}$
% y de $Z_{n}$ cuando $X_{n}$ es un proceso Bernoulli es decir es una
% succesi\'on de v.a. iid $Ber(p)$).
% \item
%
% \end{enumerate}
%
%\prb Considerem els seg\"{u}ents processos
%``\textit{autorregressius}'':
%
%$$W(n)=2 W(n-1) +X(n)\mbox{,  } W(0)=0$$
%
%$$Z(n)=\frac{1}{2} Z(n-1)+X(n) \mbox{,  } Z(0)=0$$
%
%\begin{enumerate}[a)]
%\item Llan\c{c}ar 10 vegades una moneda per obtenir la realitzaci\'o d'un
%proc\'es Bernoulli $X(n)$. Amb els valors obtinguts trobau les
%realitzacions resultants $W(n)$ i de $Z(n)$. La mitjana de la
%mostra d'aquests processos  aproxima la mitjana de cadascun dels
%processos?
%\item Expressau $W(n)$ i $Z(n)$ en termes de
%$X(n),X(n-1),\ldots,X(1)$, a partir d'aquesta  expressi\'o trobau
%$E(W(n))$ i $E(Z(n))$.
%\item Tenen $W(n)$ i $Z(n)$ increments independents,
%estacionaris?.
%\end{enumerate}

\prb Sigui $M(n)$ un proc\'es discret definit com la seq\"{u}\`encia de
les mitjanes d'una successi\'o de v.a. $X_i$ iid:

$$M(n)=\frac{X_{1}+\cdots +X_{n}}{n}$$
%\begin{enumerate}[a)]
%\item  
Trobau la mitjana,la vari\`ancia i l'autocovari\`ancia de $M(n)$.
%    \item T\'e $M(n)$ increments independents?, i
%    estacionaris?
%\end{enumerate}

%\prb Sigui $X(n)$ una successi\'o de v.a. iid normal est\`andard.
%
%\begin{enumerate}[a)]
%\item Trobau la  funci\'o de densitat de $M(n)$ definida en el
%problema anterior.
%\item Trobau la funci\'o de densitat conjunta de $M(n)$ i $M(n+k)$
%(Indicaci\'o: Utilitzau la propietat dels incremens independents de
%$S(n)$).
%\end{enumerate}

%\prb Suposem que un experiment t\'e tres possibles resultats, diguem
%que s\'on $0,1$ i $2$, que ocorren amb probabilitats $p_{0}$,
%$p_{1}$
% i $p_{2}$ respectivament. Considerem una seq\"{u}\`encia de repeticions
% independents del mateix experiment. Sigui
% $$X_{j}(n)=\left\{\begin{array}{ll}1 & \mbox{ si va surtir }
%  $j$ \mbox{ a l' experiment } n\\
% 0 & \mbox{en qualsevol altre cas }\end{array}\right.$$
%
%per a $j=0,1,2$. Sigui
%$\underline{X}(n)=(X_{0}(n),X_{1}(n),X_{2}(n))$ el vector que ens
%d\'ona  els valors de les tres v.a. Bernoulli en la $n$-\`essima
%repetici\'o de l'experiment. Considerem el seg\"{u}ent proc\'es de
%comptatge de $\underline{X}(n)$:
%
%$$\underline{S}(n)=\underline{X}(n)+\cdots +\underline{X}(1)\mbox{
%amb  } \underline{S}(0)=0$$.
%
%\begin{enumerate}[a)]
%\item Demostrau que $\underline{S}(n)$ t\'e distribuci\'o multinomial.
%\item Demostrau que $\underline{S}(n)$ t\'e increments independents.
%Trobau la funci\'o de probabilitat conjunta de $\underline{S}(n)$ i $\underline{S}(n+k)$
%\item Demostrau que les $S_{j}(n)$ s\'on un proc\'es suma binomial.
%\end{enumerate}

\prb Suposem que una secret\`aria rep cridades  que  arriben
d'acord amb un proc\'es de Poisson amb un ritme de 10 cridades per
hora. Quina \'es la probabilitat  que cap cridada es quedi sense
resposta si la secret\`aria surt de l'oficina els primers 15 i els
darrers 15 minuts d'una hora?

\prb Els clients arriben  a una m\`aquina de refrescs segons un
proc\'es Poisson de mitjana $\lambda$. Suposem que cada vegada que
un client deposita una moneda, la m\`aquina dispensa un refresc amb
probabilitat $p$. Trobau la funci\'o de probabilitat del nombre de
begudes dispensades en un temps $t$. (Nota: s'ha de suposar que la
m\`aquina cont\'e un nombre ilimitat  de refrecs).


\prb Un impuls de renou ocorr en una l\'{\i}nia telef\`onica d'acord amb
un proc\'es Poisson de par\`ametre $\lambda$ per segon.

\begin{enumerate}[a)]
\item Trobau la probabilitat  que no ocorri cap impuls
en  el transcurs d'un missatge de $t$ segons.
\item Suposem que el missatge est\`a codificat i que si s'ha
produ\"{\i}t un impuls podem corregir el missatge. Quina \'es la
probabilitat  que un missatge de $t$ segons estigui lliure
d'errors o es pugui corregir?
\end{enumerate}

\prb Els missatges arriben a un ordinador des de dues l\'{\i}nies
telef\`oniques segons dos processos de Poisson independents i amb
ritmes $\lambda_{1}$ i $\lambda_{2}$ respectivament.
\begin{enumerate}[a)]
    \item Trobau la probabilitat  que un missatge arribi primer per la
    linea $1$.
    \item  Trobau la funci\'o de densitat del temps que tarda un
    missatge en arribar per alguna de les l\'inies.
    \item Trobau la probabilitat de $N(t)$ el nombre total de
    missatges
    que arriben a l'ordinador en un interval de longitud $t$.
    \item  Generalitzau el resultat anterior quan es junten $k$
    l\'{\i}nies telef\`oniques independents Poisson amb par\`ametres
$\lambda_{1},\ldots,\lambda_{k}$ y
$N(t)=N_{1}(t)+\ldots+N_{k}(t)$.
\end{enumerate}


\prb Calculau $P(N(t-d)=j|N(t)=k)$ amb $d>0$ quan $N(t) $ es un
proc\'es de Poisson amb par\`ametre $\lambda$.





\end{document}
