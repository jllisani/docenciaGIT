\documentclass[oneside,11pt]{report}
\usepackage{enumerate}
%\input 8bitdefs
\setlength{\textwidth}{17cm} \setlength{\textheight}{24cm}
 \setlength{\oddsidemargin}{-0.3cm}
 \setlength{\evensidemargin}{1cm}
\addtolength{\headheight}{\baselineskip}
\addtolength{\topmargin}{-3cm} \pagestyle{myheadings}
 %\parskip= 1 ex
% % % % % \parindent = 10pt
%%%%%CONTADORES
\newcount\secc %%CONTADOR SECCION
\secc=0
\newcount\sbsecc %%CONTADOR SUBSECCION
\sbsecc=0
\newcount\teor%%%%CONTADOR TEOREMAS
\teor=0
\newcount\prob%%%%CONTADOR PROBLEMAS
\prob=1
\newcount\defi
\defi=0
\newcount\prop
\prop=0
%%%%%%DEFINICIONES DE LAS MACROS
% \def \sp#1{\vskip #1ex} %%SALTO INDENTADO DE #1 lineas
% \def \fsp#1{\vskip #1ex \noindent} %%SALTO NO INDENTADO DE #1 lineas

%%%FORMULAS MATEMATICAS Y SIMBOLOS OPORTUNOS
% \def \va#1{\ #1_{1},\ #1_{2}, \cdots , #1_{n}} %%VARIABLE #1 DE 1 A N
% \def \fun#1#2#3{#1 : #2 \longrightarrow #3} %%FUNCION #1:#2 FLECHA #3
% \def \E{$E\ $}%% PONE E DE MODO MATEMATICO
% \def \suc#1#2{\{ #1_#2 \}}
% \def \su#1#2{#1_#2}
% \def \ep{\varepsilon}
% \def \np{\vskip 0.25 cm}
% \def \ap{\vskip 0.15 cm}
% \def \linf{\lim_{n \rightarrow \infty}}
% \def \limfun#1#2{\lim_{#1 \rightarrow #2}}

\newcommand{\pr}[1]{P( #1 )}

% \def \X{{ X}}
% \def \Y{{\cal Y}}
%\newcommand{\Z}{{\cal Z}}


%%%%CONJUNTOS N, Z, R, Rn,[0,1].vacio
\newcommand{\N}{I\!\!N}
\newcommand{\R}{I\!\!R}
\newcommand{\C}{I\!\!\!\!C}
\newcommand{\Z}{Z\!\!\!Z}
%\def\Zp{\Z^{+}}
\newcommand{\Q}{{\rm{0\!\!\!\!Q}}}
%\def \Rn{\R^{n}}
%\def \int{[0,1]}
%\def \vac{\emptyset}


%%%OTRAS MACROS


% \def \lme#1{\ \> {\hbox{\rm {#1}}}\ \> }%%PONE TEXTO EN MODO MATEMATICO
% \def \lm#1{{\hbox{\rm {#1}}}}%%PONE TEXTO EN MODO MATEMATICO

%%%%FUENTES
% \font\mg = cmr10 scaled \magstep 4 %% FONT NORMAL AUMENTO 4
% \font\g = cmr10 scaled \magstep 2 %% FONT NORMAL AUMENTO 2
% \font\cab=cmr6 %%DEFINICION DE FONT CMR6
% \font\pie=cmr5 %%DEFINICION DE FONT CMR6

%%%% COMEN�A EL DOCUMENT
%\includeonly{probtema1}
\includeonly{prob1bis,prob2,catalanprob3}
\begin{document}
\parindent = 0pt
\pagestyle{empty}

\newcounter{problema}
\newcommand{\prb}{\addtocounter{problema}{1}
\noindent\vskip 2mm {\textbf{\theproblema ) }}}
\newcommand{\sol}[1]{{\textbf{\footnotetext[\theproblema]{Sol.: #1} }}}

\setcounter{problema}{0}

%%%Teorema central del limite
\begin{center}
   \textsc{Tema 2. Suma de variables aleat\`ories.}
\end{center}

\prb  Sigui $W = X + Y + Z$, on  $X$, $Y$ i $Z$ s\'on variables
aleat\`ories amb mitjana 0 i vari\`ancia 1, i amb $ Cov(X,Y) = 1/4,
Cov(X,Z) = 0, Cov(Y,Z) = -1/4.$

\begin{enumerate}[a)]
\item Trobau l'esperan\c{c}a i la vari\`ancia de $W$. (Sol.:$\mathbf{0; 3}$)
\item Repetiu l'apartat (a) suposant que $X, Y$ i $Z$
estan incorrelacionades. (Sol.:$\mathbf{0; 3}$)
\end{enumerate}

\prb   Siguin $X_1, \ldots, X_n \> $ variables aleat\`ories amb la
mateixa mitjana $\mu$ i covari\`ancies $$Cov(X_i,X_j) =
\cases{\sigma^2 & si $\> i=j$\cr \rho \sigma^2 & si $\>
|i-j|=1$\cr 0 & en otro caso $ $\cr}$$

on $|\rho|<1.$ Determinau la mitjana i la vari\`ancia de $ S_n =
X_1+ \cdots +X_n.$ (Sol.:$\mathbf{n\mu;
n\sigma^2+2\rho\sigma^2(n-1)}$)


\prb \'Idem que  l'anterior per\`o amb $Cov(X_{i},X_{j})=\sigma^2
\rho^{|i-j|}$ per $i,j=1,\ldots,n$ amb $\sigma>0$ i $|\rho|<1$.
(Sol.:$\mathbf{n\mu;
\frac{\sigma^2}{(1-\rho)^2}\left(n-2\rho-n\rho^2+2\rho^{n+1}\right)}$)



%\prb Sigui $\chi_{n}^2=X_{1}^2+\ldots+X_{n}^2$ on
%$X_{1},\ldots,X_{n}$ s\'on n v.a. iid. normals est\`andard.
%
%\begin{enumerate}[a)]
%\item Demostrau que $\chi_{1}^2$ segueix una distribuci\'o 
%chi-quadrat\footnote{Una v.a. $X$ segueix una distribuci\'o chi-quadrat (o ji-quadrat) amb $k$ graus de llibertat
% si la seva funci\'o de 
%densitat \'es $f_X(x)=\frac{x^{(k-2)/2} e^{-x/2}}{2^{k/2} \Gamma(k/2)}$ si $x > 0$, on $k$ \'es un enter positiu.
%La seva funci\'o car\`acter\'\i stica \'es $\Phi_X(\omega)=\left( \frac{1}{1-j2\omega} \right)^{k/2}$.
%Alguns valors t\'\i pics de la funci\'o $\Gamma$ s\'on: $\Gamma(1/2)=\sqrt{\pi}$, $\Gamma(z+1)=z\Gamma(z)$ per $z > 0$, $\Gamma(m+1)=m!$ per $m$ enter.} 
%amb $1$ grau de llibertat.
%\item Demostrau que $\chi_{n}^2$ segueix una distribuci\'o chi-quadrat amb $n$
%graus de llibertat.
%\item  Trobau la distribuci\'o de $T_{n}=\sqrt{\chi_{n}^2}$.
%  %%%%%%%  (Sol.:$\mathbf{f_{t_{n}}(t)= t e^{-\frac{t^2}{2}} \mbox{ si }
%%%%%%%%    t>0}$)
%    \item Demostrau que $T_{2}$ segueix la distribuci\'o Rayleigh.(Sol.:$\mathbf
%{f_{t_{n}}(t)=
%     t e^{-\frac{t^2}{2}} \mbox{ si }
% t>0}$)
%   \end{enumerate}
%
%
%   \prb Siguin $X$ i $Y$ dues v.a. exponencials independents amb par\`ametres 
%   $\alpha$
%   i $\beta$ respectivament. Sigui $Z=X+Y$
%
%   \begin{enumerate}[a)]
%   \item Trobau la densitat de $Z$.
%    (Sol.:$\mathbf{\frac{\alpha\beta}{\alpha-\beta}
%       \left(e^{-\beta z}-e^{-\alpha z}\right)\mbox{ si } z>0}$)
%       \item Trobau la funci\'o caracter\'{\i}stica de $Z$.
%       (Sol.:$\mathbf{\frac{\alpha}{\alpha-j \omega}
%       \frac{\beta}{\beta-j \omega}
%      }$)
%   \item Trobau la densitat de $Z$ a partir de la funci\'o
%    caracter\'{\i}stica.
%       \end{enumerate}
%
%       \prb Siguin $Z=aX+bY$ on $X$ e $Y$ s\'on v.a. cont\'{\i}nues
%       i independents i $a,b\in\R$.
%
%       \begin{enumerate}[a)]
%       \item Calculau la funci\'o caracter\'{\i}stica de $Z$ en funci\'o de
%       les de $X$ i $Y$. (Sol.:$\mathbf{\Phi_{X}(a\omega)\Phi_{Y}(b\omega)}$)
%       \item Donau les expressions del c\`alcul de l'esperan\c{c}a i la
%       vari\`ancia de $Z$ en funci\'o de les derivades de la seva funci\'o
%       caracter\'{\i}stica.
%       \end{enumerate}

       \prb Sigui $S_{k}=X_{1}+\ldots+X_{k}$ on $X_{i}$ s\'on v.a.
       independents i amb  distribuci\'o $B(n_{i},p)$ per a $i=1,\ldots,k$.
       Utilitzau la funci\'o generadora de probabilitats per demostrar
       que $S_{k}$ segueix una  distribuci\'o $B(\sum_{i=1}^k n_{i}, p)$.
       Explicau aquest resultat.

       \prb Demostrau que la suma de $n$ v.a. Poisson independents \'es
       tamb\'e Poisson amb par\`ametre la suma dels par\`ametres.

       \prb Sigui $N$ una v.a. que pren  valors enters positius. Sigui
       $X_{1},\ldots,X_{N}$ una seq\"{u}\`encia de $N$ v.a. iid.
       Considerem $S_{N}=\sum_{k=1}^N X_{k}$.

       \begin{enumerate}[a)]
           \item Calculau $E(S_{N}|N)$. (Sol.:$\mathbf{N E(X)}$).
           \item Calculau $E(S_{N})=E(E(S_{N}|N))$. (Sol.:$\mathbf{E(N) E(X)}$)
           \item Demostrau que  $E(e^{j\omega
           S_{N}}|N)=\Phi_{X_{1}}(\omega)^N$.
          \item Demostrau que $\Phi_{S_{N}}(\omega)=
          G_{N}(\phi_{X_{1}}(\omega))$
          on $G_{N}(z)=E(z^N)$ \'es la funci\'o generadora de
          probabilitats
          de $N$.(ind.: $\Phi_{S_{N}}(\omega)=E(E(e^{j\omega
           S_{N}}|N))$).
       \end{enumerate}

%       \prb El nombre de treballs que arriben a un ordinador en una hora
%       \'es una v.a. geom\`etrica amb par\`ametre $p$, i el temps d'execuci\'o
% de cada treball s\'on v.a. exponencials independents
%       amb mitjana $\frac{1}{\alpha}$. Trobau la funci\'o de densitat de
%       la suma dels temps d'execuci\'o de tots els treballs
%       arribats en una hora.(Sol.:$\mathbf{f_{S_{N}}(x)=p \delta(x)+(1-p) p\alpha       e^{-p\alpha x}}$ on $\delta$ \'es la delta de Dirac.)

       \prb Sigui $X_{1},X_{2},\ldots,X_{k},\ldots$ una seq\"{u}\`encia de v.a.
       iid. que pren valors enters. Sigui $N$ una v.a. que pren  valors enters
       positius. Sigui $S_{N}=\sum_{k=1}^N X_{k}$
       \begin{enumerate}[a)]
           \item Trobau la mitjana i la vari\`ancia de $S_{N}$.
           \item Demostrau que $G_{S_{N}}(z)=E(z^{S_N})=G_{N}(G_{X_{1}}(z))$.
        \end{enumerate}

           \prb Suposem que  el nombre de treballs que arriben a una
           tenda  en una hora segueix una distribuci\'o Poisson amb
           mitjana
           $L$. Cada treball requereix $X_{j}$ segons per completar-se,
           on cada $X_{j}$ \'es independent de les dem\'es i pren  els
           valors ``3'' o ``6'' amb la mateixa probabilitat.
           \begin{enumerate}[a)]
           \item Trobau la mitjana i la vari\`ancia   del temps total
           de treball (W) mesurat en minuts en un per\'{\i}ode d'una hora.
           (Sol.:$\mathbf{E(W)=\frac{9}{2}L, Var(W)=\frac{45}{2}L}$)
           \item Trobau $G_{W}(z)=E(z^{W})$.
           \end{enumerate}
% \prb EL tiempo de vida de un dispositivo es una v.a. que sigue una
% distribuci\'on Rayleigh. Sea $T$ el tiempo transcurrido hasta el primer
% fallo entre $n$ dispositivos independientes. Hallar la funci\'on de
% densidad de $T$. Hallar $E(T)$.
%%%%%%

\prb  Suposem que el nombre d'emissions de part\'{\i}cules per part
d'un objecte radioactiu en $t$ segons \'es una variable aleat\`oria
$N_t$ de Poisson amb mitjana $\lambda t$. Utilizau la desigualtat
de Txebicheff per obtenir una fita de la probabilitat que
$|N_t/t-\lambda|$ superi un valor donat $\varepsilon.$
(Sol.:$\mathbf{{\lambda \over \varepsilon^2 \cdot t}}$)

\prb  Suposem que el 10\% dels votants estan a favor d'una certa
legislaci\'o. Es fa una enquesta entre la poblaci\'o i s'obt\'e una
freq\"{u}\`encia relativa $f_n(A)$ com una estimaci\'o de la proporci\'o
anterior. Determinau, aplicant la desigualtat de Txebicheff,
quants de votants s'haurien d'enquestar perqu\`e  la probabilitat
que $f_n(A)$ difereixi de 0.1 menys de 0.02 sigui al menys 0.95
(Sol.:$\mathbf{4500}$).  Qu\`e podem dir si no coneixem el valor de
la proporci\'o? (Sol.:$\mathbf{12500}$) Repetiu el mateix per\`o
aplicant el teorema  del l\'{\i}mit central. (Sol.:$\mathbf{865}$
\textbf{si sabem que } $\mathbf{p=0.1}$ \textbf{i} $\mathbf{2401}$
\textbf{altrament})

\prb  Es llan\c{c}a a l'aire un dau regular 100 vegades. Aplicau la
desigualtat de Txebicheff per obtenir una fita de la probabilitat
que el nombre total de punts obtinguts estigui entre 300 i 400.
(Sol.:$\mathbf{0.883}$). Quina probabilitat s'obt\'e
 aplicant el teorema  del l\'{\i}mit central?
(Sol.:$\mathbf{0.9964}$)

\prb Es sap que, en una poblaci\'o, la talla dels individus mascles
adults \'es una variable aleat\`oria $X$ amb  mitjana $\mu_x = 170$ cm
i desviaci\'o t\'{\i}pica $\sigma_x = 7 $ cm. Es tria una mostra
aleat\`oria de 140 individus. Calculau la probabilitat que la
mitjana mostral $\overline{x}$ difereixi de $\mu_x$ en menys d'1
cm. (Sol.:$\mathbf{0.909}$)

\prb  Quantes  vegades  hem  de  llan\c{c}ar  un
 dau  sense  biaix per
tenir  com a    m\'{\i}nim  un  95\% de seguretat de que la freq\"{u}\`encia
relativa del ``6'' disti menys de 0.01 de la probabilitat te\`orica
1/6? (Sol.:$\mathbf{5336}$)

\prb Es llan\c{c}a a  l'aire una moneda sense biaix $n$ vegades.
Calculau $n$ de manera que la freq\"{u}\`encia relativa del nombre de
cares difereixi de 1/2 en menys de 0.01 amb probabilitat 0.95.
(Sol.:$\mathbf{9604}$)

\prb El nombre de missatges que arriben a un multiplexor \'es una
variable aleat\`oria de Poisson amb una mitjana de 10 missatges per
segon. Estimau la probabilitat que arribin m\'es de 650 missatges en
un minut. (Ind.: Utilizau el teorema  del l\'{\i}mit central)
(Sol.:$\mathbf{0.0207}$)



%
%%%%%%%%%va la 21
%
\prb  El nombre  d'errors d'impremta per p\`agina d'un llibre
segueix una llei de Poisson, amb un nombre mitj\`a d'errors per
p\`agina igual a 2. En un llibre de 300 p\`agines, quina \'es la
probabilitat que en una o m\'es p\`agines  hi hagi  m\'es de 5 errors?
Calculau-la directament, aproximant per una Poisson, i aproximant
per una normal. (Sol.:\textbf{ 0.9934, 0.9931, 0.9849})

\prb  Un canal de transmissi\'o binari t\'e una probabilitat d'error
igual a 0.15. Estimau la probabilitat que hi hagi  20 o menys
errors en una transmissi\'o de 100 bits. (Sol.: \textbf{0.9382, \'o
0.917 si aproximam per una Poisson})

\prb  S'ha de calcular la suma de 100 n\'umeros reals. Suposem que
els n\'umeros s'han aproximat per l'enter m\'es pr\`oxim de manera que
cada n\'umero t\'e un error  uniformement distribu\"{\i}t en $(-1/2,1/2).$
Utilitzau el teorema  del l\'{\i}mit central per estimar la
probabilitat que  l'error total en la suma dels 100 n\'umeros superi
6. (Sol.:\textbf{ 0.0376})

\prb  Un radiofar est\`a alimentat per  una bateria amb un temps de
vida \'util $T$ governat per una distribuci\'o exponencial amb una
esperan\c{c}a d'un mes. Trobau el nombre m\'{\i}nim de bateries que s'han
de suministrar al radiofar perqu\`e  que sigui operatiu al menys un
any amb probabilidad $0.99$. (Sol.:\textbf{ 24})

\prb  Si obtenim 447 cares en 1000 llan\c{c}aments d'una moneda
suposadament regular,  hi ha algun indici per suposar que no ho
\'es? (Sol.:\textbf{ S\'{\i}})

\prb  En un museu  venen di\`ariament 1000 entrades i la proporci\'o
esperada de visitants estrangers \'es del 35$\%$. Quina \'es la
probabilitat que en una setmana visitin el museu m\'es de 5000
espanyols? (Sol.:\textbf{ Pr\`acticament 0})

\prb  La mitjana de bol\'{\i}grafs que es venen di\`ariament en una
papereria \'es 30, i la desviaci\'o t\'{\i}pica 5. Aquests valors s\'on 20 i
4 per al nombre de quaderns venuts. Es sap, a m\'es, que el
coeficient de correlaci\'o entre les vendes de ambd\'os productes \'es
0.7. Quina \'es la probabilitat que el nombre total dels dos
articles venuts durant un trimestre estigui compr\`es entre 4300 i
4600 unitats? (Sol.: \textbf{0.894})

\prb  El nombre {\it N} d'usuaris que arriben a un sistema durant
un cert per\'{\i}ode \'es una variable aleat\`oria amb llei
$\mathrm{Po}(\lambda)$. Sigui $\> p \in (0,1) \> $ la probabilitat que
un usuari que arriba al sistema rebi servei. Determinau la llei de
la variable aleat\`oria que compta el nombre d'usuaris que reben
servei. (Sol.:$\mathbf{\mathrm{Po}(\lambda p)}$)


%\prb Suposem que instal.lam un dispositiu en $t=0$ i el seu temps
%de vida \'es una v.a. $X_{1}$ , si el dispositiu falla \'es
%immediatament reempla\c{c}at per un altre, el temps de vida del qual
%\'es $X_{2}$ i aix\'{\i} successivament. Sigui $N(t)$ el nombre de
%dispositius que han fallat fins el temps $t$. La freq\"{u}\`encia de
%components reempla\c{c}ats en $t$ ser\'a $\frac{N(t)}{t}$. Si denotam
%per $X_{j}$ el temps de vida del $j$-\`essim dispositiu, llavors  el
%temps transcorregut fins  la fallada del $n$-\`essim dispositiu \'es
%$S_{n}=X_{1}+\ldots+X_{n}$. Suposau que les $X_{j}$ s\'on iid amb
%$0<E(X_{j})=\mu<+\infty$
%
%\begin{enumerate}[a)]
%\item Demostrau que $\frac{S_{N(t)}}{N(t)}\leq \frac{t}{N(t)} <
%\frac{S_{N(t)+1}}{N(t)}$
%\item Demostrau que, amb probabilitat 1 $\lim_{t\to\infty}\frac{S_{N(t)}}{N(t)}=
%\lim_{t\to\infty}\frac{S_{N(t)+1}}{N(t)}=E(X)$ i per tant\newline
% $\lim_{t\to \infty}\frac{N(t)}{t}=\frac{1}{E(X)}$ amb probabilitat 1.
% Aquest \'ultim l\'{\i}mit  rep el nom de
%freq\"{u}\`encia de fallada a ``\emph{llarg termini}''.
%\end{enumerate}
%
%\prb Els clients arriben  a una estaci\'o de servei amb distribuci\'o
%de temps entre arribades iid exponencials $X_{j}$ amb
%$E(X_{j})=\frac{1}{\alpha}$. Calculau la freq\"{u}\`encia d'arribades a
%llarg termini.  (Sol.: $\alpha$).
%
%
%\prb Sigui $U_{j}$ el temps que roman funcionant de forma continua
%un sistema (\'es a dir el temps de  funcionament des de la
%$(j-1)$-\`essima aturada) i sigui $D_{j}$ el temps en qu\`e roman
%aturat, \'es a dir el temps total de reparaci\'o quan s'atura per
%$j$-\`essima vegada. Es defineix un cicle de reparaci\'o com el temps
%des de que comen\c{c}a a funcionar fins que torna a funcionar despr\'es
%d'una fallada, \'es a dir $X_{j}=U_{j}+D_{j}$. Suposau que les
%$U_{j}$ i les $D_{j}$ s\'on v.a. iid:
%
%\begin{enumerate}[a)]
%    \item Calculau el temps esperat de duraci\'o del cicle $j$. (Sol.:
%    $E(U)+E(D)$).
%    \item Calculau la freq\"{u}\`encia  de reparacions a llarg termini. (Sol.:
%    $\frac{1}{E(U)+E(D)}$)
%    \end{enumerate}
%
%    \prb Suposam que tenim un esdeveniment de forma que la distribuci\'o
%    del temps entre l'ocurr\`encia $j-1$ i la $j$ s\'on v.a. $X_{j}$ iid.
%    Sigui $C_{j}$ el cost associat a la $j$-\`essima ocurr\`encia de
%    l'esdeveniment. Si
%    $N(t)$ \'es el nombre d'esdeveniments fins el temps $t$, tindrem que
%    $C(t)=\sum_{j=1}^{N(t)} C_{j}$ \'es el cost acumulat en el temps $t$.
%    Suposau que $(X_{j},C_{j})$ s\'on vectors aleatoris
%    independents i id\`enticament distribu\"{\i}ts.
%%     (esto no significa que $C_{j}$ sea independiente
%%     de $X_{j}$).
%   \begin{enumerate}[a)]
%       \item Demostrau que $\lim_{t\to\infty}
%      \frac{\sum_{j=1}^{N(t)}C_{j}}{N(t)}=E(C)$ amb
%      probabilitat 1.
%\item Demostrau que $\lim_{t\to\infty}\frac{C(t)}{t}=\frac{E(C)}{E(X)}$
%amb probabilitat 1
%\end{enumerate}
%
%
%\prb Considerem  la situaci\'o dels problemes 30 y 31. Sigui
%$$I_{U}(t)=\left\{\begin{array}{ll}1 & \mbox{si el sistema
%funciona a l'instant } t\\ 0 &\mbox{altrament}\end{array}\right.
%$$ La proporci\'o de temps de funcionament fins a l'instant $t$ \'es
%$\frac{1}{t}\int_{0}^{t} I_{U}(t')dt'$.
%
%\begin{enumerate}[a)]
%\item Demostrau que  si $t$ \'es un instant on   el cicle s'acaba
%llavors $\frac{1}{t}\sum_{j=1}^{N(t)}
%U_{j}=\frac{1}{t}\int_{0}^{t} I_{U}(t')dt'$
%% y por lo tanto la proporci\'on de tiempo de funcionamiento a largo
%% plazo es
%% $\lim_{t\to\infty} \frac{1}{t}\sum_{j=1}^{N(t)} U_{j}=
%% \lim_{t\to\infty} \frac{1}{t}\int_{0}^{t} I_{U}(t')dt'$.
%\item  Definim $U(t)=\sum_{j=1}^{N(t)} U_{j}$. Aplicau l'apartat
%a), considerant $U(t)$ com a  un cost, i els resultats del
%problema anterior per demostrar que, amb probabilitat 1, la
%proporci\'o de temps de funcionamient a llarg termini \'es
%$\lim_{t\to\infty}\frac{U(t)}{t}=\frac{E(U)}{E(U)+E(D)}$.
%\item Suposau que el cost associat a cada aturada s\'on variables
%aleat\`ories  $C_{j}$ iid. Trobau el cost mitj\`a a llarg termini.
%(Sol.: $\mathbf{\frac{E(C)}{E(U)+E(D)}}$.)
%\end{enumerate}
%
%
%\prb El temps entre arribades de viatgers a una estaci\'o
%d'autobusos per prendre una determinada l\'{\i}nia s\'on v.a. iid
%exponencials amb mitjana $T$. Suposem que els autobusos surten de
%l'estaci\'o quan tenen m viatgers. Quina \'es la freq\"{u}\`encia de
%sortides d'autobusos a llarg termini? (Sol.:
%$\mathbf{\frac{1}{mT}}$.)
%
%\prb Un canal de comunicaci\'o alterna per\'{\i}odes lliures d'errors amb
%per\'{\i}odes en  qu\`e introdueix errors. Assumint que la duraci\'o
%d'aquests per\'{\i}odes s\'on v.a. independents amb mitjanes $\mu_{1}$ i
%$\mu_{2}$, trobau la proporci\'o de temps a llarg termini en qu\`e el
%canal no produeix errors.
%(Sol.:$\mathbf{\frac{\mu_{1}}{\mu_{1}+\mu_{2}}}$.)
%
%\prb Un empleat treballa una proporci\'o de temps $r_{1}$ quan el
%seu  cap hi \'es present i $r_{2}$ quan no hi \'es. Suposem que la
%seq\"{u}\`encia de la duraci\'o dels per\'{\i}odes de temps en qu\`e el cap hi 
%\'es
%i en les que no hi \'es s\'on v.a. exponencials, independents amb
%mitjanes $\mu_{1}$ i $\mu_{2}$. Trobau la proporci\'o de temps que
%treballa l'empleat a llarg termini.
%(Sol.:$\mathbf{\frac{r_1\mu_{1}+r_2 \mu_2}{\mu_{1}+\mu_{2}}}$.)
%
%
%\prb Els usuaris arriben a un tel\`efon p\'ublic i  l'utilitzen  de
%forma independent durant un temps determinat per a una v.a. $Y$,
%si el tel\`efon est\`a lliure. Si un usuari arriba i el tel\`efon no
%esta lliure, se'n va immediatament. Suposau que el temps entre
%arribades dels clients s\'on v.a. iid exponencials amb par\`ametre $\lambda$.
%\begin{enumerate}[a)]
%    \item Trobau la freq\"{u}\`encia d'utilitzaci\'o del tel\`efon a llarg termini.
%    (Sol.:$\mathbf{r=\frac{\lambda}{1+\lambda E(Y)}}$.)
%    \item Trobau la proporci\'o  a llarg termini  d'usuaris que se'n van
%    sense
%    utilitzar el tel\`efon. (Sol.:$\mathbf{\frac{\lambda-r}{\lambda}}$.)
%    \end{enumerate}
%
%\prb Un programa de compressi\'o de dades codifica cadenes de bits
%de la forma seg\"{u}ent:
%
%\begin{center}
%\begin{tabular}{lll}
%    \multicolumn{1}{c}{Patr\'o} & \multicolumn{1}{c}{Codificaci\'o} &
%    \multicolumn{1}{c}{Probabilitat}\\
%    1 & 100 & 0.1\\
%    01 & 101 & 0.09\\
%    001 & 110 & 0.081\\
%    0001 & 111 & 0.0729\\
%    0000 & 0 & 0.6561
%    \end{tabular}
%\end{center}
%
%Suposem que es produeix una entrada d'un bit cada milisegon.
%\begin{enumerate}[a)]
%    \item Demostrau que l'entrada dels bits \'es independent i que
%    si $X_{t}$ \'es el bit que entra en el milisegon $t$ llavors
%    $X_{t}$ \'es una v.a. Bernoulli.
%    \item Sigui $L_{j}$ la v.a. que ens d\'ona la llarg\`aria d'entrada de la
%    $j$-\`essima paraula codificada. Demostrau que la seq\"{u}\`encia de v.a.
%    $L_{j}$ s\'on iid. A m\'es a m\'es $L_{j}$ \'es el temps en milisegons entre
%    les dues codificacions. Sigui $M(t)$ el nombre de paraules
%    codificades
%    en $t$ milisegons. Calculau la freq\"{u}\`encia de paraules
%    codificades a llarg termini. (Sol.:$0.2908$.)
%    \item Sigui $S_{j}$ el cost de la codificaci\'o de la $j$-\`essima
%    paraula, \'es a dir la llarg\`aria de la paraula codificada.
%    Calculau la proporci\'o entre el nombre de bits de la
%    codificaci\'o i els de la entrada a llarg termini.(Sol.:$49.078\%$.)
%\end{enumerate}

\end{document}
