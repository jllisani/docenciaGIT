\documentclass{article}
\usepackage[catalan]{babel}
\usepackage[latin1]{inputenc}   % Permet usar tots els accents i car�ters llatins de forma directa.
\usepackage{enumerate}
\usepackage{amsfonts, amscd, amsmath, amssymb}
\usepackage{eepic}
\usepackage{graphicx}

%NOTA: si usam eepic hem de compilar a .dvi o .ps (NO PDF)

\setlength{\textwidth}{16cm}
\setlength{\textheight}{25cm}
\setlength{\oddsidemargin}{-0.3cm}
\setlength{\evensidemargin}{0.25cm} \addtolength{\headheight}{\baselineskip}
\addtolength{\topmargin}{-3cm}

\newcommand\Z{\mathbb{Z}}
\newcommand\R{\mathbb{R}}
\newcommand\N{\mathbb{N}}
\newcommand\Q{\mathbb{Q}}
\newcommand\K{\Bbbk}
\newcommand\C{\mathbb{C}}


\begin{document}

\begin{center}
\textbf{Problemes resolts Tema 1}
\end{center}

\vskip 0.5 cm
\noindent
\textbf{Problema 1.}
Es llancen a l'aire dos daus de diferent color, un blanc i
l'altre vermell. Sigui $X$ la variable aleat\`oria ``nombre de punts
obtinguts amb el dau blanc", i $Y$ la variable aleat\`oria ``nombre
m\'es gran dels punts obtinguts amb els dos daus".
\begin{enumerate}[a)]
\item Determinau la llei conjunta.
\item Obteniu les lleis marginals.
\item S\'on independents? 
\end{enumerate}

\vskip 0.3 cm
\noindent
\textbf{Soluci\'o:}

\vskip 0.2 cm
\noindent
\textbf{a)} Tenim que $\Omega_X=\Omega_Y=\{ 1, 2, 3, 4, 5, 6 \}$. Podem definir $Z$ com la
v.a. ``nombre de punts obtinguts amb el dau vermell". Podem suposar que les v.a. $X$ i $Y$ s\'on independents.
La funci\'o de probabilitat conjunta de $(X, Y)$ \'es (considerant daus equilibrats):

\[
\begin{array}{rl}
P(X=x, Y=y)=& 0 \qquad \text{si } x\notin \Omega_X \quad \text{i} \quad y\notin \Omega_Y \\ \\
P(X=1, Y=1)=& P(X=1, Z \leq 1)=P(X=1)\cdot P(Z\leq 1) = \\ 
           =&P(X=1) \cdot P(Z=1)= \frac{1}{6} \cdot \frac{1}{6} =\frac{1}{36} \\ \\
P(X=1, Y=2)=& P(X=1, Z=2)=P(X=1)\cdot P(Z=2) =\frac{1}{6} \cdot \frac{1}{6} =\frac{1}{36} \\ \\
\vdots & \vdots \\
P(X=1, Y=6)=& P(X=1, Z=6)=P(X=1)\cdot P(Z=6) =\frac{1}{6} \cdot \frac{1}{6} =\frac{1}{36} \\ \\
P(X=2, Y=1)=&0 \quad \text{(succ\'es impossible)} \\ \\
P(X=2, Y=2)=& P(X=2, Z \leq 2)=P(X=2)\cdot P(Z\leq 2) = \\ 
           =&P(X=2) \cdot (P(Z=1)+P(Z=2))= \frac{1}{6} \cdot (\frac{1}{6}+\frac{1}{6}) =\frac{2}{36} \\ \\
P(X=2, Y=3)=& P(X=2, Z=3)=P(X=2)\cdot P(Z=3) =\frac{1}{6} \cdot \frac{1}{6} =\frac{1}{36} \\ \\
\vdots & \vdots \\
P(X=3, Y=1)=&0 \quad \text{(succ\'es impossible)} \\ \\
P(X=3, Y=2)=&0 \quad \text{(succ\'es impossible)} \\ \\
P(X=3, Y=3)=& P(X=3, Z \leq 3)=P(X=3)\cdot P(Z\leq 3) = \\ 
           =& P(X=3) \cdot (P(Z=1)+P(Z=2)+P(Z=3))= \frac{1}{6} \cdot (\frac{1}{6}+\frac{1}{6}+\frac{1}{6}) =\frac{3}{36} \\ \\
\vdots & \vdots 
\end{array}
\]

\vskip 0.2 cm
\noindent
Raonant de manera similar per a tots els valors possibles de $(x, y)$ obtenim la seg\"uent taula per 
a la funci\'o de probabilitat conjunta de $(X, Y)$:

\vskip 0.2 cm
\begin{center}
\begin{tabular}{c|c|c|c|c|c|c|}
Y $\backslash$ X & 1 & 2 & 3 & 4 & 5 & 6  \\ \hline
1     & 1/36 &    0 &    0 &    0 &    0 & 0  \\ \hline
2     & 1/36 & 2/36 &    0 &    0 &    0 & 0  \\ \hline
3     & 1/36 & 1/36 & 3/36 &    0 &    0 & 0  \\ \hline
4     & 1/36 & 1/36 & 1/36 & 4/36 &    0 & 0  \\ \hline
5     & 1/36 & 1/36 & 1/36 & 1/36 & 5/36 & 0  \\ \hline
6     & 1/36 & 1/36 & 1/36 & 1/36 & 1/36 & 6/36  \\ \hline
\end{tabular}
\end{center}

\vskip 0.2 cm
\noindent
\'Es f\`acil comprovar que la suma de tots els valors de la taula \'es igual a $1$.

\vskip 0.2 cm
\noindent
\textbf{b)} Tenim que:
\[
\begin{array}{ll}
P(X=x)= & P(X=x, Y=1)+P(X=x, Y=2)+\cdots+P(X=x, Y=6) \\
P(Y=y)= & P(X=1, Y=y)+P(X=2, Y=y)+\cdots+P(X=6, Y=y)
\end{array}
\]
\noindent
Per tant:

\vskip 0.2 cm
\begin{center}
\begin{tabular}{c|c|c|c|c|c|c||c}
Y $\backslash$ X & 1 & 2 & 3 & 4 & 5 & 6 & P(Y=y) \\ \hline
1     & 1/36 &    0 &    0 &    0 &    0 & 0 & 1/36 \\ \hline
2     & 1/36 & 2/36 &    0 &    0 &    0 & 0 & 3/36 \\ \hline
3     & 1/36 & 1/36 & 3/36 &    0 &    0 & 0 & 5/36 \\ \hline
4     & 1/36 & 1/36 & 1/36 & 4/36 &    0 & 0 & 7/36 \\ \hline
5     & 1/36 & 1/36 & 1/36 & 1/36 & 5/36 & 0 & 9/36 \\ \hline
6     & 1/36 & 1/36 & 1/36 & 1/36 & 1/36 & 6/36 & 11/36 \\ \hline \hline
P(X=x) & 6/36=1/6 & 6/36=1/6 & 6/36=1/6 & 6/36=1/6 & 6/36=1/6 & 6/36=1/6 &  
\end{tabular}
\end{center}

\vskip 0.2 cm
\noindent
\textbf{c)} Si $X$ i $Y$ f\`ossin independents, per a qualsevol valor $(x, y)$
s'hauria de verificar que $P(X=x, Y=y)=P(X=x) \cdot P(Y=y)$. Podem comprovar
que, per exemple, si $(x, y)=(1, 1)$: 

\[
P(X=1, Y=1)=\frac{1}{36} \neq P(X=1) \cdot P(Y=1)=\frac{1}{6} \cdot \frac{1}{36}
\]

\vskip 0.2 cm
\noindent
Per tant $X$ i $Y$ no s\'on independents.

\newpage
\noindent
\textbf{Problema 3.} Dues persones A i B esperen trobar-se en un cert lloc entre
les 5 i les 6. Cap d'elles  esperar\`a l'altra m\'es de 10 minuts. Si
suposam que arriben independentment, calculau la probabilitat que
se trobin en cada un dels dos casos seg\"{u}ents:
\begin{enumerate}[a)]
\item Si la persona A arriba a les 5.30. 
\item Si A i B arriben en qualsevol moment, a l'atzar.
\end{enumerate}

\vskip 0.3 cm
\noindent
\textbf{Soluci\'o:} Definim les seg\"uents variables aleat\`ories:

$T_A$: hora d'arribada de A al lloc de l'encontre

$T_B$: hora d'arribada de B al lloc de l'encontre

\noindent
Com que l'hora d'arribada pot \'esser qualsevol valor entre 5 i 6 (en hores), llavors
podem considerar que $T_A \sim {\cal U}(5, 6)$ i $T_B \sim {\cal U}(5, 6)$, per tant:
\[
f_{T_A}(t)=f_{T_B}(t)=\begin{cases} 1 & \text{si } t \in (5, 6) \\ 0 & \text{en altre cas} \end{cases}
\]


\noindent
\textbf{a)} 
\[
\begin{array}{rl}
P(\{\text{trobar-se si A arriba a les 5h30}\}) &=P(\text{5h20} \leq T_B \leq \text{5h40})= \\ \\
 &= P(\frac{16}{3} \leq T_B \leq \frac{17}{3})=
 \int_{16/3}^{17/3} 1 \cdot dt = \left. t\right]_{16/3}^{17/3}=
\frac{17}{3}-\frac{16}{3}=\frac{1}{3}
\end{array}
\]
\noindent
on s'ha tengut en compte que 5h20min=5h+$\frac{1}{3}$h=$\frac{16}{3}$h i 
5h40min=5h+$\frac{2}{3}$h=$\frac{17}{3}$h.

\vskip 0.2 cm
\noindent
\textbf{b)} 
\[
P(\{\text{trobar-se si A i B arriben en qualsevol moment}\}) =  P(|T_A-T_B| < 10\text{min})=
P(|T_A-T_B| < \frac{1}{6})=P(R)
\]

\noindent
on s'ha tengut en compte que 10min=$\frac{1}{6}$h, $R$ \'es el conjunt de valors $(t_A, t_B)$ 
que cumpleixen la condici\'o $|t_A-t_B| < \frac{1}{6}$ i
\[
f_{T_A T_B} (t_A, t_B) = \text{(v.a. independents)}=f_{T_A}(t_A) \cdot f_{T_B}(t_B)=
\begin{cases} 1 & \text{si } (t_A, t_B) \in (5, 6)^2 \\ 0 & \text{en altre cas} \end{cases}
\]

\noindent
(veure figura 1).

\vskip 0.2 cm
\setcounter{figure}{0}
\begin{figure}[htbp]
\begin{center}
\begin{picture}(100, 100)
\put(10, 10){\vector(1, 0){90}}
\put(10, 10){\vector(0, 1){90}}
\put(95, 0){$t_A$}
\put(-5, 95){$t_B$}
\put(50, 50){\line(1, 0){30}}
\put(50, 80){\line(1, 0){30}}
\put(50, 50){\line(0, 1){30}}
\put(80, 50){\line(0, 1){30}}
\multiput(10, 50)(10, 0){4}{\line(5, 0){5}} 
\multiput(10, 80)(10, 0){4}{\line(5, 0){5}} 
\multiput(50, 10)(0, 10){4}{\line(0, 5){5}} 
\multiput(80, 10)(0, 10){4}{\line(0, 5){5}} 
\put(50, 0){$5$}
\put(0, 50){$5$}
\put(80, 0){$6$}
\put(0, 80){$6$}
\put(70, 70){\textbf{\textit{S}}}
\end{picture}
$\qquad \qquad$
\begin{picture}(100, 100)
\put(10, 10){\vector(1, 0){90}}
\put(10, 10){\vector(0, 1){90}}
\put(95, 0){$t_A$}
\put(-5, 95){$t_B$}
\put(20, 10){\line(1, 1){70}}
\put(20, 10){\line(-1, -1){20}}
\put(10, 20){\line(1, 1){80}}
\put(10, 20){\line(-1, -1){10}}
\multiput(5, -5)(5, 5){17}{\line(0, 1){20}}
\put(20, 0){$\frac{1}{6}$}
\put(0, 20){$\frac{1}{6}$}
\put(50, 50){\textbf{\textit{R}}}
\put(80, 60){$t_A-t_B=\frac{1}{6}$}
\put(80, 105){$t_A-t_B=-\frac{1}{6}$}
\end{picture}
$\qquad \qquad \qquad \qquad$
\begin{picture}(100, 100)
\put(10, 10){\vector(1, 0){90}}
\put(10, 10){\vector(0, 1){90}}
\put(95, 0){$t_A$}
\put(-5, 95){$t_B$}
\put(50, 50){\line(1, 0){30}}
\put(50, 80){\line(1, 0){30}}
\put(50, 50){\line(0, 1){30}}
\put(80, 50){\line(0, 1){30}}
\multiput(10, 50)(10, 0){4}{\line(5, 0){5}} 
\multiput(10, 80)(10, 0){4}{\line(5, 0){5}} 
\multiput(50, 10)(0, 10){4}{\line(0, 5){5}} 
\multiput(80, 10)(0, 10){4}{\line(0, 5){5}} 
\put(85, 55){\textbf{\textit{S}}}
\put(20, 10){\line(1, 1){70}}
\put(20, 10){\line(-1, -1){20}}
\put(10, 20){\line(1, 1){80}}
\put(10, 20){\line(-1, -1){10}}
\multiput(5, -5)(5, 5){17}{\line(0, 1){20}}
\put(20, 0){$\frac{1}{6}$}
\put(0, 20){$\frac{1}{6}$}
\put(90, 90){\textbf{\textit{R}}}
\thicklines
\put(70, 80){\line(1, 0){10}}
\put(50, 50){\line(1, 0){10}}
\put(50, 50){\line(0, 1){10}}
\put(80, 80){\line(0, -1){10}}
\put(50, 60){\line(1, 1){20}}
\put(60, 50){\line(1, 1){20}}
\put(60,60){$V$}
\put(50, 72){$T_2$}
\put(70, 52){$T_1$}
\thinlines
\multiput(60, 10)(0, 10){4}{\line(0, 5){5}} 
\multiput(10, 70)(10, 0){7}{\line(5, 0){5}} 
\put(45, 0){$5$}
\put(0, 50){$5$}
\put(80, 0){$6$}
\put(0, 80){$6$}
\put(55, 0){$\frac{31}{6}$}
\put(-3, 65){$\frac{35}{6}$}
\end{picture}
\end{center}
\caption{Esquerra: suport ($S$) de $f_{T_A T_B} (t_A, t_B)$. Centre: $R=\{ (t_A, t_B) / \, |t_A-t_B| < \frac{1}{6} \}$.
Dreta: $V=R \cap S$. (Nota: els dibuixos no estan a escala).}
\end{figure}

\[
\begin{array}{rl}
P(R)=\iint_R f_{T_A T_B} (t_A, t_B) \, dt_A dt_B & = \iint_{R \cap S} 1 \cdot dt_A dt_B =
\text{area}(R \cap S)= \\ \\
&= \text{area}(S)-\text{area}(T_1)-\text{area}(T_2)=\\ \\
&= 1-2\cdot \text{area}(T_1)=
1-2 \frac{(6-\frac{31}{6}) (\frac{35}{6}-5)}{2}=\frac{11}{36}
\end{array}
\]
\vskip 0.2 cm
\noindent
on, per simetria (veure figura 1), area($T_1$)=area($T_2$) i, a m\'es, area($S$)=$(6-5)\cdot(6-5)=1$.


\newpage
\noindent
\textbf{Problema 4.} Les variables aleat\`ories $X_1 \mbox{i } X_2$ s\'on independents
i amb densitat com\'u 
\[
f(x) = \begin{cases}1 & \text{si } 0 \leq x \leq 1 \\ 
0 & \text{en cas contrari} \end{cases}
\]
\begin{enumerate}[a)]
\item Determinau la densitat de $Y = X_1 + X_2.$
\item Determinau la densitat de $Z = X_1 - X_2.$
\end{enumerate}

\vskip 0.3 cm
\noindent
\textbf{Soluci\'o:}
\vskip 0.2 cm
\noindent
\textbf{a)} Com que $X_1$ i $X_2$ s\'on v.a. cont\'\i nues, tamb\'e ho ser\`a $Y$.
El c\`alcul de $f_Y(y)$ es pot fer de v\`aries maneres. En aquest exercici nosaltres 
ho farem en dos pasos: (1) primer calcularem $F_Y(y)$ (la funci\'o de distribuci\'o de $Y$)
i (2) trobarem $f_Y(y)$ derivant $F_Y(y)$.

\[
F_Y(y)=P(Y \leq y)=P(X_1+X_2 \leq y)=\iint_{A_y} f_{X_1 X_2}(x_1, x_2) \, dx_1 dx_2
\]
\noindent
on $A_y$ \'es el conjunt de punts $(x_1, x_2)$ que verifiquen la condici\'o $x_1+x_2 \leq y$
i s'ha aplicat la propietat del c\`alcul de la probabilitat d'un conjunt de punts.
Observem que per a cada valor de $y$ el conjunt de punts $A_y$ ser\`a diferent (veure la
Figura 1-esquerra). A m\'es, tenim que 
\[
f_{X_1 X_2}(x_1, x_2) = f_{X_1}(x_1) \cdot f_{X_2}(x_2) = 
\begin{cases} 1 & \text{si } 0 \leq x_1 \leq 1 \quad \text{i} \quad 0 \leq x_2 \leq 1 \\
0 & \text{en cas contrari} \end{cases}
\]
\noindent
ja que $X_1$ i $X_2$ s\'on v.a. independents que tenen la funci\'o de densitat donada a
l'enunciat.

\vskip 0.2 cm
Per a calcular la integral anterior hem de dibuixar primer el suport de $f_{X_1 X_2}(x_1, x_2)$ (veure la Figura 1-dreta):
\[
S=\{(x, y) \, / \quad 0 \leq x \leq 1 \quad \text{i} \quad 0 \leq y \leq 1 \}
\]

\vskip 0.2 cm
\begin{figure}[htbp]
\begin{center}
\begin{picture}(100, 100)
\put(10, 10){\vector(1, 0){90}}
\put(10, 10){\vector(0, 1){90}}
\put(95, 0){$x_1$}
\put(-5, 95){$x_2$}
\put(10, 72){\line(1, -1){75}}
\put(10, 72){\line(-1, 1){10}}
\put(72, 0){\textbf{\textit{1.5}}}
\put(-10, 72){\textbf{\textit{1.5}}}
\put(25, 25){$A_{1.5}$}
\multiput(5, 77)(5, -5){12}{\line(-1, -1){40}}
\put(65, 17){\line(-1, -1){35}}
\put(70, 12){\line(-1, -1){30}}
\put(75, 7){\line(-1, -1){25}}
\end{picture}
$\qquad$ $\qquad$
\begin{picture}(100, 100)
\put(10, 10){\vector(1, 0){90}}
\put(10, 10){\vector(0, 1){90}}
\put(95, 0){$x_1$}
\put(-5, 95){$x_2$}
\put(10, 60){\line(1, 0){50}}
\put(60, 10){\line(0, 1){50}}
\put(60, 0){\textbf{\textit{1}}}
\put(0, 60){\textbf{\textit{1}}}
\put(50, 50){\textbf{\textit{S}}}
\end{picture}
\end{center}
\caption{Esquerra: conjunt $A_y$ per al valor $y=1.5$. 
Dreta: suport de $f_{X_1 X_2}(x_1, x_2)$. }
\end{figure}

Si tenim en compte el suport $S$ la funci\'o de distribuci\'o es calcula com:
\[
F_Y(y)=P(Y \leq y)=P(X_1+X_2 \leq y)=\iint_{A_y} f_{X_1 X_2}(x_1, x_2) \, dx_1 dx_2
= \iint_{A_y \cap S} f_{X_1 X_2}(x_1, x_2) \, dx_1 dx_2
\]


En funci\'o del valor de $y$ tendrem diferents resultats per a la intersecci\'o $A_y \cap S$.
Podem trobar quatre casos diferents:
\begin{itemize}
\item $y < 0$, llavors $A_y \cap S=\emptyset$ i per tant $F_Y(y)=\iint_{\emptyset} f_{X_1 X_2}(x_1, x_2) \, dx_1 dx_2=0$
\item $y \geq 2$, llavors $A_y \cap S=S$ i per tant $F_Y(y)=\iint_{S} f_{X_1 X_2}(x_1, x_2) \, dx_1 dx_2=1$
\item $0 \leq y < 1$, llavors $A_y \cap S=T$ (figura 2-esquerra) i \newline
\[
F_Y(y)=\iint_{T} f_{X_1 X_2}(x_1, x_2) \, dx_1 dx_2=\iint_{T} 1 \cdot dx_1 dx_2=\text{area}(T)=\frac{y^2}{2}
\]
\item $1 \leq y < 2$, llavors $A_y \cap S=P$ (figura 2-dreta) i 
\[
\begin{array}{rl}
F_Y(y)=\iint_{P} f_{X_1 X_2}(x_1, x_2) \, dx_1 dx_2=\iint_{P} 1 \cdot dx_1 dx_2=&\text{area}(P)=
\text{area}(S)-\text{area}(U)=\\
=&1-\frac{(1-(y-1)^2}{2}=1-\frac{(2-y)^2}{2}
\end{array}
\]
\end{itemize}

\vskip 0.2 cm
\begin{figure}[htbp]
\begin{center}
\begin{picture}(100, 100)
\put(10, 10){\vector(1, 0){90}}
\put(10, 10){\vector(0, 1){90}}
\put(95, 0){$x_1$}
\put(-5, 95){$x_2$}
%\linethickness{0.5mm}
\thicklines
\put(10, 10){\line(1, 0){27}}
\put(10, 10){\line(0, 1){27}}
\put(10, 37){\line(1, -1){27}}
\put(10, 11){\line(1, 0){27}}
\put(11, 10){\line(0, 1){27}}
\put(10, 37.5){\line(1, -1){27}}
\put(37, 0){\textbf{\textit{y}}}
\put(0, 37){\textbf{\textit{y}}}
\put(30, 30){$A_y \cap S=T$}
\end{picture}
$\qquad$ $\qquad$
\begin{picture}(100, 100)
\put(10, 10){\vector(1, 0){90}}
\put(10, 10){\vector(0, 1){90}}
\put(95, 0){$x_1$}
\put(-5, 95){$x_2$}
\put(10, 72){\line(1, -1){75}}
\put(10, 72){\line(-1, 1){10}}
\put(72, 0){\textbf{\textit{y}}}
\put(-10, 72){\textbf{\textit{y}}}
\put(10, 60){\line(1, 0){50}}
\put(60, 10){\line(0, 1){50}}
\put(60, 0){\textbf{\textit{1}}}
\put(0, 60){\textbf{\textit{1}}}
\put(12, 0){\textbf{\textit{y-1}}}
\put(-10, 22){\textbf{\textit{y-1}}}
\thicklines
\put(10, 10){\line(1, 0){50}}
\put(10, 10){\line(0, 1){50}}
\put(60, 10){\line(0, 1){12}}
\put(10, 60){\line(1, 0){12}}
\put(22, 60){\line(1, -1){38}}
\put(10, 11){\line(1, 0){50}}
\put(11, 10){\line(0, 1){50}}
\put(60.5, 10){\line(0, 1){12}}
\put(10, 60.5){\line(1, 0){12}}
\put(22, 60.5){\line(1, -1){38}}
\thinlines
\multiput(22, 10)(0, 10){6}{\line(0, 1){5}}
\multiput(10, 22)(10, 0){6}{\line(1, 0){5}}
\put(15, 25){$A_y \cap S=P$}
\put(50, 50){\textbf{\textit{U}}}
\end{picture}
\end{center}
\caption{Esquerra: intersecci\'o del conjunt $A_y$ ($0 \leq y < 1$) amb el suport de $f_{X_1 X_2}(x_1, x_2)$
(triangle delimitat per l\'\i nies gruixades). 
Dreta: intersecci\'o del conjunt $A_y$ ($1 \leq y < 2$) amb el suport de $f_{X_1 X_2}(x_1, x_2)$
(pol\'\i gon delimitat per l\'\i nies gruixades). }
\end{figure}

En conclusi\'o:
\[
F_Y(y)=\begin{cases} 
0 & \text{si } y < 0\\
\frac{y^2}{2} & \text{si } 0 \leq y < 1\\
1-\frac{(2-y)^2}{2} & \text{si } 1 \leq  y < 2\\
1 & \text{si } y \geq 2
\end{cases}
\]

Derivant aquesta funci\'o en cada un d'aquests intervals obtenim la funci\'o de densitat de $Y$:
\[
f_Y(y)=\begin{cases}
\frac{d}{dy}0=0 & \text{si } y < 0\\
\frac{d}{dy} (\frac{y^2}{2})=y & \text{si } 0 \leq y < 1\\
\frac{d}{dy}(1-\frac{(2-y)^2}{2})=2-y & \text{si } 1 \leq y < 2\\
\frac{d}{dy}1=0 & \text{si } y \geq 2
\end{cases}
= 
\begin{cases}
0 & \text{si } y < 0 \quad \text{\'o} \quad y \geq 2\\
y & \text{si } 0 \leq y < 1\\
2-y & \text{si } 1 \leq y < 2\\
\end{cases}
\]

\vskip 0.2 cm
\noindent
\textbf{b)} La funci\'o de densitat de $Z=X_1-X_2$ es calcula de manera similar 
a l'apartat anterior. En aquest cas el conjunt $A_z=\{ (x_1, x_2) / \quad x_1-x_2 \leq z\}$
t\'e la forma que es mostra en la figura 3.
Raonant de manera similar a l'apartat anterior arribarem a:
\[
f_Z(z)=\begin{cases}
0 & \text{si } z < -1 \quad \text{\'o} \quad z \geq 1\\
1+z & \text{si }  -1 \leq z < 0\\
1-z & \text{si } 0 \leq z < 1
\end{cases}
\]

\vskip 0.2 cm
\begin{figure}[htbp]
\begin{center}
\begin{picture}(100, 100)
\put(10, 10){\vector(1, 0){90}}
\put(10, 10){\line(-1, 0){30}}
\put(10, 10){\vector(0, 1){90}}
\put(95, 0){$x_1$}
\put(-5, 95){$x_2$}
\put(10, 35){\line(1, 1){30}}
\put(10, 35){\line(-1, -1){35}}
\put(0, 35){\textbf{\textit{0.5}}}
\put(-25, 0){\textbf{\textit{-0.5}}}
\put(25, 25){$A_{-0.5}$}
\multiput(10, 35)(-5, -5){4}{\line(1, -1){40}}
\put(-10, 15){\line(1, -1){35}}
\put(-15, 10){\line(1, -1){30}}
\put(-20, 5){\line(1, -1){25}}
\multiput(10, 35)(5, 5){6}{\line(1, -1){40}}
\end{picture}
\end{center}
\caption{Conjunt $A_z$ per al valor $z=-0.5$.}
\end{figure}


\newpage
\noindent
\textbf{Problema 5.} 
Obteniu la funci\'o de distribuci\'o conjunta de les variables
aleat\`ories $X$ i $Y $ la funci\'o de densitat conjunta de les quals
\'es: 
\[
f(x,y) = \begin{cases}
\frac{6}{5}(x+y^2) & \text{si } x,y \in (0,1)\\
0 & \text{en cas contrari}\end{cases}
\]

\vskip 0.3 cm
\noindent
\textbf{Soluci\'o:}

\vskip 0.2 cm
\noindent
El suport ($S$) de $f_{XY}$ es mostra en la figura 1-esquerra. D'altra banda 
$F_{XY}(\mathrm{x}, \mathrm{y})=P(X \leq \mathrm{x}, Y \leq \mathrm{y})=P(A_{\mathrm{xy}})$, 
on $A_{\mathrm{xy}}$ \'es el conjunt de punts $(x, y)$ tals que $x \leq \mathrm{x}$ i $y \leq \mathrm{y}$ 
(veure la figura 1-dreta). 

\vskip 0.2 cm
\setcounter{figure}{0}
\begin{figure}[htbp]
\begin{center}
\begin{picture}(100, 100)
\put(10, 10){\vector(1, 0){90}}
\put(10, 10){\vector(0, 1){90}}
\put(95, 0){$x$}
\put(-5, 95){$y$}
\put(10, 60){\line(1, 0){50}}
\put(60, 10){\line(0, 1){50}}
\put(60, 0){\textbf{\textit{1}}}
\put(0, 60){\textbf{\textit{1}}}
\put(50, 50){\textbf{\textit{S}}}
\end{picture}
$\qquad$ $\qquad$
\begin{picture}(100, 100)
\put(10, 10){\vector(1, 0){90}}
\put(10, 10){\vector(0, 1){90}}
\put(95, 0){$x$}
\put(-5, 95){$y$}
\put(-20, 72){\line(1, 0){100}}
\put(72, -20){\line(0, 1){100}}
\put(72, 0){$\mathrm{x}$}
\put(-10, 77){$\mathrm{y}$}
\put(35, 35){$A_{\mathrm{x}\mathrm{y}}$}
\put(77, 77){$(\mathrm{x}, \mathrm{y})$}
\put(72, 72){\circle*{3}}
\put(-5, 72){\line(-1, -1){10}}
\put(5, 72){\line(-1, -1){20}}
\put(15, 72){\line(-1, -1){30}}
\put(25, 72){\line(-1, -1){40}}
\put(35, 72){\line(-1, -1){50}}
\put(45, 72){\line(-1, -1){60}}
\put(55, 72){\line(-1, -1){70}}
\put(65, 72){\line(-1, -1){80}}
\put(72, 70){\line(-1, -1){90}}
\put(72, 60){\line(-1, -1){80}}
\put(72, 50){\line(-1, -1){70}}
\put(72, 40){\line(-1, -1){60}}
\put(72, 30){\line(-1, -1){50}}
\put(72, 20){\line(-1, -1){40}}
\put(72, 10){\line(-1, -1){30}}
\put(72, 0){\line(-1, -1){20}}
\put(72, -10){\line(-1, -1){10}}
\end{picture}
\end{center}
\caption{Esquerra: suport de $f_{XY}(x, y)$. 
Dreta: conjunt $A_{\mathrm{x}\mathrm{y}}$ per a $\mathrm{x} > 0$ i $\mathrm{y} > 0$. }
\end{figure}

\[
F_{XY}(\mathrm{x}, \mathrm{y})=\iint_{A_{\mathrm{xy}} \cap S} f_{XY}(x, y) \, dxdy
\]
\noindent
Depenent del valor de $(\mathrm{x}, \mathrm{y})$ la intersecci\'o $A_{\mathrm{xy}} \cap S$ tendr\`a
un valor o un altre. Si tenim en compte tots els casos possibles obtenim:

\[
F_{XY}(\mathrm{x}, \mathrm{y})=\begin{cases}
\text{si } \mathrm{x} \geq 1 \quad \text{i} \quad \mathrm{y} \geq 1 & 1 \\ \\
\text{si } \mathrm{x} < 0 \quad \text{\'o} \quad \mathrm{y} < 0 & 0 \\ \\
\text{si } 0 \leq \mathrm{x} < 1 \quad \text{i} \quad 0 \leq \mathrm{y} < 1 &
\int_0^\mathrm{x} \int_0^\mathrm{y} \frac{6}{5}(x+y^2) \, dy dx =\frac{3}{5}yx^2+\frac{2}{5}y^3x\\ \\
\text{si } 0 \leq \mathrm{x} < 1 \quad \text{i} \quad \mathrm{y} \geq 1 & 
\int_0^\mathrm{x} \int_0^1 \frac{6}{5}(x+y^2) \, dy dx = \frac{3}{5}x^2+\frac{2}{5} x \\ \\
\text{si } \mathrm{x} \geq 1 \quad \text{i} \quad 0 \leq \mathrm{y} < 1 & 
\int_0^\mathrm{1} \int_0^\mathrm{y} \frac{6}{5}(x+y^2) \, dy dx = \frac{3}{5}y+\frac{2}{5} y^3 
\end{cases}
\]



\newpage
\noindent
\textbf{Problema 6.} Si $X_1 \mbox{ i } X_2 $ s\'on dues variables aleat\`ories amb
distribuci\'o de Poisson, independents i amb mitjanes respectives
$\alpha \mbox{ i } \beta$, provau que $Y = X_1 + X_2 $ tamb\'e \'es
una variable aleat\`oria Poisson (amb mitjana $\> \alpha + \beta$).
(Nota: Una variable aleat\`oria que t\'e aquesta propietat \'es diu que
\'es \emph{estable})

\vskip 0.3 cm
\noindent
\textbf{Soluci\'o:} Tenim que $X_1 \sim \mathrm{Po}(\alpha)$ i 
$X_2 \sim \mathrm{Po}(\beta)$, per tant:
\[
P(X_1=x)=\begin{cases}
\frac{\alpha^{x}}{x!} e^{-\alpha} & \text{si } x \in \Omega_{X_1}\\
0 & \text{si } x \notin \Omega_{X_1} \end{cases}
\qquad 
P(X_2=x)=\begin{cases}
\frac{\beta^{x}}{x!} e^{-\beta} & \text{si } x \in \Omega_{X_2}\\
0 & \text{si } x \notin \Omega_{X_2} \end{cases}
\]
\noindent
on $\Omega_{X_1}=\Omega_{X_2}=\{ 0, 1, 2, \cdots \}$.

\vskip 0.2 cm
\noindent
$Y=X_1+X_2$, per tant $\Omega_Y=\{0, 1, 2, \cdots\}$. 

\noindent
Podem calcular alguns valors de la funci\'o de probabilitat de $Y$ (tenint en compte que les
v.a. $X_1$ i $X_2$ s\'on independents):
\[
\begin{array}{rl}
P(Y=0)=&P(X_1+X_2=0)=P(X_1=0, X_2=0)=P(X_1=0)\cdot P(X_2=0)= \\
      =&\frac{\alpha^{0}}{0!} e^{-\alpha} \cdot \frac{\beta^{0}}{0!} e^{-\beta} = e^{-\alpha} \cdot e^{-\beta}=e^{-(\alpha+\beta)} \\ \\
P(Y=1)=&P(X_1+X_2=1)=P(X_1=1, X_2=0)+P(X_1=0, X_2=1)= \\
      =&P(X_1=1)\cdot P(X_2=0)+P(X_1=0)\cdot P(X_2=1)=\\
      =&\frac{\alpha^{1}}{1!} e^{-\alpha} \cdot \frac{\beta^{0}}{0!} e^{-\beta} + 
\frac{\alpha^{0}}{0!} e^{-\alpha} \cdot \frac{\beta^{1}}{1!} e^{-\beta}= (\alpha+\beta) e^{-(\alpha+\beta)} 
\end{array}
\]

\noindent
en general, si $y \in \Omega_Y$:
\[
\begin{array}{rl}
P(Y=y)=&P(X_1=y, X_2=0)+ P(X_1=y-1, X_2=1)+ \cdots + P(X_1=0, X_2=y)=\\ \\
      =& \sum_{i=0}^{y} P(X_1=y-i, X_2=i) = \\ \\
      =& \sum_{i=0}^{y} P(X_1=y-i) \cdot P(X_2=i) = 
      \sum_{i=0}^{y} \frac{\alpha^{y-i}}{(y-i)!} e^{-\alpha} \cdot \frac{\beta^{i}}{i!} e^{-\beta} = \\ \\
      =& e^{-(\alpha+\beta)} \sum_{i=0}^{y} \frac{1}{(y-i)! i!} \alpha^{y-i} \beta^{i}
\end{array}
\]

\noindent
Com que $\binom{y}{i}=\frac{y!}{(y-i)! i!}$, podem escriure $\frac{1}{(y-i)! i!} = \frac{1}{y!} \binom{y}{i}$,
i per tant:

\[
P(Y=y)= e^{-(\alpha+\beta)} \frac{1}{y!} \sum_{i=0}^{y} \binom{y}{i} \alpha^{y-i} \beta^{i}
\]

\noindent
L'expressi\'o dins el sumatori \'es el desenvolupament del binomi de Newton: 
$(\alpha + \beta)^y = \sum_{i=0}^{y} \binom{y}{i} \alpha^{y-i} \beta^{i}$. De manera que, finalment:

\[
P(Y=y)= \frac{(\alpha + \beta)^y}{y!} e^{-(\alpha+\beta)}
\]

\noindent
Aquesta expressi\'o correspon a la funci\'o de probabilitat d'una v.a. de Poisson amb par\`ametre
$\alpha+\beta$. Per tant podem concloure que $Y \sim \mathrm{Po}(\alpha+\beta)$.


\newpage
\noindent
\textbf{Problema 7.}
Durant un per\'{\i}ode $T$, el nombre $X$ de cridades arribades a
una centraleta \'es una variable aleat\`oria Poisson amb mitjana
$\lambda \cdot T,$ i el nombre $Y$ de cridades que surten de la
centraleta \'es una altra variable aleat\`oria Poisson amb mitjana $
\mu \cdot T. $ Suposant que $X \mbox{i } Y$ s\'on independents,
determinau la distribuci\'o condicional de $X \mbox{donat que } X +
Y = n.$ 
\vskip 0.3 cm
\noindent
\textbf{Soluci\'o:} 
\vskip 0.1 cm
\noindent
Ens demanen calcular $P(X=k| \, X+Y=n)$. 
Tenim que $X \sim \mathrm{Po}(\lambda T)$ i $Y \sim \mathrm{Po}(\mu T)$.
D'altra banda, pel resultat del problema 6 sabem que $X+Y \sim \mathrm{Po}(\lambda T+\mu T)=\mathrm{Po}((\lambda + \mu) T)$.
\[
\begin{array}{rl}
P(X=k| \, X+Y=n) &= \frac{ P(X=k \, \cap \, X+Y=n) }{P(X+Y=n)}=\frac{ P(X=k \, \cap \, Y=n-X) }{P(X+Y=n)}= \\ \\
&= \frac{ P(X=k \, \cap \, Y=n-k) }{P(X+Y=n)}=\frac{ P(X=k) \cdot P(Y=n-k) }{P(X+Y=n)}
\end{array}
\]
\noindent
on s'ha aplicat que $X$ i $Y$ s\'on v.a. independents.
\[
P(X=k)=\frac{(\lambda T)^k}{k!} e^{-\lambda T} \qquad \text{si } k \in \{ 0, 1, \cdots \} 
\]
\[
P(Y=n-k)=\frac{(\mu T)^{(n-k)}}{(n-k)!} e^{-\mu T} \qquad \text{si } n-k \in \{ 0, 1, \cdots \} 
\]
\[
P(X+Y=n)=\frac{((\lambda + \mu) T)^n}{n!} e^{-(\lambda + \mu) T} \qquad \text{si } n \in \{ 0, 1, \cdots \} 
\]

\noindent
Combinant tots aquests valors de probabilitat i simplificant les expressions arribarem a:
\[
P(X=k| \, X+Y=n)= \begin{cases} 
\frac{n!}{k! (n-k)!} \frac{\lambda^k}{(\lambda+\mu)^k} \frac{\mu^{(n-k)}}{(\lambda+\mu)^{(n-k)}} & 
\text{si } k \in \{0, 1, \cdots, n \} \\
0 & \text{en altre cas} \end{cases}
\]

\noindent
Finalment, si observam que $\binom{n}{k}=\frac{n!}{k! (n-k)!}$ i $\frac{\mu}{\lambda+\mu}=1-\frac{\lambda}{\lambda+\mu}$,
veurem que la funci\'o de probabilitat anterior es pot escriure com
\[
P(X=k| \, X+Y=n)= \begin{cases} 
\binom{n}{k}  \cdot \left( \frac{\lambda}{\lambda+\mu} \right)^k  \cdot \left( 1- \frac{\lambda}{\lambda+\mu} \right)^{(n-k)} & 
\text{si } k \in \{0, 1, \cdots, n \} \\
0 & \text{en altre cas} \end{cases}
\]
\noindent
Aquesta funci\'o de probabilitat correspon a una v.a. binomial amb par\`ametres $n$ i 
$\frac{\lambda}{\lambda+\mu}$. Per tant: 
\[
X |_{X+Y=n} \, \sim B(n, \frac{\lambda}{\lambda+\mu})
\]
 

\newpage
\noindent
\textbf{Problema 8.}
Un prove\"{\i}dor de serveis inform\`atics t\'e una quantitat $X$ de
cents d'unitats d'un cert producte al principi de cada mes. Durant
el mes es venen $Y$ cents d'unitats del producte. Suposem que $X$
 i $Y$ tenen una densitat conjunta donada per
\[
f(x,y) = \begin{cases}2/9 & \text{si } 0 < y < x < 3\\
0 & \text{en cas contrari}\end{cases}
\]
\begin{enumerate}[a)]
\item Comprovau que {\it f} \'es una densitat.
\item Determinau $\> F_{X,Y}.$
\item Calculau la probabilitat que a final de mes s'hagi venut com a
m\'{\i}nim la meitat de les unitats que hi havia inicialment.
\item Si s'han venut 100 unitats, quina \'es la probabilitat que n'hi hagu\`essin
com a m\'\i nim 200 a principi de mes?
\end{enumerate}

\vskip 0.3 cm
\noindent
\textbf{Soluci\'o:}


\vskip 0.2 cm
\noindent
\textbf{a)} Per demostrar que $f(x, y)$ \'es una funci\'o de densitat hem de 
comprovar que $f$ no agafa mai valors negatius (en efecte, els dos valors
posibles de la funci\'o, $0$ i $\frac{2}{9}$; s\'on no negatius) i que la 
seg\"uent integral es pot calcular i val $1$:
\[
\int_{-\infty}^{+\infty} \int_{-\infty}^{+\infty} f(x, y) \, dxdy=\iint_S \frac{2}{9} \,dxdy=
\frac{2}{9} \cdot \text{area}(S)=\frac{2}{9} \cdot \frac{3 \cdot 3}{2}=1
\]
\noindent
on $S$ \'es el suport de $f(x, y)$, qu\`e es mostra a la figura 1-esquerra.

\vskip 0.2 cm
\setcounter{figure}{0}
\begin{figure}[htbp]
\begin{center}
\begin{picture}(100, 100)
\put(10, 10){\vector(1, 0){90}}
\put(10, 10){\vector(0, 1){90}}
\put(95, 0){$x$}
\put(-5, 95){$y$}
\put(10, 10){\line(1, 1){70}}
\put(75, 70){$y=x$}
\multiput(10, 60)(10, 0){6}{\line(1, 0){5}}
\thicklines
\put(10, 10){\line(1, 1){50}}
\put(60, 10){\line(0, 1){50}}
\thinlines
\put(20, 10){\line(0, 1){10}}
\put(25, 10){\line(0, 1){15}}
\put(30, 10){\line(0, 1){20}}
\put(35, 10){\line(0, 1){25}}
\put(40, 10){\line(0, 1){30}}
\put(45, 10){\line(0, 1){35}}
\put(50, 10){\line(0, 1){40}}
\put(55, 10){\line(0, 1){45}}
\put(60, 0){\textbf{\textit{3}}}
\put(0, 60){\textbf{\textit{3}}}
\put(50, 40){\textbf{\textit{S}}}
\end{picture}
$\qquad$ $\qquad$
\begin{picture}(100, 100)
\put(10, 10){\vector(1, 0){90}}
\put(10, 10){\vector(0, 1){90}}
\put(95, 0){$x$}
\put(-5, 95){$y$}
\put(-20, 72){\line(1, 0){100}}
\put(72, -20){\line(0, 1){100}}
\put(72, 0){$\mathrm{x}$}
\put(-10, 77){$\mathrm{y}$}
\put(35, 35){$A_{\mathrm{x}\mathrm{y}}$}
\put(77, 77){$(\mathrm{x}, \mathrm{y})$}
\put(72, 72){\circle*{3}}
\put(-5, 72){\line(-1, -1){10}}
\put(5, 72){\line(-1, -1){20}}
\put(15, 72){\line(-1, -1){30}}
\put(25, 72){\line(-1, -1){40}}
\put(35, 72){\line(-1, -1){50}}
\put(45, 72){\line(-1, -1){60}}
\put(55, 72){\line(-1, -1){70}}
\put(65, 72){\line(-1, -1){80}}
\put(72, 70){\line(-1, -1){90}}
\put(72, 60){\line(-1, -1){80}}
\put(72, 50){\line(-1, -1){70}}
\put(72, 40){\line(-1, -1){60}}
\put(72, 30){\line(-1, -1){50}}
\put(72, 20){\line(-1, -1){40}}
\put(72, 10){\line(-1, -1){30}}
\put(72, 0){\line(-1, -1){20}}
\put(72, -10){\line(-1, -1){10}}
\end{picture}
\end{center}
\caption{Esquerra: suport ($S$) de $f_{XY}(x, y)$ (zona ratxada). Centre: regi\'o de punts $(x, y)$
per al c\`alcul de la funci\'o de distribuci\'o. Dreta: $R$, regi\'o de punts tals que $y \geq x/2$.}
\end{figure}

\vskip 0.2 cm
\noindent
\textbf{b)} 
\[
F_{XY}(\mathrm{x}, \mathrm{y})=P(X \leq \mathrm{x}, Y \leq \mathrm{y})=\iint_{A_{\mathrm{x}\mathrm{y}}} f_{XY}(x, y) \, dxdy=
\iint_{A_{\mathrm{x}\mathrm{y}} \cap S} \frac{2}{9} \, dxdy=
\frac{2}{9} \cdot \text{area}(A_{\mathrm{x}\mathrm{y}} \cap S)
\]
\noindent
on $A_{\mathrm{x}\mathrm{y}}$ \'es el conjunt de punts $(x, y)$ tals que $x \leq \mathrm{x}$ i 
$y \leq \mathrm{y}$ (veure figura 1-dreta). El conjunt intersecci\'o depenendr\`a del valor de 
$\mathrm{x}$ i de $\mathrm{y}$:
\[
F_{XY}(\mathrm{x}, \mathrm{y})=\begin{cases}
\text{si } \mathrm{x} < 0 \quad \text{\'o} \quad \mathrm{y} < 0 & 0 \\ \\
\text{si } \mathrm{x} \geq 3 \quad \text{i} \quad \mathrm{y} \geq 3 & 1 \\ \\
\text{si } 0 \leq \mathrm{x} < 3 \quad \text{i} \quad \mathrm{y} \geq  \mathrm{x} & \frac{x^2}{2} \\ \\
\text{si } 0 \leq \mathrm{x} < 3 \quad \text{i} \quad 0 \leq \mathrm{y} <  \mathrm{x} & \frac{1}{9} (2xy-y^2) \\ \\
\text{si } \mathrm{x} \geq 3 \quad \text{i} \quad 0 \leq \mathrm{y} < 3 & \frac{1}{9} (6y-y^2) 
\end{cases}
\]

\newpage
\noindent
\textbf{c)} 
\[
P(Y \geq \frac{X}{2})=P(R)=\iint_{R \cap S} \frac{2}{9} \, dxdy=\frac{2}{9} \cdot \text{area}(R \cap S)
\]
\noindent
on $R$ \'es el conjunt de punts $(x, y)$ tals que $y \geq \frac{x}{2}$ (veure figura 2-esquerra).
\[
P(Y \geq \frac{X}{2})=\frac{2}{9} \cdot (\text{area}(S)-   \text{area}(T))=
\frac{2}{9} \cdot ( \frac{9}{2} - \frac{3 \cdot 3/2}{2})=\frac{1}{2}
\]

\vskip 0.2 cm
\begin{figure}[htbp]
\begin{center}
\begin{picture}(100, 100)
\put(10, 10){\vector(1, 0){90}}
\put(10, 10){\vector(0, 1){90}}
\put(95, 0){$x$}
\put(-5, 95){$y$}
\put(10, 10){\line(1, 1){70}}
\put(75, 70){$y=x$}
\multiput(10, 60)(10, 0){6}{\line(1, 0){5}}
\thicklines
\put(10, 10){\line(1, 1){50}}
\put(60, 10){\line(0, 1){50}}
\thinlines
\put(-10, 0){\line(2, 1){95}}
\put(75, 35){$y=x/2$}
\multiput(-5, 2.5)(5, 2.5){18}{\line(-1, 2){30}}
\put(60, 0){\textbf{\textit{3}}}
\put(0, 60){\textbf{\textit{3}}}
\put(30, 70){\textbf{\textit{R}}}
\put(50, 15){\textbf{\textit{T}}}
\end{picture}
$\qquad$ $\qquad$
\begin{picture}(100, 100)
\put(10, 10){\vector(1, 0){90}}
\put(10, 10){\vector(0, 1){90}}
\put(95, 0){$x$}
\put(-5, 95){$y$}
\put(10, 10){\line(1, 1){70}}
\put(75, 70){$y=x$}
\multiput(10, 60)(10, 0){6}{\line(1, 0){5}}
\thicklines
\put(10, 10){\line(1, 1){50}}
\put(60, 10){\line(0, 1){50}}
\thinlines
\put(0, 27){\line(1, 0){85}}
\put(0, 28){\textbf{\textit{1}}}
\put(44, 0){\line(0, 1){80}}
\put(45, 0){\textbf{\textit{2}}}
\put(60, 0){\textbf{\textit{3}}}
\put(0, 60){\textbf{\textit{3}}}
\linethickness{0.7mm}
\put(44, 27){\line(1, 0){16}}
\thinlines
\put(80, 43){\vector(-2, -1){30}}
\put(80, 44){$A$}
\put(50, 45){\textbf{\textit{S}}}
\end{picture}
\end{center}
\caption{Esquerra: $R$, regi\'o de punts tals que $y \geq x/2$. 
Dreta: $A$, regi\'o de punts de $S$ tals que $x \geq 2$ i $y=1$.}
\end{figure}

\vskip 0.2 cm
\noindent
\textbf{d)} 
\[
P(X \geq 2 | Y=1)= \frac{P(X \geq 2, Y=1)}{f_Y(1)}
\]

\[
f_Y(y)=\int_{-\infty}^{+\infty} f_{XY}(x, y) \, dx = \begin{cases}
\text{si } 0 \leq y < 3 & \int_y^3 \frac{2}{9} \, dx = \frac{2}{9} (3-y) \\ \\
\text{en altre cas } & 0 
\end{cases}
\]

\[
P(X \geq 2, Y=1)=\iint_{x \geq 2 \, \cap \, y =1} f_{XY}(x, y) \, dxdy = \int_{A} f_{XY}(x, 1) \, dx = 
\int_2^3 \frac{2}{9} \, dx = \frac{2}{9}
\]
\noindent
on $A$ \'es la regi\'o de punts del suport de $f_{XY}(x, y)$ tals que $x \geq 2$ i $y=1$ (veure
la figura 2-dreta).

\noindent
El resultat final \'es:
\[
P(X \geq 2 | Y=1)= \frac{\frac{2}{9}}{\frac{2}{9} (3-1)}=\frac{1}{2}
\]



\newpage
\noindent
\textbf{Problema 11.}
Llan\c{c}am a l'aire un dau equilibrat. Considerem dues variables
aleat\`ories $X$ i $Y$ definides com:
\[
X = \begin{cases}-1 & \text{si el resultat \'es imparell}\\ 
1 & \text{si el resultat \'es parell} \end{cases} \ \ 
Y = \begin{cases}-1 & \text{si el resultat \'es 1, 2 o 3}\\
0 & \text{si el resultat \'es 4}\\ 
1 & \text{si el resultat \'es 5 o 6}\end{cases}
\]
\begin{enumerate}[a)]
\item Trobau la llei conjunta i la funci\'o de distribuci\'o de $X$
i $Y$.
\item Calculau $P(X + Y = 0 | Y \leq 0)$  i
$P(X = 1 | X + Y = 2)$. 
\end{enumerate}

\vskip 0.3 cm
\noindent
\textbf{Soluci\'o:}

\vskip 0.2 cm
\noindent
\textbf{a)} Com el dau est\`a equilibrat, llavors $P(\{i\})=\frac{1}{6}$, on $\{i\}$ denota
el succ\'es ``treure $i$''. A m\'es, tenim que $\Omega_X=\{-1, 1\}$, $\Omega_Y=\{-1, 0, 1\}$ i 
$\Omega_{XY}=\{(-1, -1), (-1, 0), (-1, 1), (1, -1), (1, 0), (1, 1)\}$.
\[
\begin{array}{rl}
P(X=x, Y=y)=& 0 $\qquad$ \text{ si } (x, y) \notin \Omega_{XY}\\ \\
P(X=-1, Y=-1)=& P(\{1, 3, 5\} \cap \{1, 2, 3\})=P(\{1, 3\})=P(\{1\} \cup \{ 3\})=P(\{1\})+P(\{3\})=
\frac{1}{6}+\frac{1}{6}=\frac{2}{6} \\ \\
P(X=-1, Y=0)=& P(\{1, 3, 5\} \cap \{4\})=P(\emptyset)=0 \\
\vdots & \vdots
\end{array}
\]
\noindent
raonant de manera similar per a tots els valors de $\Omega_{XY}$ obtenim la seg\"uent taula 
probabilitat conjunta, on tamb\'e s'indiquen els valors de les probabilitats marginals
(sumant per files o per columnes, per a valors de $X$ o de $Y$ fixats):
\vskip 0.2 cm
\begin{center}
\begin{tabular}{c|c|c|c||c}
$X \backslash Y$ & -1 & 0 & 1 & $P(X=x)$ \\ \hline
-1 & 2/6 & 0 & 1/6 & 1/2 \\ \hline
1 & 1/6 & 1/6 & 1/6 & 1/2 \\ \hline \hline
$P(Y=y)$ & 1/2 & 1/6 & 1/3
\end{tabular}
\end{center}
\noindent
es pot comprovar que la suma de tots els valors de la taula de probabilitat conjunta \'es igual a $1$.

\vskip 0.2 cm
\noindent
\textbf{Funci\'o de distribuci\'o conjunta}: 
\[
\begin{array}{rl}
F_{XY}(x, y)& = P(X \leq x, Y \leq y)=\\ \\
 & = 
\left\{ 
\begin{array}{ccccll}
x < -1 & \text{\'o} & y < -1 & \qquad & & 0 \\ \\
x \geq 1 & \text{i} & y \geq 1 & \qquad & & 1 \\ \\
-1 \leq x < 1 & \text{i} & -1 \leq y < 0 & \qquad & P(X=-1, Y=-1)=&\frac{1}{3}\\ \\
-1 \leq x < 1 & \text{i} & 0 \leq y < 1 & \qquad & P(X=-1, Y=-1)+P(X=-1, Y=0)=& \frac{1}{3}\\ \\
-1 \leq x < 1 & \text{i} &  y \geq 1 & \qquad & P(X=-1, Y=-1)+P(X=-1, Y=0)+ & \\
              &          &           & \qquad & +P(X=-1, Y=1)=& \frac{1}{2}\\ \\
x \geq 1 & \text{i} & -1 \leq y < 0 & \qquad & P(X=-1, Y=-1)+P(X=1, Y=-1)=& \frac{1}{2}\\ \\
x \geq 1 & \text{i} & 0 \leq y < 1 & \qquad & P(X=-1, Y=-1)+P(X=-1, Y=0)+ & \\
         &          &              & \qquad & +P(X=1, Y=-1)+P(X=1, Y=0)= & \frac{2}{3}
\end{array}
\right.
\end{array}
\]

\vskip 0.2 cm
\noindent
\textbf{b)} 
\[
\begin{array}{ll}
P(X+Y=0 |Y \leq 0)&=\frac{P(\{X+Y =0\} \cap \{Y \leq 0\})}{P(Y \leq 0)}=\frac{P(X=1, Y=-1)}{P(Y=-1)+P(Y=0)}=
\frac{1/6}{1/2+1/6}=\frac{1}{4}
\\ 
\\
P(X=1 | X+Y=2)&=\frac{ P(\{X=1\} \cap \{X+Y=2\})}{P(X+Y=2)}=\frac{P(X=1,Y=1)}{P(X=1, Y=1)}=1
\end{array}
\]


\newpage
\noindent
\textbf{Problema 12.} 
El nombre d'errors $X$ per p\`agina que comet un determinat
escriptor \'es una variable aleat\`oria amb llei 
\[
P(X = x) = e^{-2} \cdot {2^x \over x \> !}, \ \ x = 0,1, \ldots
\]
\noindent
Si una p\`agina t\'e {\it x} errors, el nombre de minuts $Y$ que un revisor tarda en
revisar i corregir cada p\`agina \'es una variable aleat\`oria amb
distribuci\'o:

\[
P(Y = y | X = x) = \begin{cases}
1/5 & \text{si } y = 1+x\\
3/5 & \text{si } y = 2+x\\
1/5 & \text{si } y = 3+x
\end{cases}
\]
\begin{enumerate}[a)]
\item Trobau la probabilitat que es necessitin 4 minuts per revisar
i corregir una p\`agina elegida a l'atzar. 
\item Si s'han utilitzat 4 minuts en la revisi\'o i correcci\'o d'una
p\`agina, quina \'es la probabilitat que hi hagu\'es 3 errors?
\end{enumerate}

\vskip 0.3 cm
\noindent
\textbf{Soluci\'o:} 

\vskip 0.2 cm
\noindent
\textbf{a)} Ens demanen calcular $P(Y=4)$, aplicant la f\`ormula de la probabilitat total:
\[
P(Y=4)=P(Y=4|X=0)\cdot P(X=0) + P(Y=4|X=1)\cdot P(X=1) + P(Y=4|X=2)\cdot P(X=2) + \cdots
\]

Tenim a m\'es que:
\[
P(Y = y | X = 0) = \begin{cases}
1/5 & \text{si } y = 1+0=1\\
3/5 & \text{si } y = 2+0=2\\
1/5 & \text{si } y = 3+0=3 \\
0 & \text{resta de casos}
\end{cases}
\qquad \text{per tant} \qquad P(Y = 4 | X = 0)=0
\]
\[
P(Y = y | X = 1) = \begin{cases}
1/5 & \text{si } y = 1+1=2\\
3/5 & \text{si } y = 2+1=3\\
1/5 & \text{si } y = 3+1=4 \\
0 & \text{resta de casos}
\end{cases}
\qquad \text{per tant} \qquad P(Y = 4 | X = 1)=1/5
\]
\[
P(Y = y | X = 2) = \begin{cases}
1/5 & \text{si } y = 1+2=3\\
3/5 & \text{si } y = 2+2=4\\
1/5 & \text{si } y = 3+2=5 \\
0 & \text{resta de casos}
\end{cases}
\qquad \text{per tant} \qquad P(Y = 4 | X = 2)=3/5
\]
\[
P(Y = y | X = 3) = \begin{cases}
1/5 & \text{si } y = 1+3=4\\
3/5 & \text{si } y = 2+3=5\\
1/5 & \text{si } y = 3+3=6 \\
0 & \text{resta de casos}
\end{cases}
\qquad \text{per tant} \qquad P(Y = 4 | X = 3)=1/5
\]
\[
P(Y = y | X = 4) = \begin{cases}
1/5 & \text{si } y = 1+4=5\\
3/5 & \text{si } y = 2+4=6\\
1/5 & \text{si } y = 3+4=7 \\
0 & \text{resta de casos}
\end{cases}
\qquad \text{per tant} \qquad P(Y = 4 | X = 4)=0
\]
\noindent
raonant de manera similar comprovam que $P(Y = y | X = x)=0$ si $x \geq 4$. Per tant:
\[
P(Y=4)=\frac{1}{5} e^{-2} \frac{2^1}{1!} + \frac{3}{5} e^{-2} \frac{2^2}{2!}
+ \frac{1}{5} e^{-2} \frac{2^3}{3!}=\frac{28}{15} e^{-2}=0.2526
\]

\vskip 0.2 cm
\noindent
\textbf{b)} 
\[
P(X=3|Y=4)=\frac{ P(X=3, Y=4) }{P(Y=4)}=\frac{P(Y=4|X=3) \cdot P(X=3)}{P(Y=4)}=
\frac{ \frac{1}{5} e^{-2} \frac{2^3}{3!} }{\frac{28}{15} e^{-2}}=\frac{1}{7}
\]


\newpage
\noindent
\textbf{Problema 13.}
Suposem que la variable aleat\`oria $X$ es selecciona a l'atzar
de l'interval unitat, i aleshores la variable aleat\`oria $Y$
 es selecciona a l'atzar de l'interval $(0,X).$ Determinau la
distribuci\'o de $Y.$ 
\vskip 0.3 cm
\noindent
\textbf{Soluci\'o:} Per l'enunciat dedu\"\i m que $Y$ \'es una v.a. cont\'\i nua, per tant
haurem calcular $f_Y(y)$. 

\noindent
Les dades del problema s\'on: $X \sim {\cal U}(0, 1)$ i $Y|X \sim {\cal U}(0, X)$. De manera que:
\[
f_X(x)=\begin{cases} 1 & \text{si } 0 < x < 1 \\ 0 & \text{en altre cas} \end{cases}
\qquad \qquad
f_{Y|X}(y|x)=\begin{cases} \frac{1}{x} & \text{si }  0 < y < x\\ 
0 & \text{en altre cas} \end{cases}
\]

\noindent
D'altra banda, sabem que $f_{Y|X}(y|x)=\frac{f_{XY}(x, y)}{f_X(x)}$, per tant 
\[
f_{XY}(x, y)=f_{Y|X}(y|x) \cdot f_X(x) = 
\begin{cases} \frac{1}{x} & \text{si } 0 < y < x < 1 \\ 0 & \text{en altre cas} \end{cases}
\]
\noindent
El suport d'aquesta funci\'o de densitat es mostra en la seg\"uent figura.

\vskip 0.2 cm
\setcounter{figure}{0}
\begin{figure}[htbp]
\begin{center}
\begin{picture}(100, 100)
\put(10, 10){\vector(1, 0){90}}
\put(10, 10){\vector(0, 1){90}}
\put(95, 0){$x$}
\put(-5, 95){$y$}
\put(10, 10){\line(1, 1){70}}
\put(75, 70){$y=x$}
\multiput(10, 60)(10, 0){6}{\line(1, 0){5}}
\thicklines
\put(10, 10){\line(1, 1){50}}
\put(60, 10){\line(0, 1){50}}
\thinlines
\put(20, 10){\line(0, 1){10}}
\put(25, 10){\line(0, 1){15}}
\put(30, 10){\line(0, 1){20}}
\put(35, 10){\line(0, 1){25}}
\put(40, 10){\line(0, 1){30}}
\put(45, 10){\line(0, 1){35}}
\put(50, 10){\line(0, 1){40}}
\put(55, 10){\line(0, 1){45}}
\put(60, 0){\textbf{\textit{1}}}
\put(0, 60){\textbf{\textit{1}}}
\put(50, 40){\textbf{\textit{S}}}
\end{picture}
\end{center}
\caption{Suport de $f_{XY}(x, y)$ (zona ratxada). }
\end{figure}

\[
f_Y(y)=\int_{-\infty}^{+\infty} f_{XY}(x, y) \, dx=\begin{cases}
0 < y < 1 & \int_y^1 \frac{1}{x} \, dx = \left( \mathrm{ln}(x) \right]_y^1 = \mathrm{ln}(1) - \mathrm{ln}(y)=-\mathrm{ln}(y) \\ \\
\text{en altre cas} & 0 \end{cases}
\]




\newpage
\noindent
\textbf{Problema 14.}  Un auditor selecciona a l'atzar un cert nombre $X$ de
factures d'un arxivador; $X$ \'es un nombre a l'atzar entre 5 i 8.
Sigui $Y$ el temps en minuts que tarda en revisar-les. Suposem que
$(X,Y)$ t\'e una llei conjunta donada per

\[
P(X = x, Y = y) = \begin{cases}
{1 \over 4} \cdot {10-x \over x} \cdot \left ( {x \over 10} \right )^y & \text{si } x = 5,6,7,8, y =
1,2,\ldots \\ 0 & \text{en cas contrari}
\end{cases}
\]
\begin{enumerate}[a)]
\item Trobau la distribuci\'o marginal de $Y.$
\item Trobau la distribuci\'o condicional de $X $  donat que $Y = y.$
\item  Calculau la probabilitat que hagi triat 6 factures sabent que
ha tardat m\'es de 3 minuts en revisar-les. 
\end{enumerate}

\vskip 0.3 cm
\noindent
\textbf{Soluci\'o:} 

\vskip 0.2 cm
\noindent
\textbf{a)} Tenim que $\Omega_X=\{5,6,7,8\}$ i $\Omega_Y=\{ 1, 2, \cdots \}$. 
Tenim que $P(Y=y)=0$ si $y \notin \Omega_Y$ i per als valors $y \in \Omega_Y$:
\[
\begin{array}{rl}
P(Y=y)=&\sum_{x \in \Omega_X} P(X=x, Y=y)= \\ \\
      =&P(X=5, Y=y)+P(X=6, Y=y)+P(X=7, Y=y)+P(X=8, Y=y)=\\ \\
      =& \frac{1}{4} \frac{10-5}{5} ( \frac{5}{10} )^y + \frac{1}{4} \frac{10-6}{6} ( \frac{6}{10} )^y +
      \frac{1}{4} \frac{10-7}{7} ( \frac{7}{10} )^y + \frac{1}{4} \frac{10-8}{8} ( \frac{8}{10} )^y=\\ \\
      =& \frac{1}{4} \left[ (\frac{5}{10})^y + \frac{2}{3} ( \frac{6}{10} )^y + \frac{3}{7} ( \frac{7}{10} )^y
+ \frac{1}{4} ( \frac{8}{10} )^y \right]
\end{array}
\]

\vskip 0.2 cm
\noindent
\textbf{b)} 
\[
\begin{array}{rl}
P(X=x | Y=y)=&\frac{P(X=x, Y=y)}{P(Y=y)}= \\ \\
=&\begin{cases} 
\frac{ {1 \over 4} \cdot {10-x \over x} \cdot \left ( {x \over 10} \right )^y }{ \frac{1}{4} \left[ (\frac{5}{10})^y + \frac{2}{3} ( \frac{6}{10} )^y + \frac{3}{7} ( \frac{7}{10} )^y + \frac{1}{4} ( \frac{8}{10} )^y \right] } = 
\frac{10-x}{x} \frac{x^y}{ 5^y+ \frac{2}{3} 6^y + \frac{3}{7} 7^y + \frac{1}{4} 8^y}
& 
\text{si } x\in \Omega_X, y \in \Omega_Y \\ \\
0 & \text{altrament}
\end{cases}
\end{array}
\]

\vskip 0.2 cm
\noindent
\textbf{c)}
\[
P(X=6 | Y > 3) = \frac{P(X=6, Y > 3)}{P(Y > 3)}=\frac{\sum_{y=4}^\infty P(X=6, Y=y)}{1-P(Y \leq 3)}
\]

\[
\begin{array}{rl}
P(Y \leq 3)= & P(Y=1)+P(Y=2)+P(Y=3)=  \\ \\
=& \frac{1}{4} \left[ (\frac{5}{10})^1 +\frac{2}{3}(\frac{6}{10})^1 + \frac{3}{7} (\frac{7}{10})^1 +\frac{1}{4}(\frac{8}{10})^1 \right]+
 \frac{1}{4} \left[ (\frac{5}{10})^2 +\frac{2}{3}(\frac{6}{10})^2 + \frac{3}{7} (\frac{7}{10})^2 +\frac{1}{4}(\frac{8}{10})^2 \right] +
\\ \\
& + \frac{1}{4} \left[ (\frac{5}{10})^3 +\frac{2}{3}(\frac{6}{10})^3 + \frac{3}{7} (\frac{7}{10})^3 +\frac{1}{4}(\frac{8}{10})^3 \right] 
= \frac{7}{20} + \frac{86}{400} + \frac{544}{4000} = \frac{1196}{4000}
\end{array} 
\]

\[
\sum_{y=4}^\infty P(X=6, Y=y)= \sum_{y=4}^\infty {1 \over 4} \cdot {10-6 \over 6} \cdot \left ( {6 \over 10} \right )^y=
\frac{1}{4} \cdot \frac{4}{6}  \sum_{y=4}^\infty \left( \frac{3}{5} \right)^y = 
\frac{1}{6} \frac{ \left( \frac{3}{5} \right) ^4}{1-\frac{3}{5}} = \frac{5}{12} \left( \frac{3}{5} \right) ^4
\]
\noindent
on hem aplicat la f\`ormula per a la suma d'infinits termes d'una progressi\'o geom\`etrica: $S=\frac{a_0}{1-r}$, 
amb $a_0=(\frac{3}{5})^4$ i $r=\frac{3}{5} < 1$.

\vskip 0.2 cm
\noindent
La soluci\'o final \'es:
\[
P(X=6 | Y > 3) = \frac{ \frac{5}{12} \left( \frac{3}{5} \right) ^4 }{ 1- \frac{1196}{4000} } = \cdots = \frac{54}{299}=0.1806
\]

\newpage
\noindent
\textbf{Problema 15.} Una diana consisteix en 3 cercles conc\`entrics amb radis
respectius 1 cm, 2 cm i 3 cm. En una competici\'o de tir, la
distribuci\'o dels impactes en la diana i els voltants \'es tal que
les desviacions horitzontal i vertical respecte del centre de la
diana s\'on independents i les dues segueixen una distribuci\'o normal
N(0, 1 cm.). Determinau la proporci\'o d'impactes dins cada anell de
la diana. (Ind.: Utilitzau les coordenades polars).

\vskip 0.3 cm
\noindent
\textbf{Soluci\'o:} 

\vskip 0.2 cm
\noindent
Anomenam $A$, $B$ i $C$ cada un dels anells de la diana (veure la figura 1). 
Denotam $X$, $Y$ les coordenades de l'impacte en la diana. La dist\`ancia de
cada impacte al centre de la diana \'es $R=+\sqrt{X^2+Y^2}$. Definim una
nova variable $Z=X^2+Y^2$, per tant $R=+\sqrt{Z}$.
Tenim que:
\[
\begin{array}{rl}
P(\{\text{impacte en $A$}\})=&P(R \leq 1)=P(\sqrt{Z} \leq 1)=P(Z \leq 1)=F_Z(1) \\ \\
P(\{\text{impacte en $B$}\})=&P(1 < R \leq 2)=P(1 < \sqrt{Z} \leq 2)=P(1 < Z \leq 4)=F_Z(4)-F_Z(1)\\ \\
P(\{\text{impacte en $C$}\})=&P(2 < R \leq 3)=P(2 < \sqrt{Z} \leq 3)=P(4 < Z \leq 9)=F_Z(9)-F_Z(4)
\end{array}
\]
\noindent
on $F_Z(z)$ \'es la funci\'o de distribuci\'o de $Z=X^2+Y^2$.

\[
F_Z(z)=P(Z \leq z)=P(X^2+Y^2 \leq z)=P(A_z)=\iint_{A_z \cap S} f_{XY}(x, y) \, dx dy
\]

\noindent
$A_z$ \'es el conjunt de punts $(x, y)$ tals que $x^2+y^2 \leq z$ i $S$ \'es el suport de
la funci\'o de densitat conjunta de $X$ i $Y$.

\setcounter{figure}{0}
\begin{figure}[htbp]
\begin{center}
\begin{picture}(120, 120)(-60, -60)
\put(-60, 0){\vector(1, 0){120}}
\put(0, -60){\vector(0, 1){120}}
\put(0, 0){\circle{30}}
\put(0, 0){\circle{60}}
\put(0, 0){\circle{90}}
\put(0, 0){\line(1, 2){6.5}}
\put(0, 0){\line(-1, 2){13.5}}
\put(0, 0){\line(-2, 1){40}}
\put(6,3){1}
\put(-8,20){2}
\put(-35,20){3}
\put(-8,-9){A}
\put(-21, -18){B}
\put(-34, -29){C}
\end{picture}
\end{center}
\caption{Diana}
\end{figure}


\vskip 0.2 cm
\noindent
Com que $X$ i $Y$ s\'on v.a. independents i ambdues s\'on v.a. normals $N(0, 1)$, llavors:
\[
\begin{array}{rll}
f_X(x)=&\frac{1}{\sqrt{2\pi}} e^{-x^2/2} & \qquad \Omega_X=(-\infty, +\infty)=\R \\ 
f_Y(y)=&\frac{1}{\sqrt{2\pi}} e^{-y^2/2} & \qquad \Omega_Y=(-\infty, +\infty)=\R \\ 
f_{XY}(x, y)=&f_X(x) \cdot f_Y(y) = \frac{1}{2\pi} e^{-(x^2+y^2)/2} & \qquad \Omega_{XY}=S=\R^2
\end{array}
\]

\noindent
Per tant:
\[
F_Z(z)=\frac{1}{2\pi} \iint_{x^2+y^2 \leq z} e^{-(x^2+y^2)/2} \, dx dy
\]

\noindent
Si feim un canvi de coordenades $(x, y)$ a coordenades polars $(r, \theta)$:
\[
x=r \cos \theta \qquad y=r \sin \theta \qquad x^2+y^2=r^2 \qquad dxdy=rdrd\theta \qquad 
A_z=\{ r \leq \sqrt{z}, 0 \leq \theta < 2\pi \}
\]

\[
F_Z(z)=\frac{1}{2\pi} \int_0^{2\pi} \int_0^{\sqrt{z}} e^{-r^2/2} \, rdrd\theta = 
\text{(fent el canvi de variable $t=r^2/2$)}=\cdots=1-e^{-z/2}
\]

\noindent
De manera que:
\[
\begin{array}{rl}
P(\{\text{impacte en $A$}\})=&F_Z(1)=1-e^{-1/2}=0.3934 \\
P(\{\text{impacte en $B$}\})=&F_Z(4)-F_Z(1)=(1-e^{-4/2})-(1-e^{-1/2})=0.47067\\
P(\{\text{impacte en $C$}\})=&F_Z(9)-F_Z(4)=(1-e^{-9/2})-(1-e^{-4/2})=0.124
\end{array}
\]


\newpage
\noindent
\textbf{Problema 16.}  Siguin $X$ i $Y$ dues variables aleat\`ories amb densitat
conjunta 
\[
f_{XY}(x,y) = \left\{\begin{array}{ll}\frac{1}{x} &
\mbox{si } 0 \leq y \leq x \leq 1\\ 0 & \mbox{en cas
contrari}\end{array}\right.
\]
\begin{enumerate}[a)]
\item Calculau $f_X(x)$.
\item Calculau $f_{Y}(y|x)$.
\item Calculau $E(Y)$. 
\item Calculau $E(Y|X)$, la ra\'o de correlaci\'on $\eta_{Y|X}$ i ECM. 
\item Calculau $E(E(Y | X))$ i comprovau que
coincideix amb $E(Y)$.
\item Estimeu $Y$ si sabem que $X=0.75$.
\end{enumerate}

\vskip 0.3 cm
\noindent
\textbf{Soluci\'o:} 

\vskip 0.2 cm
\noindent
El suport de $f_{XY}$ es mostra en la figura seg\"uent:

\vskip 0.2 cm
\setcounter{figure}{0}
\begin{figure}[htbp]
\begin{center}
\begin{picture}(100, 100)
\put(10, 10){\vector(1, 0){90}}
\put(10, 10){\vector(0, 1){90}}
\put(95, 0){$x$}
\put(-5, 95){$y$}
\put(10, 10){\line(1, 1){70}}
\put(75, 70){$y=x$}
\multiput(10, 60)(10, 0){6}{\line(1, 0){5}}
\thicklines
\put(10, 10){\line(1, 1){50}}
\put(60, 10){\line(0, 1){50}}
\thinlines
\put(20, 10){\line(0, 1){10}}
\put(25, 10){\line(0, 1){15}}
\put(30, 10){\line(0, 1){20}}
\put(35, 10){\line(0, 1){25}}
\put(40, 10){\line(0, 1){30}}
\put(45, 10){\line(0, 1){35}}
\put(50, 10){\line(0, 1){40}}
\put(55, 10){\line(0, 1){45}}
\put(60, 0){\textbf{\textit{1}}}
\put(0, 60){\textbf{\textit{1}}}
\put(50, 40){\textbf{\textit{S}}}
\end{picture}
\end{center}
\caption{Suport de $f_{XY}(x, y)$ (zona ratxada). }
\end{figure}

\vskip 0.2 cm
\noindent
\textbf{a)} 
\[
f_X(x)=\int_{-\infty}^{+\infty} f_{XY}(x, y) \, dy=\begin{cases}
\text{si } x \leq 0 & 0\\
\text{si } x > 1 & 0 \\
\text{si } 0 < x \leq 1 & \int_0^x \frac{1}{x} \, dy = \frac{1}{x} \left( y \right]_0^x = \frac{x-0}{x}=1
\end{cases}
\]
\noindent
Aquesta funci\'o de densitat correspon a una v.a. uniforme en l'interval $(0, 1)$, per tant: $X \sim {\cal U}(0, 1)$.

\vskip 0.2 cm
\noindent
\textbf{b)} 
\[
f_{Y|X}(y|x)=\frac{f_{XY}(x, y)}{f_X(x)}=\begin{cases}
\text{si } (x, y) \in S \, \Leftrightarrow \, 0 \leq y \leq x \leq 1 & \qquad \frac{1/x}{1}=\frac{1}{x} \\
\text{en cas contrari} & \qquad 0
\end{cases}
\]
\noindent
Aquesta funci\'o de densitat correspon a una v.a. uniforme en l'interval $(0, x)$, per tant: $Y|_X \sim {\cal U}(0, X)$.

\vskip 0.2 cm
\noindent
\textbf{c)} $E(Y)=\int_{-\infty}^{+\infty} y f_Y(y) \, dy$, d'altra banda,
\[
f_Y(y)=\int_{-\infty}^{+\infty} f_{XY}(x, y) \, dx=\begin{cases}
\text{si } y \leq 0 & 0\\
\text{si } y > 1 & 0 \\
\text{si } 0 < y \leq 1 & \int_y^1 \frac{1}{x} \, dx = \left( \mathrm{ln} x \right]_y^1=
(\mathrm{ln} 1 - \mathrm{ln} y)= -\mathrm{ln}y
\end{cases}
\]

\noindent
Per tant:
\[
E(Y)=\int_0^1 - y \mathrm{ln}y \, dy= \left( \text{per parts: }
\left\{ \begin{array}{ll} u=\mathrm{ln} y & du= \frac{1}{y} dy \\ dv=y\, dy & v=\frac{y^2}{2} \end{array} \right\} \right)
 =\cdots=
\left( \frac{y^2}{2} (\frac{1}{2}-\mathrm{ln}y) \right]_0^1=\cdots=\frac{1}{4}
\]

\vskip 0.2 cm
\noindent
\textbf{d)} El suport de $f_{Y|X}(y|x)$ \'es el mateix que el de $f_{XY}(x, y)$ (veure figura 1), per tant:
\[
E(Y|X)=\int_{-\infty}^{+\infty} y f_{Y|X}(y|x) \, dy =\begin{cases}
\text{si } 0 < x \leq 1 & \int_0^x y \cdot \frac{1}{x} \, dy = \frac{1}{x} \left( \frac{y^2}{2}\right]_0^x = 
\frac{1}{x} \frac{x^2}{2}=\frac{x}{2}\\
\text{en cas contrari} & 0 \end{cases}
\]

\vskip 0.2 cm
\noindent
Si denotam $Y^*=E(Y|X)$, llavors l'error quadr\`atic mitj\`a (ECM) \'es $\mathrm{ECM}=E((Y^* - Y)^2)$
i la ra\'o de correlaci\'o $\eta_{Y|X} = \frac{\mathrm{Var}(Y^*)}{\mathrm{Var}(Y)}=1-\frac{\mathrm{ECM}}{\mathrm{Var}(Y)}$,
per tant $\mathrm{ECM}=\mathrm{Var}(Y)(1-\eta_{Y|X})$.
\[
\begin{array}{ll}
E(Y) &=  \text{(apartat c)}=\frac{1}{4} \\ \\
E(Y^2) & = \int_0^1 - y^2 \mathrm{ln}y \, dy= \left( \text{per parts: }
\left\{ \begin{array}{ll} u=y \mathrm{ln} y & du= (\mathrm{ln}y+1) dy \\ dv=y\, dy & v=\frac{y^2}{2} \end{array} 
\right\} \right)= \\ 
&= \cdots= \left(  \frac{y^3}{3} (\frac{1}{3}-\mathrm{ln}y) \right]_0^1=\cdots=\frac{1}{9} \\ \\
\mathrm{Var}(Y) &=E(Y^2)-E(Y)^2 = \frac{1}{9} - \frac{1}{16}=\frac{7}{144}\\ \\
E(Y^*) & =E(E(Y|X))=\text{(apartat e)}=E(Y)=\frac{1}{4} \\ \\
E({Y^*}^2) & = \int_{-\infty}^{+\infty} E(Y|X)^2 f_X(x) \, dx = \int_0^1 \frac{x^2}{4} \, dx = 
\left( \frac{x^3}{12} \right]_0^1 = \frac{1}{12} \\ \\
\mathrm{Var}(Y^*) &=E({Y^*}^2)-E(Y^*)^2 = \frac{1}{12} - \frac{1}{16} =\frac{1}{48} \\ \\
\eta_{Y|X} & = \frac{\mathrm{Var}(Y^*)}{\mathrm{Var}(Y)}= \frac{1/48}{7/144}=\frac{3}{7} \\ \\
\mathrm{ECM}& =\mathrm{Var}(Y)(1-\eta_{Y|X}) = \frac{7}{144} (1-\frac{3}{7})=\frac{1}{36}
\end{array}
\]

\vskip 0.2 cm
\noindent
\textbf{e)} 
\[
E(E(Y|X))=\int_{-\infty}^{+\infty} E(Y|X) f_X(x) \, dx = \int_0^1 \frac{x}{2} \, dx = \left( \frac{x^2}{4} \right]_0^1=
\frac{1}{4}
\]
\noindent
Observem que aquest resultat coincideix amb $E(Y)$ (recordem que $E(E(Y|X))=E(Y)$).

\vskip 0.2 cm
\noindent
\textbf{f)} $E(Y|X=0.75)=\text{(a partir del resultat de l'apartat d)}=\frac{0.75}{2}= \frac{3}{8}$.



\newpage
\noindent
\textbf{Problema 17.} 
Considerem un experiment amb tres possibles resultats $E_1,
E_2 \mbox{ i } E_3$ amb probabilitats respectives $p,q$ i $r$ tals
que $p+q+r = 1.$ En una seq\"{u}\`encia de $n$ proves independents de
l'experiment, denotem per $X$ el nombre de vegades que ocorr
$E_1$, i per $Y$ el nombre de vegades que ocorr $E_2$. Es sap que
el vector $(X,Y)$ t\'e una distribuci\'o anomenada \textbf{trinomial}
que ve donada per: $$P(X=i,Y=j) = {n! \over i! \> j! \> k!} \>
p^i \> q^j \> r^k \> , \ \ \ k = n-i-j$$
\begin{enumerate}[a)]
\item Provau que les distribucions marginals s\'on binomials.
\item Determinau el vector mitjana i la matriu de covari\`ancies de
$(X,Y)$.
\item Provau que les distribucions condicionals s\'on binomials.
\item Obteniu l'expressi\'o de $E(Y | X = i) \ \ \forall i = 0,
 \ldots , n$.
\item Comprovau que el coeficient de correlaci\'o entre $E(Y | X)$
i $X$ \'es igual a -1 i, per tant, $E(Y | X)$ \'es una funci\'o lineal
de $X$, $E(Y | X) = a X + b$, amb $a < 0$.
\item Determinau els coeficients $a$ i $b$ a partir de
l'expressi\'o obtinguda a l'apartat (d).
\end{enumerate}

\vskip 0.3 cm
\noindent
\textbf{Soluci\'o:} 

\vskip 0.2 cm
\noindent
\textbf{a)} Per a valors de $i \notin \{0, 1, 2, \cdots, n\}$, $P(X=i)=0$.
Per a valors $i \in \{0, 1, 2, \cdots, n\}$:
\[
P(X=i)=\sum_{j \in \Omega_Y} P(X=i, Y=j)= \sum_{j=0}^{n-i} P(X=i, Y=j) = 
\sum_{j=0}^{n-i} \frac{n!}{i! j! (n-i-j)!} p^i q^j r^{(n-i-j)}
\]
\noindent
on s'ha tengut en compte que si $X=i$ llavors $Y$ no pot prendre un valor
superior a $n-i$.
\[
P(X=i)=\frac{n!}{i!} p^i \sum_{j=0}^{n-i} \frac{1}{j! (n-i-j)!} q^j r^{(n-i-j)} =
 \frac{n!}{i! (n-i)!} p^i \sum_{j=0}^{n-i} \binom{n-i}{j} q^j r^{(n-i-j)}
\]
\noindent
on s'ha aplicat la igualtat: $\binom{n-i}{j}=\frac{(n-i)!}{j! (n-i-j)!}$, per tant:
$\frac{1}{j! (n-i-j)!}=\frac{1}{(n-i)!} \binom{n-i}{j}$.

\vskip 0.2cm
\noindent
Si ara aplicam la definici\'o del binomi de Newton: 
$(q+r)^{(n-i)}=\sum_{j=0}^{n-i} \binom{n-i}{j} q^j r^{(n-i-j)}$, obtenim:
\[
P(X=i)= \frac{n!}{i! (n-i)!} p^i (q+r)^{(n-i)}
\]
\noindent
i com que $q+r=1-p$, i a m\'es $\frac{n!}{i! (n-i)!}=\binom{n}{i}$, llavors:
\[
P(X=i)=\binom{n}{i} p^i (1-p)^{(n-i)}
\]
\noindent
Aquesta expressi\'o correspon a la funci\'o de probabilitat d'una v.a. 
binomial amb par\`ametres $n$ i $p$. Per tant podem concloure que $X \sim B(n, p)$.

\vskip 0.3 cm
\noindent
Seguint un raonament similar per a la funci\'o de probabilitat marginal de $Y$ 
arribariem a: $Y \sim B(n, q)$.

\vskip 0.3 cm
\noindent
\textbf{b)} 
Vector de mitjanes: $(E(X), E(Y))$. A l'apartat anterior s'ha demostrat que $X \sim B(n, p)$ 
i $Y \sim B(n, q)$. Per tant, mirant les taules d'esperances per a una v.a. binomial trobam:
$E(X)=n p$, $E(Y)=nq$.

\vskip 0.3 cm
Matriu de covari\`ancies: 
$K=\begin{pmatrix} \sigma_{XX} & \sigma_{XY} \\ \sigma_{XY} & \sigma_{YY} \end{pmatrix}$.
\vskip 0.3 cm
Tenim que $\sigma_{XX}=\mathrm{Cov}(X, X)=\mathrm{Var}(X)=\sigma_X^2=\text{(taules)}=np(1-p)$ i
$\sigma_{YY}=\mathrm{Cov}(Y, Y)=\mathrm{Var}(Y)=\sigma_Y^2=\text{(taules)}=nq(1-q)$. 

$$\sigma_{XY}=\mathrm{Cov}(X, Y)=\mathrm{Cov}(Y, X)=E(XY)-E(X)E(Y)$$

\[
E(XY)=\sum_{i=0}^n \sum_{j=0}^n i j P(X=i, Y=j)
\]

Observam que quan $i=0$ o $j=0$ el factor $i j P(X=i, Y=j)$ ser\`a igual a zero, per tant basta comen\c{c}ar els
sumatoris amb el valor 1. Cal observar tamb\'e que el segon sumatori no pot arribar fins a $n$, ja que, si s'obtenen $i$
resultats de tipus $E_1$, nom\'es es poden obtenir un m\`axim de $n-i$ resultats de tipus $E_2$. 
Pel mateix motiu, com el segon sumatori comen\c{c}a en $j=1$, el primer pot prendre com a m\`axim el valor $n-1$.

A m\'es, podem fer les substitucions: $k=n-i-j$ i $r=1-p-q$. 

\[
\begin{split}
E(XY) & = \sum_{i=1}^{n-1} \sum_{j=1}^{n-i} i j {n! \over i!  j!  k!}  p^i  q^j  r^k = \\ 
&= \sum_{i=1}^{n-1} i {n! \over i!} p^i \sum_{j=1}^{n-i} j {1 \over j! (n-i-j)!} q^j  (1-p-q)^{n-i-j} = \\ 
&= \sum_{i=1}^{n-1} {n! \over (i-1)!} p^i \sum_{j=1}^{n-i} {1 \over (j-1)! (n-i-j)!} q^j  (1-p-q)^{n-i-j} =
\end{split}
\]

Descomposam $q^j$ en $q^{j-1} q$ i feim el canvi de variable $m=j-1$. 

\[
E(XY)= q \sum_{i=1}^{n-1} {n! \over (i-1)!} p^i \sum_{m=0}^{n-i-1} {1 \over m! (n-i-m-1)!} q^m  (1-p-q)^{n-i-m-1}
\]

Utilitzam la relaci\'o: $\binom{n-i-1}{m}=\frac{(n-i-1)!}{m! (n-i-1-m)!}$.

\[
E(XY)= q \sum_{i=1}^{n-1} {n! \over (i-1)! (n-i-1)!} p^i \sum_{m=0}^{n-i-1} \binom{n-i-1}{m} q^m  (1-p-q)^{n-i-m-1}
\]

Observam que el segon sumatori \'es igual al binomi de Newton: \[ (q + (1-p-q) )^{n-i-1} = (1-p)^{n-i-1} \]

Per tant:

\[
E(XY)= q \sum_{i=1}^{n-1} {n! \over (i-1)! (n-i-1)!} p^i (1-p)^{n-i-1}
\]

Descomposam $p^i$ en $p^{i-1} p$ i feim el canvi de variable $s=i-1$. 

\[
\begin{split}
E(XY)& = qp \sum_{s=0}^{n-2} {n! \over s! (n-s-2)!} p^s (1-p)^{n-s-2} = \\
&= qp n (n-1) \sum_{s=0}^{n-2} {(n-2)! \over s! (n-s-2)!} p^s (1-p)^{n-s-2}
\end{split}
\]

Aplicant l'expressi\'o del binomi de Newton a aquest sumatori obtenim:

\[
E(XY)=qp n (n-1) (p+1-p)^{n-2}=qpn(n-1) 1^{n-2}=qpn(n-1)
\]


En conclusi\'o: $$\sigma_{XY}=qpn(n-1)-qpn^2=-qpn$$

i la matriu de covari\`ancies queda:
\[
K=\begin{pmatrix} np(1-p) & -npq \\ -npq & nq(1-q) \end{pmatrix}
\]


\vskip 0.2 cm
\noindent
\textbf{c)}
\[
P(X=i | Y=j)=\frac{ P(X=i, Y=j) }{P(Y=j)} = 
\begin{cases}
\frac{ \frac{n!}{i! j! (n-i-j)!} p^i q^j r^{(n-i-j)} }{ \frac{n!}{j! (n-j)!} q^j (1-q)^{(n-j)}} & 
\text{si } j \in \{ 0, 1, \cdots, n \}, i \in \{ 0, 1, \cdots, n-j\} \\
0 & \text{altrament}
\end{cases}
\] 

\noindent
simplificant l'expressi\'o anterior:
\[
\begin{array}{rl}
P(X=i | Y=j)=&\frac{(n-j)!}{i! (n-i-j)!} p^i \frac{r^{n-i-j}}{(1-q)^{n-j}}=
\binom{n-j}{i} p^i \frac{r^{n-i-j}}{(1-q)^{(n-i-j)} (1-q)^i}=
\binom{n-j}{i} \left( \frac{p}{1-q} \right)^i \left( \frac{r}{1-q} \right)^{(n-i-j)}= \\ \\
=& \binom{n-j}{i} \left( \frac{p}{1-q} \right)^i \left( \frac{1-p-q}{1-q} \right)^{(n-i-j)}=
\binom{n-j}{i} \left( \frac{p}{1-q} \right)^i \left( 1 - \frac{p}{1-q} \right)^{(n-i-j)}
\end{array}
\]
\noindent
per a valors $j \in \{ 0, 1, \cdots, n \}$, $i \in \{ 0, 1, \cdots, n-j\}$. 
Aquesta funci\'o de probabilitat correspon a una v.a. binomial amb par\`ametres
$n-j$ i $\frac{p}{1-q}$, per tant podem concloure:
\[
X|_{Y=j} \sim B(n-j, \frac{p}{1-q})
\]

\noindent
De manera similar demostrariem: $Y|_{X=i} \sim B(n-i, \frac{q}{1-p})$

\vskip 0.2 cm
\noindent
\textbf{d)} Com que $X|_{Y=j} \sim B(n-j, \frac{p}{1-q})$ i $Y|_{X=i} \sim B(n-i, \frac{q}{1-p})$ 
mirant les taules de moments de la v.a. binomial tenim que:
\[
\begin{array}{rll}
E(X|_{Y=j}) &= (n-j) \frac{p}{1-q} & \text{si } j=0, 1, \cdots, n \\ \\
\mathrm{Var}(X|_{Y=j}) &= (n-j) \frac{p}{1-q} (1-\frac{p}{1-q}) & \text{si } j=0, 1, \cdots, n \\ \\
E(Y|_{X=i}) &=(n-i) \frac{q}{1-p} & \text{si } i=0, 1, \cdots, n \\ \\
\mathrm{Var}(Y|_{X=i}) &=(n-i) \frac{q}{1-p} (1-\frac{q}{1-p}) & \text{si } i=0, 1, \cdots, n 
\end{array}
\]

\vskip 0.2 cm
\noindent
\textbf{e)} Si denotam $Y^*=E(Y|_X)$, llavors
\[
\eta_{E(Y|_X) X}=\eta_{Y^*X}=\frac{ \mathrm{Cov}(Y^*, X) }{\sqrt{\mathrm{Var}(Y^*)} \sqrt{\mathrm{Var}(X)} }
\]

\noindent
on $\mathrm{Var}(X)=np(1-p)$.

\vskip 0.2 cm
\noindent
D'altra banda tenim que $Y^*=E(Y|_X)=(n-X) \frac{q}{1-p}$, per tant
\[ 
\begin{array}{rl}
E(Y^*)& =E(E(Y|_X))=\text{(propietat)}=E(Y)=nq \\ \\
\mathrm{Var}(Y^*)& =\mathrm{Var}\left( (n-X) \frac{q}{1-p} \right)= 
\mathrm{Var}\left( n \frac{q}{1-p} - \frac{q}{1-p} X \right) = \\
 &= \frac{q^2}{(1-p)^2} \mathrm{Var}(X)=  \frac{q^2}{(1-p)^2}  np(1-p) = \frac{npq^2}{1-p} \\ \\
\mathrm{Cov}(Y^*, X) &= E(Y^*X)-E(Y^*)E(X) \\ \\
E(Y^*X) &= E \left(  (n-X) \frac{q}{1-p} X \right) =n \frac{q}{1-p} E(X) - \frac{q}{1-p} E(X^2) =
n \frac{q}{1-p}  np - \frac{q}{1-p} (np-np^2+n^2p^2) = \\ 
&= \frac{q}{1-p} ( n^2p - np+np^2-n^2p^2 )=\frac{npq}{1-p} (n-1+p-np)
\end{array}
\]
\noindent
on s'ha aplicat que $E(X^2)=\mathrm{Var}(X)+E(X)^2= np(1-p)+(np)^2$.

\vskip 0.2 cm
\noindent
Simplificant les expressions anteriors obtenim: $\mathrm{Cov}(Y^*, X) =-npq$,
de manera que:

\[
\eta_{E(Y|_X) X}=\frac{-npq}{\sqrt{\frac{npq^2}{1-p}} \sqrt{np(1-p)}} = \frac{-npq}{npq} = -1
\]

\noindent
que \'es el que ens demanaven demostrar.


\vskip 0.2 cm
\noindent
\textbf{f)} $E(Y|_X)=(n-X) \frac{q}{1-p}=-\frac{q}{1-p} X + n \frac{q}{1-p}=aX+b$, on $a=-\frac{q}{1-p}$
i $b=n \frac{q}{1-p}$.

\newpage
\noindent
\textbf{Problema 18.} 
En l'exercici 1,
\begin{enumerate}[a)]
\item Determinau $E(X | Y)$. 
\item Calculau $E(E(X | Y))$, i comprovau que
coincideix amb $E(X)$. 
\end{enumerate}

\vskip 0.3 cm
\noindent
\textbf{Soluci\'o:}

\vskip 0.2 cm
\noindent
\textbf{a)} $E(X | Y)=E(X|_{Y=y})=\sum_{x \in \Omega_X} x \cdot P(X=x | Y=y)$
on $P(X=x | Y=y)= \frac{P(X=x, Y=y)}{P(Y=y)}$.
A partir de les taules de les funcions de probabilitat conjunta i marginals
(veure soluci\'o del problema 1) obtenim la seg\"uent taula per a la
funci\'o de probabilitat condicionada:

\vskip 0.2 cm
\begin{center}
\begin{tabular}{c|c|c|c|c|c|c|}
Y $\backslash$ X & 1 & 2 & 3 & 4 & 5 & 6  \\ \hline
1     & $\frac{1/36}{1/36}=1$ &    0 &    0 &    0 &    0 & 0  \\ \hline
2     & $\frac{1/36}{3/36}=1/3$ & $\frac{2/36}{3/36}=2/3$ &    0 &    0 &    0 & 0  \\ \hline
3     & $\frac{1/36}{5/36}=1/5$ & $\frac{1/36}{5/36}=1/5$ & $\frac{3/36}{5/36}=3/5$ &    0 &    0 & 0  \\ \hline
4     & $\frac{1/36}{7/36}=1/7$ & $\frac{1/36}{7/36}=1/7$ & $\frac{1/36}{7/36}=1/7$ & $\frac{4/36}{7/36}=4/7$ &    0 & 0  \\ \hline
5     & $\frac{1/36}{9/36}=1/9$ & $\frac{1/36}{9/36}=1/9$ & $\frac{1/36}{9/36}=1/9$ & $\frac{1/36}{9/36}=1/9$ & $\frac{5/36}{9/36}=5/9$
 & 0  \\ \hline
6     & $\frac{1/36}{11/36}=1/11$ & $\frac{1/36}{11/36}=1/11$ & $\frac{1/36}{11/36}=1/11$ & $\frac{1/36}{11/36}=1/11$
 & $\frac{1/36}{11/36}=1/11$ & $\frac{6/36}{11/36}=6/11$  \\ \hline
\end{tabular}
\end{center}

\vskip 0.2 cm
\noindent
Aplicant la f\`ormula de l'esperan\c{c}a condicionada obtenim, per a cada valor de $Y$:
\vskip 0.2 cm
\begin{center}
\begin{tabular}{c|l|}
Y & $E(X|_{Y=y})$ \\ \hline
1 & $1 \cdot 1 = 1$ \\ \hline 
2 & $1 \cdot \frac{1}{3} + 2 \cdot \frac{2}{3}=\frac{5}{3}$ \\ \hline
3 & $1 \cdot \frac{1}{5} + 2 \cdot \frac{1}{5} + 3 \cdot \frac{3}{5} =\frac{12}{5}$ \\ \hline
4 & $1 \cdot \frac{1}{7} + 2 \cdot \frac{1}{7} + 3 \cdot \frac{1}{7} + 4 \cdot \frac{4}{7} =\frac{22}{7}$ \\ \hline
5 & $1 \cdot \frac{1}{9} + 2 \cdot \frac{1}{9} + 3 \cdot \frac{1}{9} + 4 \cdot \frac{1}{9} + 5 \cdot \frac{5}{9} =\frac{35}{9}$ \\ \hline
6 & $1 \cdot \frac{1}{11} + 2 \cdot \frac{1}{11} + 3 \cdot \frac{1}{11} + 4 \cdot \frac{1}{11} + 5 \cdot \frac{1}{11} 
+ 6 \cdot \frac{6}{11} =\frac{51}{11}$ \\ \hline
\end{tabular}
\end{center}

\vskip 0.3 cm
\noindent
\textbf{b)} 
\[
\begin{array}{rl}
E(E(X|Y)) = & E(X|_{Y=1}) \cdot P(Y=1) + E(X|_{Y=2}) \cdot P(Y=2) + E(X|_{Y=3}) \cdot P(Y=3) + \\ \\
 & + E(X|_{Y=4}) \cdot P(Y=4) + E(X|_{Y=5}) \cdot P(Y=5) + E(X|_{Y=6}) \cdot P(Y=6) = \\ \\
 = & 1 \cdot \frac{1}{36} + \frac{5}{3} \cdot \frac{3}{36} + \frac{12}{5} \cdot \frac{5}{36} + \frac{22}{7} \cdot \frac{7}{36}
 + \frac{35}{9} \cdot \frac{9}{36} + \frac{51}{11} \cdot \frac{11}{36} = \frac{7}{2}
\end{array}
\]

\noindent
D'altra banda:
\[
\begin{array}{rl}
E(X) & = 1 \cdot P(X=1) + 2 \cdot P(X=2) + 3 \cdot P(X=3) + 4 \cdot P(X=4) + 5 \cdot P(X=5) + 6 \cdot P(X=6) = \\ \\
 & = 1 \cdot \frac{1}{36} + 2 \cdot \frac{1}{36} + 3 \cdot \frac{1}{36} + 4 \cdot \frac{1}{36} + 5 \cdot \frac{1}{36} + 6 \cdot \frac{1}{36} =
\frac{7}{2}
\end{array}
\]

\newpage
\noindent
\textbf{Problema 22.}
Es llancen a l'aire dos daus sense biaix. Siguin $N_1 \mbox{
i } N_2$ els valors obtinguts en els dos daus. Posem $X = N_1 +
N_2$ i $Y = \ \mid N_1 - N_2 \mid.$ Obteniu la distribuci\'o
conjunta de $X$ i $Y$ i comprovau que $X$ i $Y$ estan
incorrelacionades. S\'on independents?
\vskip 0.3 cm
\noindent
\textbf{Soluci\'o:} 

\vskip 0.2 cm
\noindent
Tenim que $\Omega_{N_1}=\Omega_{N_2}=\{ 0, 1, 2, 3, 4, 5, 6 \}$ i que $P(N_1=i)P(N_2=i)=\frac{1}{6}$
si $i \in \Omega_{N_1}$, ja que els daus no tenen biaix. A m\'es, com poden considerar que els
daus s\'on independents, llavors $P(N_1=i, N_2=j)=P(N_1=i) \cdot P(N_2=j)$.

\vskip 0.2 cm
\noindent
Cada parell de valors $(N_1, N_2)$ genera un parell de valors $(X, Y)$. El conjunt de tots els valors
$(X, Y)$ possibles es mostra en la taula seg\"uent:
\vskip 0.2 cm
\begin{center}
\begin{tabular}{c|c|c|c|c|c|c|}
$N_2 \backslash N_1$ & $1$ & $2$ & $3$ & $4$ & $5$ & $6$ \\ \hline
$1$ & $X=2$ & $X=3$ & $X=4$ & $X=5$ & $X=6$ & $X=7$  \\ 
    & $Y=0$ & $Y=1$ & $Y=2$ & $Y=4$ & $Y=5$ & $Y=6$  \\ \hline
$2$ & $X=3$ & $X=4$ & $X=5$ & $X=6$ & $X=7$ & $X=8$  \\ 
    & $Y=1$ & $Y=0$ & $Y=1$ & $Y=2$ & $Y=3$ & $Y=4$  \\ \hline
$3$ & $X=4$ & $X=5$ & $X=6$ & $X=7$ & $X=8$ & $X=9$  \\ 
    & $Y=2$ & $Y=1$ & $Y=0$ & $Y=1$ & $Y=2$ & $Y=3$  \\ \hline
$4$ & $X=5$ & $X=6$ & $X=7$ & $X=8$ & $X=9$ & $X=10$  \\ 
    & $Y=3$ & $Y=2$ & $Y=1$ & $Y=0$ & $Y=1$ & $Y=2$  \\ \hline
$5$ & $X=6$ & $X=7$ & $X=8$ & $X=9$ & $X=10$ & $X=11$  \\ 
    & $Y=4$ & $Y=3$ & $Y=2$ & $Y=1$ & $Y=0$ & $Y=1$  \\ \hline
$6$ & $X=7$ & $X=8$ & $X=9$ & $X=10$ & $X=11$ & $X=12$  \\ 
    & $Y=5$ & $Y=4$ & $Y=3$ & $Y=2$ & $Y=1$ & $Y=0$  \\ \hline
\end{tabular}
\end{center}
 
\noindent
Per tant, el conjunt de valors possibles per a $X$ i $Y$ s\'on: $\Omega_X=\{ 2, 3, \cdots, 12 \}$ i 
$\Omega_Y=\{ 0, 1, \cdots, 5 \}$. I les probabilitats de cada parell $(x, y)$ es calculen a partir
de les probalitats de $N_1$ i $N_2$ observant la taula anterior. Per exemple: 
\[
\begin{array}{l}
P(X=2, Y=0)= P(N_1=1, N_2=1)= P(N_1=1) \cdot P(N_2=1)=\frac{1}{36} \\ \\
P(X=2, Y=1)= 0 \\ \\
P(X=3, Y=0)= 0 \\ \\
P(X=3, Y=1)=P(N_1=2, N_2=1)+P(N_1=1, N_2=2)=\frac{1}{36}+\frac{1}{36}=\frac{2}{36} \\
\qquad \qquad \vdots \qquad \qquad \vdots \qquad \qquad \vdots
\end{array}
\]

\noindent
Els valors de la funci\'o de probabilitat conjunta de $(X, Y)$ es mostren en la taula seg\"uent. Tamb\'e 
es mostren els valors de les probabilitats marginals.
\vskip 0.1 cm
\begin{center}
\begin{tabular}{c|c|c|c|c|c|c|c|c|c|c|c||c}
$Y \backslash X$ & $2$ & $3$ & $4$ & $5$ & $6$ & $7$ & $8$ & $9$ & $10$ & $11$ & $12$ & $P(X=i)$ \\ \hline
$0$ & $\frac{1}{36}$ & $0$ & $\frac{1}{36}$ & $0$ & $\frac{1}{36}$ &  $0$ & $\frac{1}{36}$ & $0$ & $\frac{1}{36}$ & 
$0$ & $\frac{1}{36}$ & $\frac{6}{36}$ \\ \hline
$1$ & $0$ & $\frac{2}{36}$ & $0$ & $\frac{2}{36}$ & $0$ & $\frac{2}{36}$ &  $0$ & $\frac{2}{36}$ & $0$ & $\frac{2}{36}$ & 
$0$  & $\frac{10}{36}$ \\ \hline
$2$ & $0$ & $0$ & $\frac{2}{36}$ & $0$ & $\frac{2}{36}$ & $0$ & $\frac{2}{36}$ &  $0$ & $\frac{2}{36}$ & $0$ & $0$  & $\frac{8}{36}$ \\ \hline
$3$ & $0$ & $0$ & $0$ & $\frac{2}{36}$ & $0$ & $\frac{2}{36}$ & $0$ & $\frac{2}{36}$ & $0$ & $0$ & $0$  & $\frac{6}{36}$ \\ \hline
$4$ & $0$ & $0$ & $0$ & $0$ & $\frac{2}{36}$ & $0$ & $\frac{2}{36}$ & $0$ & $0$ & $0$ & $0$  & $\frac{4}{36}$ \\ \hline
$5$ & $0$ & $0$ & $0$ & $0$ & $0$ & $\frac{2}{36}$ & $0$ & $0$ & $0$ & $0$ & $0$  & $\frac{2}{36}$ \\ \hline
$P(Y=j)$ & $\frac{1}{36}$ & $\frac{2}{36}$ & $\frac{3}{36}$ & $\frac{4}{36}$ & $\frac{5}{36}$ & 
$\frac{6}{36}$ & $\frac{5}{36}$ & $\frac{4}{36}$ & $\frac{3}{36}$ & $\frac{2}{36}$ & $\frac{1}{36}$ & \\ \hline
\end{tabular}
\end{center}

\vskip 0.1 cm
\noindent
Podem comprovar que $X$ i $Y$ no s\'on independents ja que, per exemple, 
$P(X=2, Y=2)=0 \neq P(X=2) \cdot P(Y=2) = \frac{2}{36} \cdot \frac{8}{36}$.

\vskip 0.1 cm
\noindent
Per comprovar si $X$ i $Y$ s\'on incorrelades hem de demostrar que $\mathrm{Cov}(X, Y)=0$, i per tant $E(XY)=E(X) \cdot E(Y)$.
\[
\begin{array}{rl}
E(XY) & =\sum_{i \in \Omega_X} \sum_{j \in \Omega_Y} i \cdot j \cdot P(X=i, Y=j) =  \\ 
      & =0 \cdot 2 \cdot \frac{1}{36} +  0 \cdot 4 \cdot \frac{1}{36} + \cdots + 
      1 \cdot 3 \cdot \frac{2}{36} + 1 \cdot 5 \cdot \frac{2}{36} + \cdots + 5 \cdot 7 \cdot \frac{2}{36}=\frac{490}{36} \\
E(X) &= \sum_{i \in \Omega_X} i \cdot P(X=i) = 2 \cdot \frac{1}{36} + 3 \cdot \frac{2}{36} + \cdots + 12 \cdot \frac{1}{36}=7 \\
E(Y) &= \sum_{j \in \Omega_Y} j \cdot P(Y=j) = 0 \cdot \frac{6}{36} + 1 \cdot \frac{10}{36} + \cdots + 5 \cdot \frac{2}{36}=\frac{70}{36}
\end{array}
\]      

\noindent
Comprovam que $E(X) \cdot E(Y)=7 \cdot \frac{70}{36}=\frac{490}{36}=E(XY)$, per tant les variables estan \textbf{incorrelades}.


\newpage
\noindent
\textbf{Problema 23.}
El nombre de clients que arriben a una estaci\'o de servei
durant un temps {\it t} \'es una variable aleat\`oria de Poisson amb
par\`ametre $\beta \cdot t.$ El temps necessari per servir cada
client \'es una variable aleat\`oria exponencial amb par\`ametre
$\alpha$. Determinau la llei de la variable aleat\`oria que d\'ona el
nombre de clients que arriben durant el temps de servei {\it T}
d'un determinat client. Es suposa que les arribades de clients s\'on
independents del temps de servei dels clients. (Nota: $
\displaystyle \> \int_o^\infty r^k \> e^{-r} \> dr = \Gamma(k+1) =
k!$)

\vskip 0.3 cm
\noindent
\textbf{Soluci\'o:} Denotam:

$X_t$: nombre de clients que arriben a una estaci\'o de servei en un temps $t$, $X_t \sim \mathrm{Po}(\beta t)$

$T$: temps necessari per a servir a cada client, $T \sim \mathrm{Exp}(\alpha)$

$Z$: nombre de clients que arriben durant el temps de servei $T$ d'un client


\noindent
$T$ \'es un v.a. cont\'\i nua amb funci\'o
de densitat:
\[
f_T(t)=\begin{cases} \alpha e^{-\alpha t} & \text{si } t > 0 \\ 0 & \text{si } t \leq 0 \end{cases} 
\]

\noindent
$X_t$ \'es una v.a. discreta amb funci\'o de probabilitat:
\[
P(X_t=i)=\begin{cases} \frac{(\beta t)^i}{i!} e^{-\beta t} & \text{si } i \in \{0, 1, \cdots \} \\
0 & \text{en cas contrari} \end{cases} 
\]


Observem que $Z$ \'es una v.a. discreta i el seu conjunt de valors possibles \'es 
$\Omega_Z=\{ 0, 1, 2, \cdots \}$. 
Si $i \notin \Omega_Z$, $P(Z=i)=0$. Si $i \in \Omega_Z$, aplicam la f\`ormula de
la probabilitat total per a calcular $P(Z=i)$, tenint en compte que $T$ \'es una v.a. cont\'\i nua:
\[
P(Z=i)=\int_0^{+\infty} P(Z=i | T=t) \cdot f_T(t) \, dt 
\]

\noindent
D'altra banda, $Z=i |_{T=t}$ \'es el succ\'es ``arriben $i$ clients a l'estaci\'o
de servei en un temps $t$'', per tant \'es equivalent al succ\'es $X_t=i$. De
manera que
\[
P(Z=i | T=t)=P(X_t=i)=\frac{(\beta t)^i}{i!} e^{-\beta t} \qquad \text{si } i \in \{0, 1, \cdots \}
\]

\noindent
La integral queda:
\[
P(Z=i)=\int_0^{+\infty} \frac{(\beta t)^i}{i!} e^{-\beta t} \cdot \alpha e^{-\alpha t} \, dt=
\frac{\alpha \beta^i}{i!} \int_0^{+\infty} t^i e^{-(\alpha+\beta)t} \, dt
\]
\noindent
Per aplicar la nota de l'enunciat hem de fer el canvi de variable:
\[
r=(\alpha+\beta)t \qquad t=\frac{r}{\alpha+\beta} \qquad dr=\alpha + \beta
\]
\noindent
i la integral s'escriu:
\[
P(Z=i)= \frac{\alpha \beta^i}{i!} \frac{1}{(\alpha+\beta)^{i+1}} \int_0^{+\infty} r^i e^{-r} \, dr = 
\frac{\alpha \beta^i}{i!} \frac{1}{(\alpha+\beta)^{i+1}} i! = 
\frac{\alpha}{\alpha+\beta} \left( \frac{\beta}{\alpha+\beta} \right)^i = 
\frac{\alpha}{\alpha+\beta} \left( 1-\frac{\alpha}{\alpha+\beta} \right)^i
\]

\noindent
Observem que aquesta funci\'o de probabilitat correspon a la d'una v.a. geom\`etrica amb par\`ametre
$\frac{\alpha}{\alpha+\beta}$ que comen\c{c}a a 0:
\[
Z \sim \mathrm{Geom}(\frac{\alpha}{\alpha+\beta})
\]

\newpage
\noindent
\textbf{Problema 25.}  
Una persona contreu el virus d'una certa malaltia que
requereix hospitalitzaci\'o. Sigui $X$ la variable aleat\`oria que
d\'ona el temps, en setmanes, que tarda en manifestar-se la
malaltia. Sigui $Y$ la variable aleat\`oria que expressa el temps
total, tamb\'e en setmanes, des de que el virus s'ha contret fins
que el pacient \'es donat d'alta. Se suposa que en el moment que
apareixen els s\'{\i}mptomes el malalt \'es hospitalitzat. Se sap que $X$
i $Y$ tenen una densitat conjunta donada per: $$f_{XY}(x,y) = \begin{cases}x
\cdot {e}^{-y} & \text{si   } 0 \leq x \leq y < +\infty \\ 0 & \text{en cas
contrari}\end{cases}$$

\begin{enumerate}[a)]
\item Trobau les corbes i les rectes de regressi\'o i estudiau la
bondat dels ajustaments.
\item Calculau la probabilitat que un pacient estigui hospitalitzat
menys d'una setmana. ($\mathbf{1-{ e}^{-1}}$)
\item Suposant que el pacient \'es donat d'alta avui despr\'es de 5 dies
hospitalitzat, quina \'es la probabilitat que contragu\'es el virus fa
menys de dues setmanes? ({\bf Ind.}: Fes un canvi de les variables
$(X,Y)$ a les noves variables $({\cal U},{\cal V}) = (Y - X, Y)$.)
($\mathbf{1-\frac{16}{7}e^{-\frac{9}{7}}=0.37}$)
\end{enumerate}


\vskip 0.3 cm
\noindent
\textbf{Soluci\'o:}

\vskip 0.2 cm
\noindent
\textbf{a)} Calcularem primer les corbes de regressi\'o $Y^*=E(Y|X)$ i
$X^*=E(X|Y)$.

\[
E(Y|X)= \int_{-\infty}^{+\infty} y f_{Y|X}(y|x) dy
\]

Necessitam calcular primer la funci\'o de densitat de $Y$ condicionada a $X$. Aplicam la definici\'o:

\[
f_{Y|X}(y|x)= \frac{ f_{XY}(x, y) }{f_X(x)}
\]

La funci\'o de densitat marginal de $X$ es calcula com:

\[
f_X(x)= \int_{-\infty}^{+\infty} f_{XY}(x, y) dy = 
\begin{cases} x \geq 0 & \int_x^{+\infty} x e^{-y} dy = x e^{-x} \\ \\
x < 0 & 0 \end{cases}
\]

Per tant:

\[
f_{Y|X}(y|x)= \begin{cases} \frac{ x e^{-y} } { x e^{-x} } = e^{x-y} & \text{si   } 0 \leq x \leq y < +\infty \\
0 & \text{en cas contrari}\end{cases}
\]

Finalment,

\[
E(Y|X)= \int_{x}^{+\infty} y e^{x-y} dy = e^x \int_{x}^{+\infty} y e^{-y} dy = (\text{integram per parts})=1+x
\]


\vskip 0.5 cm
Seguim un procediment similar per calcular $E(X|Y)$:

\[
E(X|Y)= \int_{-\infty}^{+\infty} x f_{X|Y}(x|y) dx
\]

\[
f_{X|Y}(x|y)= \frac{ f_{XY}(x, y) }{f_Y(y)}
\]

\[
f_Y(y)= \int_{-\infty}^{+\infty} f_{XY}(x, y) dx = 
\begin{cases} y \geq 0 & \int_0^y x e^{-y} dx = \frac{y^2}{2} e^{-y} \\ \\
y < 0 & 0 \end{cases}
\]

\[
f_{X|Y}(x|y)= \begin{cases} \frac{ x e^{-y} } { \frac{y^2}{2} e^{-y} } = \frac{2x}{y^2} 
& \text{si   } 0 \leq x \leq y < +\infty \\
0 & \text{en cas contrari}\end{cases}
\]

\[
E(X|Y)= \int_{0}^{y} x \frac{2x}{y^2} dx = \frac{2}{3} y
\]


\vskip 0.5 cm
Podem calcular tamb\'e les rectes de regressi\'o:

\[
Y^{**}=aX+b \qquad \text{amb } a=\frac{\mathrm{Cov}(X, Y)}{\mathrm{Var}(X)} \qquad \text{i} \qquad b=E(Y)-a E(X) 
\]

\[
X^{**}=a'X+b' \qquad \text{amb } a'=\frac{\mathrm{Cov}(X, Y)}{\mathrm{Var}(Y)} \qquad \text{i} \qquad b'=E(X)-a' E(Y) 
\]

No obstant, com l'equaci\'o de la corba de regressi\'o $Y^*=E(Y|X)$ \'es l'equaci\'o d'una recta, sabem que 
$Y^{**}=Y^*$. A m\'es, com l'equaci\'o de la corba de regressi\'o $X^*=E(X|Y)$ \'es tamb\'e l'equaci\'o d'una recta,
tendrem que $X^{**}=X^*$.

\vskip 0.5 cm
\noindent
\textbf{b)} Anomenam $Z$ el temps d'hospitalitzaci\'o. Aquest temps es calcula com la difer\`encia entre el temps que
 tarda un pacient en \'esser donat d'alta i el temps en qu\`e es manifesten els primers simptomes. Per tant
 $Z=Y-X$. Volem calcular $P(Z < 1)$.
 
 \[
 P(Z < 1) = P(Y-X < 1) = \iint\limits_R f_{XY}(x, y) dxdy
 \]
 
on $R$ \'es la regi\'o del pla formada pels punts $(x, y)$ que verifiquen la condici\'o $y-x < 1$:
\[
R=\{ (x, y) \; / \; y-x < 1 \}
\]

\setcounter{figure}{0}
\begin{figure}[htbp]
\begin{center}
\begin{picture}(100, 100)
\thicklines
\put(10, 10){\vector(1, 0){90}}
\put(10, 10){\vector(0, 1){90}}
\thinlines
\put(95, 0){$x$}
\put(-5, 95){$y$}
\put(10, 30){\line(1, 1){70}}
\put(10, 10){\line(1, 1){85}}
\put(85, 100){$y-x=1$}
\put(95, 90){$y=x$}
\put(10, 20){\line(1, 0){10}}
\put(10, 25){\line(1, 0){15}}
\put(10, 30){\line(1, 0){20}}
\put(10, 35){\line(1, 0){25}}
\put(10, 40){\line(1, 0){30}}
\put(10, 45){\line(1, 0){35}}
\put(10, 50){\line(1, 0){40}}
\put(10, 55){\line(1, 0){45}}
\put(10, 60){\line(1, 0){50}}
\put(10, 65){\line(1, 0){55}}
\put(10, 70){\line(1, 0){60}}
\put(10, 75){\line(1, 0){65}}
\put(10, 80){\line(1, 0){70}}
\put(10, 85){\line(1, 0){75}}
\put(10, 90){\line(1, 0){80}}
\put(10, 30){\line(0, -1){45}}
\put(15, 35){\line(0, -1){50}}
\put(20, 40){\line(0, -1){55}}
\put(25, 45){\line(0, -1){60}}
\put(30, 50){\line(0, -1){65}}
\put(35, 55){\line(0, -1){70}}
\put(40, 60){\line(0, -1){75}}
\put(45, 65){\line(0, -1){80}}
\put(50, 70){\line(0, -1){85}}
\put(55, 75){\line(0, -1){90}}
\put(60, 80){\line(0, -1){95}}
\put(65, 85){\line(0, -1){100}}
\put(70, 90){\line(0, -1){105}}
\put(75, 95){\line(0, -1){110}}
\multiput(90, 50)(0, 5){4}{\line(1, 0){15}}
\put(110, 55){suport de $f_{XY}$}
\multiput(90, 30)(5, 0){4}{\line(0, 1){10}}
\put(110, 30){$R$}
\put(50, -30){(a)}
\end{picture}
\hspace{4cm}
\begin{picture}(100, 100)
\thicklines
\put(10, 10){\vector(1, 0){90}}
\put(10, 10){\vector(0, 1){90}}
\thinlines
\put(95, 0){$z$}
\put(-5, 95){$y$}
\put(-10, 50){\line(1, 0){105}}
\put(-5, 55){$2$}
\multiput(-5, 50)(5, 0){20}{\line(0, -1){65}}
\multiput(25, 10)(0, 5){18}{\line(0, 1){3}}
\put(24, 0){$\frac{5}{7}$}
\linethickness{0.5mm}
\put(25, -15){\line(0, 1){65}}
\thinlines
\put(65, 35){$y < 2$}
\put(35, 63){$A=\{ y < 2 \cap z=\frac{5}{7} \}$}
\put(35, 60){\vector(-1, -2){10}}
\put(50, -30){(b)}
\end{picture}
\end{center}
\vskip 0.4 cm
\caption{}
\end{figure}



La integral anterior prendr\`a un valor diferent de zero nom\'es en la regi\'o d'intersecci\'o de $R$ amb el
suport de la funci\'o (veure figura 1a), de manera que tendrem:

\[
P(Z < 1) = \int_0^{+\infty} \int_x^{1+x} x e^{-y} dy dx = \cdots = 1-\frac{1}{e}
\]


\vskip 0.5 cm
\noindent
\textbf{c)} Ens demanen calcular 
\[
P(Y < 2 |_{Z=\frac{5}{7}})=\frac{ P( Y < 2 \cap Z=\frac{5}{7} )}{f_Z(\frac{5}{7})}=\frac{P(A)}{f_Z(\frac{5}{7})}
\]

on $A$ \'es el conjunt de punts $(z, y)$ que verifiquen les condicions $y < 2$ i $z=\frac{5}{7}$ (veure figura 1b).
De manera que: 
\[
P(A)=\int_{-\infty}^2 f_{ZY}(\frac{5}{7}, y) dy
\]

Observam com per calcular els valors anteriors necessitam con\`eixer $f_Z$ i $f_{ZY}$. Tenim dues maneres de fer-ho:
c\`alcul directe o utilitzant transformacions. Ho farem de les dues maneres:

\vskip 0.5 cm
\noindent
\textbf{C\`alcul directe de $f_Z$ i $f_{ZY}$}

Com $Y$ i $Z$ s\'on v.a. cont\'\i nues hem de calcular en primer lloc les funcions de distribuci\'o. A partir d'aquestes
podrem obtenir, per derivaci\'o, les funcions de densitat.

\[
F_Z(z)=P(Z \leq z) = P(Y-X \leq z)=P(R_z)=\iint\limits_{R_z} f_{XY}(x, y) dx dy
\]

on $R_z$ \'es la regi\'o del pla formada pels punts $(x, y)$ que verifiquen la condici\'o $y-x \leq z$. Aquesta regi\'o
varia per a cada valor de $z$ i t\'e una forma similar a la mostrada en la figura 1a. La intersecci\'o entre $R_z$ 
i el suport de la funci\'o $f_{XY}$ ser\`a nul.la per a valors de $z < 0$. Per tant:

\[
F_Z(z)=\begin{cases} \int_0^{+\infty} \int_x^{z+x} x e^{-y} dy dx = \cdots = 1-e^{-z} & \text{si } z \geq 0 \\
0 & \text{si } z < 0 \end{cases}
\]

Derivant aquesta funci\'o obtenim la funci\'o de densitat de $Z$:

\[
f_Z(z)=\frac{d F_Z(z)}{dz}=\begin{cases} e^{-z} & \text{si } z \geq 0 \\
0 & \text{si } z < 0 \end{cases}
\]

\vskip 0.3 cm
Per obtenir $f_{ZY}$ seguim un procediment similar a l'anterior:

\[
F_{ZY}(z, y)=P(Z \leq z, Y \leq y)=P(Y-X \leq z, Y \leq y)=P(R_{zy})=\iint\limits_{R_{zy}} f_{XY}(x, y) dx dy
\]

\begin{figure}[htbp]
\begin{center}
\begin{picture}(100, 100)
\thicklines
\put(10, 10){\vector(1, 0){90}}
\put(10, 10){\vector(0, 1){90}}
\thinlines
\put(95, 0){$x'$}
\put(-5, 95){$y'$}
\put(10, 30){\line(1, 1){70}}
\put(10, 10){\line(1, 1){85}}
\put(85, 100){$y'-x'=z$}
\put(95, 90){$y'=x'$}
\put(0, 25){$z$}
\put(10, 20){\line(1, 0){10}}
\put(10, 25){\line(1, 0){15}}
\put(10, 30){\line(1, 0){20}}
\put(10, 35){\line(1, 0){25}}
\put(10, 40){\line(1, 0){30}}
\put(10, 45){\line(1, 0){35}}
\put(10, 50){\line(1, 0){40}}
\put(10, 55){\line(1, 0){45}}
\put(10, 60){\line(1, 0){50}}
\put(10, 65){\line(1, 0){55}}
\put(10, 70){\line(1, 0){60}}
\put(10, 75){\line(1, 0){65}}
\put(10, 80){\line(1, 0){70}}
\put(10, 85){\line(1, 0){75}}
\put(10, 90){\line(1, 0){80}}
\put(10, 30){\line(0, -1){45}}
\put(15, 35){\line(0, -1){50}}
\put(20, 40){\line(0, -1){55}}
\put(25, 45){\line(0, -1){60}}
\put(30, 50){\line(0, -1){65}}
\put(35, 55){\line(0, -1){70}}
\put(40, 60){\line(0, -1){75}}
\put(45, 65){\line(0, -1){80}}
\put(50, 70){\line(0, -1){85}}
\put(55, 75){\line(0, -1){90}}
\put(60, 80){\line(0, -1){95}}
\put(65, 85){\line(0, -1){100}}
\put(70, 90){\line(0, -1){105}}
\put(75, 95){\line(0, -1){110}}
\thicklines
\put(0, 50){\line(1, 0){100}}
\thinlines
\put(5, 50){\line(1, -1){65}}
\put(10, 50){\line(1, -1){65}}
\put(15, 50){\line(1, -1){65}}
\put(20, 50){\line(1, -1){60}}
\put(25, 50){\line(1, -1){55}}
\put(30, 50){\line(1, -1){50}}
\put(35, 50){\line(1, -1){45}}
\put(40, 50){\line(1, -1){40}}
\put(45, 50){\line(1, -1){35}}
\put(50, 50){\line(1, -1){30}}
\put(55, 50){\line(1, -1){25}}
\put(60, 50){\line(1, -1){20}}
\put(65, 50){\line(1, -1){15}}
\put(70, 50){\line(1, -1){10}}
\put(5, 45){\line(1, -1){60}}
\put(5, 40){\line(1, -1){55}}
\put(5, 35){\line(1, -1){50}}
\put(5, 30){\line(1, -1){45}}
\put(5, 25){\line(1, -1){40}}
\put(5, 20){\line(1, -1){35}}
\put(5, 15){\line(1, -1){30}}
\put(5, 10){\line(1, -1){25}}
\put(5, 5){\line(1, -1){20}}
\put(5, 0){\line(1, -1){15}}
\put(100, 55){$y'=y$}
\put(-10, 45){$y$}
\multiput(140, 50)(0, 5){4}{\line(1, 0){15}}
\put(160, 55){suport de $f_{XY}$}
\multiput(140, 30)(5, 0){4}{\line(0, 1){10}}
\put(160, 30){$y'-x'\leq z$}
\put(140, 15){\line(1, -1){20}}
\put(145, 15){\line(1, -1){15}}
\put(150, 15){\line(1, -1){10}}
\put(140, 10){\line(1, -1){15}}
\put(140, 5){\line(1, -1){10}}
\put(162, 5){$y'\leq y$}
\put(15, 30){$\mathbf{R_{zy}}$}
\end{picture}
\end{center}
\caption{}
\end{figure}


on $R_{zy}$ \'es la regi\'o del pla formada pels punts $(x', y')$ que verifiquen les condicions  $y' \leq y$ i 
$y'-x' \leq z$. Aquesta regi\'o varia per a cada valor $y$ i $z$ (veure figura 2).
En particular, La intersecci\'o entre $R_{zy}$ i el suport de la funci\'o $f_{XY}$ ser\`a nul.la quan
$y < 0$ o b\'e $z < 0$. Per tant:

\[
F_{ZY}(z, y)=\begin{cases} 
0 & \text{si } y < 0 \; \text{o} \; z < 0 \\ \\
\int_0^{y-z} \int_x^{x+z} x e^{-y} dy dx + \int_{y-z}^y \int_x^y x e^{-y} dy dx = \cdots = 
1-e^{-z}-ze^{-y}(1+y-\frac{z}{2}) & \text{si } y \geq z \geq 0 \\ \\
\int_0^y \int_x^y xe^{-y} dy dx = \cdots = 1 - e^{-y} (\frac{y^2}{2}+y+1) & \text{si } z > y \geq 0
\end{cases}
\]

Derivant aquesta funci\'o obtenim $f_{YZ}$:

\[
f_{ZY}(z, y)=\frac{\partial^2 F_{ZY}(z, y)}{\partial z \partial y}=
\begin{cases} 
0 & \text{si } y < 0 \; \text{o} \; z < 0 \\ \\
e^{-y} (y-z) & \text{si } y \geq z \geq 0 \\ \\
0 & \text{si } z > y \geq 0
\end{cases} = 
\begin{cases} e^{-y} (y-z) & \text{si } y \geq z \geq 0 \\ \\ 0 & \text{en cas contrari} \end{cases}
\]


\newpage
\noindent
\textbf{C\`alcul de $f_Z$ i $f_{ZY}$ utilitzant transformacions de v.a.}
\vskip 0.2 cm

Feim el canvi de variables $({\cal U},{\cal V}) = (Y - X, Y)$. En notaci\'o matricial aquest canvi s'escriu
com:

\[
\begin{pmatrix} {\cal U} \\ {\cal V} \end{pmatrix} = \begin{pmatrix} -1 & 1 \\ 0 & 1 \end{pmatrix} 
\begin{pmatrix} X \\ Y \end{pmatrix}
\]

Amb aquest canvi tenim que $Z={\cal U}$ i per tant $f_{ZY}=f_{{\cal U}{\cal V}}$ i $f_Z=f_{\cal U}$.

La funci\'o de densitat conjunta de ${\cal U}$ i ${\cal V}$ es pot calcular a partir de la funci\'o de densitat 
conjunta de $X$ i $Y$  amb la f\`ormula:

\[
f_{{\cal U}{\cal V}}(u, v) =
\frac{ f_{XY} ( A^{-1} \begin{pmatrix} u \\ v \end{pmatrix} } { | \mathrm{det}(A) |}
\]

on $A$ \'es la matriu associada a la transformaci\'o lineal. Tenim doncs:
\[
A= \begin{pmatrix} -1 & 1 \\ 0 & 1 \end{pmatrix} 
\qquad 
A^{-1}= \begin{pmatrix} -1 & 1 \\ 0 & 1 \end{pmatrix} 
\qquad 
\text{i}
\quad 
\mathrm{det}(A)=-1
\]

Per tant:

\[
f_{{\cal U}{\cal V}}(u, v) =\frac{ f_{XY} (v-u, v) }{|-1|}=f_{XY} (v-u, v) =
\begin{cases} (v-u) e^{-v} & \text{si } 0 \leq v-u \leq v \\ \\ 0 & \text{en cas contrari} \end{cases}
\]

La condici\'o $0 \leq v-u \leq v$ \'es equivalent a $v \geq u \geq 0$, de manera que obtenim la mateixa 
funci\'o de densitat conjunta que fent el c\`alcul directe.

$f_{\cal U}$ es calcula com una llei marginal a partir de $f_{{\cal U}{\cal V}}$:
\[
f_{\cal U}(u)=\int_{-\infty}^{+\infty} f_{{\cal U}{\cal V}}(u, v)dv = 
\begin{cases} \int_u^{+\infty} (v-u) e^{-v} dv = \cdots = e^{-u} & \text{si } u \geq 0 \\ \\
0 & \text{en cas contrari} \end{cases}
\]

que \'es el mateix resultat obtingut amb el c\`alcul directe.




\vskip 1.5 cm
Ara ja podem resoldre l'apartat c):

\[
P(Y < 2 |_{Z=\frac{5}{7}})=\frac{ \int_{-\infty}^2 f_{ZY}(\frac{5}{7}, y) dy } {f_Z(\frac{5}{7})}=
\frac{ \int_{5/7}^2 \; e^{-y} (y-\frac{5}{7}) dy}{e^{-5/7}}=
\frac{ e^{-5/7} - \frac{16}{7} e^{-2} }{e^{-5/7}} = 1-\frac{16}{7} e^{-9/7}
\]


\newpage
\noindent
\textbf{Problema 28.}  
Siguin $X_{1}$, $X_{2}$ variables aleat\`ories independents
tals que

$$P(X_{i}=1)=p=1-P(X_{i}=2) \mbox{ si } i=1,2$$ Sigui $Y_{k}$ la
v.a. que designa el nombre de variables $X_{i}$ iguals a $k$ amb
$k=1,2$. Trobau la distribuci\'o conjunta de $(Y_{1},Y_{2})$ i el
seu vector de mitjanes i la seva matriu de covari\`ancies.

\vskip 0.3 cm
\noindent
\textbf{Soluci\'o:} Per l'enunciat del problema sabem que:
\[
\begin{array}{l}
\Omega_{X_1}=\{ 1, 2 \} \qquad P(X_1=1)=p \qquad P(X_1=2)=1-p
\\
\\
\Omega_{X_2}=\{ 1, 2 \} \qquad P(X_2=1)=p \qquad P(X_2=2)=1-p
\end{array}
\]

\noindent
Com que $X_1$ i $X_2$ s\'on independents, llavors $P(X_1=i, X_2=j)=P(X_1=i) \cdot P(X_2=j)$.

\vskip 0.2 cm
\noindent
A m\'es, per la definici\'o de $Y_1$ i $Y_2$ tenim que:
\[
\begin{array}{l}
\Omega_{Y_1}=\{0, 1, 2\} \qquad \Omega_{Y_2}=\{0, 1, 2\} \\ \\
P(Y_1=i, Y_2=j) =0 \qquad \text{si } i \notin \Omega_{Y_1} \quad \text{ \'o } \quad j \notin \Omega_{Y_2} \\ \\
P(Y_1=0, Y_2=0) =P(\text{cap variable $X_1$, $X_2$ igual a 1 o a 2})=0 \quad \text{(impossible)} \\ \\
P(Y_1=0, Y_2=1) =P(\text{cap variable $X_1$, $X_2$ igual a 1, una \'unica igual a 2})=0 \quad \text{(impossible)} \\ \\
P(Y_1=0, Y_2=2) =P(X_1=2, X_2=2)=P(X_1=2) \cdot P(X_2=2)=(1-p)^2 \\ \\
P(Y_1=1, Y_2=0) =P(\text{cap variable $X_1$, $X_2$ igual a 2, una \'unica igual a 1})=0 \quad \text{(impossible)} \\ \\
P(Y_1=1, Y_2=1) =P(X_1=1, X_2=2)+P(X_1=2, X_2=1)= \\
\qquad \qquad \qquad  \qquad  =P(X_1=1) \cdot P(X_2=2) + P(X_1=2) \cdot P(X_2=1)= 2p(1-p) \\ \\
P(Y_1=1, Y_2=2) =P(\text{una \'unica variable $X_1$, $X_2$ igual a 1, dues iguals a 2})=0 \quad \text{(impossible)} \\ \\
P(Y_1=2, Y_2=0) =P(X_1=1, X_2=1)=P(X_1=1) \cdot P(X_2=1)=p^2 \\ \\
P(Y_1=2, Y_2=1) =P(\text{una \'unica variable $X_1$, $X_2$ igual a 2, dues iguals a 1})=0 \quad \text{(impossible)} \\ \\
P(Y_1=2, Y_2=2) =P(\text{dues variables $X_1$, $X_2$ iguals a 1 i dues iguals a 2})=0 \quad \text{(impossible)} 
\end{array}
\]

\noindent
Aquests resultats es poden resumir en la seg\"uent taula, on es mostren tamb\'e les funcions de
probabilitat marginals.

\vskip 0.2 cm
\begin{center}
\begin{tabular}{c|c|c|c||c}
$Y_2 \backslash Y_1$ & $0$ & $1$ & $2$ & $P(Y_2=j)$ \\ \hline
$0$ & $0$ & $0$ & $p^2$ & $p^2$ \\ \hline
$1$ & $0$ & $2p(1-p)$ & $0$ & $2p(1-p)$ \\ \hline
$2$ & $(1-p)^2$ & $0$ & $0$ & $(1-p)^2$ \\ \hline \hline
$P(Y_1=i)$ & $(1-p)^2$ & $2p(1-p)$ & $p^2$ 
\end{tabular} 
\end{center}

\vskip 0.2 cm
\noindent
Es pot comprovar que la suma de tots els valors de la probabilitat conjunta, aix\'i com
la suma de les probabilitats marginals, \'es igual a $1$.

\vskip 0.3 cm
\noindent
El vector de mitjanes $(E(Y_1), E(Y_2))$ \'es el seg\"uent:
\[
\begin{array}{l}
E(Y_1)=0 \cdot P(Y_1=0) + 1 \cdot P(Y_1=1) + 2 \cdot P(Y_1=2)=2p(1-p)+2p^2=2p \\ \\
E(Y_2)=0 \cdot P(Y_2=0) + 1 \cdot P(Y_2=1) + 2 \cdot P(Y_2=2)=2p(1-p)+2(1-p)^2=2-2p=2(1-p) \\ \\
\end{array}
\]

\vskip 0.3 cm
\noindent
La matriu de covari\`ancies \'es: $K = \begin{pmatrix} \sigma_{Y_1}^2 & \sigma_{Y_1 Y_2} \\ \sigma_{Y_1 Y_2} & \sigma_{Y_2}^2 \end{pmatrix}$

\[
\begin{array}{l}
E(Y_1^2)=0^2 \cdot P(Y_1=0) + 1^2 \cdot P(Y_1=1) + 2^2 \cdot P(Y_1=2)=2p(1-p)+4p^2=2p(1+p) \\ \\
\sigma_{Y_1}^2=E(Y_1^2)-E(Y_1)^2= 2p+2p^2 - (2p)^2=2p(1-p) \\ \\
E(Y_2^2)=0^2 \cdot P(Y_2=0) + 1^2 \cdot P(Y_2=1) + 2^2 \cdot P(Y_2=2)=2p(1-p)+4(1-p)^2=2(1-p)(2-p) \\ \\
\sigma_{Y_2}^2=E(Y_2^2)-E(Y_2)^2=  2(1-p)(2-p) - (2-2p)^2 = 2(1-p)p\\ \\
E(Y_1Y_2)=0 \cdot 2 \cdot P(Y_1=0, Y_2=2) +  1 \cdot 1 \cdot P(Y_1=1, Y_2=1) + 2 \cdot 0 \cdot P(Y_1=2, Y_2=0)=2p(1-p) \\ \\
\sigma_{Y_1 Y_2}=E(Y_1Y_2) - E(Y_1) \cdot E(Y_2) = 2p(1-p) - 2p \cdot 2(1-p) = -2p(1-p)
\end{array}
\]

\vskip 0.2 cm
\noindent
De manera que la matriu de covari\`ancies queda finalment:
\[
K=\begin{pmatrix} 2p(1-p) & -2p(1-p) \\ \\ -2p(1-p) & 2p(1-p) \end{pmatrix} = 2p(1-p) \begin{pmatrix} 1 & -1 \\ \\ -1 & 1 \end{pmatrix}
\]



\vskip 0.2cm
\noindent
Si calculam el coeficient de correlaci\'o:
\[
\rho_{Y_1 Y_2}= \frac{ \sigma_{Y_1 Y_2} }{ \sigma_{Y_1} \sigma_{Y_2} }=\frac{-2p(1-p)}{\sqrt{2p(1-p)} \sqrt{2p(1-p)}}=-1
\]

\noindent
Aquest valor de $\rho_{Y_1 Y_2}$ indica que existeix una relaci\'o lineal entre les variables $Y_1$ i $Y_2$.




\newpage
\noindent
\textbf{Problema 29.}  
Sigui $(X,Y)$ un vector aleatori que t\'e distribuci\'o uniforme
a $(0,1)\times(0,1)$. Calculau la distribuci\'o conjunta i les
marginals de $(U,V)$ on $U=\max(X,Y)$ i $V=\min(X,Y)$.
\vskip 0.3 cm
\noindent
\textbf{Soluci\'o:} Calcularem primer $F_{UV}(u, v)$ i a continuaci\'o derivarem
per a trobar les funcions de densitat conjunta i marginals.

\[
F_{UV}(u, v)=P(U \leq u, V \leq v)=P(\max(X, Y) \leq u, \min(X, Y) \leq v)=P(A_{uv})
\]
\noindent
on $A_{uv}$ \'es el conjunt de punts $(x, y)$ tals que $\max(x, y) \leq u$ i $\min(X, Y) \leq v$.

\noindent
Podem escriure $A$ com la intersecci\'o de dos conjunts: $A_{uv}=A_{u}^{\max} \cap A_{v}^{\min}$,
on 
\[
A_{u}^{\max}=\{ (x, y) \, / \max(x, y) \leq u \} \qquad \text{i} \qquad
A_{v}^{\min}=\{ (x, y) \, / \min(x, y) \leq v \}
\]
\noindent
En la figura 1 es mostra la forma que tenen aquests conjunts. 

\vskip 0.2 cm
\noindent
D'altra banda,
\[
f_{XY}(x, y)=\begin{cases} K & \text{si } 0 \leq x \leq 1 \quad \text{i} \quad 0 \leq y \leq 1 \\ 
0 & \text{en cas contrari} \end{cases}
\]
\noindent
El suport d'aquesta funci\'o es mostra en la figura 1. El valor de $K$ es troba aplicant la propietat
$\iint_{\R^2} f_{XY}(x, y) \, dxdy=\iint_{\Omega_{XY}} f_{XY}(x, y) \, dxdy= 1$. Fent la integral comprovam que $K=1$.

\vskip 0.2 cm
\setcounter{figure}{0}
\begin{figure}[htbp]
\begin{center}
\begin{picture}(100, 100)
\put(10, 10){\vector(1, 0){90}}
\put(10, 10){\vector(0, 1){90}}
\put(95, 0){$x$}
\put(-5, 95){$y$}
\put(10, 60){\line(1, 0){50}}
\put(60, 10){\line(0, 1){50}}
\put(60, 0){\textbf{\textit{1}}}
\put(0, 60){\textbf{\textit{1}}}
\put(50, 50){\textbf{\textit{S}}}
\end{picture}
$\qquad$ $\qquad$
\begin{picture}(100, 100)
\put(10, 10){\vector(1, 0){90}}
\put(10, 10){\vector(0, 1){90}}
\put(95, 0){$x$}
\put(-5, 95){$y$}
\put(-20, 72){\line(1, 0){100}}
\put(72, -20){\line(0, 1){100}}
\put(72, 0){$u$}
\put(-10, 77){$u$}
\put(35, 35){$A_u^{\mathrm{max}}$}
\put(77, 77){$(u, u)$}
\put(72, 72){\circle*{3}}
\put(-5, 72){\line(-1, -1){10}}
\put(5, 72){\line(-1, -1){20}}
\put(15, 72){\line(-1, -1){30}}
\put(25, 72){\line(-1, -1){40}}
\put(35, 72){\line(-1, -1){50}}
\put(45, 72){\line(-1, -1){60}}
\put(55, 72){\line(-1, -1){70}}
\put(65, 72){\line(-1, -1){80}}
\put(72, 70){\line(-1, -1){90}}
\put(72, 60){\line(-1, -1){80}}
\put(72, 50){\line(-1, -1){70}}
\put(72, 40){\line(-1, -1){60}}
\put(72, 30){\line(-1, -1){50}}
\put(72, 20){\line(-1, -1){40}}
\put(72, 10){\line(-1, -1){30}}
\put(72, 0){\line(-1, -1){20}}
\put(72, -10){\line(-1, -1){10}}
\end{picture}
$\qquad$ $\qquad$
\begin{picture}(100, 100)
\put(10, 10){\vector(1, 0){90}}
\put(10, 10){\vector(0, 1){90}}
\put(95, 0){$x$}
\put(-5, 95){$y$}
\put(-20, 42){\line(1, 0){120}}
\put(42, -20){\line(0, 1){120}}
\put(42, 0){$v$}
\put(-10, 47){$v$}
\put(20, 20){$A_v^{\mathrm{min}}$}
\put(47, 47){$(v, v)$}
\put(42, 42){\circle*{3}}
\multiput(42, 95)(0, -10){6}{\line(-1, -1){50}}
\put(37, 100){\line(-1, -1){45}}
\put(27, 100){\line(-1, -1){35}}
\put(17, 100){\line(-1, -1){25}}
\put(52, 42){\line(-1, -1){60}}
\put(62, 42){\line(-1, -1){60}}
\put(72, 42){\line(-1, -1){60}}
\put(82, 42){\line(-1, -1){60}}
\put(92, 42){\line(-1, -1){60}}
\put(100, 40){\line(-1, -1){58}}
\put(100, 30){\line(-1, -1){48}}
\put(100, 20){\line(-1, -1){38}}
\end{picture}
\end{center}
\caption{Esquerra: suport de $f_{XY}(x, y)$. 
Centre: conjunt $A_{u}^{\mathrm{max}}$ per a $u > 0$. 
Dreta: conjunt $A_{v}^{\mathrm{min}}$ per a $v > 0$.}
\end{figure}


\vskip 0.2 cm
\noindent
Ara ja podem calcular $F_{UV}(u, v)$:
\[
F_{UV}(u, v)=\iint_{A_{u}^{\max} \cap A_{v}^{\min} \cap \Omega_{XY}} f_{XY}(x, y) \, dxdy= 
\iint_{A_{u}^{\max} \cap A_{v}^{\min} \cap \Omega_{XY}} 1 \, dxdy=\mathrm{area}(A_{u}^{\max} \cap A_{v}^{\min} \cap \Omega_{XY})
\]

\vskip 0.2 cm
\noindent
L'\`area de la intersecci\'o d'aquests conjunts dependr\`a dels valors de $u$ i$v$. Tenint en compte 
totes les possibilitats arribarem al seg\"uent resultat:
\[
F_{UV}(u, v)=\begin{cases}
\text{si } u < 0 \, \text{ \'o } \, v < 0 & 0 \\
\text{si } 0 \leq u \leq 1  \, \text{ i } \, 0 \leq v < u & 2uv-v^2 \\
\text{si } 0 \leq u \leq 1  \, \text{ i } \, v \geq u & u^2 \\
\text{si } u > 1 \, \text{ i } \,   v \leq 1  & 2v-v^2 \\
\text{si } u > 1 \, \text{ i } \, v > 1 & 1 
\end{cases}
\]
 
\vskip 0.2 cm
\noindent
Dividint aquesta funci\'o en cada interval obtenim la funci\'o de densitat conjunta:
\[
f_{UV}(u, v)=\frac{\partial^2 F_{UV}(u, v)}{\partial u \, \partial v}=\begin{cases}
\text{si } u < 0 \, \text{ \'o } \, v < 0 & \frac{\partial^2 0}{\partial u \, \partial v}=0 \\
\text{si } 0 \leq u \leq 1  \, \text{ i } \, 0 \leq v < u & \frac{\partial^2 2uv-v^2}{\partial u \, \partial v}=2  \\
\text{si } 0 \leq u \leq 1  \, \text{ i } \, v \geq u & \frac{\partial^2 u^2}{\partial u \, \partial v} =0 \\
\text{si } u > 1 \, \text{ i } \,   v \leq 1  & \frac{\partial^2 2v-v^2}{\partial u \, \partial v}=0 \\
\text{si } u > 1 \, \text{ i } \, v > 1 &  \frac{\partial^2 1}{\partial u \, \partial v} = 0
\end{cases}
= \begin{cases} 2 & \text{si } 0 \leq v < u \leq 1 \\ 0 & \text{en cas contrari} \end{cases}
\]

Les funcions de densitat marginals s'obtenen integrant aquesta funci\'o:
\[
f_U(u)=\int_{-\infty}^{+\infty} f_{UV}(u, v) \,dv = \begin{cases} \text{si } 0 \leq u \leq 1 & 2u \\ \text{altrament} & 0 \end{cases}
\qquad 
f_V(v)=\int_{-\infty}^{+\infty} f_{UV}(u, v) \,du = \begin{cases} \text{si } 0 \leq v \leq 1 & 2-2v \\ \text{altrament} & 0 \end{cases}
\]

\newpage
\noindent
\textbf{Problema 30.} 
Sigui $(X,Y)$ un vector aleatori distribu\"{\i}t uniformement al
cercle unitat. Siguin les variables aleat\`ories $U=\sqrt{X^2+Y^2}$
i $V=\arctan{\frac{Y}{X}}$.
\begin{enumerate}[a)]
\item Demostrau que $U$ i $V$ s\'on v.a. independents.
\item Siguin $R$ i $\Theta$ dues variables aleat\`ories independents
 amb distribuci\'o uniforme a l'interval
unitat i a l'interval $(0,2\pi)$ respectivament. El vector aleatori
$(R,\Theta)$ t\'e la mateixa distribuci\'o que el $(U,V)$? 
\end{enumerate}

\vskip 0.3 cm
\noindent
\textbf{Soluci\'o:} 

\vskip 0.2 cm
\noindent
\textbf{a)} Per demostrar que $U$ i $V$ s\'on independents hem de provar que
$f_{UV}(u, v)=f_U(u) \cdot f_V(v)$. Primer calcularem $f_{UV}(u, v)$ i a partir
d'aqu\'i trobarem les densitats marginals.

Calcularem $f_{UV}(u, v)$ utilitzant la f\`ormula:
\[
f_{UV}(u, v)=f_{XY}(h_1(u, v), h_2(u, v)) \cdot |J_{h_1h_2}(u, v)| \qquad \text{on }\quad
J_{h_1h_2}(u, v)=\mathrm{det} \begin{pmatrix} 
\frac{\partial h_1(u, v)}{\partial u} & \frac{\partial h_1(u, v)}{\partial v} \\
\frac{\partial h_2(u, v)}{\partial u} & \frac{\partial h_2(u, v)}{\partial v} 
\end{pmatrix} 
\]

\noindent
on les funcions $h_1(u, v)$ i $h_2(u, v)$ s\'on les inverses de $u=g_1(x, y)=\sqrt{x^2+y^2}$ i 
$v=g_2(x, y)=\mathrm{arctan}\frac{y}{x}$. Tenim que $x=h_1(u, v)=u \cdot \cos v$ i
$y=h_2(u, v)=u \cdot \sin v$. Per tant:

\[
J_{h_1h_2}(u, v)=\mathrm{det} \begin{pmatrix} \cos v & -u \cdot \sin v \\ \sin v & u \cdot \cos v \end{pmatrix}=
u \cdot \cos^2 v + u \cdot \sin^2 v = u
\]

\noindent
La funci\'o de densitat de $(X, Y)$ \'es de la forma
\[
f_{XY}(x, y)=\begin{cases} K & \text{si } (x, y) \in S \\ 0 & \text{en cas contrari} \end{cases}
\]
\noindent
El suport $S$ d'aquesta funci\'o es mostra en la figura seg\"uent:

\setcounter{figure}{0}
\begin{figure}[htbp]
\begin{center}
\begin{picture}(120, 120)(-60, -60)
\put(-40, 0){\vector(1, 0){100}}
\put(0, -40){\vector(0, 1){100}}
\put(60, -5){$x$}
\put(-5, 60){$y$}
\put(0, 0){\circle{60}}
\put(0, 0){\line(-1, 2){13.5}}
\put(-8,20){1}
\put(-21, -18){S}
\end{picture}
\end{center}
\vskip -1.2 cm
\caption{$S$: suport de $f_{XY}(x, y)$.}
\end{figure}
 
\vskip 0.2 cm
\noindent
El valor de $K$ es troba aplicant la propietat
$\iint_{\R^2} f_{XY}(x, y) \, dxdy=\iint_{S} f_{XY}(x, y) \, dxdy= 1$. 
Com que $\iint_{S} f_{XY}(x, y) \, dxdy = \iint_{S} K \, dxdy=K \cdot \mathrm{area}(S)$ i $\mathrm{area}(S)=\pi \cdot 1^2$,
llavors $K=\frac{1}{\pi}$.

\vskip 0.4 cm
\noindent
De manera que:
\[
f_{UV}(u, v)=u \cdot f_{XY}(u \cdot \cos v, u \cdot \sin v)=\begin{cases}
\frac{u}{\pi} & \text{si } (u \cdot \cos v, u \cdot \sin v) \in S \\ 0 & \text{en cas contrari} \end{cases}=
\begin{cases}
\frac{u}{\pi} & \text{si } 0 \leq u \leq 1 \quad 0 \leq v \leq 2 \pi \\ 0 & \text{en cas contrari} \end{cases}
\]

\noindent
Les funcions de densitat marginals s\'on:
\[
f_U(u)=\int_{-\infty}^{+\infty} f_{UV}(u, v) \,dv = \begin{cases} \text{si } 0 \leq u \leq 1 & 2u \\ \text{altrament} & 0 \end{cases}
\qquad 
f_V(v)=\int_{-\infty}^{+\infty} f_{UV}(u, v) \,du = \begin{cases} \text{si } 0 \leq v \leq 2\pi  & \frac{1}{2\pi} 
\\ \text{altrament} & 0 \end{cases}
\]

\noindent
Es comprova que $f_U(u) \cdot f_V(v) = f_{UV}(u, v)$, per tant hem demostrat que $U$ i $V$ s\'on v.a. independents.

\vskip 0.2 cm
\noindent
\textbf{b)} Si $R \sim {\cal U}(0, 1)$ i $\Theta \sim {\cal U}(0, 2\pi)$, llavors:
$
f_R(r)= \begin{cases} \text{si } 0 \leq r \leq 1 & 1 \\ \text{altrament} & 0 \end{cases}
$ i
$
f_\Theta(\theta)=\begin{cases} \text{si } 0 \leq \theta \leq 2\pi  & \frac{1}{2\pi} 
\\ \text{altrament} & 0 \end{cases}
$

\noindent
Com que $R$ i $\Theta$ s\'on v.a. independents, la seva funci\'o de densitat conjunta \'es:
\[
f_{R\Theta}(r, \theta)=f_R(r) \cdot f_\Theta(\theta) = \begin{cases}
\text{si } 0 \leq r \leq 1 \quad 0 \leq \theta \leq 2\pi  & \frac{1}{2\pi} 
\\ \text{altrament} & 0 \end{cases}
\]

\noindent
Com que $f_{R\Theta}$ \'es diferent de $f_{UV}$ podem afirmar que $(R, \Theta)$ i $(U, V)$ tenen distribucions diferents.
 
\newpage
\noindent
\textbf{Problema 31.}   Siguin $X$ i $Y$ dues v.a. exponencials independents. Trobau
la funci\'o de densitat de $Z=|X-Y|$.

\vskip 0.3 cm
\noindent
\textbf{Soluci\'o:} Calcularem primer $F_Z(z)$ i a continuaci\'o derivarem per a trobar $f_Z(z)$.

\[
F_Z(z)=P(Z \leq z)=P(|X-Y| \leq z)=\begin{cases} P(-z \leq X-Y \leq z) & \text{si } z \leq 0 \\ 0 & \text{en cas contrari} \end{cases}
\]
\noindent
Per al cas $z \geq 0$ denotam $R$ el conjunt de punts $(x, y)$ tals que $-z \leq x-y \leq z$ (figura 1), de manera que
\[
P(-z \leq X-Y \leq z)=\iint_R f_{XY}(x, y) \, dxdy
\]

\setcounter{figure}{0}
\begin{figure}[htbp]
\begin{center}
\begin{picture}(100, 100)
\thicklines
\put(10, 10){\vector(1, 0){90}}
\put(10, 10){\vector(0, 1){90}}
\thinlines
\put(95, 0){$x$}
\put(-5, 95){$y$}
\put(0, 20){\line(1, 1){90}}
\put(0, -20){\line(1, 1){90}}
\put(0, 30){$z$}
\put(30, 0){$z$}
\put(85, 100){$x-y=-z$}
\put(85, 60){$x-y=z$}
\multiput(10, 15)(0, 5){15}{\line(1, 0){80}}
\multiput(5, -15)(5, 5){17}{\line(0, 1){40}}
\multiput(150, 50)(0, 5){4}{\line(1, 0){15}}
\put(170, 55){suport de $f_{XY}$}
\multiput(150, 30)(5, 0){4}{\line(0, 1){10}}
\put(170, 30){$R$}
\end{picture}
\end{center}
\caption{}
\end{figure}

\vskip 0.2 cm
\noindent
D'altra banda, $X$ i $Y$ s\'on v.a. exponencials independents. Si denotam els seus par\`ametres com $\lambda$ i $\alpha$,
respectivament, llavors:
\[
f_X(x)=\begin{cases} \lambda e^{-\lambda x} & \text{si } x>0  \\
0 & \text{si } x \leq 0 \end{cases} 
\qquad
f_Y(y)=\begin{cases} \alpha e^{- \alpha y} & \text{si } y>0  \\
0 & \text{si } y \leq 0 \end{cases} 
\]
\[
f_{XY}(x, y)=f_X(x) \cdot f_Y(y)=
\begin{cases} \lambda \alpha e^{-\lambda x - \alpha y} & \text{si } x>0 \text{ i } y > 0 \\
0 & \text{en cas contrari} \end{cases}
\]

\noindent
El suport de $f_{XY}(x, y)$ s\'on els valors de $x$ i $y$ positius, per tant el c\`alcul de $P(-z \leq X-Y \leq z)$ queda:
\[
\begin{array}{ll}
P(-z \leq X-Y \leq z) & =\displaystyle \iint_{R \cap \Omega_{XY}} \lambda \alpha e^{-\lambda x - \alpha y} \, dxdy=\\ \\
 & = \displaystyle  \int_0^z \left( \int_0^{x+z} \alpha e^{-\lambda x - \alpha y} \, dy \right) \, dx +
\int_z^{+\infty} \left( \int_{x-z}^{x+z} \alpha e^{-\lambda x - \alpha y} \, dy \right) \, dx = \\ \\
 & = \cdots = 1-e^{-\lambda z} + \frac{\lambda}{\lambda + \alpha} (e^{-\lambda z} - e^{-\alpha z})
\end{array}
\]

\vskip 0.2 cm
\noindent
En conclusi\'o:
\[
F_Z(z)=\begin{cases} 
1-e^{-\lambda z} + \frac{\lambda}{\lambda + \alpha} (e^{-\lambda z} - e^{-\alpha z}) & \text{si } z \geq 0\\ \\
0 & \text{en cas contrari} \end{cases}
\]
\vskip 0.2 cm
\noindent
I la funci\'o de densitat de $Z$ \'es per tant:
\[
f_Z(z)=\frac{d F_Z(z)}{dz}=\begin{cases} 
\frac{d \quad}{dz} \left(  1-e^{-\lambda z} + \frac{\lambda}{\lambda + \alpha} (e^{-\lambda z} - e^{-\alpha z}) \right)
= \frac{\lambda \alpha}{\lambda + \alpha} (e^{-\lambda z} + e^{-\alpha z})
 & \text{si } z \geq 0\\ \\
\frac{d 0}{dz}=0 & \text{en cas contrari} \end{cases}
\]






\end{document}

