\documentclass{report}
\usepackage[catalan]{babel}
\usepackage[latin1]{inputenc}   % Permet usar tots els accents i carï¿œters llatins de forma directa.
\usepackage{enumerate}
\usepackage{amsfonts, amscd, amsmath, amssymb}

\setlength{\textwidth}{16.5cm}
\setlength{\textheight}{27cm}
\setlength{\oddsidemargin}{-0.3cm}
\setlength{\evensidemargin}{0.25cm} \addtolength{\headheight}{\baselineskip}
\addtolength{\topmargin}{-3cm}

\newcommand\Z{\mathbb{Z}}
\newcommand\R{\mathbb{R}}
\newcommand\N{\mathbb{N}}
\newcommand\Q{\mathbb{Q}}
\newcommand\K{\Bbbk}
\newcommand\C{\mathbb{C}}

\begin{document}

\begin{center}
\textsc{Control 2 Probabilitat i Processos Aleatoris}
\end{center}

\vspace{1 cm}
\noindent\textbf{P1.-}
L'emissor d'un sistema de comunicacions emet, de manera equiprobable, quatre
tipus de s\'imbols, cada un d'ells format per un parell de valors:
$(-1/2, -1/2)$, $(-1/2, 1/2)$, $(1/2, -1/2)$, $(1/2, 1/2)$. Degut al renou en el canal de
transmissi\'o el receptor rep el parell $(\hat{U}, \hat{V})=(U+X, V+Y)$ quan
s'emet el s\'imbol $(U, V)$, on $(X, Y)$ \'es un renou additiu amb funci\'o de
densitat conjunta
\[
f_{XY}(x, y)=\begin{cases}
K x^2 e^{-y} & \text{si } -1 \leq x \leq 1 \quad \text{ i } \quad -1 \leq y \leq 1 \\ \\
0 & \text{resta}
\end{cases}
\]

La decisi\'o sobre el s\'imbol rebut es fa seguint els seg\"uents criteris:
\begin{itemize}
\item es decideix que s'havia enviat $(-1/2, -1/2)$ si $\hat{U} < 0$ i $\hat{V} < 0$
\item es decideix que s'havia enviat $(-1/2, 1/2)$ si $\hat{U} < 0$ i $\hat{V} \geq 0$
\item es decideix que s'havia enviat $(1/2, -1/2)$ si $\hat{U} \geq 0$ i $\hat{V} < 0$
\item es decideix que s'havia enviat $(1/2, 1/2)$ si $\hat{U} \geq 0$ i $\hat{V} \geq 0$
\end{itemize}

Responeu les seg\"uents q\"uestions:
\begin{enumerate}[a)]
\item Trobau el valor de la constant $K$.\ \hfill{\textbf{ 1 pt.}}
\item Quina \'es la probabilitat de transmetre $(-1/2, -1/2)$ i decidir $(1/2, 1/2)$?\ \hfill{\textbf{ 2 pt.}}
\item Quina \'es la probabilitat de decidir $(1/2, -1/2)$ si s'ha transm\'es $(-1/2, 1/2)$?\ \hfill{\textbf{ 2 pt.}}
\end{enumerate}


\vspace{0.75 cm}

\noindent\textbf{P2.-}
El temps (en segons) que tarda un ordinador en processar cada un dels treballs que li arriben segueix una llei
exponencial amb par\`ametre $\lambda=20$. Es suposa que els treballs s\'on independents
entre s\'\i .
\begin{enumerate}[a)]
\item Donau una estimaci\'o el m\'es aproximada possible de la probabilitat que l'ordinador
tardi entre 10 i 12 segons en processar 200 treballs.\ \hfill{\textbf{ 2 pt.}}
\item Donau una estimaci\'o el m\'es aproximada possible del nombre m\`axim de treballs que pot processar
l'ordinador en menys de 25 segons amb una probabilitat superior al $95\%$.

$ $ \ \hfill{\textbf{3 pt.}}
\end{enumerate}

\vspace{0.75 cm}

\hrule

\vspace{0.5 cm}

\noindent Duraci\'o del control 1h30 hores.\newline

\end{document}
