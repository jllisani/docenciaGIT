\documentclass{report}
\usepackage[catalan]{babel}
\usepackage[latin1]{inputenc}   % Permet usar tots els accents i car�cters llatins de forma directa.
\usepackage{enumerate}
\usepackage{amsfonts, amscd, amsmath, amssymb}

\setlength{\textwidth}{16.5cm}
\setlength{\textheight}{27cm}
\setlength{\oddsidemargin}{-0.3cm}
\setlength{\evensidemargin}{0.25cm} \addtolength{\headheight}{\baselineskip}
\addtolength{\topmargin}{-3cm}

\newcommand\Z{\mathbb{Z}}
\newcommand\R{\mathbb{R}}
\newcommand\N{\mathbb{N}}
\newcommand\Q{\mathbb{Q}}
\newcommand\K{\Bbbk}
\newcommand\C{\mathbb{C}}

\begin{document}

\begin{center}
\textsc{Control 1 Probabilitat i Processos Aleatoris.
Telem\`{a}tica\\
curs 2008/09}
\end{center}

\vspace{1 cm}

Siguin dues v.a. exponencials independents $X$ i $Y$ amb par\`ametres respectius 
$\alpha$ i $\beta$. Definim una nova variable $Z=X+Y$.
\begin{enumerate}[a)]
\item Calculau la funci\'o de densitat conjunta $f_{ZX}(z, x)$. 
\item Calculau la llei marginal $f_Z(z)$. 
\item Calculau la funci\'o de densitat condicionada $f_{X|Z}(x|z)$. 
\end{enumerate}

\end{document}