\documentclass{report}
\usepackage[catalan]{babel}
\usepackage[latin1]{inputenc}   % Permet usar tots els accents i car�ters llatins de forma directa.
\usepackage{enumerate}
\usepackage{amsfonts, amscd, amsmath, amssymb}

\setlength{\textwidth}{16.5cm}
\setlength{\textheight}{27cm}
\setlength{\oddsidemargin}{-0.3cm}
\setlength{\evensidemargin}{0.25cm} \addtolength{\headheight}{\baselineskip}
\addtolength{\topmargin}{-3cm}

\newcommand\Z{\mathbb{Z}}
\newcommand\R{\mathbb{R}}
\newcommand\N{\mathbb{N}}
\newcommand\Q{\mathbb{Q}}
\newcommand\K{\Bbbk}
\newcommand\C{\mathbb{C}}

\begin{document}

\begin{center}
\textsc{Examen Probabilitat i Processos Aleatoris.
Telem\`{a}tica\\
febrer 2009}
\end{center}

\vspace{1 cm}
\noindent\textbf{P1.-}
El sistema de c�pies de seguretat d'una empresa fa una c�pia di�ria
de cada un dels dos discs durs centrals en un moment aleatori entre 
les 3h i les 5h. Les c�pies de cada disc dur es fan de manera independent.
Per un error en la configuraci�, si la difer�ncia
entre els inicis de c�pia de cada disc dur �s inferior a 2 minuts
es produeix un error en el sistema.
\begin{enumerate}[a)]
\item Definiu les variables aleat�ries associades al problema i escriviu les
seves funcions de probabilitat. \ \hfill{\textbf{ 0.5pt.}}
\item Escriviu la funci� de probabilitat conjunta i dibuixau el seu suport. 
\ \hfill{\textbf{ 0.5 pt.}}
\item Quina �s la probabilitat que es produeixi un error en el sistema
un dia qualsevol? \ \hfill{\textbf{ 1.5 pt.}}
\end{enumerate}
\vspace{0.75 cm}




\vspace{0.75 cm}

\noindent\textbf{P2.-}
Per fer un estudi estad�stic sobre la relaci� entre la latitud i la temperatura mitjana de diferents
ciutats es modelen ambdues magnituds mitjan�ant les variables aleat�ries discretes $T$ i $L$,
respectivament i les dades estad�stiques s'utilitzen per a crear la seg�ent taula de 
probabilitats conjuntes:

\vskip 0.2 cm
\begin{center}
\begin{tabular}{|c|c|c|c|c|c|c|}
$T \, \backslash \, L$& $5$ & $15$ & $25$ & $35$ & $45$ & $55$ \\
\hline
$2$ & $0$ & $0$ & $0$ & $0$ & $1/40$ & $3/40$ \\ \hline
$7$ & $0$ & $0$ & $0$ & $0$ & $0$ & $2/40$ \\ \hline
$12$ & $1/40$ & $0$ & $0$ & $1/40$ & $5/40$ & $1/40$  \\ \hline
$17$ & $0$ & $1/40$ & $0$ & $5/40$ & $0$ & $0$ \\ \hline
$22$ & $0$ & $2/40$ & $5/40$ & $1/40$ & $0$ & $0$\\ \hline
$27$ & $k/40$ & $4/40$ & $2/40$ & $0$ & $0$ & $0$\\ \hline
\end{tabular}
\end{center}

\vskip 0.2 cm
\begin{enumerate}[a)]
\item Calculau $k$. \ \hfill{\textbf{ 0.25 pt.}}
\item Calculau les probabilitats marginals. \ \hfill{\textbf{ 0.25 pt.}}
\item Calculau el vector de mitjanes. \ \hfill{\textbf{ 0.5 pt.}}
\item Calculau la matriu de covari�ncies. \ \hfill{\textbf{ 0.5 pt.}}
\item Calculau la recta de regressi� de $T$ sobre $L$. \ \hfill{\textbf{ 0.5 pt.}}
\item Quantificau el grau de relaci� lineal entre les variables.\ \hfill{\textbf{ 0.5 pt.}}
\end{enumerate}

\vspace{0.75 cm}


\noindent\textbf{P3.-}.
El departament de qualitat d'una empresa que fabrica
descodificadors de TDT vol estimar la probabilitat $p$ de que
els aparells s'espenyin en 1 any o menys. Per a aix�
es seleccionen $n$ aparells, es proven durant 1 any i es
calcula la mitjana mostral dels que no funcionen al cap
d'aquest temps. Quin ha d'�sser el valor
m�nim de $n$ per assegurar que la difer�ncia entre la mitjana
mostral calculada i el valor real $p$ sigui inferior a 0.01
amb una probabilitat, com a m�nim, del $95\%$?
\ \hfill{\textbf{ 2.5 pt.}}

\vspace{0.75 cm}

\noindent\textbf{P4.-} La transmissi� d'un senyal binari est� contaminada
per un renou blanc Gaussi� de mitjana $0$ i densitat 
espectral de pot�ncia $5$. Aquest renou s'afegeix (renou additiu)
al senyal transm�s $v(t)$. 

\noindent
$v(t)$ pren el valor $+A$ quan es vol 
transmetre un `1' i $-A$ si es vol transmetre un `0' $(A > 0)$. 
En recepci� es decideix que s'ha rebut un `1' si el senyal rebut �s positiu
i un `0' en cas contrari. Quin �s el valor m�nim de $A$ que assegura 
que, com a m�nim, 999 de cada 1000 bits es reben correctament?
\ \hfill{\textbf{1.25 pt.}}


\vspace{0.75 cm}

\noindent\textbf{P5.-} 
Sigui $S_n$ un proc�s de tipus passejada aleat�ria.
Calculau $P(S_{n+3}=0|_{S_n=1})$.
\ \hfill{\textbf{1.25 pt.}}



\vspace{0.75 cm}

\hrule

\vspace{0.5 cm}

\noindent Duraci\'o de l'examen 4 hores.\newline

\end{document}
