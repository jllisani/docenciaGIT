\documentclass{report}
\usepackage[catalan]{babel}
\usepackage[latin1]{inputenc}   % Permet usar tots els accents i car�ters llatins de forma directa.
\usepackage{enumerate}
\usepackage{amsfonts, amscd, amsmath, amssymb}

\setlength{\textwidth}{16.5cm}
\setlength{\textheight}{27cm}
\setlength{\oddsidemargin}{-0.3cm}
\setlength{\evensidemargin}{0.25cm} \addtolength{\headheight}{\baselineskip}
\addtolength{\topmargin}{-3cm}

\newcommand\Z{\mathbb{Z}}
\newcommand\R{\mathbb{R}}
\newcommand\N{\mathbb{N}}
\newcommand\Q{\mathbb{Q}}
\newcommand\K{\Bbbk}
\newcommand\C{\mathbb{C}}

\begin{document}

\begin{center}
\textsc{Examen Probabilitat i Processos Aleatoris.
Telem\`{a}tica\\
setembre 2009}
\end{center}

\vspace{0.5 cm}
\noindent\textbf{P1.-}
Un malalt que ha contret un virus arriba a l'hospital universitari
de Princetown. El doctor House \'es el seu metge i fa el seg\"uent
raonament per determinar el nombre de dies que haur\`a d'estar 
hospitalitzat: anomena $X$ a la v.a. aleat\`oria que compta el temps,
en dies, que tarda en manifestar-se la malaltia. Anomena $Y$ al
temps total, en dies, des de que el virus s'ha contret fins que el pacient
\'es donat d'alta. Si suposam que en el moment que apareixen els 
s\'\i mptomes el pacient \'es hospitalitzat, llavors la v.a. $Z=Y-X$
compta el temps d'hospitalitzaci\'o. House, que va llegir un estudi
estad\'\i stic sobre la evoluci\'o del virus, 
sap que la funci\'o de densitat conjunta de $X$ i $Y$ \'es:
\[
f_{XY}(x, y)=\begin{cases} 4 x \mathrm{e}^{-y} & \text{si } 0 \leq 2x \leq y \\ \\
0 & \text{altrament}
\end{cases}
\]

\noindent
Ajudau al doctor House i responeu a les seg\"uents q\"uestions:
\begin{enumerate}[a)]
\item Calculau la funci\'o de densitat conjunta de les variables $Z$ i $Y$ i dibuixau el seu suport. 
\ \hfill{\textbf{ 1.5 pt.}}
\item Calculau la funci\'o de densitat de $Z$. \ \hfill{\textbf{ 0.75 pt.}}
\item Suposant que el pacient \'es donat d'alta despr\'es de 3 dies hospitalitzat,
quina \'es la probabilitat que contragu\'es el virus menys de 5 dies abans d'\'esser donat d'alta?.
 \ \hfill{\textbf{ 0.75 pt.}}
\end{enumerate}

\vskip 0.2 cm
\noindent
\textbf{Indicaci\'o:} $\int x \mathrm{e}^{-x} \, dx = - \mathrm{e}^{-x} (1+x) + C $





\vspace{0.75 cm}

\noindent\textbf{P2.-}
Tenim una urna amb tres bolles blanques i dues negres. Es fan tres extraccions sense reposici\'o.
Sigui $N$ la variable aleat\`oria que compta el nombre de bolles blanques extretes, i sigui
$M$ la variable que compta el nombre de bolles negres extretes {\it abans} de la primera bolla blanca.
Es demana:
\begin{enumerate}[a)]
\item Calculau la funci\'o de probabilitat conjunta de $N$ i $M$.\ \hfill{\textbf{ 0.5 pt.}}
\item Calculau $P(|M-N|\leq 2)$. \ \hfill{\textbf{ 0.5 pt.}}
\item Calculau l'esperan\c{c}a i la vari\`ancia de $N$. \ \hfill{\textbf{ 0.25 pt.}}
\item Calculau l'esperan\c{c}a i la vari\`ancia de $M$. \ \hfill{\textbf{ 0.25 pt.}}
\item Calculau la covari\`ancia i el coeficient de correlaci\'o de $N$ i $M$. \ \hfill{\textbf{ 0.5 pt.}}
\end{enumerate}

\vspace{0.75 cm}


\noindent\textbf{P3.-}.
El motor que permet orientar una antena parab\`olica produeix un error en l'orientaci\'o de
$\varepsilon$ graus cada vegada que s'acciona, on $\varepsilon \sim N(0, 1)$. Els errors en
l'orientaci\'o s'acumulen despr\'es de cada actuaci\'o del motor i s\'on independents entre s\'i.
\begin{enumerate}[a)]
\item Quina \'es la probabilitat que l'error d'orientaci\'o sigui superior a $5$ graus despr\'es de
$100$ actuacions del motor?\ \hfill{\textbf{ 1.25 pt.}}
\item Quan l'error acumulat (en valor absolut) \'es superior a $10$ graus l'antena s'ha de recalibrar.
Quin \'es el nombre m\`axim d'actuacions del motor que es poden fer si es vol garantir, amb una
probabilitat del $95\%$, que l'antena no necessita \'esser recalibrada?\ \hfill{\textbf{ 1.25 pt.}}
\end{enumerate}

\vspace{0.75 cm}

\noindent\textbf{P4.-} La transmissi� d'un senyal binari est� contaminada
per un renou blanc Gaussi� de mitjana $0$ i densitat 
espectral de pot�ncia $5$. Aquest renou s'afegeix (renou additiu)
al senyal transm�s $v(t)$. 

\noindent
$v(t)$ pren el valor $+A$ quan es vol 
transmetre un `1' i $-A$ si es vol transmetre un `0' $(A > 0)$. 
En recepci� es decideix que s'ha rebut un `1' si el senyal rebut �s positiu
i un `0' en cas contrari. Quin �s el valor m�nim de $A$ que assegura 
que, com a m�nim, 999 de cada 1000 bits es reben correctament?
\ \hfill{\textbf{1.25 pt.}}


\vspace{0.75 cm}

\noindent\textbf{P5.-}
Sigui $S_n$ el proc\'es suma seg\"uent: $S_n=X_1+X_2+\cdots+X_n$.
Les variables aleat\`ories $X_i$ s\'on variables discretes iid que prenen valors $-1$, $0$ o $1$ amb 
probabilitats respectives $\frac{1}{4}$, $\frac{1}{2}$ i $\frac{1}{4}$.
Calculau $P(S_3=k)$ per a tots els valors possibles de $k$. 
\ \hfill{\textbf{ 1.25 pt.}}



\vspace{0.5 cm}

\hrule

\vspace{0.5 cm}

\noindent Duraci\'o de l'examen 4 hores.\newline

\end{document}
