\documentclass{article}
\usepackage[catalan]{babel}
\usepackage[latin1]{inputenc}   % Permet usar tots els accents i car�ters llatins de forma directa.
\usepackage{enumerate}
\usepackage{amsfonts, amscd, amsmath, amssymb}

\setlength{\textwidth}{16cm}
\setlength{\textheight}{25cm}
\setlength{\oddsidemargin}{-0.3cm}
\setlength{\evensidemargin}{0.25cm} \addtolength{\headheight}{\baselineskip}
\addtolength{\topmargin}{-3cm}

\newcommand\Z{\mathbb{Z}}
\newcommand\R{\mathbb{R}}
\newcommand\N{\mathbb{N}}
\newcommand\Q{\mathbb{Q}}
\newcommand\K{\Bbbk}
\newcommand\C{\mathbb{C}}

\newcounter{exctr}
\newenvironment{exemple}
{ \stepcounter{exctr} 
\hspace{0.2cm} 
\textit{Exemple  \arabic{exctr}: }
\it
\begin{quotation}
}{\end{quotation}}


\begin{document}

\textit{
Sabent que a un canal donat la probabilitat de transmetre correctament una trama �s PRR. Com es podria calcular el nombre total de trames necess�ries a transmetre per assegurar, amb certa probabilitat, que es transmetran n trames correctament? 
}


\vskip 0.5 cm
\noindent
\textbf{Plantejament:}


\vskip 0.3 cm
\noindent
Denotam:


\noindent
$p$: probabilitat d'�xit en la transmissi� d'una trama (PRR)

\noindent
$P_{min}$: probabilitat m�nima de transmetre correctament $n$ trames


\noindent
$X$: nombre de trames correctes en $N$ transmissions


\vskip 0.3 cm
$X$ �s una v.a. de tipus binomial: $X \sim B(N, p)$ i el problema consisteix en calcular
el valor de $N$ m�nim que assegura que
\[
P(X \geq n) \geq P_{min}
\]


\vskip 0.5 cm
\noindent
\textbf{Resoluci�:}
\vskip 0.3 cm
\noindent
La dificultat del problema �s que no es pot donar una f�rmula per al c�lcul del valor
de $N$ ja que els valors de estan tabulats. No obstant, es pot escriure un petit
programa que calculi una bona aproximaci� num�rica, no �s dificil i si est�s 
interesat pots passar pel despatx perqu� t'ho expliqui.

\vskip 0.3 cm
\noindent
Et donc dos exemples de resoluci� utilitzant taules:

\vskip 0.5 cm
\begin{exemple}
Si $n$ �s petit es pot utilitzar la taula de la binomial. Per exemple, suposem $p=0.9$,
$P_{min}=0.9$ i $n=10$:

\[
\begin{array}{l}
P(X \geq 10)=1-P(X < 10)=1-P(X \leq 9) \geq 0.9 \\ \\
P(X \leq 9) \leq 1-0.9=0.1
\end{array}
\]

\noindent
Hauriem de mirar la columna de la taula binomial corresponent a $p=0.9$ i per a $k=9$,
per� a la taula de la binomial nom�s surten valors de $p$ fins a $p=0.5$.
Per solucionar aix� definim una nova variable $Y$=nombre transmisions err�nies de $N$,
com que $Y=N-X$, llavors:
\[
\begin{array}{l}
P(X \geq 10)=P(N-Y \geq 10)=P(Y \leq N-10) \geq 0.9 
\end{array}
\]

\noindent
Miram la columna de la taula binomial corresponent a $p=0.9$ i per a $k=N-10$
per a valors de $N=10, 11, \cdots$. En aquest cas trobam que la soluci� �s $N=13$
ja que en aquest cas $P(X \geq 10)=P(Y \leq 3)=0.9658 > 0.9$.

\end{exemple}


\vskip 0.5 cm
\begin{exemple}
Si $n$ �s gran, la v.a. binomial es pot aproximar per una Gaussiana, que tamb� est� tabulada.
Per exemple, suposem $p=0.9$, $P_{min}=0.9$ i $n=100$:

\[
\begin{array}{l}
P(X \geq 100)=1-P(X < 100)=1-P(X \leq 99) \geq 0.9 \\ \\
P(X \leq 99) \leq 1-0.9=0.1 \\ \\
P(X \leq 99) \approx P(X' \leq 99.5)=F_{X'}(99.5) 
\end{array}
\]

\noindent
on $X' \sim N(N \cdot p, N \cdot p \cdot (1-p))=N(0.9 \cdot N, 0.09 \cdot N)$. Per tant:

\[
F_{X'}(99.5)=F_Z(\frac{99.5 - 0.9 \cdot N}{\sqrt{0.09 \cdot N}}) \leq 0.1
\]

\noindent
Com el valor $0.1$ no surt a la taula de la normal est�ndar aplicam la propietat
$F_Z(z)=1-F_Z(-z)$:

\[
F_Z(\frac{99.5 - 0.9 \cdot N}{\sqrt{0.09 \cdot N}}) =
1- F_Z(-\frac{99.5 - 0.9 \cdot N}{\sqrt{0.09 \cdot N}}) \leq 0.1 
\]

\noindent
Per tant 
\[
F_Z(-\frac{99.5 - 0.9 \cdot N}{\sqrt{0.09 \cdot N}}) \geq 0.9
\]

\noindent
Mirant la taula:
\[
-\frac{99.5 - 0.9 \cdot N}{\sqrt{0.09 \cdot N}} \geq 1.29
\]

\noindent
Resolent aquesta equaci� trobarem $N$:
\[
\begin{array}{l}
-99.5+0.9N \geq 1.29 \sqrt{0.09N} \\ \\
(-99.5+0.9N)^2 \geq 1.29^2 \cdot 0.09N \\ \\
0.81N^2 - 179.25 N + 9900.25 \geq 0
\end{array}
\]

\noindent
Resolent l'equaci� de segon grau:
\[
(N-115.17)(N-106.12) \geq 0
\]

\noindent
La inequaci� t� dues solucions: $N < 106.12$ i $N > 115.17$.
La primera no �s v�lida ja que d�na valors de probabilitat molt baixos.
La soluci� final �s per tant $N=116$.



\end{exemple}

\end{document}
