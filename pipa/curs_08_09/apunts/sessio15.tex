\documentclass{article}
\usepackage[catalan]{babel}
\usepackage[latin1]{inputenc}   % Permet usar tots els accents i car\`acters llatins de forma directa.
\usepackage{enumerate}
\usepackage{amsfonts, amscd, amsmath, amssymb}
\usepackage{fancyheadings}

\setlength{\textwidth}{16cm}
\setlength{\textheight}{25cm}
\setlength{\oddsidemargin}{-0.3cm}
\setlength{\evensidemargin}{0.25cm} \addtolength{\headheight}{\baselineskip}
\addtolength{\topmargin}{-3cm}

\newcommand\Z{\mathbb{Z}}
\newcommand\R{\mathbb{R}}
\newcommand\N{\mathbb{N}}
\newcommand\Q{\mathbb{Q}}
\newcommand\K{\Bbbk}
\newcommand\C{\mathbb{C}}

\newcounter{exctr}
\setcounter{exctr}{6}
\newenvironment{exemple}
{ \stepcounter{exctr} 
\hspace{0.2cm} 
\textit{Exemple  \arabic{exctr}: }
\it
\begin{quotation}
}{\end{quotation}}

\pagestyle{fancy}
\markboth{Tema 3. Processos aleatoris}{}
\setcounter{page}{4}
\setlength{\headrulewidth}{0pt}



\begin{document}

\textbf{\Large Processos aleatoris t\'ipics}

\vskip 0.5 cm
\textbf{Proc\'es Gaussi\`a}

Un proc\'es aleatori es diu Gaussi\`a si per 
a qualsevol elecci\'o de mostres $t_1, \cdots, t_k$ i qualsevol valor de $k$,
les variables $X(t_1), \cdots, X(t_k)$ s\'on v.a. conjuntament Gaussianes.

\vskip 0.2 cm
Propietats:
\begin{enumerate}
\item Un proc\'es Gaussi\`a ho pot ser tant en temps discret com continu.
\item La funci\'o de densitat d'un proc\'es Gaussi\`a queda totalment 
determinada per les esperances i les covari\`ancies de les variables que el formen.
\item En un proc\'es Gaussi\`a cada una de les v.a. $X(t_i)$ \'es Gaussiana.
\item Qualsevol operaci\'o lineal (suma, derivaci\'o, integraci\'o, etc) damunt
un proc\'es Gaussi\`a d\'ona lloc a un nou proc\'es Gaussi\`a.
\item Molts senyals i renous en Comunicacions es poden modelar com a processos Gaussians,
per aquest motiu \'es el tipus de proc\'es aleatori m\'es utilitzat en processament
de senyals. 
\end{enumerate}

\vskip 0.4 cm
Exercicis proposats: 9, 10


\vskip 0.5 cm
\textbf{\large Temps Discret}

\vskip 0.3 cm
\begin{itemize}
\item \textbf{Proc\'es suma}. \'Es un proc\'es discret que s'obt\'e
com a suma de variables aleat\`ories i.i.d.:
\[
S_n=X_1+X_2+\cdots+X_n \qquad \qquad n=1, 2, \cdots
\]


\vskip 0.2 cm
Propietats: 
\begin{enumerate}
\item $S_n=S_{n-1}+X_n$
\item Si $X_i$ s\'on Bernouilli amb par\`ametre $p$, llavors $S_n \sim B(n, p)$
i el proc\'es es diu de \textbf{suma binomial}.
\end{enumerate}

\vskip 0.3 cm
\begin{exemple}
(Setembre 2005). Sigui $S_n$ el proc\'es suma seg\"uent: $S_n=X_1+X_2+\cdots+X_n$.
Les variables aleat\`ories $X_i$ s\'on variables discretes iid que prenen valors $-1$, $0$ o $1$ amb 
probabilitats respectives $\frac{1}{4}$, $\frac{1}{2}$ i $\frac{1}{4}$.
Calculau $P(S_3=k)$ per a tots els valors possibles de $k$. 
\end{exemple}

\vskip 0.4 cm
Exercicis proposats: 11


\item \textbf{Proc\'es de passejada aleat\`oria}. \'Es un proc\'es de
suma binomial on les $X_i$ prenen valors $+1$ o $-1$.

\vskip 0.3 cm
\begin{exemple}
(Exercici 12). Trobau $P(S(n)=0)$ per al proc\'es de la passejada aleat\`oria.
\end{exemple}


\end{itemize}


\vskip 0.4 cm
Exercicis proposats: 13

\newpage
\textbf{\large Temps Continu}

\vskip 0.3 cm
\begin{itemize}

\item \textbf{Proc\'es de Poisson}. Si $N(t)$ \'es la funci\'o que compta
el nombre d'\textit{esdeveniments} que s'han produit fins a l'instant de temps $t$
(\'es a dir, en l'interval $(0, t)$) i el promig d'\textit{esdeveniments} per
unitat de temps \'es $t$, llavors $N(t)$ s'anomena proc\'es de Poisson.

\vskip 0.2 cm
Propietat: $N(t) \sim \mathrm{Po}(\lambda t)$, per tant, 
$\displaystyle P(N(t) = k)=\frac{(\lambda t)^k}{k!} e^{-\lambda t}$, $k=0, 1, \cdots$


\vskip 0.3 cm
\begin{exemple}
(Exercici 14).
Suposem que una secret\`aria rep cridades  que  arriben
d'acord amb un proc\'es de Poisson amb un ritme de 10 cridades per
hora. Quina \'es la probabilitat  que cap cridada es quedi sense
resposta si la secret\`aria surt de l'oficina els primers 15 i els
darrers 15 minuts d'una hora?
\end{exemple}

\vskip 0.3 cm
\begin{exemple}
(Exercici 16).
Un impuls de renou ocorr en una l\'{\i}nia telef\`onica d'acord amb
un proc\'es Poisson de par\`ametre $\lambda$ per segon.
\begin{enumerate}[a)]
\item Trobau la probabilitat  que no ocorri cap impuls
en  el transcurs d'un missatge de $t$ segons.
\item Suposem que el missatge est\`a codificat i que si s'ha
produ\"{\i}t un impuls podem corregir el missatge. Quina \'es la
probabilitat  que un missatge de $t$ segons estigui lliure
d'errors o es pugui corregir?
\end{enumerate}
\end{exemple}


\vskip 0.3 cm
\begin{exemple}
(Exercici 17).
Els missatges arriben a un ordinador des de dues l\'{\i}nies
telef\`oniques segons dos processos de Poisson independents i amb
ritmes $\lambda_{1}$ i $\lambda_{2}$ respectivament.
\begin{enumerate}[a)]
\item Trobau la probabilitat  que un missatge arribi primer per la
  linea $1$.
\item  Trobau la funci\'o de densitat del temps que tarda un
  missatge en arribar per alguna de les l\'inies.
\item Trobau la probabilitat de $N(t)$ el nombre total de
  missatges
  que arriben a l'ordinador en un interval de longitud $t$.
\item  Generalitzau el resultat anterior quan es junten $k$
  l\'{\i}nies telef\`oniques independents Poisson amb par\`ametres
  $\lambda_{1},\ldots,\lambda_{k}$ y
  $N(t)=N_{1}(t)+\ldots+N_{k}(t)$.
\end{enumerate}
\end{exemple}

\vskip 0.5 cm
Exercicis proposats: 15, 18

\end{itemize}



\end{document}