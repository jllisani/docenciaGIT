\documentclass{article}
\usepackage[catalan]{babel}
\usepackage[latin1]{inputenc}   % Permet usar tots els accents i car\`acters llatins de forma directa.
\usepackage{enumerate}
\usepackage{amsfonts, amscd, amsmath, amssymb}
\usepackage{fancyheadings}

\setlength{\textwidth}{16cm}
\setlength{\textheight}{25cm}
\setlength{\oddsidemargin}{-0.3cm}
\setlength{\evensidemargin}{0.25cm} \addtolength{\headheight}{\baselineskip}
\addtolength{\topmargin}{-3cm}

\newcommand\Z{\mathbb{Z}}
\newcommand\R{\mathbb{R}}
\newcommand\N{\mathbb{N}}
\newcommand\Q{\mathbb{Q}}
\newcommand\K{\Bbbk}
\newcommand\C{\mathbb{C}}

\newcounter{exctr}
\setcounter{exctr}{2}
\newenvironment{exemple}
{ \stepcounter{exctr} 
\hspace{0.2cm} 
\textit{Exemple  \arabic{exctr}: }
\it
\begin{quotation}
}{\end{quotation}}

\pagestyle{fancy}
\markboth{Tema 3. Processos aleatoris}{}
\setcounter{page}{2}
\setlength{\headrulewidth}{0pt}



\begin{document}

\textbf{\Large Relaci\'o entre els processos aleatoris i les variables aleat\`ories}

Si representam varies realitzacions d'un proc\'es aleatori i consideram els 
valors de cada realitzaci\'o per a un valor constant de $t_i$, obtenim un conjunt de
nombres aleatoris. Cadascun d'aquests nombres est\`a associat a un \'unic resultat de
l'experiment aleatori. 
\[
\begin{array}{rccc}
X(t_i): & \Omega & \longrightarrow  & \R\\
   & \omega_1 & \rightarrow & X(\omega_1, t_i) \\
   & \omega_2 & \rightarrow & X(\omega_2, t_i) \\
   & \vdots   & \vdots      &   \vdots 
\end{array}
\]

Recordem que una variable aleat\`oria es definia com un conjunt de nombres, cada un d'ells
associats a un resultat d'un experiment aleatori. De manera que 
\textbf{$X(t_i)$ \'es una variable aleat\`oria, associada a l'instant de temps $t_i$ 
d'un proc\'es aleatori}. 

Per a cada instant de temps (tamb\'e anomenat \'index) que considerem tendrem una 
nova variable aleat\`oria, per aquest motiu es diu que \textbf{un proc\'es aleatori 
\'es una col.lecci\'o indexada de variables aleat\`ories}.

\vskip 0.2 cm
\textbf{Notaci\'o}. Hi ha v\`aries maneres de denotar els processos aleatoris.
$X(t)$ o $X_{t}$ s'utilitza per a denotar la v.a. associada al temps $t$ en el 
cas continu. En el cas discret \'es m\'es freq\"uent utilitzar la notaci\'o 
$X_i$, que representa la v.a. associada a l'instant de temps $t_i$.

\vskip 0.2 cm
\textbf{Observaci\'o}. Les v.a. associades a un proc\'es aleatori discret no s\'on
necess\`ariament discretes ni les associades a un proc\'es continu s\'on cont\'inues.



\vskip 0.5 cm
\textbf{\Large Caracteritzaci\'o dels processos aleatoris}

Com que un proc\'es aleatori \'es una colecci\'o de variables aleat\`ories, una
forma de caracteritzar-lo (definir les seves propietats) \'es mitjan\c{c}ant
la funci\'o de probabilitat (v.a. discretes) o de densitat (v.a. cont\'inues) conjunta
de totes aquestes variables:
\[
\begin{array}{ll}
P(X_1=x_1, X_2=x_2, \cdots) & \text{(cas discret)} \\ \\
f_{X_1 X_2 \cdots}(x_1, x_2, \cdots) & \text{(cas continu)}
\end{array}
\]

\vskip 0.3 cm
En general \'es impossible calcular la funci\'o de probabilitat o densitat conjunta
d'un proc\'es que pot estar definit per una infinitat de v.a. Per aquest motiu \'es
m\'es habitual calcular altres caracter\'istiques d'aquestes v.a. com ara els 
seus moments (esperan\c{c}a i vari\`ancia) i els seus moments conjunts (autocorrelaci\'o
i autocovari\`ancia).

\vskip 0.5 cm
Donat un proc\'es aleatori $X(t)$, definim:
\begin{itemize}
\item \textbf{Mitjana} o \textbf{esperan\c{c}a} de $X(t)$: $m_X(t)=E(X(t))$
\item \textbf{Autocorrelac\'o} de $X(t)$: $R_X(t_1, t_2)=E(X(t_1) \cdot X(t_2))$
\item \textbf{Autocovari\`ancia} de $X(t)$: $C_X(t_1, t_2)=R_X(t_1, t_2)- m_X(t_1) \cdot m_X(t_2)$
\item \textbf{Vari\`ancia} de $X(t)$: $\mathrm{Var}(X(t))=C_X(t, t)$
\item \textbf{Coeficient de correlaci\'o} de $X(t)$: 
$\displaystyle \rho_X(t_1, t_2)=\frac{C_X(t_1, t_2)}{\sqrt{C_X(t_1, t_1)} \sqrt{C_X(t_2, t_2)}}$
\item Propietats: 

 $R_X(t_1, t_2)=R_X(t_2, t_1)$

 $C_X(t_1, t_2)=C_X(t_2, t_1)$
\end{itemize}

\vskip 0.5 cm
Donat dos processos aleatoris $X(t)$ i $Y(t)$, definim:
\begin{itemize}
\item \textbf{Correlaci\'o creuada} de $X(t)$ i $Y(t)$: $R_{XY}(t_1, t_2)=E(X(t_1) \cdot Y(t_2))$
\item \textbf{Covari\`ancia creuada} de $X(t)$ i $Y(t)$: 
$C_{XY}(t_1, t_2)=R_{XY}(t_1, t_2)-m_X(t_1) \cdot m_Y(t_2)$
\item Els processos es diuen \textbf{ortogonals} si $R_{XY}(t_1, t_2)=0$, $\forall t_1, t_2$
\item Els processos es diuen \textbf{incorrelats} si $C_{XY}(t_1, t_2)=0$, $\forall t_1, t_2$
\item Els processos es diuen \textbf{independents} si 
els vectors aleatoris $(X(t_1), X(t_2), \cdots, X(t_k))$ i
$(Y(t'_1), Y(t'_2), \cdots, Y(t'_j))$ s\'on independents per a tot $k$ i $j$
i qualsevol elecci\'o de $t_1, \cdots, t_k$ i $t'_1, \cdots, t'_j$.
\end{itemize}

\vskip 0.5 cm
\begin{exemple}
(Exercici 2).Considerem el proc\'es aleatori en temps  discret $X(n)$
definit a continuaci\'o. Es llan\c{c}a una moneda a l'aire;  si surt
cara $X(n)=(-1)^n$ i $X(n)=(-1)^{n+1}$ si surt creu, per a tot $n$.

\begin{enumerate}[a)]
\item Dibuixau alguns camins de mostra del proc\'es.
\item Calculau la funci\'o de probabilitat de $X(n)$.
\item Calculau la funci\'o de probabilitat conjunta  de $X(n)$ i
$X(n+k)$.
\item Calculau $\mu_{X}(n)$ y $C_{X}(n,m)$.
\end{enumerate}
\end{exemple}

\vskip 0.5 cm
\begin{exemple}
(Exercici 3).Sigui $g(t)$ un pols rectangular a l'interval $(0,1)$, \'es a
dir $g(t)=1$ si $t\in(0,1)$ i zero a la resta de casos. Considerem
el proc\'es aleatori definit per $X(t)=Ag(t)$ on $A=\pm 1$ amb la
mateixa probabilitat.

\begin{enumerate}[a)]
\item Calculau la funci\'o de probabilitat de $X(t)$.
\item Calculau $\mu_{X}(t)$
\item Calculau la funci\'o de probabilitat conjunta  de $X(t)$ i
$X(t+d)$, amb $d > 0$.
\item Calculau  $C_{X}(t,t+d)$ amb $d>0$.
\end{enumerate}
\end{exemple}

\vskip 0.5 cm
\begin{exemple}
(Exercici 6).Sigui $Z(t)=A t+B$ on $A$ i $B$ s\'on v.a. independents.
\begin{enumerate}[a)]
\item Calculau la funci\'o de densitat de $Z(t)$.
\item Trobau $\mu_{Z}(t)$ i $C_{Z}(t_{1},t_{2})$.
\end{enumerate}
\end{exemple}

\vskip 0.5 cm
\begin{exemple}
Calculau $C_{XY}(t_1, t_2)$ per als processos $X(t)=\cos(\omega t + \Theta)$
i $Y(t)=\sin(\omega t + \Theta)$, on $\Theta \sim {\cal U}(-\pi, \pi)$.
\end{exemple}

\vskip 0.5 cm
\noindent
Exercicis proposats: 1, 4, 5, 7, 8

\end{document}
