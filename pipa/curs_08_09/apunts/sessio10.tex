\documentclass{article}
\usepackage[catalan]{babel}
\usepackage[latin1]{inputenc}   % Permet usar tots els accents i car�ters llatins de forma directa.
\usepackage{enumerate}
\usepackage{amsfonts, amscd, amsmath, amssymb}
\usepackage{fancyheadings}

\setlength{\textwidth}{16cm}
\setlength{\textheight}{25cm}
\setlength{\oddsidemargin}{-0.3cm}
\setlength{\evensidemargin}{0.25cm} \addtolength{\headheight}{\baselineskip}
\addtolength{\topmargin}{-3cm}

\newcommand\Z{\mathbb{Z}}
\newcommand\R{\mathbb{R}}
\newcommand\N{\mathbb{N}}
\newcommand\Q{\mathbb{Q}}
\newcommand\K{\Bbbk}
\newcommand\C{\mathbb{C}}

\newcounter{exctr}
\setcounter{exctr}{2}
\newenvironment{exemple}
{ \stepcounter{exctr} 
\hspace{0.2cm} 
\textit{Exemple  \arabic{exctr}: }
\it
\begin{quotation}
}{\end{quotation}}

\pagestyle{fancy}
\markboth{Tema 2. Suma de variables aleat\`ories}{}
\setcounter{page}{2}
\setlength{\headrulewidth}{0pt}



\begin{document}

\noindent
\textbf{\large Esperan\c{c}a i vari\`ancia d'una suma de variables aleat\`ories}
\vskip 0.2cm
Si $S_n=X_1 + X_2 + \cdots + X_n$, on $X_1, \cdots, X_n$ s\'on variables aleat\`ories i $n$ \'es 
constant, llavors:

\[
\begin{array}{l}
\displaystyle E(S_n)=E(X_1 + X_2 + \cdots + X_n)=E(X_1)+E(X_2)+\cdots+E(X_n)=\sum_{i=1}^n E(X_i) \\ \\
\displaystyle \mathrm{Var}(S_n)=\mathrm{Var}(X_1 + X_2 + \cdots + X_n)=
\mathrm{Cov}(X_1, X_1)+ \mathrm{Cov}(X_1, X_2) + \cdots + \mathrm{Cov}(X_n, X_n)=
\sum_{i=1}^n \sum_{j=1}^n \mathrm{Cov}(X_i, X_j) 
\end{array}
\]

\vskip 0.2 cm
\noindent
Propietats:
\begin{itemize}
\item si $X_1, \cdots, X_n$ s\'on \textbf{independents}, llavors:  
\[
\mathrm{Var}(X_1 + X_2 + \cdots + X_n)=\mathrm{Var}(X_1)+\mathrm{Var}(X_2)+\cdots+\mathrm{Var}(X_n)
\]
\item si $X_1, \cdots, X_n$ s\'on \textbf{i.i.d.}\footnote{
Les v.a. $X_1, \cdots, X_n$ es diu que s\'on \textbf{i.i.d.} si s\'on \textbf{independents}
i estan \textbf{id\`enticament distribu\"\i des}, \'es a dir, si totes tenen la mateixa funci\'o
de densitat o probabilitat, i en conseq\"u\`encia $E(X_1)=E(X_2)=\cdots=E(X_n)=E(X)$ i
$\mathrm{Var}(X_1)=\mathrm{Var}(X_2)=\cdots=\mathrm{Var}(X_n)=\mathrm{Var}(X)$.}
, llavors 

\[
\begin{array}{l}
E(X_1 + X_2 + \cdots + X_n)=n \cdot E(X) \\ \\
\mathrm{Var}(X_1 + X_2 + \cdots + X_n)=n \cdot \mathrm{Var}(X)
\end{array}
\]
\end{itemize}

\vskip 0.2 cm
\begin{exemple}
(Exercici 1).
Sigui $W = X + Y + Z$, on  $X$, $Y$ i $Z$ s\'on variables
aleat\`ories amb mitjana 0 i vari\`ancia 1, i amb $ Cov(X,Y) = 1/4,
Cov(X,Z) = 0, Cov(Y,Z) = -1/4.$
\begin{enumerate}[a)]
\item Trobau l'esperan\c{c}a i la vari\`ancia de $W$. 
\item Repetiu l'apartat (a) suposant que $X, Y$ i $Z$
estan incorrelades. 
\end{enumerate}
\end{exemple}

\vskip 0.2 cm
\begin{exemple}
(Exercici 3).
Siguin $X_1, \ldots, X_n \> $ variables aleat\`ories amb la
mateixa mitjana $\mu$ i covari\`ancies
$Cov(X_{i},X_{j})=\sigma^2
\rho^{|i-j|}$ per $i,j=1,\ldots,n$ amb $\sigma>0$ i $|\rho|<1$.
Determinau la mitjana i la vari\`ancia de $ S_n =
X_1+ \cdots +X_n.$
\end{exemple}

\vskip 1.5 cm
\noindent
Exercicis proposats: 2


\newpage
\noindent
\textbf{\large Suma d'un nombre aleatori de v.a.}
\vskip 0.2cm
Sigui $S_N=X_1 + X_2 + \cdots + X_N$, on $X_1, \cdots, X_N$ s\'on variables aleat\`ories i $N$ \'es 
un nombre aleatori enter positiu, llavors, si les variables s\'on i.i.d.:
\[
\begin{array}{l}
E(X_1 + X_2 + \cdots + X_N)=E(N) \cdot E(X) \\ \\
\mathrm{Var}(X_1 + X_2 + \cdots + X_N)=E(N) \cdot \mathrm{Var}(X) + E(X)^2 \cdot \mathrm{Var}(N)
\end{array}
\]

\vskip 0.2 cm
\begin{exemple}
(Exercici 6).
Sigui $N$ una v.a. que pren  valors enters positius. Sigui
$X_{1},\ldots,X_{N}$ una seq\"{u}\`encia de $N$ v.a. iid.
Considerem $S_{N}=\sum_{k=1}^N X_{k}$.
\begin{enumerate}[a)]
\item Calculau $E(S_{N}|N)$. (Sol.:$\mathbf{N E(X)}$).
\item Calculau $E(S_{N})=E(E(S_{N}|N))$. (Sol.:$\mathbf{E(N) E(X)}$)
\item Demostrau que  $E(e^{j\omega S_{N}}|N)=\Phi_{X_{1}}(\omega)^N$.
\item Demostrau que $\Phi_{S_{N}}(\omega)=G_{N}(\phi_{X_{1}}(\omega))$
on $G_{N}(z)=E(z^N)$ \'es la funci\'o generadora de probabilitats
de $N$.(ind.: $\Phi_{S_{N}}(\omega)=E(E(e^{j\omega S_{N}}|N))$).
\end{enumerate}
\end{exemple}

\vskip 0.2 cm
\begin{exemple}
(Exercici 7).
Sigui $X_{1},X_{2},\ldots,X_{k},\ldots$ una seq\"{u}\`encia de v.a.
iid. que pren valors enters. Sigui $N$ una v.a. que pren  valors enters
positius. Sigui $S_{N}=\sum_{k=1}^N X_{k}$
\begin{enumerate}[a)]
\item Trobau la mitjana i la vari\`ancia de $S_{N}$.
\item Demostrau que $G_{S_{N}}(z)=E(z^{S_N})=G_{N}(G_{X_{1}}(z))$.
\end{enumerate}
\end{exemple}

\vskip 0.2 cm
\begin{exemple}
(Exercici 23).
El nombre {\it N} d'usuaris que arriben a un sistema durant
un cert per\'{\i}ode \'es una variable aleat\`oria amb llei
$\mathrm{Po}(\lambda)$. Sigui $\> p \in (0,1) \> $ la probabilitat que
un usuari que arriba al sistema rebi servei. Determinau la llei de
la variable aleat\`oria que compta el nombre d'usuaris que reben
servei. 
\end{exemple}

\vskip 1.5 cm
\noindent
Problemes proposats: 8


\end{document}