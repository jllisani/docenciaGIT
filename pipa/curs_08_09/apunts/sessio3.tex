\documentclass{article}
\usepackage[catalan]{babel}
\usepackage[latin1]{inputenc}   % Permet usar tots els accents i car�ters llatins de forma directa.
\usepackage{enumerate}
\usepackage{amsfonts, amscd, amsmath, amssymb}
\usepackage{fancyheadings}

\setlength{\textwidth}{16cm}
\setlength{\textheight}{25cm}
\setlength{\oddsidemargin}{-0.3cm}
\setlength{\evensidemargin}{0.25cm} \addtolength{\headheight}{\baselineskip}
\addtolength{\topmargin}{-3cm}

\newcommand\Z{\mathbb{Z}}
\newcommand\R{\mathbb{R}}
\newcommand\N{\mathbb{N}}
\newcommand\Q{\mathbb{Q}}
\newcommand\K{\Bbbk}
\newcommand\C{\mathbb{C}}

\newcounter{exctr}
\setcounter{exctr}{8}
\newenvironment{exemple}
{ \stepcounter{exctr} 
\hspace{0.2cm} 
\textit{Exemple  \arabic{exctr}: }
\it
\begin{quotation}
}{\end{quotation}}

\pagestyle{fancy}
\markboth{Tema 1. Variables aleat\`ories vectorials}{}
\setcounter{page}{3}
\setlength{\headrulewidth}{0pt}

\begin{document}

\noindent
\textbf{\large Probabilitats condicionades}

\vskip 0.2 cm
\noindent
Recordatori:
\vskip 0.1 cm
donats dos successos qualssevol $A$ i $B$, la probabilitat de $A$ condicionada a $B$ es calcula com:
\[
P(A|B)=\frac{P(A \cap B)}{P(B)}
\]

Si els successos s\'on independents, llavors $P(A|B)=P(A) \Leftrightarrow P(A \cap B)=P(A) \cdot P(B)$.


\vskip 0.5cm
\noindent
En aquest tema:


\vskip 0.2cm
\noindent
\textbf{Cas discret}
\vskip 0.1 cm
Siguin dues v.a. $X$ i $Y$ discretes amb funci\'o de probabilitat conjunta $P(X=x, Y=y)$, la 
\textbf{funci\'o de probabilitat de $Y$ condicionada per $X$} es defineix com:
\[
P(Y=y|X=x)=\frac{P(X=x, Y=y)}{P(X=x)}
\]

\vskip 0.2 cm
\noindent
\textbf{Propietat:} les variables aleat\`ories discretes $X$ i $Y$ s\'on \textbf{independents} si i nom\'es si 
\[
P(X=x, Y=y)=P(X=x) \cdot P(Y=y) \qquad \forall (x, y)
\]

\vskip 0.2 cm
\noindent
La \textbf{funci\'o de distribuci\'o de $Y$ condicionada per $X$} es defineix com:
\[
P(Y \leq y|X=x)=\frac{P(X=x, Y \leq y)}{P(X=x)}
\]
\noindent
I, en general, donat un succ\'es qualsevol $A$:
\[
P(A|X=x)=\frac{P(A)}{P(X=x)}=\frac{\sum_{y \in A} P(X=x, Y=y)}{P(X=x)}
\]

Les funcions de probabilitat i distribuci\'o de $X$ condicionades per $Y$ es defineixen de manera
an\`aloga i es denoten $P(X=x|Y=y)$, $P(X \leq x|Y=y)$.

\vskip 0.2cm
\noindent
\textbf{Cas continu}
\vskip 0.1 cm
Siguin dues v.a. $X$ i $Y$ cont\'\i nues amb funci\'o de densitat conjunta $f_{XY}(x, y)$, la 
\textbf{funci\'o de densitat de $Y$ condicionada per $X$} es defineix com:
\[
f_{Y|X}(y|x)=\frac{f_{XY}(x, y)}{f_X(x)}
\]
\vskip 0.2 cm
\noindent
\textbf{Propietat:} les variables aleat\`ories cont\'\i nues $X$ i $Y$ s\'on \textbf{independents} si i nom\'es si 
\[
f_{XY}(x, y)=f_X(x) \cdot f_Y(y) \qquad \forall (x, y)
\]

\vskip 0.2 cm
\noindent
La \textbf{funci\'o de distribuci\'o de $Y$ condicionada per $X$} es defineix com:
\[
F_{Y|X}(y|x)=P(Y \leq y | X=x)=\frac{\int_{-\infty}^y f_{XY}(x, y) \, dy}{f_X(x)}
\]
\noindent
I, en general, donat un succ\'es qualsevol $A$:
\[
P(A|X=x)=\frac{\int_{y \in A} f_{XY}(x, y) \, dy}{f_X(x)}
\]
Les funcions de densitat i distribuci\'o de $X$ condicionades per $Y$ es defineixen de manera
an\`aloga i es denoten $f_{X|Y}(x|y)$, $F_{X|Y}(x|y)$.




\newpage
\begin{exemple}
(Exercici 11b). Llan\c{c}am a l'aire un dau equilibrat. Considerem dues variables
aleat\`ories $X$ i $Y$ definides com:
\[
X = \begin{cases}-1 & \text{si el resultat \'es imparell}\\ 
1 & \text{si el resultat \'es parell} \end{cases} \qquad
Y = \begin{cases}-1 & \text{si el resultat \'es 1, 2 o 3}\\
0 & \text{si el resultat \'es 4}\\ 
1 & \text{si el resultat \'es 5 o 6}\end{cases}
\]
Calculau $P(X+Y=0|Y \leq 0)$ i $P(X=1|X+Y=2)$.
\end{exemple}
  
\vskip 0.2 cm
\begin{exemple}
(Exercici 14bc). Un auditor selecciona a l'atzar un cert nombre $X$ de
factures d'un arxivador; $X$ \'es un nombre a l'atzar entre 5 i 8.
Sigui $Y$ el temps en minuts que tarda en revisar-les. Suposem que
$(X,Y)$ t\'e una llei conjunta donada per

\[
P(X = x, Y = y) = \begin{cases}
{1 \over 4} \cdot {10-x \over x} \cdot \left ( {x \over 10} \right )^y & \text{si } x = 5,6,7,8, y =
1,2,\ldots \\ 0 & \text{en cas contrari}
\end{cases}
\]
\begin{enumerate}[a)]
\item Trobau la distribuci\'o condicional de $X $  donat que $Y = y.$
\item  Calculau la probabilitat que hagi triat 6 factures sabent que
ha tardat m\'es de 3 minuts en revisar-les. 
\end{enumerate}
\end{exemple}

\vskip 0.3 cm
\begin{exemple}
(Exercici 16b). Siguin $X$ i $Y$ dues variables aleat\`ories amb densitat
conjunta 
\[
f_{XY}(x,y) = \left\{\begin{array}{ll}\frac{1}{x} &
\mbox{si } 0 \leq y \leq x \leq 1\\ 0 & \mbox{en cas
contrari}\end{array}\right.
\]
Calculau $f_{Y}(y|x)$.
\end{exemple}
  
  
\vskip 0.3 cm
\begin{exemple}
(Exercici 9). Siguin $X$ i $Y$ dues variables aleat\`ories conjuntament
absolutament cont\'{\i}nues. Suposem que
\[
f_{X}(x) = \begin{cases}4x^3 & \text{si } 0 < x < 1 \\ 0 & \text{en cas contrari} \end{cases}
\qquad
\text{i que}
\qquad
f_{Y}(y|x) = \begin{cases}{2y \over x^2} & \text{si } 0 < y < x \\ 0 & \text{en cas contrari} \end{cases}
\]
\begin{enumerate}[a)]
\item  Determinau $f_{X,Y}$.
\item Obteniu la distribuci\'o de $Y.$
\item  Trobau $f_{X}(x|y).$
\end{enumerate}
\end{exemple}


\vskip 0.5 cm
\noindent
Exercicis proposats: 12, 8d, 13, 17c, 7, 23

\end{document}