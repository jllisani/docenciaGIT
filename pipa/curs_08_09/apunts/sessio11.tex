\documentclass{article}
\usepackage[catalan]{babel}
\usepackage[latin1]{inputenc}   % Permet usar tots els accents i car�ters llatins de forma directa.
\usepackage{enumerate}
\usepackage{amsfonts, amscd, amsmath, amssymb}
\usepackage{fancyheadings}

\setlength{\textwidth}{16cm}
\setlength{\textheight}{25cm}
\setlength{\oddsidemargin}{-0.3cm}
\setlength{\evensidemargin}{0.25cm} \addtolength{\headheight}{\baselineskip}
\addtolength{\topmargin}{-3cm}

\newcommand\Z{\mathbb{Z}}
\newcommand\R{\mathbb{R}}
\newcommand\N{\mathbb{N}}
\newcommand\Q{\mathbb{Q}}
\newcommand\K{\Bbbk}
\newcommand\C{\mathbb{C}}

\newcounter{exctr}
\setcounter{exctr}{7}
\newenvironment{exemple}
{ \stepcounter{exctr} 
\hspace{0.2cm} 
\textit{Exemple  \arabic{exctr}: }
\it
\begin{quotation}
}{\end{quotation}}

\pagestyle{fancy}
\markboth{Tema 2. Suma de variables aleat\`ories}{}
\setcounter{page}{4}
\setlength{\headrulewidth}{0pt}



\begin{document}

\noindent
\textbf{\large Mitjana mostral i llei dels grans nombres}
\vskip 0.2cm
Consideram una v.a. $X$ associada a un experiment aleatori.

(Per exemple, $X$=nombre de cares en el llan\c{c}ament d'una moneda,
llavors $\Omega_X=\{0, 1\}$).

\vskip 0.2cm
Repetim $n$ vegades l'experiment i denotam $X_i$ la v.a. associada a la
repetici\'o i-\`essima de l'experiment.

(Observem que les $X_i$ estan id\`enticament distribu\"ides ja que totes
estan definides de manera id\`entica per al mateix experiment. A m\'es,
si les repeticions de l'experiment s\'on independents tenim que les $X_i$
s\'on i.i.d, i per tant $E(X_i)=E(X)$ i $\mathrm{Var}(X_i)=\mathrm{Var}(X)$).

\vskip 0.2 cm
Definim la \textbf{mitjana mostral de $X$} com:
\[
M_n=\frac{X_1+X_2+\cdots+X_n}{n}
\]

$M_n$ \'es una nova v.a. que t\'e les seg\"uents \textbf{propietats}:
\begin{itemize}
\item $E(M_n)=E(X)$
\item $\mathrm{Var}(M_n)=\displaystyle \frac{\mathrm{Var}(X)}{n}$
\end{itemize}

\textbf{Interpretaci\'o:}
Les anteriors propietats ens diuen que la mitjana dels valors de $M_n$ \'es igual a
la mitjana de $X$ i que la dispersi\'o dels valors de $M_n$ tendeix cap a zero
quan $n$ augmenta.


\vskip 0.3 cm
La relaci\'o entre la mitjana (esperan\c{c}a) i la dispersi\'o (vari\`ancia) dels valors 
d'una variable aleat\`oria qualsevol v\'e donada per la \textbf{desigualtat de Txebitxeff}:
\[
P(|Y-E(Y)| \geq \varepsilon) \leq \frac{\mathrm{Var}(Y)}{\varepsilon^2} \qquad 
\forall \varepsilon > 0 \qquad
\forall \text{v.a. } Y
\]
\noindent
que tamb\'e se pot escriure com:
\[
P(|Y-E(Y)| < \varepsilon) \geq 1- \frac{\mathrm{Var}(Y)}{\varepsilon^2} \qquad 
\forall \varepsilon > 0 \qquad
\forall \text{v.a. } Y
\]


\vskip 0.2 cm
Aplicant l'anteri\'o relaci\'o al cas de la mitjana mostral tenim:
\[
P(|M_n-E(M_n)| < \varepsilon) \geq 1- \frac{\mathrm{Var}(M_n)}{\varepsilon^2} \quad
\Longleftrightarrow \quad 
P(|M_n-E(X)| < \varepsilon) \geq 1- \frac{\mathrm{Var}(X)}{n \varepsilon^2}
\qquad \forall \varepsilon > 0 
\]

\vskip 0.3 cm
Com a conseq\"u\`encia d'aquesta relaci\'o, tenim que si $n$ \'es molt gran ($n \rightarrow \infty$):
\[
\lim_{n\rightarrow \infty} P(|M_n-E(X)| < \varepsilon) = 1 \qquad \quad \forall \varepsilon > 0 
\]
\vskip 0.2 cm
Aquesta expressi\'o es coneix com \textbf{Llei d\`ebil dels grans nombres} i ens diu que
\'es segur (amb probabilitat 1) que si repetim un nombre molt gran de vegades un experiment,
la mitjana mostral ser\`a igual a la mitjana te\`orica de la v.a. $X$.

\vskip 0.5 cm
\begin{exemple}
(Exercici 10).
Suposem que el 10\% dels votants estan a favor d'una certa
legislaci\'o. Es fa una enquesta entre la poblaci\'o i s'obt\'e una
freq\"{u}\`encia relativa $f_n(A)$ com una estimaci\'o de la proporci\'o
anterior. Determinau, aplicant la desigualtat de Txebicheff,
quants de votants s'haurien d'enquestar perqu\`e  la probabilitat
que $f_n(A)$ difereixi de 0.1 menys de 0.02 sigui al menys 0.95
Qu\`e podem dir si no coneixem el valor de
la proporci\'o? 
\end{exemple}


\begin{exemple}
(Exercici 11). 
Es llan\c{c}a a l'aire un dau regular 100 vegades. Aplicau la
desigualtat de Txebicheff per obtenir una fita de la probabilitat
que el nombre total de punts obtinguts estigui entre 300 i 400.
\end{exemple}

\begin{exemple}
(Exercici 12).
Es sap que, en una poblaci\'o, la talla dels individus mascles
adults \'es una variable aleat\`oria $X$ amb  mitjana $\mu_x = 170$ cm
i desviaci\'o t\'{\i}pica $\sigma_x = 7 $ cm. Es tria una mostra
aleat\`oria de 140 individus. Calculau la probabilitat que la
mitjana mostral $\overline{x}$ difereixi de $\mu_x$ en menys d'1
cm.
\end{exemple}

\vskip 0.5 cm
\noindent
Problemes proposats: 9, 13, 14


\end{document}