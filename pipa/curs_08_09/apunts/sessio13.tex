\documentclass{article}
\usepackage[catalan]{babel}
\usepackage[latin1]{inputenc}   % Permet usar tots els accents i car\`acters llatins de forma directa.
\usepackage{enumerate}
\usepackage{amsfonts, amscd, amsmath, amssymb}
\usepackage{fancyheadings}

\setlength{\textwidth}{16cm}
\setlength{\textheight}{25cm}
\setlength{\oddsidemargin}{-0.3cm}
\setlength{\evensidemargin}{0.25cm} \addtolength{\headheight}{\baselineskip}
\addtolength{\topmargin}{-3cm}

\newcommand\Z{\mathbb{Z}}
\newcommand\R{\mathbb{R}}
\newcommand\N{\mathbb{N}}
\newcommand\Q{\mathbb{Q}}
\newcommand\K{\Bbbk}
\newcommand\C{\mathbb{C}}

\newcounter{exctr}
\newenvironment{exemple}
{ \stepcounter{exctr} 
\hspace{0.2cm} 
\textit{Exemple  \arabic{exctr}: }
\it
\begin{quotation}
}{\end{quotation}}

\pagestyle{fancy}
\markboth{Tema 3. Processos aleatoris}{}
\setcounter{page}{1}
\setlength{\headrulewidth}{0pt}



\begin{document}

\textbf{\Large Tema 3. Processos aleatoris}

\vskip 0.2 cm
En els temes anteriors hem estudiat les \textit{variables aleat\`ories}, que s\'on
nombres aleatoris associats a un experiment aleatori. En aquest tema estudiarem
el \textit{processos aleatoris o estoc\`astics}, que s\'on \textbf{seq\"u\`encies de nombres aleatoris}
associades a un experiment.

\vskip 0.3 cm
Exemples de processos aleatoris s\'on:
\begin{enumerate}
\item el s\'o que rebem en un receptor de r\`adio quan sintonitzam una freq\"u\`encia buida
\item el d\`ebil senyal el\`ectric que detectam amb un volt\'imetre en un circuit sense
alimentaci\'o
\item la seq\"u\`encia de bits que rebem quan transmetem un senyal binari per un canal amb renou
\end{enumerate}

Veurem en el tema seg\"uent que el \textit{renou} en un sistema de comunicacions
es modela en termes matem\`atics com un 
proc\'es aleatori, per aix\`o, l'estudi de les propietats dels processos aleatoris ens
ser\`a \'util per aprendre a controlar el nivell de renou en les comunicacions.

\vskip 0.5 cm
\textbf{Formalitzaci\'o dels processos aleatoris}

Cada una de les seq\"u\`encies de nombres que obtenim quan repetim v\`aries vegades
el mateix experiment aleatori es diu \textbf{realitzaci\'o del proc\'es aleatori}.
Formalment aix\`o ho podem escriure de la seg\"uent manera:
\[
\begin{array}{rccc}
X: & \Omega & \longrightarrow  & \text{conjunt de funcions}\\
   & \omega_1 & \rightarrow & X(\omega_1, t) \\
   & \omega_2 & \rightarrow & X(\omega_2, t) \\
   & \vdots   & \vdots      &   \vdots 
\end{array}
\]

\noindent
on $\Omega=\{ \omega_1, \omega_2, \cdots \}$ \'es l'espai mostral de l'experiment (el conjunt
de tots els resultats possibles). $X$ denota el proc\'es aleat\`ori i les funcions 
$X(\omega_1, t)$, $X(\omega_2, t)$, etc. representen
les seq\"u\`encies de nombres associades a cada repetici\'o de l'experiment.

Cadascuna d'aquestes funcions \'es una \textit{realitzaci\'o} del proc\'es.
Si aquestes funcions s\'on cont\'inues deim que el proc\'es \'es un 
\textbf{proc\'es aleatori en temps continu}, mentre que si les funcions s\'on discretes
parlam de \textbf{proc\'es aleatori en temps discret}. 

El terme ``temps'' fa refer\`encia a l'\'index $t$ de les funcions  
$X(\omega_1, t)$, $X(\omega_2, t)$, etc i s'utilitza per ab\'us de llenguatge, ja que en molts 
de casos les seq\"u\`encies de nombres associades al proc\'es representen la seva evoluci\'o
temporal.

\vskip 0.5 cm
\begin{exemple}
Sigui $\xi$ un nombre aleatori seleccionat a l'atzar en l'interval $[-1, 1]$.
Definim el seg\"uent proc\'es aleatori: $X(\xi, t)=\xi \cos(2\pi t)$. Dibuixau
algunes realitzacions del proc\'es.
\end{exemple}

\vskip 0.5 cm
\begin{exemple}
Una part\'icula es despla\c{c}a d'esquerra a dreta amb un moviment constant i de dalt a baix
de manera aleat\`oria segons la seg\"uent llei: $(x_{n+1}, y_{n+1})=(x_n+1, y_n + A)$,
on $A$ \'es una v.a. que pren valors equiprobables $+1$ o $-1$.
Dibuixau algunes de les possibles traject\`ories de la part\'icula i raonau per qu\`e
es pot considerar un proc\'es aleatori.
\end{exemple}


\end{document}