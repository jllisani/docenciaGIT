\documentclass{article}
\usepackage[catalan]{babel}
\usepackage[latin1]{inputenc}   % Permet usar tots els accents i car�ters llatins de forma directa.
\usepackage{enumerate}
\usepackage{amsfonts, amscd, amsmath, amssymb}
\usepackage{fancyheadings}

\setlength{\textwidth}{16cm}
\setlength{\textheight}{25cm}
\setlength{\oddsidemargin}{-0.3cm}
\setlength{\evensidemargin}{0.25cm} \addtolength{\headheight}{\baselineskip}
\addtolength{\topmargin}{-3cm}

\newcommand\Z{\mathbb{Z}}
\newcommand\R{\mathbb{R}}
\newcommand\N{\mathbb{N}}
\newcommand\Q{\mathbb{Q}}
\newcommand\K{\Bbbk}
\newcommand\C{\mathbb{C}}

\newcounter{exctr}
\newenvironment{exemple}
{ \stepcounter{exctr} 
\hspace{0.2cm} 
\textit{Exemple  \arabic{exctr}: }
\it
\begin{quotation}
}{\end{quotation}}

\pagestyle{fancy}
\markboth{Tema 2. Suma de variables aleat\`ories}{}
\setcounter{page}{1}
\setlength{\headrulewidth}{0pt}



\begin{document}

\textbf{\Large Tema 2. Suma de variables aleat\`ories}

\vskip 0.2 cm
En aquest tema estudiarem les propietats de la suma de $n$ variables aleat\`ories $S_n=X_1 + X_2 + \cdots + X_n$,
on $n$ pot \'esser un valor constant o un valor aleatori.

\vskip 0.3 cm
\noindent
\textbf{\large Funcions de probabilitat o densitat d'una suma de variables aleat\`ories}
\vskip 0.2cm
Si definim $S_n=X_1 + X_2 + \cdots + X_n$, on $X_1, \cdots, X_n$ s\'on variables aleat\`ories i $n$ \'es 
constant, en general \'es dif\'\i cil calcular la funci\'o de probabilitat (si les $n$ variables s\'on discretes)
o de densitat (si les $n$ variables s\'on cont\'inues) de $S_n$ a partir de les funcions de probabilitat
o densitat de $X_1, \cdots, X_n$. La manera general de fer-ho seria la seg\"uent:

\vskip 0.2 cm
cas discret:
\[
P(S_n=s)={\sum \cdots \sum}_{(x_1, x_2, \cdots, x_n) \in A} P(X_1=x_1, X_2=x_2, \cdots, X_n=x_n) 
\]
\indent
on $A=\{ x_1, x_2, \cdots, x_n) / x_1+x_2+\cdots+x_n=s \}$

\vskip 0.2 cm
cas continu:
\[
F_{S_n}(s)=P(S_n \leq s)=
{\int \cdots \int}_{(x_1, x_2, \cdots, x_n) \in A} f_{X_1 X_2 \cdots X_n}(x_1, x_2, \cdots, x_n) \, dx_1 dx_2 \cdots dx_n
\]
\indent
on $A=\{ x_1, x_2, \cdots, x_n) / x_1+x_2+\cdots+x_n \leq s \}$ i $f_{S_n}(s)=\frac{d f_{S_n}(s)}{ds}$.

\vskip 0.5 cm
Una manera alternativa de calcular la probabilitat o densitat de $S_n$ \'es utilitzant funcions caracter\'\i stiques
o funcions generadores de probabilitat:

\vskip 0.2 cm
La \textbf{funci\'o caracter\'\i stica} d'una v.a. (continua o discreta) $X$ es defineix com $\Phi_X(\omega)=E(e^{j\omega X})$.

La \textbf{funci\'o generadora de probabilitat} d'una v.a. discreta $X$ ($X \geq 0$) \'es $G_X(z)=E(z^X)$.

Propietats: 
\begin{itemize}
\item $\Phi_{S_n}(\omega)=\Phi_{X_1}(\omega) \cdot \Phi_{X_2}(\omega) \cdots \Phi_{X_n}(\omega)$ si les v.a. $X_i$ 
s\'on independents.
\item $G_{S_n}(z)=G_{X_1}(z) \cdot G_{X_2}(z) \cdots G_{X_n}(z)$ si les v.a. $X_i$ 
s\'on independents.
\end{itemize}

\vskip 0.5 cm
En alguns casos concrets les funcions de probabilitat o densitat es poden calcular f\`acilment a partir de les
densitats o probabilitats de les v.a. sumades:
\begin{itemize}
\item la suma de $n$ v.a. binomials independents $X_i \sim B(n_i, p)$ \'es una v.a. binomial $$S_n \sim B(n_1+n_2+\cdots+n_n, p)$$
\item la suma de $n$ v.a. de Poisson independents $X_i \sim \mathrm{Po}(\lambda_i)$ \'es una v.a. de Poisson 
$$S_n \sim \mathrm{Po}(\lambda_1+\lambda_2+\cdots+\lambda_n)$$
\item la suma de $n$ v.a. Gaussianes independents $X_i \sim N(\mu_i, \sigma_i^2)$ \'es una v.a. Gaussiana
$$S_n \sim N(\mu_1+\mu_2+\cdots+\mu_n, \sigma_1^2+\sigma_2^2+\cdots+\sigma_n^2)$$

\end{itemize}


\vskip 0.5 cm
\begin{exemple}
Sabent que la funci\'o caracter\'\i stica d'una v.a. Gaussiana $N(\mu, \sigma^2)$ \'es 
$\displaystyle \Phi(\omega)=e^{\frac{j\mu\omega - \sigma^2 \omega^2}{2}}$, demostrau que la 
suma de $n$ v.a. Gaussianes independents $X_i \sim N(\mu_i, \sigma_i^2)$ \'es tamb\'e Gaussiana
amb par\`ametres $\mu_1+\mu_2+\cdots+\mu_n$ i $\sigma_1^2+\sigma_2^2+\cdots+\sigma_n^2$.
\end{exemple}

\vskip 0.2 cm
\begin{exemple}
(Exercici 4).Sigui $S_{k}=X_{1}+\ldots+X_{k}$ on $X_{i}$ s\'on v.a.
independents i amb  distribuci\'o $B(n_{i},p)$ per a $i=1,\ldots,k$.
Utilitzau la funci\'o generadora de probabilitats per demostrar
que $S_{k}$ segueix una  distribuci\'o $B(\sum_{i=1}^k n_{i}, p)$.
Explicau aquest resultat.
\end{exemple}

\vskip 0.5 cm
\noindent
Exercicis proposats: 5 (indicaci\'o: si $X\sim \mathrm{Po}(\lambda)$, llavors $G_X(z)=e^{\lambda (z-1)}$).

\end{document}