\documentclass[a4paper,12pt]{report}
\usepackage[latin1]{inputenc}   % Permet usar tots els accents i car�ters llatins de forma directa.
\usepackage[spanish]{babel}
%\decimalspanish{.}
\usepackage{latexsym}
\usepackage{hyperref}
\usepackage{theorem}
\usepackage{enumerate}
\usepackage{amsfonts, amscd, amsmath, amssymb}
\usepackage[pdftex]{graphicx}
\usepackage{epstopdf}
\usepackage{url, hyperref}


\setlength{\textwidth}{16.5cm}
\setlength{\textheight}{25cm}
\setlength{\oddsidemargin}{-0.3cm}
\setlength{\evensidemargin}{0.25cm} \addtolength{\headheight}{\baselineskip}
\addtolength{\topmargin}{-3cm}

\pagestyle{empty}
\begin{document}
\begin{center}
\textbf{\large Estad�stica Aplicada}

\textbf{\large Seguretat i Ci�ncies Policials}

\vskip 0.5 cm
\textbf{\large Activitats Avaluaci� Bloc I}
\end{center}

\vskip 1cm
Volem fer un estudi sobre el nombre d'empreses dels municipis de les
Illes Balears. Utilitzau la fulla de c�lcul OpenOffice Calc per a respondre
les seg�ents q�estions.

\begin{enumerate}
\item Recopilaci� de dades \textit{brutes}: les dades es troben
en l'\textit{Anuari Estad�stic Municipal de les Illes Balears}. Per accedir-hi:

\begin{enumerate}[a)]
\item Visitau la web de l'\textit{Observatori del Treball} del Govern
de les Illes Balears:

\url{http://www.caib.es/sacmicrofront/contenido.do?idsite=282&cont=10654}

\item Al men� de la part esquerra heu de fer click damunt \textit{Anuari estad�stic municipal}.
Apareix la llista de municipis de les Illes Balears.

\item Si seleccionau un municipi es mostra diferent informaci� corresponent a l'any 2009.
Per exemple, la superf�cie del municipi d'Alaior �s 109.86 $\mathrm{km}^2$, la seva poblaci� 
total 9.257 persones, el nombre d'empreses en R�gim general 321, etc.

\item Ens centram en la informaci� del `Nombre d'empreses en R�gim general'. Per a cada un dels
municipis de les Illes (67 municipis en total) heu de recopilar aquesta dada. 
Aquest conjunt de 67 valors �s el que heu d'utilitzar per fer la pr�ctica.
\end{enumerate}

\item Calculau a partir de les dades anteriors els seg�ents par�metres estad�stics 
(si �s possible): moda, mitjana, mediana, primer i tercer quartils.
Si no �s possible o pensau que no t� sentit calcular algun d'aquests par�metres
comentau-ho. Indicau com feu els c�lculs.

\item Construcci� de la taula de freq��ncies:
\begin{enumerate}[a)]
\item agrupau les dades anteriors en els seg�ents intervals:
\textit{menys de 50}, $[50, 250)$, $[250, 500)$, $[500, 750)$, $[750, 1000)$, \textit{m�s de 1000}. 
\item constru�u una taula de freq�encies que contengui: les freq��ncies absolutes, les freq��ncies
absolutes acumulades, les relatives, les relatives acumulades i els percentatges.
\end{enumerate}
\item Dibuixau un diagrama de barres de freq��ncies absolutes i un diagrama de tarta de percentatges.

\item Calculau a partir de la taula de freq��ncies els seg�ents par�metres estad�stics: 
moda, mitjana, mediana, primer i tercer quartils. Indicau com feu els c�lculs.
(Nota: per al c�lcul de la mitjana utilitzau com a marques de classe del primer i
darrer intervals $25$ i $2000$, respectivament).

\end{enumerate}

\end{document}
