\chapter*{Prefacio:\newline 
Estad\'istica en  Seguridad y Ciencias Policiales}


El enfoque racional de los problemas en cualquier �rea de las 
Ciencias implica un paso previo de recopilaci�n de datos y an�lisis
de los mismos que permite conocer en profundidad el problema
y eventualmente solucionarlo.

Ya sea para descubrir c�mo se mueven los planetas,
c�mo se desintegran los �tomos, c�mo distribuir el presupuesto 
municipal o c�mo asignar los recursos policiales para reducir 
la criminalidad, en el origen de la soluci�n est� la recopilaci�n 
de datos sobre el problema.
Las Ciencias Sociales, en las que se enmarcan las Ciencias Policiales 
y de Seguridad, no son ajenas a este m�todo de trabajo.

Sin embargo, los datos por s� solos poco aportan a la soluci�n 
de los problemas. Es su organizaci�n lo que permite descubrir
tendencias, singularidades, etc. que conduciran a la soluci�n.
La \textbf{Estad�stica Descriptiva} es la rama de las Matem�ticas que proporciona 
las t�cnicas necesarias para recopilar, organizar, representar, analizar e 
interpretar los datos.

En la mayor�a de ocasiones, y por razones pr�cticas,
el an�lisis estad�stico se hace sobre un 
conjunto de datos inferior al total disponible.
El ejemplo t�pico son las encuentas sobre intenci�n de voto, 
realizadas sobre unos pocos miles de personas que \textit{representan} 
al total de la poblaci�n. 
Los resultados obtenidos se generalizan despu�s a toda la poblaci�n.
La \textbf{Estad�stica In\-fe\-ren\-cial} permite conocer
el grado de fiabilidad de estas generalizaciones. 

En este curso estudiaremos las t�cnicas b�sicas de la Estad�stica Descriptiva
e Inferencial, de modo que aprenderemos a organizar y analizar 
conjuntos de datos y a conocer el grado de fiabilidad de las generalizaciones
realizadas a partir de ellos.

\vskip 0.5 cm
El curso se organiza en tres m�dulos:

\begin{enumerate}
\item El m�dulo I da una introducci�n a la notaci�n habitual de la
estad�stica y presenta las t�cnicas b�sicas de la Estad�stica Descriptiva.

\item En el m�dulo II se avanza en el estudio de la Estad�stica
Descriptiva y se dan los fundamentos de Probabilidad necesarios
para la Estad�stica Inferencial.

\item El m�dulo III se dedica al estudio de la Estad�stica Inferencial
\end{enumerate}

 
 