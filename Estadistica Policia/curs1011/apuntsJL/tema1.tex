\chapter{El lenguaje de la estad�stica}

En m\'ultiples �mbitos de las Ciencias o la Gesti�n es necesario
el estudio de temas relacionados con cada disciplina.
Por ejemplo:
\begin{itemize}
\item un experto en Seguridad puede estar interesado en estudiar
la criminalidad en una ciudad;
\item  un Economista, el nivel de paro de un pa�s;
\item un Ecologista, la deforestaci\'on de la selva amaz\'onica;
\item un F\'\i sico, la desintegraci\'on de elementos radioactivos en una central nuclear;
\item un Ingeniero, la calidad de las piezas producidas en una f�brica, etc.
\end{itemize}

Sea cual sea el tema a estudiar, la informaci\'on acerca del mismo
se obtiene realizando m\'ultiples \textbf{observaciones} de algunas de 
sus caracter\'isticas. Estas observaciones proporcionan una serie de \textbf{datos} 
que son organizados, analizados e interpretados usando unas t\'ecnicas estad\'\i sticas
comunes.

Antes de comenzar a explicar cualquier t\'ecnica estad\'istica es necesario
definir algunos conceptos b\'asicos y un vocabulario estad\'istico elemental. 
Entre otros aspectos, en este tema estudiaremos las nociones de poblaci\'on y muestra, la
diferencia entre estad\'istica descriptiva e inferencial y las distinci\'on
entre datos y variables.


\section{Poblaci�n y muestra}
El t\'ermino \textbf{poblaci\'on} hace referencia al conjunto total de \textit{elementos}
objeto del estudio estad\'istico. Por ejemplo:
\begin{itemize}
\item el n\'umero total de personas v\'ictimas de un delito, en un estudio sobre la criminalidad;
\item el n\'umero de personas en paro, en un estudio sobre el desempleo;
\item el n\'umero de \'arboles talados en el \'ultimo a\~no, en un estudio sobre la de\-fo\-res\-taci\'on;
\item el n\'umero de part\'iculas subat\'omicas generadas por un reactor nuclear, en un estudio
sobre la desintegraci\'on de elementos radiaoctivos;
\item el n\'umero de piezas defectuosas fabricadas, en un estudio sobre la ca\-li\-dad de la producci\'on.  
\end{itemize}

En la mayor\'ia de ocasiones es demasiado dif\'icil, caro o imposible obtener informaci\'on de 
todos los elementos de la poblaci\'on, por lo que el estudio estad\'istico se realiza
sobre un subconjunto de la poblaci\'on. Este subconjunto se denomina \textbf{muestra}.

Como veremos en el apartado siguiente, es frecuente que se deseen extraer conclusiones
relativas a la poblaci\'on total a partir de los datos de la muestra. Para que estas 
conclusiones sean fiables es necesario que la muestra sea \textbf{representativa} de 
la poblaci\'on. Se considera que si los elementos de la muestra han sido
elegidos al azar entre toda la poblaci\'on (\textbf{muestra aleatoria simple}), 
se obtiene una muestra representativa, aunque hay otras formas de tomar muestras. 
Otro factor a tener en cuenta es el tama\~no 
de la muestra, cuanto mayor sea mejor representar\'a al conjunto de la poblaci\'on.


\section{Tipos de estad�sticas: descriptiva e inferencial}
Tradicionalmente la disciplina de Estad\'istica se dividide en dos ramas:
descriptiva e inferencial. 

\vskip 0.2cm
La Estad\'istica Descriptiva tiene por objeto la descripci\'on de los 
datos recopilados, para ello proporciona t�cnicas que permiten:
\begin{itemize}
\item la \textbf{organizaci�n} de los datos mediante tablas y  representaciones gr�ficas;
\item el \textbf{an�lisis} de los datos mediante el c\'alculo de valores representativos 
como son las medidas de tendencia central y  de dispersi�n.
\end{itemize}


\vskip 0.2cm
La Estad\'istica Inferencial parte de los datos obtenidos a partir de una
muestra e intenta extraer conclusiones de las caracter\'\i sticas generales
de toda la poblaci\'on. Tambi\'en proporciona m\'etodos para medir la
fiabilidad de las conclusiones obtenidas y relacionar esta fiabilidad con
el tama\~no de la muestra estudiada.


\vskip 0.3 cm
Por ejemplo, en el caso del experto en Seguridad que hace un estudio sobre
la criminalidad en una ciudad, la manera de hacer el estudio puede ser la
siguiente:
\begin{enumerate}
\item En lugar de recopilar los datos relativos a los delitos cometidos
en toda la ciudad, recoge los datos de unos pocos barrios (elegidos
al azar o siguiendo alg\'un criterio determinado);
\item Organiza y analiza los datos recogidos utilizando t\'ecnicas de 
estad\'istica des\-crip\-ti\-va;
\item Utiliza t\'ecnicas de estad\'istica inferencial para generalizar
los resultados obtenidos a toda la ciudad y estimar el grado de fiabilidad
de esta generalizaci\'on.
\end{enumerate}


\section{Datos y variables}

Los \textbf{datos} se definen como las unidades de informaci\'on recopiladas
al hacer un estudio estad\'istico. Una \textbf{variable} es una caracter\'\i stica
o atributo que permite clasificar en diferentes categor\'\i as 
los elementos de la muestra o poblaci\'on en funci\'on de los
datos recopilados. 
Por ejemplo, el conjunto de personas v\'ictimas de un crimen se puede 
clasificar en funci\'on del sexo de las personas (hombre/mujer), la edad 
(menores de 20 a\~nos, entre 20 y 50, mayores de 50), etc. Cada uno 
de estos atributos (sexo, edad, etc.) es una variable.

Existen distintos tipos de variables que se pueden clasificar siguiendo 
tres criterios:

\begin{itemize}
\item Tipo de dato
\begin{itemize}
\item Nominales (cualitativas o de atributos):  cuando los datos no son num\'ericos 
y la comparaci�n entre sus valores s�lo puede ser de igualdad o desigualdad.

Por ejemplo: sexo, color de los ojos, afiliaci�n pol�tica, lugar de residencia, etc,...
\item Ordinales: cuando los datos no son num�ricos pero la
comparaci�n entre ellos establece un orden.

Por ejemplo: estado de �nimo (valores posibles: depresivo, normal y euf�rico ), estudios
(valores posibles: ninguno, primarios, secundarios, su\-pe\-rio\-res), etc...
\item Cuantitativas: cuando los datos son
num�ricos. Dentro de las variables cuantitativas hay dos tipos m�s
\begin{itemize}
\item Discretas:  cuando entre dos posibles valores no hay otro.
Por ejemplo: n�mero de  hijos de una familia, n�mero de letras de una palabra en un
texto, etc,...
\item Continuas:  cuando  entre dos posibles valores, siempre podemos encontrar otro valor
posible. Por ejemplo: altura, intereses de una cuenta bancaria, etc,...
\end{itemize}
\end{itemize}
\item Dimensi�n
\begin{itemize}
\item Unidimensionales: si s�lo se considera una �nica
caracter�stica.

Ejemplos: altura, edad, etc,...
\item Multidimensionales: si se consideran conjuntamente varias
caracter�sticas.

Ejemplos: edad y altura, altura y peso, edad, altura y sexo, etc,...
\end{itemize}
\item Tiempo
\begin{itemize}
\item Atemporales:
cuando los datos no est�n referidos, o no se con\-si\-de\-ra, el momento de tiempo en el que
fueron obtenidos.

Ejemplos: color de los ojos de cierto conjunto de individuos, peso de los  estudiantes del
curso de Estad\'istica, etc,...
\item Temporales o series cronol�gicas: en caso contrario.

Ejemplos: P.I.B. anual de Espa\~{n}a durante el periodo 1980 hasta 2004, n�mero de turistas
llegados al aeropuerto de  Palma durante los a\~{n}os 1970 al 2004, etc,...
\end{itemize}
\end{itemize}




\section{Ejercicios propuestos}


\noindent
\textbf{Ejercicio 1} 

Identificar la poblaci\'on y la muestra estudiados 
en los siguientes casos:
\begin{enumerate}[a)]
\item En un estudio sobre el consumo de drogas en un instituto se hace
una encuesta al $30\%$ por ciento de los alumnos de $2^\text{o}$ de 
bachillerato.
\item En un estudio sobre el consumo de drogas entre los j\'ovenes de
Ciutadella (menores de 35 a\~nos) se entrevista al $10\%$ de los clientes 
de los principales locales de copas.
\item En un estudio a nivel nacional sobre la influencia del alcohol 
en los accidentes de tr\'afico se realizan 10.000 controles de alcoholemia
en diferentes carreteras del pa\'is.
\end{enumerate}

\vskip 0.2 cm
\noindent
\textbf{Ejercicio 2} 

Decidir si para estudiar los siguientes casos se utilizan
herramientas de estad\'istica descriptiva o inferencial:
\begin{enumerate}[a)]
\item Un profesor de universidad debe proporcionar a su jefe de Departamento
un informe sobre el n\'umero de alumnos matriculados y sus calificaciones 
en el periodo 2005-2007. Para ello utilizar\'a estad\'istica ............... .
\item Una empresa desea conocer los h\'abitos de trabajo de sus trabajadores.
Para ello les hace rellenar una encuesta sobre sus horas de llegada y salida,
tiempo dedicado a responder el tel\'efono o el correo electr\'onico, 
tiempo dedicado a reuniones de trabajo con los jefes u otros compa\~neros, etc.
Los datos obtenidos se organizar\'an y analizar\'an usando estad\'istica ............. .
\item La Direcci\'on General de Tr\'afico desea evaluar la eficiencia a nivel
nacional de
la \'ultima campa\~na de prevenci\'on de accidentes a partir de los 
datos en una serie de municipios. Para ello utilizar\'a
herramientas de la estad\'istica ................ . 
\end{enumerate}

\vskip 0.2 cm
\noindent
\textbf{Ejercicio 3} 

Identificar al menos tres variables que pueden aparecer
en los siguientes estudios estad\'isticos:
\begin{enumerate}[a)]
\item Consumo de drogas en una ciudad.
\item Satisfacci\'on laboral de los empleados de una empresa.
\item Notas obtenidas por los alumnos de una asignatura. 
\end{enumerate}

\vskip 0.2 cm
\noindent
\textbf{Ejercicio 4} 

Clasificar las siguientes variables seg\'un su tipo,
dimensi\'on y nivel temporal:
\begin{enumerate}[a)]
\item N\'umero de personas que han sufrido un accidente de tr\'afico en los \'ultimos 5 a\~nos.
\item Nivel profesional de un militar (por ejemplo: soldado, cabo, sargento, etc.).
\item N\'umero de goles conseguidos por un jugador de f\'utbol a lo largo de la temporada 2006-07. 
\item Religi\'on de un individuo (por ejemplo: cat\'olico, musulm\'an, budista, etc.).
\item El peso y la altura de las participantes en un desfile de moda.
\item Cantidad de dinero gastada por una Administraci\'on en obras p\'ublicas a lo largo del \'ultimo a\~no. 
\end{enumerate}


%\section{Soluciones de los ejercicios}
%
%\noindent
%\textbf{Ejercicio 1}
%\begin{enumerate}[a)]
%\item Poblaci\'on: el total de los alumnos del instituto. 
%Muestra: el $30\%$ por ciento de los alumnos de $2^\text{o}$ de bachillerato
%\item Poblaci\'on: los habitantes de Ciutadella menores de 35 a\~nos.
%Muestra: el $10\%$ de los clientes de los principales locales de copas.
%\item Poblaci\'on: todos los conductores del pa\'is. Muestra: 10.000 conductores.
%\end{enumerate}
%
%\vskip 0.2 cm
%\noindent
%\textbf{Ejercicio 2}\begin{enumerate}[a)]
%\item Descriptiva.
%\item Descriptiva.
%\item Inferencial.
%\end{enumerate}
%
%\vskip 0.2 cm
%\noindent
%\textbf{Ejercicio 3}
%\begin{enumerate}[a)]
%\item Sexo, edad, nivel de estudios, nivel de ingresos, etc.
%\item Sexo, edad, antig\"uedad en la empresa, categor\'ia profesional, salario, etc.
%\item Tiempo dedicado al estudio, compaginaci\'on de estudios
%y trabajo, n\'umero de veces que se ha cursado la asignatura, 
%nota en los examenes de acceso a la universidad, etc.
%\end{enumerate}
%
%\vskip 0.2 cm
%\noindent
%\textbf{Ejercicio 4}
%\begin{enumerate}[a)]
%\item Cuantitativa discreta, unidimensional, temporal.
%\item Ordinal, unidimensional, atemporal.
%\item Cuantitativa discreta, unidimensional, temporal.
%\item Nominal, unidimensional, atemporal.
%\item Cuantitativa continua, multidimensional (bidimensional), atemporal.
%\item Cuantitativa continua, unidimensional, temporal.
%\end{enumerate}
%
%
