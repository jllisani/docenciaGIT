\documentclass[12pt]{report}
%\overfullrule=10pt
\usepackage[latin1]{inputenc}   % Permet usar tots els accents i car�cters llatins de forma directa.
\usepackage[spanish]{babel}
   
\spanishdecimal{,}
%\usepackage{multimedia}linkcolor
\usepackage{latexsym}
\usepackage[pdftex,%
            colorlinks=true,
            linkcolor=cyan,
            citecolor=cyan,
            bookmarksnumbered=true,
           bookmarksopen=true,            
           pdftitle={Notas de Clase de Estadistica Aplicada},    % title
           pdfauthor={Ricardo Alberich, Jose Luis Lisani},     % author
            pdfsubject={UIB. 2007-2008. Graduado en Seguridad y Ciencias Policiales}
            ]{hyperref}
\usepackage[all]{hypcap}
\usepackage{theorem}
\usepackage{enumerate}
\usepackage{amsfonts, amscd, amsmath, amssymb}
\usepackage[pdftex]{graphicx} 
\usepackage{amsmath}
\usepackage{epstopdf}
%\usepackage{epsdice}

  

\newtheorem{definition}{Definici\'on}
\newtheorem{theorem}[definition]{Teorema}
\newtheorem{proposition}[definition]{Proposici\'on}
\newtheorem{corollary}[definition]{Corolario}
\newtheorem{lemma}[definition]{Lema}
\newtheorem{example}[definition]{Ejemplo}
\newtheorem{Rem}[definition]{Nota:}
\newcommand{\va}{variable aleatoria }
\def\N{I\!\!N}
\def\R{I\!\!R}
\def\Z{Z\!\!\!Z}
\def\Q{O\!\!\!\!Q}
\def\C{I\!\!\!\!C}
%\decimalpoint


%\setlength{\textwidth}{17cm} 
%\setlength{\textheight}{24cm}
\setlength{\oddsidemargin}{3cm}
\setlength{\evensidemargin}{3cm}
% \addtolength{\headheight}{\baselineskip}
% \addtolength{\topmargin}{-3cm}

%\pagestyle{plain}

%\usepackage{multimedia}

%\markboth{ESTAD�STICA APLICADA curso 07-08}{ESTAD�STICA APLICADA curso 07-08}
\makeindex

%Cambiar Cuadros por Tablas y lista de...
\renewcommand{\listtablename}{�ndice de tablas}
\renewcommand{\tablename}{Tabla} 
%%%%Definici�n de las cabeceras y los pies de p�gina
\usepackage{fancyhdr}
\lhead{}
\chead{}
\rhead{}
\lfoot{Campus Extens}
\cfoot{}
\rfoot{ \thepage}
\renewcommand{\headrulewidth}{0pt}
\renewcommand{\footrulewidth}{0pt}
\pagestyle{fancy}


%%%%%%%%%%%%Para los dibujos de portada y fondo.
\usepackage{fancybox}
%\fancyput(2cm,-19cm){\includegraphics[height=24,width=17cm]{fons.pdf}}
\fancyput(0cm,-24cm){\includegraphics[height=24cm]{fons.pdf}}

%\includeonly{tema0,tema1}
%\tableofcontents
%\listoftables
\begin{document}
\tableofcontents
\listoftables
\thisfancyput(0cm,0cm){\includegraphics{portada.pdf}}
%\thisfancyput(2cm,-19cm){\includegraphics[width=15cm]{portada.pdf}}
\part{M�dulo II: Interpretaci�n de datos estad�sticos bivariantes.\\ Introducci�n a al probabibilidad y las variables aleatorias}
\author{Ricardo Alberich, Jos� Luis Lisani}


En este m�dulo se culmina es estudio de la Estad�stica Descriptiva con el
estudio de las relaciones entre varias variables estad�sticas.
Al igual que en m�dulo anterior se explica c�mo resolver los problemas con
la ayuda de aplicaciones inform�ticas.

En la segunda parte del tema se dan los fundamentos de Probabilidad necesarios
para el estudio de la Estad�stica Inferencial.

\documentclass{article}
\usepackage[catalan]{babel}
\usepackage[latin1]{inputenc}   % Permet usar tots els accents i car�ters llatins de forma directa.
\usepackage{enumerate}
\usepackage{amsfonts, amscd, amsmath, amssymb}
\usepackage[pdftex]{graphicx}
\usepackage{longtable}

\setlength{\textwidth}{16cm}
\setlength{\textheight}{24.5cm}
\setlength{\oddsidemargin}{-0.3cm}
\setlength{\evensidemargin}{0.25cm} \addtolength{\headheight}{\baselineskip}
\addtolength{\topmargin}{-3cm}

\newcommand\Z{\mathbb{Z}}
\newcommand\R{\mathbb{R}}
\newcommand\N{\mathbb{N}}
\newcommand\Q{\mathbb{Q}}
\newcommand\K{\Bbbk}
\newcommand\C{\mathbb{C}}

\newcounter{exctr}
\newenvironment{exemple}
{ \stepcounter{exctr} 
\hspace{0.2cm} 
\textit{Exemple  \arabic{exctr}: }
\it
\begin{quotation}
}{\end{quotation}}


\begin{document}

\textbf{\Large Tema 5. Filtres digitals}

\vskip 0.3 cm
En aquest tema s'estudia el comportament freq�encial dels sistemes LTI 
i s'analitzen els sistemes que modifiquen les caracter�stiques freq�encials 
dels senyals d'entrada d'una manera preestablerta. Aquest tipus de sistemes s'anomenen \textbf{filtres}. 

\vskip 0.5 cm
\noindent
\textbf{\large An�lisi freq�encial de sistemes LTI}

%resposta d'un sistema LTI a una exponencial complexa i a una sinusoide
%exemple

Per a estudiar el comportament freq�encial dels sistemes LTI s'analitza
la sortida d'aquests sistemes quan l'entrada �s un to pur (una exponencial 
complexa o una sinusoide).

En general, la sortida d'un sistema LTI amb resposta impulsional $h[n]$ 
quan s'excita amb una entrada $x[n]$ �s
\[
y[n]=x[n] * h[n] = \sum_{k=-\infty}^\infty h[k] x[n-k]
\]

En el cas que l'entrada sigui una exponencial complexa $x[n]=A e^{j \omega n}$:
\[
y[n]=\sum_{k=-\infty}^\infty h[k] A e^{j \omega (n-k)}=\left( \sum_{k=-\infty}^\infty h[k] e^{-j \omega k} \right) A e^{j \omega n}=
H(\omega) A e^{j \omega n}
\]

\noindent
on $H(\omega)$ �s la transformada de Fourier de $h[n]$. Aquesta transformada existeix si el sistema �s estable 
(�s a dir, si $\sum_{k=-\infty}^\infty |h[n]| < \infty$).

Si escrivim $H(\omega)$ en funci� del seu m�dul i la seva fase ($H(\omega)=|H(\omega)| e^{j \Theta(\omega)}$) llavors
\begin{equation}
y[n]=A |H(\omega)| e^{j (\omega n + \Theta(\omega))}
\end{equation}

Si $h[n]$ pren valors reals $|H(\omega)|$ �s una funci� parell ($|H(\omega)|=|H(-\omega)|$) i 
$\Theta(\omega)$ �s una funci� imparell ($\Theta(\omega)=-\Theta(-\omega)$).
En aquest cas la resposta a una sinusoide (sinus o cosinus) �s:
\begin{equation}
\begin{array}{l}
y[n]=h[n] * A \cos(\omega n)=A |H(\omega)| \cos(\omega n + \Theta(\omega)) \\ \\
y[n]=h[n] * A \sin(\omega n)=A |H(\omega)| \sin(\omega n + \Theta(\omega)) 
\end{array}
\end{equation}

La conclusi� d'aquesta an�lisi �s que la resposta d'un sistema LTI davant un senyal sinuso�dal
v� totalment caracteritzada per $|H(\omega)|$ i $\Theta(\omega)$. 
$|H(\omega)|$ determina l'amplificaci� ($|H(\omega)| > 1$) o atenuaci� ($|H(\omega)| < 1$)
de l'amplitud de la sinusoide, mentre que $\Theta(\omega)$ determina el despla�ament de fase.
$H(\omega)$ rep el nom de \textbf{resposta freq�encial} del sistema.

\vskip 0.3 cm
\noindent
\textbf{Exemple:} Determinau la resposta del sistema amb resposta impulsional $h[n]=(\frac{1}{2})^n u[n]$
al senyal d'entrada $x[n]=10 - 5 \sin \frac{\pi}{2} n + 20 \cos \pi n$


\vskip 0.5 cm
\noindent
\textbf{\large An�lisi freq�encial de sistemes LTI descrits per una funci� de transfer�ncia racional}

%Expressi� de la densitat espectral d'energia en termes de la T.Z.
%Relaci� entre la resposta freq�encial i la posici� de pols i zeros.

Consideram ara el cas dels sistemes LTI que es poden descriure amb una funci� de transfer�ncia
(transformada Z de la resposta impulsional) racional.

\[
H(z)=\frac{\sum_{k=0}^M b_k z^{-k}}{1+ \sum_{k=0}^N a_k z^{-k}}=
b_0 \frac{ \Pi_{k=1}^M (1-z_k z^{-1}) }{ \Pi_{k=1}^N (1-p_k z^{-1}) }
\]

Recordem que si la ROC de $H(z)$ cont� el cercle unitat llavors $H(\omega)=H(z)|_{z=e^{j\omega}}$, per tant
\[
H(\omega)=b_0 \frac{ \Pi_{k=1}^M (1-z_k e^{-j\omega}) }{ \Pi_{k=1}^N (1-p_k e^{-j\omega}) }
\]

Tamb� podem trobar l'expressi� de $|H(\omega)|^2$ en funci� de la transformada Z per a sistemes amb $h[n]$ real:
\[
|H(\omega)|^2=H(\omega) H^*(\omega)=H(\omega) H(-\omega)=H(z) H(z^{-1})|_{z=e^{j\omega}}
\]


Per �ltim, a partir de l'expressi� 
\[
H(\omega)=b_0 \frac{ \Pi_{k=1}^M (1-z_k e^{-j\omega}) }{ \Pi_{k=1}^N (1-p_k e^{-j\omega}) }=
b_0 e^{j \omega (N-M)} \frac{ \Pi_{k=1}^M (e^{j\omega}-z_k) }{ \Pi_{k=1}^N (e^{j\omega}-p_k) }
\]

podem interpretar la relaci� entre la posici� dels pols i zeros de $H(z)$ i la resposta freq�encial
del sistema. Si $e^{j\omega} \simeq z_k$ per a algun $z_k$, llavors $H(\omega) \simeq 0$, per tant 
els zeros pr�xims al cercle unitat fan que la resposta a sinuosides amb freq��ncia propera al zero
sigui petita. En canvi, si $e^{j\omega} \simeq p_k$ per a algun $p_k$, llavors $H(\omega) \simeq \infty$, per tant 
els pols pr�xims al cercle unitat fan que la resposta a sinuosides amb freq��ncia propera al pol
sigui gran (veure Figura \ref{polszerosHomega}).

\begin{figure}[htbp]
\begin{center}
\includegraphics[width=5cm]{polszerosHomega.png}
\end{center}
\caption{Relaci� entre la posici� dels pols i zeros del sistema amb la seva resposta en freq��ncia.}
\label{polszerosHomega}
\end{figure}

\vskip 0.3 cm
\noindent
\textbf{Exemple:} Raonau perqu� el sistema descrit per la funci� de transfer�ncia $H(z)=\frac{1}{1-0.8z^{-1}}$
t� un pic de la magnitud de la resposta freq�encial a $\omega=0$.
 

\vskip 0.5 cm
\noindent
\textbf{\large Sistemes LTI com a filtres selectius en freq��ncia}

%Caracter�stiques dels filtres ideals: passa baix, passa banda, passa alt, banda eliminada i passa tot.
%
%Pols i zeros de filtres passa baix, passa banda i passa alt.
%Pols i zeros de filtres passa tot.
%
%Filtres de fase lineal i fase lineal generalitzada
%
%Filtres invertibles, filtres de fase m�nima. 
%Descomposici� d'un filtre en filtre de fase m�nima i filtre passa-tot.


La paraula \textit{filtre} fa refer�ncia, en general, a un dispositiu
que discrimina, d'acord amb qualque atribut, els elements que passen
a trav�s d'ell.

Per analogia, es parla de filtres freq�encials quan ens referim als sistemes
que permeten passar (amb poca o cap atenuaci�) senyals d'unes determinades freq��ncies
(\textbf{banda de pas})
i que atenuen (total o parcialment) senyals d'altres freq�encies (\textbf{banda de stop}).

Els \textbf{filtres ideals} s�n aquells que atenuen totalment una banda de freq��ncies
i deixen passar sense modificaci� una altra banda de freq��ncies. La magnitud 
de la seva resposta freq�encial es de la forma:
\[
|H(\omega)|=\begin{cases} 1 & \text{si } \omega_1 < \omega < \omega_2 \\ \\
0 & \text{altrament} \end{cases}
\]

En funci� dels valors de $\omega_1$ i $\omega_2$ es parla de filtres passa-baix,
passa-banda, passa-alt, de banda eliminada o passa-tot (veure Figura \ref{idealfilters}).

\begin{figure}[htbp]
\begin{center}
\includegraphics[width=12cm]{idealfilters.png}
\end{center}
\caption{Tipus de filtres ideals.}
\label{idealfilters}
\end{figure}

\vskip 0.3 cm
\noindent
\textbf{Diagrama de pols i zeros i tipus de filtres}

A partir de l'analisi de la relaci� entre la posici� dels pols i zeros del sistema i la seva
resposta freq�encial (veure secci� anterior) podem deduir si un sistema �s passa-baix, passa-alt
o passa-banda observant el seu diagrama de pols i zeros. La Figura \ref{polszerosfiltres}
mostra un exemple per als casos dels filtres passa-baix i passa-alt.

\begin{figure}[htbp]
\begin{center}
\includegraphics[width=7cm]{polszerosfiltres.png}
\end{center}
\caption{Diagrames de pols i zeros per a filtres passa-baix (dalt) i passa-alt (baix).}
\label{polszerosfiltres}
\end{figure}

\vskip 0.2 cm
Per al cas dels filtres passa tot tenim que 
\[
|H(\omega)|=1 \quad \forall \omega \qquad \text{equivalent a } \quad |H(\omega)|^2=1 \quad \forall \omega
\]

Si la funci� de transfer�ncia del sistema s'escriu com una funci� racional l'anterior relaci�
implica que $H(z)$ �s de la forma 
\[
H(z)=z^{-N} \frac{A(z^{-1})}{A(z)}
\]

\noindent
ja que llavors
\[
|H(\omega)|^2=H(z) H(z^{-1})|_{z=e^{j\omega}}= z^{-N} \frac{A(z^{-1})}{A(z)} z^{N} \frac{A(z)}{A(z^{-1})}|_{z=e^{j\omega}}= 1
\]

En aquest cas, si $z_0$ \'es un pol de $H(z)$, llavors $\frac{1}{z_0}$ ha d'�sser un zero de $H(z)$ i viceversa.
La Figura \ref{polszerospassatot} mostra dos exemples de diagrames de pols i zeros per a filtres passa-tot.

\begin{figure}[htbp]
\begin{center}
\includegraphics[width=8cm]{polszerospassatot.png}
\end{center}
\caption{Diagrames de pols i zeros per a filtres passa-tot.}
\label{polszerospassatot}
\end{figure}

\vskip 0.3 cm
\noindent
\textbf{Filtres de fase lineal}

En les seccions anteriors s'han definit els filtres ideals com aquells que tenen una magnitud 
de la resposta freq�encial igual a $1$ en una determinada banda de freq��ncies i $0$ fora
d'ella, per� no hem parlat de la fase de la resposta freq�encial.

Direm que un filtre �s de \textbf{fase lineal generalitzada} si $\Theta(\omega)=-\omega \alpha + \beta$.
Si $\beta=0$ direm que el filtre �s de fase lineal. Anem a estudiar el comportament
dels filtres amb fase lineal generalitzada.

Sigui un filtre digital amb $h[n]$ real i resposta freq�encial
\[
H(\omega)=\begin{cases} C e^{-j( \omega \alpha - \beta)} & \text{si } \omega_1 < \omega < \omega_2 \\ \\
0 & \text{altrament} \end{cases}
\]

La resposta d'aquest filtre a un senyal sinuso�dal amb freq�encia $\omega \in (\omega_1, \omega_2)$ �s:
\begin{equation}
\begin{array}{l}
y[n]=h[n] * \cos(\omega n)=A C \cos(\omega (n - \alpha) + \beta) \\ \\
y[n]=h[n] * \sin(\omega n)=A C \sin(\omega (n - \alpha)+ \beta)
\end{array} 
\end{equation}

�s a dir, la sortida del sistema �s una versi� atenuada, retardada i despla�ada en fase del senyal d'entrada.
Malgrat que aquest comportament del filtre estigui lluny del comportament ideal (voldriem un senyal id�ntic
al d'entrada a la sortida), el fet que per a totes les freq��ncies $\omega \in (\omega_1, \omega_2)$ es tengui
la mateixa atenuaci�, retard i despla�ament de fase fa que, per a senyals compostos per v�ries freq��ncies,
el senyal de sortida tengui la mateixa \textit{forma} que el d'entrada (no hi ha \textit{distorsi�}). 
Aix� no passaria si cada freq��ncia
es vei�s afectada de forma diferent pel filtre. �s per aquest motiu que els filtres amb fase lineal generalitzada
es consideren \textit{suficientment bons} per al filtratge de senyal i no s'exigeix que els filtres ideals
tenguin fase $0$ ($\Theta(\omega)=0$).


\vskip 0.3 cm
\noindent
\textbf{Filtres de fase m�nima. Filtres invertibles}

Un filtre �s diu de fase m�nima si tots els pols i els zeros de la seva funci� de transfer�ncia
es troben a l'interior del cercle unitat. 

Si la funci� de transfer�ncia �s de la forma $H(z)=\frac{B(z)}{A(z)}$ i el filtre �s
de fase m�nima llavors:
\begin{itemize}
\item el filtre �s estable i causal
\item el filtre �s invertible, �s a dir, existeix una funci� $H_I(z)$ tal que $H(z) H_I(z)=1$.
En aquest cas $H_I(z)=\frac{1}{H(z)}=\frac{A(z)}{B(z)}$
\item el filtre invers �s estable i causal
\end{itemize}

Una darrera propietat relativa als filtres de fase m�nima �s que la funci� de transfer�ncia de
qualsevol filtre amb fase no m�nima �s pot escriure com el producte de les funcions de
transfer�ncia d'un filtre de fase m�nima i un filtre passa tot:

\[
H(z)=H_{\text{min}}(z) H_{\text{pt}}(z)
\]

En particular, si el filtre �s estable i causal amb $H(z)=\frac{B(z)}{A(z)}$ i 
$B(z)$ es pot descomposar com $B(z)=B_1(z) B_2(z)$,
on $B_1(z)$ t� totes les seves arrels dins el cercle unitat i $B_2(z)$ fora del cercle unitat,
llavors 
\[
H_{\text{min}}(z)=\frac{ B_1(z) B_2(z^{-1}) }{A(z)}
\qquad 
H_{\text{pt}}(z)=\frac{B_2(z)}{B_2(z^{-1})}
\]
\noindent
on $B_2(z^{-1})$ �s un polinomi amb totes les seves arrels dins el cercle unitat i $A(z)$ t� tamb� 
totes les seves arrels dins el cercle unitat ja que el filtre �s estable i causal. 

 

\vskip 0.5 cm
\noindent
\textbf{\large Disseny de filtres digitals}

%El problema de la no-causalitat dels filtres ideals
%
%Caracter�stiques dels filtres causals i realitzables
%
%Propietats de simetria dels filtres FIR de fase lineal generalitzada
%
%Disseny de filtres FIR de fase lineal mitjan�ant finestres
%Disseny de filtres FIR de fase lineal per mostreig de freq��ncies
%Disseny de filtres IIR a partir de filtres anal�gics: filtres Butterworth i Txebyshev
%

\vskip 0.2 cm
\noindent
\textbf{El problema de la no-causalitat dels filtres ideals}

Considerem un filtre passa-baix ideal amb funci� de transfer�ncia
\[
H(\omega)=\begin{cases} 1 & \text{si } |\omega| \leq \omega_c \\ 0 & \text{altrament} \end{cases}
\]

La transformada inversa d'aquesta funci� d�na com a resposta impulsional
\[
h[n]=\begin{cases} \frac{\omega_c}{n} & \text{si } n=0 \\ 
\frac{\omega_c}{n} \mathrm{sinc}(\omega_c n) & \text{altrament} \end{cases}
\]

Observam (Figura \ref{passabaixideal}) que aquesta resposta impulsional correspon a un filtre no causal i per tant �s \textbf{irrealitzable}.
En general obtenim el mateix resultat per a qualsevol filtre ideal: qualsevol filtre per al qual
$|H(\omega)|=0$ en un interval de freq��ncies (cas de tots els filtres ideals) �s un filtre no causal i per
tant irrealitzable.

\begin{figure}[htbp]
\begin{center}
\includegraphics[width=7cm]{passabaixideal.png}
\end{center}
\caption{Resposta impulsional d'un filtre passa-baix ideal.}
\label{passabaixideal}
\end{figure}

\vskip 0.3 cm
De manera que a l'hora de dissenyar un filtre realitzable (i per tant causal) no podem esperar obtenir 
una resposta en freq��ncia com les mostrades en la Figura \ref{idealfilters} sino que ens haurem de 
conformar amb una resposta com la que es mostra en la Figura \ref{realizablefilters}.
En aquesta figura els valors de $\delta_1$, $\delta_2$, $\omega_p$ i $\omega_c$ indiquen els marges
de toler�ncia permesos en el disseny del filtre.

\begin{figure}[htbp]
\begin{center}
\includegraphics[width=7cm]{realizablefilters.png}
\end{center}
\caption{Magnitud de la resposta freq�encial d'un filtre realitzable}
\label{realizablefilters}
\end{figure}

\vskip 0.3 cm
\noindent
\textbf{Disseny de filtres FIR}

En aquestes seccions ens centram en l'estudi d'algunes t�cniques per al disseny de filtres
causals amb resposta impulsional finita (FIR) i amb fase lineal generalitzada.

\vskip 0.3 cm
\noindent
\textbf{Propietats de simetria dels filtres FIR de fase lineal generalitzada}

Considerem el cas general d'un filtre FIR causal de longitud $M+1$:
\[
h[n]=\{ h[0], h[1], \cdots, h[M-1], h[M] \}
\]

Es pot demostrar que el filtre �s de fase lineal generalitzada si 
\[
h[n]=\pm h[M-n] \qquad \forall n=0, 1, \cdots, M
\]

(la implicaci� inversa no �s certa, �s a dir, un filtre pot �sser de fase lineal
sense verificar l'anterior relaci�).

Aquesta condici� implica que la transformada Z de $h[n]$ ha de verificar:
\[
z^{-M}H(z^{-1})=\pm H(z)
\]
\noindent
�s a dir, les arrels del polinomi $H(z)$ tamb� s�n arrels de $H(z^{-1})$, la qual cosa significa que si
$z_i$ �s zero de $H(z)$ llavors $\frac{1}{z_i}$ tamb� ho �s. A m�s, si $h[n]$ �s real llavors
les arrels complexes han de formar parells conjugats i per tant, si $z_i$ �s zero de $H(z)$ llavors
$z_i^*$,$1/z_i$ i $1/z_i^*$ tamb� s�n arrels. La Figura \ref{zerosFIRlineal} mostra un exemple
de la distribuci� dels zeros en un filtre FIR causal real de fase lineal.

\begin{figure}[htbp]
\begin{center}
\includegraphics[width=5cm]{zerosFIRlineal.png}
\end{center}
\caption{Exemple de la distribuci� dels zeros en un filtre FIR causal real de fase lineal.}
\label{zerosFIRlineal}
\end{figure}

\vskip 0.2 cm
Si $h[n]=h[M-n]$ �s diu que el filtre FIR de fase lineal �s \textbf{sim�tric}.
Si $h[n]=-h[M-n]$ �s diu que el filtre �s \textbf{antisim�tric}. En ambdos casos es pot distingir
entre els filtres amb un nombre de mostres ($M+1$) parell o imparell. Aix� d�na lloc
a quatre configuracions possibles per als filtres FIR de fase-lineal:
tipus I (sim�tric imparell), tipus II (sim�tric parell), tipus III (antisim�tric imparell)
i tipus IV (antisim�tric parell). La Figura \ref{tipusFIR} mostra un exemple de cada tipus
i la Figura \ref{tipusFIR_H} mostra la magnitud de les seves respostes en freq��ncia.
Observam que els filtres tipus I i II s�n filtres passa-baix, els tipus III passa banda 
i el tipus IV passa-alt.

\begin{figure}[htbp]
\begin{center}
\includegraphics[width=12cm]{tipusFIR.png}
\end{center}
\caption{Exemples de configuracions de filtres FIR de fase lineal. Font: Discret-Time Signal
Processing, A. Oppenheim, W. Schafer, Prentice-Hall, 1989}
\label{tipusFIR}
\end{figure}

\begin{figure}[htbp]
\begin{center}
\includegraphics[width=10cm]{tipusFIR_H.png}
\end{center}
\caption{$|H(\omega)|$ per a les configuracions de filtres FIR de fase lineal de la figura \ref{tipusFIR}.
Font: Discret-Time Signal
Processing, A. Oppenheim, W. Schafer, Prentice-Hall, 1989}
\label{tipusFIR_H}
\end{figure}



\vskip 0.3 cm
\noindent
\textbf{Disseny de filtres FIR de fase lineal mitjan�ant finestres}

A partir d'una resposta en freq��ncia especificada $H_d(\omega)$ l'objectiu �s
trobar un filtre, amb una resposta impulsional $h[n]$, causal i que tengui 
una resposta en freq��ncia $H(\omega)$ similar a $H_d(\omega)$.

El filtre ideal $h_d[n]$ es pot obtenir fent l'antitransformada de Fourier de $H_d(\omega)$:
\[
h_d[n]=\frac{1}{2\pi} \int_{-\pi}^\pi H_d(\omega) e^{j\omega n} \, d\omega
\]

\noindent
com $h_d[n]$ t� una resposta impulsional infinita i per tant no causal, s'aproxima mitjan�ant
un filtre causal $h[n]$ de la seg�ent manera:
\[
h[n]=\begin{cases} h_d[n] & \text{si } n \geq 0 \\ 0 & \text{altrament} \end{cases}
\]

Aquest proc�s �s equivalent a multiplicar $h_d[n]$ per una \textit{finestra} 
$w[n]=\begin{cases} 1 & \text{si } n \geq 0 \\ 0 & \text{altrament} \end{cases}$:
\[
h[n]=h_d[n] w[n]
\]

\noindent
Per tant:
\[
H(\omega)={\cal F}^{-1} \{ h_d[n] w[n] \} = H_d(\omega) * W(\omega) 
\]

\noindent
$h[n]$ es troba fent l'antitransformada d'aquesta funci�.

La Figura \ref{exdissenyfinestra} mostra alguns resultats d'aplicar aquest m�tode al disseny
d'un filtre passa-baix ideal. A part de la finestra rectangular $w[n]$ definida m�s amunt es poden definir
finestres amb formes diferents que permeten aproximar millor el comportament dels filtres ideals.


\begin{figure}[htbp]
\begin{center}
\includegraphics[width=10cm]{exdissenyfinestra.png}
\end{center}
\caption{Exemples de disseny d'un filtre passa-baix amb finestres.}
\label{exdissenyfinestra}
\end{figure}



\vskip 0.3 cm
\noindent
\textbf{Disseny de filtres FIR de fase lineal per mostreig de freq��ncies}

En aquest cas s'especifica $H_d(\omega)$ per a un conjunt discret de freq��ncies $\omega_k$
i es troben els valors de $h[n]$ resolent un sistema d'equacions.

\vskip 0.3 cm
\noindent
\textbf{Disseny de filtres FIR de fase lineal �ptims}

Els m�todes de disseny anterior no permeten controlar de manera precisa les freq��ncies de
separaci� de la banda de pas i la de stop.
Els filtres FIR de fase lineal �ptims minimitzen la difer�ncia entre les respostes del filtre
ideal i el filtre dissenyat en cada una de les bandes. L'equaci� d'aquest filtres es troba
resolent un problema de minimitzaci�, seguint una t�cnica formulada per Txebytxev.



\vskip 0.3 cm
\noindent
\textbf{Disseny de filtres IIR}

Les t�cniques de disseny de filtres IIR es basen en la conversi� de filtres anal�gics a filtres
digitals. Els filtres anal�gics es dissenyen per a verificar una s�rie d'especificacions,
seguint unes t�cniques de disseny ben conegudes en l'�rea del processament anal�gic: disseny
tipus Butterworth o disseny tipus Txebyshev. La figura \ref{ButtTxeby} mostra un exemple
de la forma de la resposta freq�encial de cada tipus de filtre. La conversi� de les expresions
obtingudes al cas discret es fa mitjan�ant un canvi de variables (transformaci� bilineal).

\begin{figure}[htbp]
\begin{center}
\includegraphics[width=12cm]{ButtTxeby.png}
\end{center}
\caption{Exemples de respostes en freq��ncia de filtres anal�gics Butterworth (esquerra) i Txebytxev (dreta).}
\label{ButtTxeby}
\end{figure}


\vskip 7cm
\noindent
\textbf{Nota:} la majoria de figures d'aquest cap\'itol s'han tret de Digital Signal Processing, J. Proakis, D. Manolakis, Pearson Prentice Hall, 2007.




\end{document}
%\documentclass{article}
\usepackage{enumerate}
\usepackage{graphicx}
\usepackage{amsfonts, amscd, amsmath, amssymb}
\usepackage[spanish]{babel}

\oddsidemargin -0.3cm
\evensidemargin 0cm
\textwidth 16.5cm
\textheight 21cm

\def\N{\mathbb N}
\def\Z{\mathbb Z}
\def\R{\mathbb R}
\def\C{\mathbb C}

\title{An\`alisi d'imatges}
\date{}

\begin{document}

\maketitle
\tableofcontents

\section{Contorns}

\subsection{Definici\'o de contorn}

\subsection{Filtratge gaussi\`a}

\subsection{M\`etodes multi-escala}


\section{Segmentaci\'o}

\subsection{T\`ecniques b\`asiques de segmentaci\'o}

\subsection{Segmentaci\'o basada en la minimitzaci\'o d'un funcional}

\subsubsection{Segmentaci\'o de Mumford-Shah}

\section{An\`alisi morfol\`ogic: el mapa topogr\`afic}

\subsection{El principi d'invari\`ancia per canvi de contrast}

Una propietat important de la visi\'o humana, enunciada pels psic\`olegs gestaltistes ja en 1923
\'es que la percepci\'o dels objectes d'una escena \'es independent de la il.luminaci\'o o del
color. La Figura \ref{fig_invcontrast} il.lustra aquesta propietat.


\begin{figure}[htbp]
\begin{tabular}{ccc}
\includegraphics[width=5cm]{imatges6/lombardo_valsecchi2.eps} &
\includegraphics[width=5cm]{imatges6/lombardo_grey.eps} &
\includegraphics[width=5cm]{imatges6/lombardo_grey2.eps}
\end{tabular}
\caption{Invari\`ancia de la percepci\'o als canvis de color o contrast: en les tres imatges
som capa\c{c}os de reconeixer els objectes que hi apar\`eixen.}
\label{fig_invcontrast}
\end{figure}


En la Figura \ref{fig_invcontrast} observam tres imatges que representen la mateixa escena. Una d'elles
\'es una imatge en color, i les altres dues s\'on imatges de nivell de gris amb il.luminacions diferents.
A pesar d'aquestes difer\`encies en el color i la il.luminaci\'o no tenim cap dificultat en recon\`eixer
com a id\`entics els objectes que apareixen en les tres imatges. En aquest sentit deim que la informaci\'o
de {\it forma} continguda en les imatges \'es invariant als canvis de color i contrast.

\vskip 0.3 cm
Si ens centram en el cas de les imatges de nivell de gris, ens podem demanar si els models per a
l'an\`alisi d'imatges presentats en les seccions anteriors (contorns i segmentaci\'o) tenen en compte
aquesta invari\`ancia. Dit en altres paraules, les caracter\'\i stiques b\`asiques d'una imatge, 
obtingudes per segmentaci\'o o extracci\'o de contorns, s\'on les mateixes quan la imatge \'es sotmesa a un
canvi de contrast? La resposta \'es ``no'' i les seg\"uents figures ho il.lustren.


\begin{figure}[htbp]
\begin{center}
\begin{tabular}{cc}
\includegraphics[width=5cm]{imatges6/lombardo_grey_edges.eps} &
\includegraphics[width=5cm]{imatges6/lombardo_grey2_edges.eps}
\end{tabular}
\end{center}
\caption{Els contorns obtinguts per a dues imatges relacionades per un canvi de contrast s\'on diferents.
En conseq\"u\`encia, podem afirmar que la representaci\'o de les imatges basada en els contorns no \'es 
invariant als canvis de contrast.}
\label{fig_contrast_contorns}
\end{figure}


\begin{figure}[htbp]
\begin{center}
\begin{tabular}{cc}
\includegraphics[width=5cm]{imatges6/lombardo_grey_seg.eps} &
\includegraphics[width=5cm]{imatges6/lombardo_grey2_seg.eps}
\end{tabular}
\end{center}
\caption{Les segmentacions obtingudes per a dues imatges relacionades per un canvi de contrast s\'on diferents.
En conseq\"u\`encia, podem afirmar que la representaci\'o de les imatges basada en la segmentaci\'o 
no \'es invariant als canvis de contrast.}
\label{fig_contrast_segmentacio}
\end{figure}



Les figures anteriors mostren com els models presentats fins ara per a l'an\`alisi d'imatges s\'on incapa\c{c}os 
de complir un dels requeriments b\`asics de la percepci\'o humana, el de la invari\`ancia per canvi de contrast. 
Per aquest motiu cercam un nou model que s\'\i $ $ el compleixi. Aquest model est\`a basat en la representaci\'o 
de la imatge pels seus conjunts de nivell.

\subsubsection{Canvis de contrast i filtratge d'imatges}
Abans de passar a estudiar aquesta nova representaci\'o de les imatges dedicam unes l\'\i nies a parlar dels
canvis de contrast des del punt de vista de la seva modelitzaci\'o matem\`atica. El model m\'es simple
consisteix en considerar un canvi de contrast com una funci\'o mon\`otona creixent  - estrictament o no - 
(veure la Figura \ref{canvicontrast}, centre). 
\[
g:\R \longrightarrow \R \qquad \text{$g$ mon\`otona creixent} \qquad \Longleftrightarrow \qquad 
\text{$g$ \'es un canvi de contrast}
\]

Aquest model captura la caracter\'\i stica esencial d'un canvi
de contrast: els valors de nivell de gris s\'on modificats, per\`o no la seva relaci\'o d'ordre. En aquest sentit,
si un objecte \'es m\'es clar que un altre en la imatge original, el mateix passar\`a en la nova imatge
(veure les imatges esquerra i dreta de la Figura \ref{canvicontrast}). Com a molt, si la funci\'o de canvi de
contrast no \'es estrictament mon\`otona pot passar que dos nivells que abans eren diferents tenguin ara assignat
el mateix valor.


\begin{figure}[htbp]
\begin{tabular}{ccc}
\includegraphics[width=5cm]{imatges6/lombardo_grey.eps} &
\includegraphics[width=5cm]{imatges6/lombardo_canvi.eps} &
\includegraphics[width=5cm]{imatges6/lombardo_grey2.eps}
\end{tabular}
\caption{Imatge original, canvi de contrast i imatge resultant d'aplicar el canvi de contrast a la primera imatge.}
\label{canvicontrast}
\end{figure}


\noindent
{\bf Definici\'o}. Deim que un {\bf operador} (filtre) $T$ que associa a una imatge $u$ una nova imatge $Tu$ (dins
un determinat espai d'imatges) \'es {\bf invariant per canvi de contrast} (o {\bf morfol\`ogic}) si compleix la propietat
\begin{equation}
\label{prop_inv_contrast}
T(g(u))=g(Tu)
\end{equation}

\noindent
on $g$ \'es un canvi de contrast.

\noindent
\'Es a dir, $T$ \'es invariant per canvi de contrast si commuta amb totes les funciones mon\`otones creixents $g$.

\vskip 0.3 cm
\noindent
{\bf Exercici}. Recordem (Tema 5) que el filtratge gaussi\`a es pot modelar 
mitjan\c{c}ant l'equaci\'o de la calor:

\[
\frac{\partial u(x, y, t)}{\partial t} = \nabla^2 u(x, y, t) 
\]

\noindent
on $u_0=u(x, y, 0)$ \'es la imatge original, $\nabla^2$ \'es l'operador laplaci\`a i 
$u(x, y, t)$ \'es la imatge filtrada a 
l'escala $t$, la qual \'es soluci\'o de l'anterior equaci\'o en derivades parcials.

\vskip 0.2 cm
\noindent
Demostrar que l'operador $T_t$ que assigna a l'imatge original $u_0=u(x, y, 0)$ 
la nova imatge $T_t u(x, y, 0)=u(x, y, t)$ no \'es invariant per canvi de contrast.

\vskip 0.2 cm
\noindent
{\it Soluci\'o}. Si $T_t$ \'es invariant per canvi de contrast s'ha de complir que 
\[
T_t g(u(x, y, 0))=g(T_t u(x, y, 0))=g(u(x, y, t)
\]
\noindent
\'es soluci\'o de l'equaci\'o de la calor.
Aplicant la regla de la cadena i el fet que $u(x, y, t)$ \'es soluci\'o de l'equaci\'o de la calor 
obtenim (suposant que $g$ \'es $C^2$):
\begin{equation}
\label{exgauss1}
\frac{\partial g(u(x, y, t))}{\partial t}=g'(u) \frac{\partial u(x, y, t)}{\partial t}=g'(u) \nabla^2 u(x, y, t) 
\end{equation}

\begin{equation}
\label{exgauss2}
\nabla^2 g(u(x, y, t)) = g'(u) \nabla^2 u(x, y, t) + g''(u) \|\nabla u\|^2
\end{equation}

\noindent
Comparant aquestes dues equacions podem veure com, en general, 
$\frac{\partial g(u(x, y, t))}{\partial t} \neq \nabla^2 g(u(x, y, t))$, i per tant el filtratge gaussi\`a
no \'es invariant per canvis de contrast.


\vskip 0.7 cm

\noindent
{\bf Observacions}. 
\begin{enumerate}
\item En general, les operacions damunt imatges definides per una convoluci\'o no s\'on invariants
per canvis de contrast. Aix\'o implica que  les operacions amb imatges realitzades amb filtres lineals
i invariants espacials no s\'on invariants per canvis de contrast, ja que aquestes operacions es modelen com la
convoluci\'o de la imatge original per la resposta impulsional del filtre (veure el Tema 3).

\item Si a m\'es de la invari\`ancia per canvi de contrast exigim als filtres altres propietats relacionades
amb la percepci\'o, com la invari\`ancia per translaci\'o, per rotaci\'o, zoom, etc. podem arribar a deduir
l'expressi\'o general de les operacions perceptualment acceptables damunt les imatges. S'ha demostrat [AGLM] que 
aquestes operacions es poden relacionar amb la fam\'\i lia d'equacions diferencials estudiades
en el Tema 5. Recordem que aquestes equacions tenen la forma
\[
\frac{\partial u}{\partial t}=|\nabla u| \text{curv}(u)^\alpha
\]

\end{enumerate}

\subsection{Conjunts de nivell i canvis de contrast} 
\noindent
Donada una imatge $u(x, y): \Omega \subseteq \R^2 \longrightarrow \R$ definim el {\bf conjunt de nivell (superior)} 
$\lambda$ de $u$ com:

\begin{equation}
\label{upperlevelset}
{\cal X}_\lambda u = {\cal X}_\lambda^\text{sup} u = \{ (x, y) \in \Omega : u(x, y) \geq \lambda \}
\end{equation}

\vskip 0.2 cm
\noindent
De manera an\`aloga podem definir el {\bf conjunt de nivell inferior} $\lambda$ com:

\begin{equation}
\label{lowerlevelset}
{\cal X}_\lambda^\text{inf} u = \{ (x, y) \in \Omega : u(x, y) \leq \lambda \}
\end{equation}


\vskip 0.3 cm
\noindent
{\bf Observaci\'o}. Un conjunt de nivell (superior o inferior) pot estar format per una o v\`aries 
components connexes\footnote{Es diu que un espai topol\`ogic $X$ \'es {\bf connex}
si tot subconjunt $C$ de $X$ 
no es pot escriure com la uni\'o de dos conjunts disjunts no buits. En aquest cas es diu que $C$ \'es connex.
Un conjunt $C$ de $X$ 
s'anomena {\bf component connexa} si \'es connex i per a tot subconjunt connex $C_1$ de $X$ 
tal que $C \subseteq C_1$, llavors $C_1=C$. La noci\'o de connexitat en el cas d'imatges digitals
ha estat definida en el Tema 4.} (veure la Figura \ref{levelset3D}).

\vskip 0.2 cm
\noindent
La Figura \ref{levelset3D} mostra un exemple de conjunt de nivell superior d'una imatge.

\begin{figure}[htbp]
\begin{center}
\begin{tabular}{cc}
\includegraphics[width=4cm]{imatges6/im3dorigg.eps} & 
\includegraphics[width=75mm]{imatges6/im3d.eps} \\
\includegraphics[width=4cm]{imatges6/im3dlevelorig.eps} & 
\includegraphics[width=75mm]{imatges6/im3dlevelsetmark.eps} 
\end{tabular}
\end{center}
\caption{Representaci\'o tridimensional d'una imatge i d'un dels seus conjunts de nivell.
En la part superior, imatge original i representaci\'o 3D. En la part inferior, un conjunt de
nivell de la imatge i la seva representaci\'o tridimensional. Podem observar con el conjunt
de nivell est\`a compost per v\`aries components connexes.}
\label{levelset3D}
\end{figure}

\newpage
%\vskip 0.3 cm
\noindent
{\bf Propietats}. Donada una imatge $u: \Omega \subseteq \R^2 \longrightarrow \R$, si 
${\cal X}_\lambda u$ denota el conjunt de nivell superior $\lambda$ de $u$, llavors
\begin{enumerate}[i)]
\item {\bf Propietats d'inclusi\'o}.
\begin{enumerate}[a)]
\item ${\cal X}_\lambda u \subseteq {\cal X}_\mu u$, per a tot $\lambda > \mu$
\item ${\cal X}_\lambda u =\cap_{\mu < \lambda} {\cal X}_\mu$ u, per a tot $\lambda \in \R$
\end{enumerate}
\item {\bf F\`ormula de reconstrucci\'o}. 
\begin{equation}
\label{levelsetreconst}
u(x, y)=\sup \{ \lambda \in \R : (x, y) \in {\cal X}_\lambda u \}
\end{equation}
\end{enumerate}

\vskip 0.3 cm
\noindent
{\it Demostraci\'o}. 
\begin{enumerate}
\item[] {\it Propietat i)a)}. La propietat es dedueix directament de la definici\'o de conjunt de nivell,
ja que, com $\lambda > \mu$, 
\[
\begin{split}
{\cal X}_\mu u & = \{ (x, y) \in \Omega : u(x, y) \geq \mu \} = \\ \\
& = \{ (x, y) \in \Omega : u(x, y) \geq \lambda \} \cup \{ (x, y) \in \Omega : \mu \geq u(x, y) < \lambda \} 
 \supseteq \{ (x, y) \in \Omega : u(x, y) \geq \lambda \} = {\cal X}_\lambda u
\end{split}
\]

\item[] {\it Propietat i)b)}. (Exercici).


\item[] {\it Propietat ii)}. (Exercici).

\end{enumerate}


\vskip 0.4 cm
\noindent
{\bf Exercici}. {\bf Conjunts de nivell d'una imatge digital}.
En el cas d'una imatge digital, el conjunt de valors que pren $u(x, y)$ est\`a quantitzat i, normalment,
$u(x, y) \in \{0, 1, 2, \dots, 255\}$. Es demana demostrar que en una imatge digital existeixen un m\`axim de 
255 conjunts de nivell diferents.

\vskip 0.2 cm
\noindent
{\it Demostraci\'o}. 
Basta demostrar que per a tot $0 < x < 1$, $i \in \{ 0, 1, \dots, 254\}$ : 
\[
\begin{split}
{\cal X}_{i+x} u & = \{ (x, y) \in \Omega : u(x, y) \geq i+x \}= \\ \\
& = \{ (x, y) \in \Omega : i+x \leq u(x, y) < i+1 \} \, \cup \, \{ (x, y) \in \Omega : u(x, y) \geq i+1 \} = \\ \\
& = \emptyset \, \cup \, \{ (x, y) \in \Omega : u(x, y) \geq i+1 \} = {\cal X}_{i+1} u
\end{split}
\]

\vskip 0.5 cm
La Figura \ref{ex_levelsets} mostra (marcats en blanc) alguns exemples de conjunts de nivell per a una imatge. 
Observar la propietat d'inclusi\'o d'aquests conjunts a mida que augmenta $\lambda$. Observar que cada 
conjunt de nivell est\`a format per una o v\`aries components connexes.


\begin{figure}[htbp]
\begin{center}
\begin{tabular}{cc}
\includegraphics[width=55mm]{imatges6/ny.eps} &
\includegraphics[width=55mm]{imatges6/ny100.eps} \\
\includegraphics[width=55mm]{imatges6/ny150.eps} &
\includegraphics[width=55mm]{imatges6/ny200.eps}
\end{tabular}
\end{center}
\caption{Imatge original i conjunts de nivell superiors per a $\lambda=100$, $150$ i $200$. Els pixels que pert\`anyen
al conjunt de nivell s'han marcat en color blanc. Observar la propietat d'inclusi\'o entre els conjunts i 
l'exist\`encia de v\`aries components connexes en cada un d'ells.}
\label{ex_levelsets}
\end{figure}

\vskip 0.5 cm
\noindent
{\bf Teorema} [GuichardMorel]. {\bf Conjunts de nivell i canvis de contrast}. Donada una imatge $u$ i una funci\'o cont\'\i nua estrictament creixent $g$ (canvi de contrast continu estrictament creixent), llavors 
\[
\forall \lambda \in \R \quad \exists \mu \in \R \quad \text{tal que} \quad {\cal X}_\lambda u = {\cal X}_\mu g(u)
\qquad \text{ i }
\]
\[
\forall \mu \in \R \quad \exists \lambda \in \R \quad \text{tal que} \quad {\cal X}_\lambda u = {\cal X}_\mu g(u)
\]

\noindent
\'Es a dir, tot conjunt de nivell de $u$ \'es un conjunt de nivell de $g(u)$ i viceversa. Per tant, $u$ i $g(u)$
tenen, globalment, els mateixos conjunts de nivell.

\vskip 0.2 cm
\noindent
{\it Demostraci\'o}.
Definim $g^{(-1)}(\lambda)$ com
\[
g^{(-1)}(\lambda)=\inf \{ r : g(r) \geq \lambda \}
\]
\noindent
Com que $g$ \'es cont\'\i nua, tamb\'e ser\`a cont\'\i nua per la dreta i tenim que per a tota successi\'o de 
nombres $r_n$ tals que $r_n > g^{(-1)}(\lambda)$
\[
\lim_{r_n \rightarrow g^{(-1)}(\lambda)} g(r_n)=g(g^{(-1)}(\lambda))
\]
\noindent
Com que $r_n > g^{(-1)}(\lambda)$, per la definici\'o de $g^{(-1)}(\lambda)$ tenim que $g(r_n) \geq \lambda$, 
llavors $\lim_{r_n \rightarrow g^{(-1)}(\lambda)} g(r_n) \geq \lambda$ i per tant
\[
g(g^{(-1)}(\lambda)) \geq \lambda
\]
\noindent
\'Es f\`acil demostrar ({\it exercici}) que si aquesta condici\'o es compleix i $g$ \'es no decreixent, llavors 
\[
g(s) \geq \lambda \qquad \Longleftrightarrow \qquad s \geq g^{(-1)}(\lambda) 
\]

\noindent
En conseq\"u\`encia, podem escriure la igualtat seg\"uent:
\[
{\cal X}_\lambda(g(u))= \{ (x, y) : g(u(x, y)) \geq \lambda \}= 
\{ (x, y) : u(x, y) \geq g^{(-1)}(\lambda) \} = {\cal X}_\mu u
\]
\noindent
on $\mu=g^{(-1)}(\lambda)$. 

\noindent
Per una altra banda, si $g$ \'es estrictament creixent, llavors
\[
\{ (x, y) : u(x, y) \geq \mu \} = \{ (x, y) : g(u(x, y)) \geq g(\mu) \}
\]
\noindent
i seguint un raonament similar a l'anterior tenim que
\[
{\cal X}_\mu u = \{ (x, y) : u(x, y) \geq \mu \} = \{ (x, y) : g(u(x, y)) \geq g(\mu) \} =
\{ (x, y) : u(x, y) \geq g^{(-1)}(g(\mu)) \} = {\cal X}_\lambda 
\]
\noindent
on $\lambda = g^{(-1)}(g(\mu))$.

%\newpage
\vskip 0.5 cm
\noindent
{\bf Propietats}.
\begin{enumerate}

\item Donades dues imatges $u$ i $v$ que tenen globalment els mateixos conjunts de nivell, podem trobar un
canvi de contrast $g$  creixent tal que $v=g(u)$. A m\'es, $g$ es pot definir per la f\`ormula
\[
g(\lambda)=\sup \{ \mu, {\cal X}_\mu v \supset {\cal X}_\lambda u \}
\] 

\item Si $g$ no \'es estrictament creixent ni cont\'\i nua per\`o \'es cont\'\i nua per la dreta\footnote{Recordem 
que una funci\'o $g$ \'es cont\'\i nua per la dreta en un punt $s$ si
\[
\lim_{r \rightarrow s, r > s} g(r)=g(s)
\]}
, llavors podem al menys assegurar que tots els conjunts de nivell
de $g(u)$ tamb\'e ho s\'on de $u$, per\`o no a la inversa.

\end{enumerate}


\vskip 0.5 cm
La Figura \ref{ex_inv_levelsets} mostra dues imatges relacionades per un canvi de contrast i un conjunt 
de nivell com\'u a totes dues, en el primer cas corresponent al nivell 235 i en l'altre al nivell 130.

\begin{figure}[htbp]
\begin{center}
\begin{tabular}{ccc}
\includegraphics[width=5cm]{imatges6/orig_contrast_inv1.eps} & 
\includegraphics[width=5cm]{imatges6/orig_contrast_inv2.eps} &
\includegraphics[width=5cm]{imatges6/ex_contrast_inv1.eps}
\end{tabular}
\end{center}
\caption{D'esquerra a dreta, dues imatges relacionades per un canvi de contrast i un conjunt de nivell com\'u
a ambdues. El conjunt de nivell correspon al nivell 235 per a la primera imatge i al nivell 130 per a la segona.}
\label{ex_inv_levelsets}
\end{figure}

\vskip 0.5 cm
\noindent
{\bf Conclusi\'o}. Hem demostrat que els conjunts de nivell ens donen una descripci\'o de la imatge invariant 
per canvis de contrast. Significa aix\'o que els podem considerar com les estructures b\`asiques de la imatge?
En la secci\'o seg\"uent veurem que \'es possible simplificar la descripci\'o de la imatge donada pels conjunts de
nivell si ens quedam \'unicament amb les fronteres d'aquests conjunts. Aquesta simplificaci\'o d\'ona lloc
al {\bf mapa topogr\`afic} de la imatge.


\subsection{L\'\i nies de nivell i mapa topogr\`afic}

Donada una imatge $u:\Omega \subseteq \R^2 \longrightarrow \R$, definida en un compacte $\Omega$ de $\R^2$,
anomenam {\bf l\'\i nia de nivell} $\lambda$ de $u$ al conjunt de punts
\begin{equation}
\label{levelline}
u^{-1}(\lambda)=\{ (x, y) \in \Omega : u(x, y)=\lambda \}
\end{equation}


\vskip 0.3 cm
\noindent
{\bf Teorema}. Si $u:\Omega \subseteq \R^2 \longrightarrow \R$, amb $\Omega$ compacte,
 \'es $C^1$ llavors, per a quasi tots els nivells $\lambda$, la l\'\i nia de
nivell 
\[
u^{-1}(\lambda)=\{ (x, y) \in \Omega : u(x, y)=\lambda \}
\]
\noindent
est\`a formada per una uni\'o finita de corbes de Jordan\footnote{Una corba cont\'\i nua es 
diu de Jordan si no t\'e cap autointersecci\'o, excepte possiblement en els seus punts extrems. M\'es 
formalment, $C$ \'es una corba de Jordan si podem trobar una aplicaci\'o cont\'\i nua 
\[
\boldsymbol{x}: [a, b] \subseteq \R \longrightarrow \R^2
\]

\noindent
tal que $\boldsymbol{x}$ \'es bijectiva en $(a, b)$ i $\boldsymbol{x}([a, b])=C$. 
Si $\boldsymbol{x}(a)=\boldsymbol{x}(b)$
llavors es diu que la corba de Jordan \'es tancada. Si $\boldsymbol{x}$ \'es de classe $C^1$ en $[a, b]$, 
llavors es diu que la corba de Jordan \'es de classe $C^1$.} tancades de classe $C^1$.
Els \'unics nivells $\lambda$ per als quals no es compleix aquesta propietat s\'on aquells per als quals 
existeix un $(x, y) \in u^{-1}(\lambda)$ tal que $|\nabla u|(x, y) = 0$. Aquests nivell reben el nom de 
{\bf nivells singulars}.

\vskip 0.2 cm
\noindent
{\it Demostraci\'o}. El teorema \'es conseq\"u\`encia dels teoremes de la funci\'o impl\'\i cita i de 
Sard\footnote{{\bf Teorema de Sard}\cite{??}. Sigui $u$ una funci\'o real $C^1$ definida en un compacte $R$
de $\R^2$, llavors per a quasi tot $\lambda \in \R$ el conjunt $u^{-1}(\lambda)$ \'es no singular,
\'es a dir, $\forall (x, y) \in u^{-1}(\lambda)$, $|\nabla u (x, y)| \neq 0$.}.


\vskip 0.3 cm
\noindent
{\bf Observaci\'o}. La hip\`otesi de continu\"\i tat de $u$ i de les seves derivades \'es perfectament aplicable
ja que, tal com s'ha argumentat en el Tema 4, una imatge digital s'ha d'interpretar com la versi\'o
discreta d'una funci\'o cont\'\i nua subjacent, d'amplada de banda limitada, la qual \'es interpolable a partir 
dels valors discrets mitjan\c{c}ant la interpolaci\'o de Shannon (Tema 2), i per tant, de classe $C^\infty$.

\vskip 0.3 cm
\noindent
{\bf Definici\'o}. Anomenam {\bf mapa topogr\`afic} d'una imatge $u:\Omega \subseteq \R^2 \longrightarrow \R$
al conjunt de corbes de Jordan obtingudes en calcular les l\'\i nies de nivell per a tots
els valors de $\lambda$ no singulars de $u$.


\vskip 0.3 cm
\noindent
{\bf Teorema de les corbes de Jordan} [DoCarmo]. Una corba tancada de Jordan $C$ divideix el pla en dues
components connexes, una acotada i l'altra no. Anomenam {\bf interior} de $C$, i ho denotam $\text{Int}(C)$,
a la component connexa acotada. Anomenam {\bf exterior} de la corba a la component connexa no acotada.

\vskip 0.3 cm
\noindent
Una conseq\"u\`encia del teorema anterior \'es que podem assignar a cada corba del mapa topogr\`afic una 
{\bf orientaci\'o}, segons si el vector $\nabla u (x, y)$ est\`a dirigit cap a l'interior o l'exterior de 
la l\'\i nia de nivell en tots els seus punts. En el primer cas parlam d'orientaci\'o {\bf positiva} i en
el segon d'orientaci\'o {\bf negativa}.


\subsubsection{Organitzaci\'o del mapa topogr\`afic en una estructura d'arbre}

{\bf Propietat}. Les corbes del mapa topogr\`afic mai es creuen. O b\'e estan incloses unes dins les altres, o
b\'e els seus interiors s\'on disjunts (veure la Figura \ref{propcorbesmap}).

\vskip 0.3 cm
\noindent
{\it Demostraci\'o}. (Exercici).

\vskip 0.3 cm

\begin{figure}[htbp]
\begin{center}
\includegraphics[width=6cm]{imatges6/exllines.eps} $\qquad \qquad$
\includegraphics[width=6cm]{imatges6/exnollines.eps}
\end{center}
\caption{La primera imatge mostre alguns exemples de corbes del mapa topogr\`afic. Les corbes estan incloses unes
dins les altres o b\'e els seus interiors s\'on disjunts. La segona imatge mostra una corba que no pot pert\`anyer
al mapa topogr\`afic, ja que es creua amb ella mateixa.}
\label{propcorbesmap}
\end{figure}


Aquesta propietat ens permet organitzar les l\'\i nies de nivell del mapa topogr\`afic dins una estructura
d'arbre. La Figura \ref{ex_arbre} mostra un exemple de mapa topogr\`afic per a una imatge senzilla i l'estructura
d'arbre que se'n deriva.

\begin{figure}[htbp]
\begin{center}
\begin{tabular}{ccc}
\includegraphics[width=45mm]{imatges6/extreeimD.eps} &
\includegraphics[width=45mm]{imatges6/extreeB.eps} &
\includegraphics[width=45mm]{imatges6/extreeA.eps}
\end{tabular}
\end{center}
\caption{D'esquerra a dreta, imatge original, algunes l\'\i nies del seu mapa topogr\`afic i estructura 
d'arbre del mapa topogr\`afic. La corba etiquetada com {\it Z} representa el contorn de la imatge.}
\label{ex_arbre}
\end{figure}

\subsubsection{Mapa topogr\`afic i conjunts de nivell}

{\bf Teorema}. Si $u$ \'es de classe $C^1$, la l\'\i nia de nivell $\lambda$ de $u$, $u^{-1}(\lambda)$, 
del mapa topogr\`afic \'es la frontera topol\`ogica del conjunt de nivell ${\cal X}_\lambda u$:
\[
u^{-1}(\lambda)=\partial {\cal X}_\lambda u
\]

\vskip 0.3 cm
La Figura \ref{ex_levellines} mostra un conjunt de nivell i la seva l\'\i nia de nivell associada.

\begin{figure}[htbp]
\begin{center}
\begin{tabular}{ccc}
\includegraphics[width=55mm]{imatges6/whitetigergrey.eps} & 
\includegraphics[width=55mm]{imatges6/whitetigergrey_set.eps} &
\includegraphics[width=55mm]{imatges6/whitetigergrey_level.eps}
\end{tabular}
\end{center}
\caption{D'esquerra a dreta: imatge original, un dels seus conjunts de nivell i la l\'\i nia de nivell del
conjunt.}
\label{ex_levellines}
\end{figure}


\vskip 0.3 cm
El conjunt de nivell ${\cal X}_\lambda u$ es pot obtenir a partir de la seva frontera. Si anomenam 
$c_i^+$ les corbes de $u^{-1}(\lambda)$ orientades positivament i $c_{j}^-$ les corbes 
orientades negativament, llavors
\[
{\cal X}_\lambda u=\cup_i \left( \text{Int}(c_i^+) - 
\left( \cup_{c_j^- \subset \text{Int}(c_i^+)} \text{Int}(c_j^-) \right) \right)
\]

Intuitivament, l'anterior expressi\'o equival a `omplir' primer les corbes $c_i$ per llevar despres els seus 
forats.

\vskip 0.3 cm
Com que el conjunt de $\lambda$'s no singulars per als quals hem calculat el mapa topogr\`afic \'es
dens (pel teorema de Sard, el conjunt de nivells singulars \'es de mesura nula), 
les anteriors propietats ens asseguren que podrem trobar els conjunts de nivell de la imatge per a quasi
tots els possibles $\lambda$. A partir d'aquest conjunts de nivell podrem reconstruir la imatge original
mitjan\c{c}ant la f\'ormula de reconstrucci\'o (\ref{levelsetreconst}). En conclusi\'o:

\vskip 0.2 cm
\noindent
{\bf Proposici\'o}. Sigui $u$ una funci\'o $C^1$ definida en un conjunt compacte de $\R^2$. Llavors $u$
es pot recuperar a partir del seu mapa topogr\`afic.



\vskip 0.3 cm
\noindent
{\bf Conclusions}. El conjunt de l\'\i nies de nivell d'una imatge (el mapa topogr\`afic) hereta les propietats
d'invari\`ancia als canvis de contrast dels conjunts de nivell. 
A m\'es, el mapa topogr\`afic representa una estructura m\'es simple que la dels conjunts de nivell, a partir de la 
qual \'es sempre possible reconstruir aquests conjunts de nivell. Per aquest motiu 
{\bf el mapa topogr\`afic \'es el model que utilitzarem per representar les imatges de manera invariant per canvi de contrast}. Tamb\'e direm que el mapa topogr\`afic cont\'e la {\bf informaci\'o geom\`etrica o morfol\`ogica} 
de la imatge, en el sentit que descriu la {\it forma} dels objectes que apareixen dins l'escena,
independentment del seu nivell de gris o color.


\subsubsection{Mapa topogr\`afic d'una imatge digital}

Una manera pr\`actica de calcular el mapa topogr\`afic d'una imatge digital \'es a partir de la 
seva versi\'o cont\'\i nua obtinguda per interpolaci\'o bilineal a trossos. 

Per a cada grup de quatre 
p\'\i xels $(i, j)$, $(i+1, j)$, $(i, j+1)$, $(i+1, j+1)$ amb nivells de gris respectius $\lambda_1$,
$\lambda_2$, $\lambda_3$ i $\lambda_4$, es defineix una funci\'o $u_{ij}(x, y)=axy+bx+cy+d$ tal que
$u(0, 0)=\lambda_1$, $u(1, 0)=\lambda_2$, $u(0, 1)=\lambda_3$ i $u(1, 1)=\lambda_4$. La seg\"uent 
figura mostra un exemple d'aquesta interpolaci\'o.

\begin{figure}[htbp]
\begin{center}
\includegraphics[width=40mm]{imatges6/exQpixel.eps} $\quad$
\includegraphics[width=60mm]{imatges6/IntQpixel_b.eps}
\end{center}
\caption{Exemple d'interpolaci\'o bilineal dels valors de quatre pixels.}
\label{ex_bilinear4}
\end{figure}

El conjunt de funcions $u_{ij}$ obtingudes per a cada grup de quatre pixels de la imatge d\'ona lloc a una 
funci\'o bilineal a trossos com la mostrada en la Figura \ref{ex_bilinear}.

\begin{figure}[htbp]
\begin{center}
\includegraphics[width=50mm]{imatges6/testBilinear.eps} $\quad$
\includegraphics[width=70mm]{imatges6/IntImg3D_1b.eps}
\end{center}
\caption{D'esquerra a dreta, imatge original (per a cada pixel s'indica el seu valor de nivell de gris) 
i imatge obtinguda per interpolaci\'o bilineal.}
\label{ex_bilinear}
\end{figure}



L'avantatge d'aquest m\`etode
\'es que \'es f\`acil de calcular i no crea nous m\`axims o m\'\i nims en la imatge, ja que, per a cada grup
de pixels,  els valors de la funci\'o interpolada estan compressos
entre els valors m\`axim i m\'\i nim dels  p\'\i xels.
Les l\'\i nies de nivell obtingudes a partir de la funci\'o interpolada
 s\'on cont\'\i nues i estan formades per la concatenaci\'o
d'hip\`erboles. A m\'es, les propietats d'inclusi\'o comentades en les seccions anteriors es continuen cumplint.
Les l\'\i nies de nivell es calculen per a un conjunt discret de valors no singulars $\lambda$ i s'organitzen en una
estructura d'arbre. 

La Figura \ref{ex_bilinear_ll}-esquerra mostra un exemple d'algunes de les l\'\i nies de nivell obtingudes 
per a la imatge de la Figura \ref{ex_bilinear}.
Finalment, comentar que les l\'\i nies que arriben al contorn de la imatge es tanquen seguint el cam\'\i $ $ m\'es
curt damunt d'aquest contorn, tal com mostra la Figura \ref{ex_bilinear_ll}-dreta.
%\ref{tancament}.

\begin{figure}[htbp]
\begin{center}
\begin{tabular}{ccc}
\includegraphics[width=40mm]{imatges6/LvImg_2b.eps} & &
\includegraphics[width=30mm]{imatges6/interior_open.eps}
\end{tabular}
\end{center}
\caption{Esquerra, mapa topogr\`afic de la imatge de la Figura \ref{ex_bilinear}.
Dreta, criteri per al tancament de les corbes que arriben al contorn de la imatge.}
\label{ex_bilinear_ll}
\end{figure}


%\begin{figure}[htbp]
%\begin{center}
%\includegraphics[width=40mm]{imatges6/LvImg_2b.eps}
%\end{center}
%\caption{Mapa topogr\`afic de la imatge de la Figura \ref{ex_bilinear}.}
%\label{ex_bilinear_ll}
%\end{figure}
%
%\begin{figure}[htbp]
%\begin{center}
%\includegraphics[width=30mm]{imatges6/interior_open.eps}
%\end{center}
%\caption{Criteri per al tancament de les corbes que arriben al contorn de la imatge.}
%\label{tancament}
%\end{figure}


\newpage
\subsection{Aplicacions del mapa topogr\`afic}

\subsubsection{Comparaci\'o morfol\`ogica d'imatges}

\vskip 0.3 cm
Siguin dues imatges digitals $u_1$ i $u_2$ de la mateixa escena tals que, per a tots els pixels $(i, j)$, 
els valors $u_1(i, j)$ i $u_2(i, j)$ corresponen a la intensitat de la llum en el mateix punt 
de l'escena. En tal cas es diu que les imatges estan perfectament {\bf posades en correspond\`encia}
({\it registered}). 
Volem detectar els canvis en l'escena deguts a la pres\`encia de nous objectes,
per\`o no els deguts a canvis en la il.luminaci\'o. Aquesta situaci\'o \'es t\'\i pica en la comparaci\'o 
d'imatges de sat\`el.lit preses en dates diferents i en les quals volem detectar la construcci\'o
o destrucci\'o d'edificis, la variaci\'o dels camps de cultiu, etc.

Les t\`ecniques de comparaci\'o basades en variacions de nivell de gris no permetran la detecci\'o 
de tals canvis, ja que entre dates diferents les condicions d'il.luminaci\'o poden variar bastant.
En aquests casos, la comparaci\'o dels mapes topogr\`afics \'es un m\`etode efica\c{c} per trobar els
canvis. La seg\"uent propietat ens permet dissenyar un algortme r\`apid detecci\'o:

\vskip 0.3 cm
\noindent
{\bf Propietat}. Per un teorema cl\`asic de l'an\`alisi de funcions de v\`aries variables podem afirmar
que les l\'\i nies de nivell s\'on perpendiculars a la direcci\'o del vector gradient a la funci\'o en tots els
punts en qu\`e aquest gradient existeix. 

\vskip 0.3 cm
\noindent
La Figura \ref{fig_canvi_grad} il.lustra la propietat anterior. La difer\`encia de l'angle 
del vector gradient entre dos punts de correspond\`encia de les imatges ens d\'ona una mesura de la variaci\'o
local del mapa topogr\`afic en el punt.

\begin{figure}[htbp]
\begin{center}
\includegraphics[width=6cm]{imatges6/fig_canvi_grad.eps}
\end{center}
\caption{La figura mostra, en diferents colors, dues l\'\i nies de nivell corresponents a dues imatges de la 
mateixa escena. Suposam que les imatges estan perfectament posades en correspond\`encia, 
de manera que el pixel (x, y) de la 
primera imatge es correspon amb el pixel (x, y) de la segona. 
La variaci\'o local del mapa topogr\`afic en aquest punt es pot quantificar mesurant la variaci\'o de l'angle 
del vector gradient a cada imatge en el punt.}
\label{fig_canvi_grad}
\end{figure}

\vskip 0.2 cm
\noindent
Aprofitant aquesta propietat s'ha dissenyat el seg\"uent algoritme:
\begin{enumerate}
\item Calcular el gradient en cada punt de les imatges (utilitzar els m\`etodes de discretitzaci\'o del
valor del gradient descrits en el Tema 5).
\item Calcular el m\`odul i l'angle del gradient (respecte a la direcci\'o horitzontal) en cada punt de les
imatges. Anomenam $M_1(i, j)$ i $\alpha_1(i, j)$ al m\`odul i l'angle del gradient del pixel $(i, j)$ de la 
primera imatge; respectivament, $M_2(i, j)$ i $\alpha_2(i, j)$ representen el 
m\`odul i l'angle del gradient del pixel $(i, j)$ de la segona imatge. 
\item Si $M_1(i, j)$ i $M_2(i, j)$ s\'on majors que un cert llindar (t\'\i picament $5$), calculam 
la difer\`encia entre els seus angles, $\Delta A (i, j)=|A_1(i, j)-A_2(i, j)|$.
\item La difer\`encia anterior mesura la variaci\'o del mapa topogr\`afic en cada punt de la imatge.
Representam amb el seg\"uent codi de colors la imatge `difer\`encia': blanc, els pixels amb 
$\Delta A (i, j) < \delta$; els pixels amb $\Delta A (i, j) \geq \delta$; i gris els p\'\i xels tals
que $M_1(i, j)$ o $M_2(i, j)$ \'es inferior al llindar de gradient. $\delta$ \'es un llindar que
determina a partir de quina difer\`encia d'angle consideram diferents les orientacions del gradient
(t\'\i picament $20^\text{o}$.
\item La imatge resultant ens d\'ona una idea gr\`afica del grau de variaci\'o dels mapes topogr\`afics
de les imatges: com m\'es punts blacs, m\'es s'assemblen les imatges.
\end{enumerate}


Les seg\"uents figures mostren els resultats de la comparaci\'o per a dues imatges d'una ciutat i per
un llindar de comparaci\'o de $20^\text{o}$ i un gradient m\'\i nim de $5$.

\vskip 1 cm

\begin{figure}[htbp]
\begin{center}
\includegraphics[width=110mm]{imatges6/toulouse1.eps}
\end{center}
\caption{Primera de les imatges a comparar.}
\label{fig_changes_orig1}
\end{figure}

\begin{figure}[htbp]
\begin{center}
\includegraphics[width=110mm]{imatges6/toulouse2.eps}
\end{center}
\caption{Segona de les imatges a comparar. Les imatges han estat presses en dates diferents.}
\label{fig_changes_orig2}
\end{figure}

\begin{figure}
\begin{center}
\includegraphics[width=110mm]{imatges6/diffgradtoulouse.eps}
\end{center}
\caption{Resultat de la comparaci\'o de les imatges anteriors. S'ha mesurat el canvi en 
la direcci\'o dels seus gradients
amb llindars $5$ i $20^\text{o}$ per al m\`odul i la difer\`encia d'angle dels gradients, respectivament.
Els punts grisos representen pixels amb m\`odul de gradient insuficient en alguna de les imatges;
els punts blans s\'on punts on l'orientaci\'o del gradient no canvia i els negres on l'orientaci\'o canvia.}
\label{fig_changes}
\end{figure}

%\begin{figure}[htbp]
%\begin{center}
%\begin{tabular}{cc}
%\includegraphics[width=85mm]{imatges6/toulouse1.eps} &
%\includegraphics[width=85mm]{imatges6/toulouse2.eps}
%\end{tabular}
%
%\includegraphics[width=85mm]{imatges6/diffgradtoulouse.eps}
%\end{center}
%\caption{Imatges originals (a dalt) i resultat de la comparaci\'o de la direcci\'o dels seus gradients
%amb llindars $5$ i $20^\text{o}$ per al m\`odul i la dife\`encia d'angle dels gradients, respectivament.
%Els punts grisos representen pixels amb m\`odul de gradient insuficient en alguna de les imatges;
%els punts blans s\'on punts on l'orientaci\'o del gradient no canvia i els negres on l'orientaci\'o canvia.}
%\label{fig_changes}
%\end{figure}


\newpage
\subsubsection{Filtratge: filtre de gra}

En general, les formes petites de la imatge (moltes vegades degudes al renou) s\'on dif\'\i cilment percebudes. 
Podem aprofitar aquesta caracter\'\i stica de la visi\'o per simplificar les imatges i eliminar-ne el renou
[Vincent].


L'organitzaci\'o en forma d'arbre del mapa topogr\`afic resulta especialment indicada per eliminar les
formes petites de la imatge. Una forma petita per\`o ben contrastada en la imatge donar\`a lloc a una component
connexa per a algun conjunt de nivell de la imatge; el contorn d'aquesta component connexa ser\`a una corba 
del mapa topogr\`afic. Ja que la forma \'es petita, la corba tamb\'e ho ser\`a i 
es trobar\`a, generalment, en les `fulles' de l'estructura d'arbre del mapa topogr\`afic. Per tant
un algoritme que `podi' l'arbre aconseguir\`a eliminar aquestes formes [GuichardMonasse]. 
Resta el problema d'assignar un nou 
nivell de gris als pixels pertanyents a les formes eliminades. El criteri seguit \'es assignar-los el nivell
de gris de la corba de la mateixa orientaci\'o m\'es petita en la qu\`e estava contingut el contorn de la forma. 
L'\'unic par\`ametre d'aquest filtre \'es l'\`area m\`axima de les formes eliminades.
La Figura \ref{ex_tree_grain} mostra un exemple senzill de simplificaci\'o del mapa topogr\`afic.

\begin{figure}[htbp]
\begin{center}
\begin{tabular}{cc}
\includegraphics[width=5cm]{imatges6/extreeA.eps} &
\includegraphics[width=5cm]{imatges6/extreeB.eps} \\
\includegraphics[width=5cm]{imatges6/extreeAsim.eps} &
\includegraphics[width=5cm]{imatges6/extreeBsim.eps}
\end{tabular}
\end{center}
\caption{Exemple de simplificaci\'o del mapa topogr\`afic mitjan\c{c}ant el filtre de gr\`a.
Les imatges de la part superior mostren un mapa topogr\`afic i la seva estructura d'arbre.
El contorn de la imatge, que constitueix l'arrel de l'arbre, s'ha marcat amb la lletra Z. 
A m\'es, per a aquest exemple suposam que l'orientaci\'o de totes les corbes \'es la mateixa. Donat
un par\`ametre del filtre $\gamma$, si l'interior de les corbes G, N, S, D i K t\'e un \`area inferior
a $\gamma$, llavors aquestes corbes s'eliminen de l'arbre. Les imatges de la part inferior
mostren el resultat del filtratge. Els nous nivells de gris assignats als pixels de l'interior de les
corbes eliminades s\'on els de les corbes m\'es petites que els contenen. Aix\'\i $ $, als pixels
continguts en la corba G s'els assigna el nivell de la corba F, als de la corba N els de la corba M, etc.}
\label{ex_tree_grain}
\end{figure}

La Figura \ref{grainfiltering} mostra l'efecte d'aplicar el filtre de gra, amb par\`ametre d'\`area $10$, sobre
una imatge amb renou. La Figura \ref{ex_grain} mostra la simplificaci\'o de la imatge produida en aplicar 
el filtre de gra a una imatge amb valors creixents del par\`ametre d'\`area.

\begin{figure}
\begin{center}
\begin{tabular}{cc}
\includegraphics[width=80mm]{imatges6/hopperSPgrey.eps} &
\includegraphics[width=80mm]{imatges6/hopperSPfilter10.eps} 
\end{tabular}
\end{center}
\caption{D'esquerra a dreta, imatge amb renou i imatge filtrada amb un filtre de gra de par\`ametre 10. El renou
queda pr\`acticament eliminat.}
\label{grainfiltering}
\end{figure}

\begin{figure}[htbp]
\begin{center}
\begin{tabular}{cc}
\includegraphics[width=63mm]{imatges6/rain.eps} &
\includegraphics[width=63mm]{imatges6/rainll20.eps}
\end{tabular}
\begin{tabular}{cc}
\includegraphics[width=63mm]{imatges6/rain5.eps} &
\includegraphics[width=63mm]{imatges6/rain5ll20.eps}
\end{tabular}
\begin{tabular}{cc}
\includegraphics[width=63mm]{imatges6/rain10.eps} &
\includegraphics[width=63mm]{imatges6/rain10ll20.eps}
\end{tabular}
\begin{tabular}{cc}
\includegraphics[width=63mm]{imatges6/rain40.eps} &
\includegraphics[width=63mm]{imatges6/rain40ll20.eps}
\end{tabular}
\end{center}
\caption{De dalt a baix, imatge original i imatges resultants d'eliminar les formes de
tamany $5$, $10$ i $40$. Les l\'\i nies de nivell dels mapes topogr\`afics corresponents, 
tamb\'e es mostren per a nivells m\'utiples de 20.
Observar com el nivell de detall de la imatge es va reduint i com el mapa topogr\`afic es va simplificant.}
\label{ex_grain}
\end{figure}

\newpage
\subsubsection{Posada en correspond\`encia d'imatges ({\it registration})}

Posar en correspond\`encia ({\it register}) dues imatges significa calcular la transformaci\'o geom\`etrica que 
aplica  els pixels d'una imatge en els de l'altra imatge. Aix\'o significa que, una vegada posades en correspond\`encia, 
el pixel $(i, j)$ de cada imatge representa el nivell d'il.luminaci\'o procedent del mateix punt de l'espai 
tridimensional.

S'han desenvolupat dos m\`etodes de {\it registration} basats en el mapa topogr\`afic de les imatges.

\vskip 0.4 cm
\noindent
{\bf {\it Registration} basat en {\it formes}} [Monasse]

Definim una {\it forma} de la imatge com una component connexa d'algun dels seus conjunts de nivell.
Per tant, una forma no \'es m\'es que un subconjunt de $\R^2$ i pot \'esser caracteritzada per una
s\`erie de valors: \`area, posici\'o del centre de masses, moments de segon ordre, etc. La similitud
entre dues formes pot \'esser llavors mesurada en termes de les difer\`encies entre aquests valors.

Donades dues imatges, si som capa\c{c}os de trobar les correspond\`encies entre les formes d'una i l'altra,
podrem estimar com es mouen els objectes d'una imatge respecte als de l'altra i, si la majoria dels objectes es
desplacen amb el mateix moviment, ser\`a llavors possible estimar el moviment global entre les imatges.


L'algoritme de posada en correspond\`encia basat en les formes \'es el seg\"uent:
\begin{enumerate}
\item Calcular el mapa topogr\`afic de les dues imatges

\item Per a cada corba tancada del mapa, considerar el seu interior, el qual \'es una component connexa ({\it forma})
d'algun conjunt de nivell. 

\item Per a cada {\it forma} calcular els seg\"uents moments: 
\`area (moment d'ordre 0), centre de masses (moments d'ordre 1)
i moments centrats de segon ordre. 

En una imatge digital, 
el moment d'ordre $ij$ d'una forma $A$ es calcula com
\[
m_{ij}=\sum_{(x, y) \in A} x^i y^j
\]

i el moment centrat d'ordre $ij$ \'es
\[
\mu_{ij}=\sum_{(x, y) \in A} (x-m_{10})^i (y-m_{01})^j
\]


\vskip 0.2 cm
Els exponents $i$ i $j$ en les expressions anteriors fan que el c\`alcul de moments d'ordres superiors a 2
sigui molt sensible a petites variacions en els valor de $(x, y)$, degudes al renou. 


\item Calcular els {\it invariants} de cada forma. La informaci\'o proporcionada pels moments no \'es 
suficient per comparar dues formes sotmeses a un moviment arbitrari. Per exemple, el valor de l'\`area
no varia quan la forma es sotmet a una translaci\'o o rotaci\'o, per\`o si quan hi ha un canvi d'escala (zoom).
Per aquest motiu es cerquen combinacions de moments que siguin el m\'es invariants possibles. Els seg\"uents
valors s\'on invariants a translacions, rotacions i canvis d'escala [Reiss]:
\[
\begin{split}
s_1 & =\frac{\mu_{20}+\mu_{02}}{m_{00}^2} \\ \\
s_2 & =\frac{\mu_{20} \mu_{02}- \mu_{11}^2}{m_{00}^2}
\end{split}
\]

Per trobar invariants a altres tipus de moviment (com, per exemple, transformacions afins) hauriem d'utilitzar
moments d'ordre superior que, com ja hem dit, s\'on poc fiables per \'esser molt sensitius al renou.
De manera que el m\`etode servir\`a \'unicament per registrar imatges sotmeses a una transformaci\'o de similitud
(translaci\'o, rotaci\'o i canvi d'escala).


\item Comparar les formes. Es defineixen uns llindars de toler\`ancia $\epsilon_1$ i $\epsilon_2$ tals que,
donades dues formes $A$ i $B$, si $|s_1^A-s_1^B| < \epsilon$ i $|s_2^A-s_2^B| < \epsilon_2$, llavors
consideram que $A$ i $B$ s\'on la mateixa forma. 

Per a cada forma de la primera imatge obtendrem una llista de formes similars en la segona imatge.


\item Calcular el moviment entre parells de formes. Els moments calculats en els punts anteriors no ens permeten
calcular el moviment entre les formes d'una i altra imatge. Com a m\`axim, si coneixem que la forma $A$ de la primera
imatge es correspon amb la forma $A'$ de la segona imatge, podriem calcular la translaci\'o entre els seus centres 
de masses. Per calcular la rotaci\'o i el canvi d'escala entre les formes feim el seg\"uent:

\begin{enumerate}
\item Consideram les formes $A$ i $B$ en la primera imatge i les formes $A'$ i $B'$ de la segona, tals que
$A'$ \'es similar a $A$ i $B'$ \'es similar a $B$ (per als llindars establerts en el punt anterior).
\item Formam els vectors $\vec{AB}$ i $\vec{A'B'}$ i calculam la transformaci\'o (translaci\'o, rotaci\'o i zoom)
que aplica un vector damunt l'altre.
\end{enumerate}

\item Calcular el moviment global entre les imatges. Per a cada parell de formes de la primera imatge repetim el
proc\'es anterior. Enmagatzemam els valors de rotaci\'o, translaci\'o i zoom trobats en un histograma i
calculam el m\`axim pic de l'histograma. Aquest pic ens d\'ona el moviment dominant entre les imatges.


\item Posar en correspond\`encia les imatges amb el valor de moviment trobat.

\end{enumerate}


Les seg\"uents figures il.lustren els diferents pasos de l'algoritme:

\begin{figure}[htbp]
\begin{center}
\begin{tabular}{cc}
\includegraphics[width=70mm]{imatges6/mosaic1.eps} &
\includegraphics[width=70mm]{imatges6/mosaic2.eps}
\end{tabular}
\end{center}
\caption{Imatges originals. Observar l'exist\`encia d'una part com\'u entre les imatges.}
\end{figure}


\begin{figure}[htbp]
\begin{center}
\begin{tabular}{cc}
\includegraphics[width=70mm]{imatges6/mosaic1shapesB.eps} &
\includegraphics[width=70mm]{imatges6/mosaic2shapesB.eps} 
\end{tabular}
\end{center}
\caption{Exemple d'algunes de les formes que es corresponen entre les imatges.}
\end{figure}


\begin{figure}[htbp]
\begin{center}
\begin{tabular}{cc}
\includegraphics[width=70mm]{imatges6/mosaic1vecB.eps} &
\includegraphics[width=70mm]{imatges6/mosaic2vecB.eps} 
\end{tabular}
\end{center}
\caption{Exemple d'un dels parells de formes utilitzats per calcular el moviment entre les imatges.}
\end{figure}


\begin{figure}[htbp]
\begin{center}
\begin{tabular}{cc}
\includegraphics[width=7cm]{imatges6/histZR.eps} &
\includegraphics[width=7cm]{imatges6/histTxy.eps}
\end{tabular}
\end{center}
\caption{Histogrames (representats en tres dimensions) de rotaci\'o-zoom i translaci\'o horitzontal-vertical
que ens permetran calcular el moviment global entre
les imatges.}
\end{figure}

\begin{figure}[htbp]
\begin{center}
\includegraphics[width=10cm]{imatges6/mosaic.eps} 
\end{center}
\caption{Mosaic: resultat de aplicar el moviment trobat a la primera imatge.}
\end{figure}

\newpage
%\vskip 0.4 cm
\noindent
{\bf {\it Registration} basat en peces de l\'\i nies de nivell} [LMMM]

El m\`etode de posada en correspond\`encia basat en les formes presenta dos inconvenients principals:
\begin{enumerate}
\item No permet calcular transformacions afins entre les imatges.
\item \'Es sensible a oclusions parcials.
\end{enumerate}

\noindent
Per il.lustrar aquest darrer inconvenient consideram les imatges de la Figura \ref{ex_oclusio}.

\begin{figure}[htbp]
\begin{center}
\begin{tabular}{cc}
\includegraphics[width=6cm]{imatges6/boat1.eps} &
\includegraphics[width=6cm]{imatges6/boat2.eps}
\end{tabular}
\end{center}
\caption{Dues formes entre les quals hi ha una ocluci\'o parcial. La part com\'una entre les l\'\i nies de nivell
s'ha marcat en color vermell.}
\label{ex_oclusio}
\end{figure}

\noindent
Observam com les formes mostrades en les imatges corresponen als mateixos objectes, no obstant la segona
d'elles presenta una oclusi\'o que impedeix compararla amb la primera amb el m\`etode dels moments.
Aquesta comparaci\'o ser\`a possible \'unicament si som capa\c{c}os de codificar de manera local el
contorn de la forma.


\vskip 0.3 cm
L'observaci\'o anterior ens duu a desenvolupar un nou m\`etode de {\it registration} basat en la codificaci\'o local de
les l\'\i nies de nivell. Aquesta codificaci\'o es fa de forma invariant a transformacions afins del pla 
de les imatges i permet comparar les peces de l\'\i nies de nivell d'una i altra imatge. La informaci\'o de
moviment calculada per a cada pe\c{c}a de corba s'enmagatzema en un histograma i es calcula el moviment global entre
les imatges de manera similar al m\ `etode anterior.

\vskip 0.3 cm  
\noindent
Les seg\"uents figures mostren alguns exemples de l'aplicaci\'o del m\`etode.

\begin{figure}[htbp]
\begin{center}
\begin{tabular}{cc}
\includegraphics[width=7cm]{imatges6/satellite1.eps} &
\includegraphics[width=7cm]{imatges6/satellite2.eps}
\end{tabular}
\end{center}
\caption{Imatges originals. L'imatge de la dreta presenta una part com\'u amb la imatge de l'esquerra (cant\'o
superior esquerre). Les imatges pert\`anyen a canals espectrals diferents (blau i vermell) i existeix una
transformaci\'o af\'\i  $ $ entre elles.}
\end{figure}


\begin{figure}[htbp]
\begin{center}
\begin{tabular}{cc}
\includegraphics[width=10cm]{imatges6/satellite_match.eps} &
\includegraphics[width=7cm]{imatges6/satellite_mosaic.eps}
\end{tabular}
\end{center}
\caption{Peces de l\'\i nies de nivell comunes entre les imatges i mosaic de les imatges anteriors obtengut a partir de la informaci\'o de moviment proporcionada pel nou m\`etode de posada en correspond\`encia.}
\end{figure}

\begin{figure}[htbp]
\begin{center}
\begin{tabular}{cc}
\includegraphics[width=5cm]{imatges6/superman1.eps} &
\includegraphics[width=5cm]{imatges6/superman2.eps}
\end{tabular}
\end{center}
\caption{Imatges originals.}
\end{figure}

\begin{figure}[htbp]
\begin{center}
\begin{tabular}{cc}
\includegraphics[width=6cm]{imatges6/superman_match.eps} &
\includegraphics[width=6cm]{imatges6/superman_recal1.eps}
\end{tabular}
\end{center}
\caption{L\'\i nies de nivell comunes i resultat d'aplicar el moviment estimat a una de les imatges.}
\end{figure}


\newpage
\section{Refer\`encies}

\begin{enumerate}
\item[]
[AGLM] L. Alvarez, F. Guichard, P.L. Lions, J.M. Morel. {\it Axioms and fundamentals equations of image processing}.
Archive for Rational Mechanics and Analysis, 16 (9), pp. 200-257, 1993.
\item[]
[GuichardMorel] F. Guichard, J.M. Morel. {\it Iterative smoothing and PDE's}. Llibre en preparaci\'o, 2002.
\item[]
[DoCarmo] M.P. do Carmo. {\it Geometr\'\i a diferencial de curvas y superficies}. Alianza Universidad Textos, 1994.
\item[]
[GuichardMonasse] F. Guichard, P. Monasse. {\it Fast computation of a contrast-invariant image representation}.
IEEE Transactions on Image Processing, vol. 20, gener, 1999.
\item[]
[Monasse] P. Monasse. {\it Contrast invariant image registration}. Proc. Int. Conference on Acoustics, Speech 
and Signal Processing, vol. 6, pp. 3221-3224, 1999. 
\item[]
[LMMM] J.L. Lisani, L. Moisan, P. Monasse, J.M. Morel. {\it On the theory of planar shape}. 
SIAM Multiscale, Model. Simul. Vol. 1, No. 1, pp. 1-24, Gener, 2003. 
\item[]
[Vincent] L. Vincent. {\it Greyscale area openings and closings, their implementation and applications}. 
First Workshop on Mathematical Morphology and its Applications to Signal Processing. pp. 22-27.
\end{enumerate}


\end{document} 

\end{document}

