\documentclass[a4paper,12pt]{report}
%\usepackage[latin1]{inputenc}   % Permet usar tots els accents i car�ters llatins de forma directa.
\usepackage[utf8]{inputenc}
\usepackage[spanish]{babel}
%\decimalspanish{.}
\usepackage{latexsym}
\usepackage{hyperref}
\usepackage{theorem}
\usepackage{enumerate}
\usepackage{amsfonts, amscd, amsmath, amssymb}
\usepackage[pdftex]{graphicx}
\usepackage{epstopdf}
\usepackage{url, hyperref}


\setlength{\textwidth}{16.5cm}
\setlength{\textheight}{25cm}
\setlength{\oddsidemargin}{-0.3cm}
\setlength{\evensidemargin}{0.25cm} \addtolength{\headheight}{\baselineskip}
\addtolength{\topmargin}{-3cm}

\pagestyle{empty}
\begin{document}
\begin{center}
\textbf{\large Estadística Aplicada}

\textbf{\large Seguretat i Ciències Policials}

\vskip 0.5 cm
\textbf{\large Activitats Avaluació Bloc I}
\end{center}

\vskip 1cm
Volem fer un estudi sobre el nombre d'empreses dels municipis de les
Illes Balears. Utilitzau la fulla de càlcul OpenOffice Calc per a respondre
les següents qüestions.

\begin{enumerate}
\item Recopilació de dades \textit{brutes}: les dades es troben
en l'\textit{Anuari Estadístic Municipal de les Illes Balears}. Per accedir-hi:

\begin{enumerate}[a)]
\item Visitau la web de l'\textit{Observatori del Treball} del Govern
de les Illes Balears:

\url{http://www.caib.es/sacmicrofront/contenido.do?idsite=282&cont=10654}

\item Al menú de la part esquerra heu de fer click damunt \textit{Anuari estadístic municipal}.
Apareix la llista de municipis de les Illes Balears.

\item Si seleccionau un municipi es mostra diferent informació corresponent a l'any 2010.
Per exemple, la superfície del municipi d'Alaior és 109.86 $\mathrm{km}^2$, la seva població 
total 9.399 persones, el `Total comptes cotització amb personal assalariat' (nombre d'empreses) 342, etc.

\item Ens centram en la informació del `Total comptes cotització amb personal assalariat'. Per a cada un dels
municipis de les Illes (67 municipis en total) heu de recopilar aquesta dada. 
Aquest conjunt de 67 valors és el que heu d'utilitzar per fer la pràctica.
\end{enumerate}

\item Calculau a partir de les dades anteriors els següents paràmetres estadístics 
(si és possible): moda, mitjana, mediana, primer i tercer quartils.
Si no és possible o pensau que no té sentit calcular algun d'aquests paràmetres
comentau-ho. Indicau com feu els càlculs.

\item Construcció de la taula de freqüències:
\begin{enumerate}[a)]
\item agrupau les dades anteriors en els següents intervals:
\textit{menys de 50}, $[50, 250)$, $[250, 500)$, $[500, 750)$, $[750, 1000)$, \textit{més de 1000}. 
\item construïu una taula de freqüencies que contengui: les freqüències absolutes, les freqüències
absolutes acumulades, les relatives, les relatives acumulades i els percentatges.
\end{enumerate}
\item Dibuixau un diagrama de barres de freqüències absolutes i un diagrama de tarta de percentatges.

\item Calculau a partir de la taula de freqüències els següents paràmetres estadístics: 
moda, mitjana, mediana, primer i tercer quartils. Indicau com feu els càlculs.
(Nota: per al càlcul de la mitjana utilitzau com a marques de classe del primer i
darrer intervals $25$ i $2000$, respectivament).

\end{enumerate}

\end{document}
