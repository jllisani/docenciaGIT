\documentclass[a4paper,12pt]{report}
%\usepackage[latin1]{inputenc}   % Permet usar tots els accents i car�ters llatins de forma directa.
\usepackage[utf8]{inputenc}
\usepackage[spanish]{babel}
%\decimalspanish{.}
\usepackage{latexsym}
\usepackage{hyperref}
\usepackage{theorem}
\usepackage{enumerate}
\usepackage{amsfonts, amscd, amsmath, amssymb}
\usepackage[pdftex]{graphicx}
\usepackage{epstopdf}
\usepackage{url, hyperref}


\setlength{\textwidth}{16.5cm}
\setlength{\textheight}{25cm}
\setlength{\oddsidemargin}{-0.3cm}
\setlength{\evensidemargin}{0.25cm} \addtolength{\headheight}{\baselineskip}
\addtolength{\topmargin}{-3cm}

\pagestyle{empty}
\begin{document}
\begin{center}
\textbf{\large Estadística Aplicada}

\textbf{\large Seguretat i Ciències Policials}

\vskip 0.5 cm
\textbf{\large Activitats Avaluació Bloc II}
\end{center}

\vskip 1cm
Volem fer un estudi relacionant el nombre d'empreses dels municipis de les
Illes Balears amb la seva superfície. Utilitzau la fulla de càlcul OpenOffice Calc per a respondre
les següents qüestions.


\begin{enumerate}
\item Recopilació de dades \textit{brutes}: les dades es troben
en l'\textit{Anuari Estadístic Municipal de les Illes Balears}. Per accedir-hi:

\begin{enumerate}[a)]
\item Visitau la web de l'\textit{Observatori del Treball} del Govern
de les Illes Balears:

\url{http://www.caib.es/sacmicrofront/contenido.do?idsite=282&cont=10654}

\item Al menú de la part esquerra heu de fer click damunt \textit{Anuari estadístic municipal}.
Apareix la llista de municipis de les Illes Balears.

\item Si seleccionau un municipi es mostra diferent informació corresponent a l'any 2010.
Per exemple, la superfície del municipi d'Alaior és 109.86 $\mathrm{km}^2$, la seva població 
total 9.399 persones, el `Total comptes cotització amb personal assalariat' (nombre d'empreses) 342, etc.

\item Ens centram en les informacions del `Total comptes cotització amb personal assalariat' i de `Superfície'. 
Per a cada un dels municipis de les Illes (67 municipis en total) heu de recopilar aquesta dues dades. 
Aquests dos conjunts de 67 valors són els que heu d'utilitzar per fer la pràctica.
\end{enumerate}

\item Per a la variable `Total comptes cotització amb personal assalariat' calculau, a partir de les dades anteriors, els 
següents paràmetres estadístics (si és possible): rang, rang interquartílic, variància i
desviació típica.

\item Per a la variable `Total comptes cotització amb personal assalariat' calculau els següents paràmetres del diagrama de capsa:
mediana, primer i tercer quartils, límit superior entre valors típics i atípics,
límit inferior entre valors típics i atípics, límit superior entre valors atípics i extrems,
límit inferior entre valors atípics i extrems, màxim valor típic, mínim valor típic, valors
atípics i valors extrems. \textbf{No fa falta dibuixar el diagrama}.

\item Per a la variable `Superfície' calculau, a partir de les dades brutes, els 
següents paràmetres estadístics (si és possible): mitjana, variància i
desviació típica.

\item Dibuixau el diagrama de dispersió per a les variables `Total comptes cotització amb personal assalariat' i `Superfície'.

\item Calculau la covariància i el coeficient de correlació entre les variables 
`Total comptes cotització amb personal assalariat' i 'Superfície'. Existeix una forta relació lineal entre les variables?



\end{enumerate}

\end{document}
