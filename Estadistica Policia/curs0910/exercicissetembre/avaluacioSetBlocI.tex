\documentclass[a4paper,12pt]{report}
\usepackage[latin1]{inputenc}   % Permet usar tots els accents i car�ters llatins de forma directa.
\usepackage[spanish]{babel}
%\decimalspanish{.}
\usepackage{enumerate}
\usepackage{amsfonts, amscd, amsmath, amssymb}
\usepackage{eepic}
\usepackage{graphicx}
\usepackage{url}
\usepackage[pdftex]{hyperref}


\setlength{\textheight}{23cm}
\addtolength{\topmargin}{-1.5cm}

\pagestyle{empty}
\begin{document}
\begin{center}
\textbf{\large Estad�stica Aplicada}

\textbf{\large Seguretat i Ci�ncies Policials}

\vskip 0.5 cm
\textbf{\large Activitats Avaluaci� Bloc I Setembre}
\end{center}

\vskip 1cm
Volem fer un estudi sobre 
la temperatura a Palma de Mallorca (observatori de Son Sant Joan) 
durant el mes de juny de l'any 2009.
 Utilitzau la fulla de c�lcul OpenOffice Calc per a respondre
les seg�ents q�estions.

\begin{enumerate}
\item Recopilaci� de dades \textit{brutes}: 
accediu a a la web \url{http://www.tutiempo.net/clima/Palma_De_Mallorca_Son_San_Juan/83060.htm}
i cercau les dades corresponents al mes de juny de 2009 (considerau les dades 
de temperatura mitjana $T$ per a cadascun dels dies del mes).

\item Calculau a partir de les dades anteriors els seg�ents par�metres estad�stics 
(si �s possible): moda, mitjana, mediana, primer i tercer quartils.
Si no �s possible o pensau que no t� sentit calcular algun d'aquests par�metres
comentau-ho. Indicau com feu els c�lculs.

\item Construcci� de la taula de freq��ncies:
agrupau les dades anteriors en intervals de $1^o C$ (entre el m�nim i el m�xim)
i constru�u la taula de freq�encies que contengui: les freq��ncies absolutes, les freq��ncies
absolutes acumulades, les relatives, les relatives acumulades i els percentatges.
(Nota: si per a algun interval no hi ha dades no fa falta que surti a la taula).

\item Dibuixau un diagrama de barres de freq��ncies absolutes i un diagrama de tarta de percentatges.

\item Calculau a partir de la taula de freq��ncies els seg�ents par�metres estad�stics 
(si �s possible): moda, mitjana, mediana, primer i tercer quartils. Indicau com feu els c�lculs.

\end{enumerate}

\end{document}
