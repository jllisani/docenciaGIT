\documentclass[a4paper,12pt]{report}
\usepackage[latin1]{inputenc}   % Permet usar tots els accents i car�ters llatins de forma directa.
\usepackage[spanish]{babel}
%\decimalspanish{.}
\usepackage{enumerate}
\usepackage{amsfonts, amscd, amsmath, amssymb}
\usepackage{eepic}
\usepackage{graphicx}
\usepackage{url}
\usepackage[pdftex]{hyperref}

\setlength{\textheight}{23cm}
\addtolength{\topmargin}{-1.5cm}

\pagestyle{empty}
\begin{document}
\begin{center}
\textbf{\large Estad�stica Aplicada}

\textbf{\large Seguretat i Ci�ncies Policials}

\vskip 0.5 cm
\textbf{\large Activitats Avaluaci� Bloc II Setembre}
\end{center}

\vskip 1cm
Volem fer un estudi sobre 
la temperatura a Palma de Mallorca (observatori de Son Sant Joan) 
durant el mes de juny de l'any 2009.
 Utilitzau la fulla de c�lcul OpenOffice Calc per a respondre
les seg�ents q�estions.

\begin{enumerate}
\item Recopilaci� de dades \textit{brutes}: 
accediu a a la web \url{http://www.tutiempo.net/clima/Palma_De_Mallorca_Son_San_Juan/83060.htm}
i cercau les dades corresponents al mes de juny de 2009 (considerau les dades 
de temperatura mitjana $T$ per a cadascun dels dies del mes).

\item Calculau a partir de les dades anteriors els 
seg�ents par�metres estad�stics (si �s possible): rang, rang interquart�lic, vari�ncia i
desviaci� t�pica.

\item Calculau els seg�ents par�metres del diagrama de capsa:
mediana, primer i tercer quartils, l�mit superior entre valors t�pics i at�pics,
l�mit inferior entre valors t�pics i at�pics, l�mit superior entre valors at�pics i extrems,
l�mit inferior entre valors at�pics i extrems, m�xim valor t�pic, m�nim valor t�pic, valors
at�pics i valors extrems. \textbf{No fa falta dibuixar el diagrama}.

\item Considerau ara les dades corresponents al mes de juny de 1979 i calculau a partir de les dades brutes els 
seg�ents par�metres estad�stics (si �s possible): mitjana, vari�ncia i
desviaci� t�pica.

\item Dibuixau el diagrama de dispersi� per a les 
temperatures dels mesos de juny de 2009 i 1979.

\item Calculau la covari�ncia i el coeficient de correlaci� entre les 
temperatures dels mesos de juny de 2009 i 1979. Existeix una forta relaci� lineal entre les dades?



\end{enumerate}

\end{document}
