\documentclass[a4paper,12pt]{report}
\usepackage[latin1]{inputenc}   % Permet usar tots els accents i car�ters llatins de forma directa.
\usepackage[spanish]{babel}
%\decimalspanish{.}
\usepackage{latexsym}
\usepackage{hyperref}
\usepackage{theorem}
\usepackage{enumerate}
\usepackage{amsfonts, amscd, amsmath, amssymb}
\usepackage[pdftex]{graphicx}
\usepackage{epstopdf}


\setlength{\textheight}{23cm}
\addtolength{\topmargin}{-1.5cm}

\pagestyle{empty}
\begin{document}
\begin{center}
\textbf{\large Estad�stica Aplicada}

\textbf{\large Seguretat i Ci�ncies Policials}

\vskip 0.5 cm
\textbf{\large Activitats Avaluaci� Bloc I}
\end{center}

\vskip 1cm
Volem fer un estudi sobre el nombre de locals de restauraci� (bars i restaurants) dels municipis de les
Illes Balears. Utilitzau la fulla de c�lcul OpenOffice Calc per a respondre
les seg�ents q�estions.

\begin{enumerate}
\item Recopilaci� de dades \textit{brutes}: 
accediu a les \textit{Fitxes municipals} de la web del Institut d'Estad�stica de
les Illes Balears i accediu a les dades del nombre de llic�ncies de restauraci� i
bars de
tots els municipis de les Illes Balears (Mallorca, Menorca, Eivissa i Formentera).
(Indicaci�: hi ha un total de 67 municipis en les 4 illes i estan disponibles les dades de
l'any 2004).

\item Calculau a partir de les dades anteriors els seg�ents par�metres estad�stics 
(si �s possible): moda, mitjana, mediana, primer i tercer quartils.
Si no �s possible o pensau que no t� sentit calcular algun d'aquests par�metres
comentau-ho. Indicau com feu els c�lculs.

\item Construcci� de la taula de freq��ncies:
agrupau les dades anteriors en intervals de 20 (de $0$ a $19$, de $20$ a $39$, etc.)
i constru�u la taula de freq�encies que contengui: les freq��ncies absolutes, les freq��ncies
absolutes acumulades, les relatives, les relatives acumulades i els percentatges.
(Nota: si per a algun interval no hi ha dades no fa falta que surti a la taula).

\item Dibuixau un diagrama de barres de freq��ncies absolutes i un diagrama de tarta de percentatges.

\item Calculau a partir de la taula de freq��ncies els seg�ents par�metres estad�stics 
(si �s possible): moda, mitjana, mediana, primer i tercer quartils. Indicau com feu els c�lculs.

\end{enumerate}

\end{document}
