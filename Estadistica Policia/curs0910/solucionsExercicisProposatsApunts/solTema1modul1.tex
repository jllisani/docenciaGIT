\documentclass[a4paper,12pt]{report}
\usepackage[latin1]{inputenc}   % Permet usar tots els accents i car�ters llatins de forma directa.
\usepackage[spanish]{babel}
%\decimalspanish{.}
\usepackage{latexsym}
\usepackage{hyperref}
\usepackage{theorem}
\usepackage{enumerate}
\usepackage{amsfonts, amscd, amsmath, amssymb}
\usepackage[pdftex]{graphicx}
\usepackage{epstopdf}
\usepackage{epsdice}

\newtheorem{definition}{Definici\'on}
\newtheorem{theorem}[definition]{Teorema}
\newtheorem{proposition}[definition]{Proposici\'on}
\newtheorem{corollary}[definition]{Corolario}
\newtheorem{lemma}[definition]{Lema}
\newtheorem{example}[definition]{Ejemplo}
\newtheorem{Rem}[definition]{Nota:}
\newcommand{\va}{variable aleatoria }
\def\N{I\!\!N}
\def\R{I\!\!R}
\def\Z{Z\!\!\!Z}
\def\Q{O\!\!\!\!Q}
\def\C{I\!\!\!\!C}
\decimalpoint


%\setlength{\textwidth}{15cm} \setlength{\textheight}{23cm}
\setlength{\textwidth}{17cm} \setlength{\textheight}{24cm}
%%\setlength{\textwidth}{15cm} \setlength{\textheight}{20cm}
\addtolength{\oddsidemargin}{-0.75cm}
\addtolength{\evensidemargin}{-0.75cm}
%\addtolength{\headheight}{\baselineskip}
\addtolength{\topmargin}{-1cm}
%\setlength{\topmargin}{0cm}
%\setlength{\footskip}{2.8cm}
\pagestyle{myheadings}

\begin{document}

\textbf{\Large Soluciones de los ejercicios propuestos}

\vskip 0.5 cm
\noindent
\textbf{Ejercicio 1}
\begin{enumerate}[a)]
\item Poblaci\'on: el total de los alumnos del instituto. 
Muestra: el $30\%$ por ciento de los alumnos de $2^\text{o}$ de bachillerato
\item Poblaci\'on: los habitantes de Ciutadella menores de 35 a\~nos.
Muestra: el $10\%$ de los clientes de los principales locales de copas.
\item Poblaci\'on: todos los conductores del pa\'is. Muestra: 10.000 conductores.
\end{enumerate}

\vskip 0.2 cm
\noindent
\textbf{Ejercicio 2}\begin{enumerate}[a)]
\item Descriptiva.
\item Descriptiva.
\item Inferencial.
\end{enumerate}

\vskip 0.2 cm
\noindent
\textbf{Ejercicio 3}
\begin{enumerate}[a)]
\item Sexo, edad, nivel de estudios, nivel de ingresos, etc.
\item Sexo, edad, antig\"uedad en la empresa, categor\'ia profesional, salario, etc.
\item Tiempo dedicado al estudio, compaginaci\'on de estudios
y trabajo, n\'umero de veces que se ha cursado la asignatura, 
nota en los examenes de acceso a la universidad, etc.
\end{enumerate}

\vskip 0.2 cm
\noindent
\textbf{Ejercicio 4}
\begin{enumerate}[a)]
\item Cuantitativa discreta, unidimensional, temporal.
\item Ordinal, unidimensional, atemporal.
\item Cuantitativa discreta, unidimensional, temporal.
\item Nominal, unidimensional, atemporal.
\item Cuantitativa continua, multidimensional (bidimensional), atemporal.
\item Cuantitativa continua, unidimensional, temporal.
\end{enumerate}


\end{document}
