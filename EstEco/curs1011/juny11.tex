\documentclass[a4paper,10pt]{article}
%\usepackage[active]{srcltx}      	%% necesario para pasar del dvi al tex%
\usepackage[spanish]{babel}
\usepackage[latin1]{inputenc}
%\usepackage{sw20res1}
\usepackage{amsmath}
\usepackage{amsfonts}
\usepackage{amssymb}
\usepackage{graphicx}
\usepackage{enumerate}
\setlength{\topmargin}{-2.5cm}	%%formato de pagina que ocupa todo
\setlength{\textwidth}{16cm}
\setlength{\textheight}{26cm}
\setlength{\oddsidemargin}{-1cm}

\pagestyle{empty}
\newcounter{prbcont}
\stepcounter{prbcont}
\setcounter{prbcont}{0}
\newtheorem{problema}[prbcont]{Problema}

\begin{document}

\begin{center}
\textbf{{\large {Examen de Estad\'{\i}stica Econ�mica. Juny 2011}}}

\vspace{0.3cm}

\textbf{Contesta a tres de los cuatro problemas propuestos.\\}
\textbf{Utiliza hojas separadas para problemas diferentes.\\}
\textbf{Todos los problemas punt\'uan lo mismo.\\}
\textbf{Se penalizar\'a a quien entregue los cuatro problemas.\\}
\vspace{0.3cm}
\end{center}


\begin{problema}
En una oficina trabajan 3 contables. 
El primero hace un $40 \%$ de las operaciones contables, 
pero se equivoca en un $5 \%$ de sus c\'alculos. El segundo 
hace un $35 \%$ de las operacions y se equivoca en un $3 \%$
de los c\'alculos. El �ltimo contable s�lo lleva un $25 \%$ de la
contabilidad, pero se equivoca muy poco (un $1 \%$ de las veces).

\begin{enumerate}[a)]
\item Identifica los sucesos m\'as relevantes y las probabilidades asignadas en el enunciado.
Distingue claramente las probabilidades condicionadas de las no condicionadas. 
\item Si un inspector revisa la contabilidad de la empresa y se fija en una
operaci\'on al azar,  ?`Cu\'al es la probabilidad de que esta operaci\'on sea err�nea? 
\item Si la operaci\'on es err�nea, ?`Cu\'al es la probabilidad de que la haya hecho
el primer contable? 
\item Si se revisan 10 operaciones, ?`Cu\'al es la probabilidad de que 3 o m�s sean err�neas? 
\end{enumerate}

\end{problema}


\vskip 0.5cm
\begin{problema}
Un restaurante de cocina ind\'u ofrece un servicio de entrega de men\'us
a domicilio. Por experiencia se sabe que el $80\%$ de los pedidos se entregan en menos de
una hora.
Si cada dia se sirven 15 men\'us a domicilio, se pide calcular:
\begin{enumerate}[a)]
\item La probabilidad de que en un dia se retrasen (se entreguen en 1 hora o m\'as) 4 pedidos. 
\item La probabilidad de que se retrasen 5 o m\'as pedidos.
\item La probabilidad de que se retrasen m\'as de 5 pedidos y menos de 10. 
\end{enumerate}

\noindent
Con el objeto de aumentar el volumen de negocio se lanza una promoci�n consistente en
regalar el men� siempre que el pedido tarde en entregarse una hora o m�s.
El precio de venta de un men\'u es de 
15 euros y los costes del servicio a domicilio constan de una parte fija de 60 euros al dia y
de 4 euros
por la elaboraci\'on de cada men\'u. Sabiendo que con la nueva promoci\'on se sirven 20 men\'us diarios:
\begin{enumerate}[d)]
\item Calcular el beneficio diario esperado. 
\item Calcular la probabilitat de que en un dia el beneficio sea de 120 euros o m\'as. \ \end{enumerate}
\end{problema} 

\vskip 0.5cm
\begin{problema}
El due\~no de una tienda de discos ha comprobado que el $20\%$ de los clientes que entran en su tienda
realizan una compra. Cierta ma\~nana entran en su tienda 180 personas, que pueden ser consideradas
como una muestra aleatoria de todos sus clientes.
\begin{enumerate}[a)]
\item ?`Cu\'al es la media de la proporci\'on muestral de clientes que realizan alguna compra?
\item ?`Cu\'al es la varianza de la proporci\'on muestral?
\item ?`Cu\'al es la probabilidad de que la proporci\'on muestral sea mayor que $0,15$?
\item ?`Cu\'al debe ser el tama\~no m\'inimo de la muestra para asegurarnos de que la desviaci\'on t\'ipica
de la proporci\'on muestral sea inferior a $0,02$?
\end{enumerate}
\end{problema}


\vskip 0.5cm
\begin{problema}
La cantidad de horas que duermen los estadounidenses cada noche var\'{\i}a mucho. Consideremos la siguiente muestra de las horas que duermen cada noche 16 personas.
\[
\begin{array}{ccccccc}
6.9	& &	7.6	& &	6.5	& &	6.2	\\
7.8	& &	7.0	& &	5.5	& &	7.6	\\
7.3	& &	6.6	& &	7.1	& &	6.9	\\
6.8	& &	6.5	& &	7.2	& &	5.8	\\

\end{array}
\]
\begin{itemize}
\item[(a)] Calcula una estimaci\'on puntual para la media de horas que se duerme cada noche y para la desviaci\'on t\'{\i}pica. ?`Qu\'e estimadores utilizas? ?`Por qu\'e?
\end{itemize}
Suponer ahora que la poblaci\'on sigue una distribuci\'on normal.
\begin{itemize}
\item[(b)] Determinar un intervalo de confianza del $80\%$ para la media de horas que se duerme cada noche.
\item[(c)] Determinar un intervalo de confianza del $95\%$ para la varianza.
\end{itemize}
\end{problema}


\newpage
\textbf{Variables aleatorias usuales}
\vskip 0.2 cm

\begin{tabular}{|c|cl|c|c|l|}
V.A. (X) & $f_X(x)$ & & $E(X)$ & $Var(X)$ & Otras propiedades \\
\hline
Binomial $B(n, p)$ & $\binom{n}{x} p^x (1-p)^{n-x}$ & si $x\in \Omega_X$ &
 $np$ & $np(1-p)$ & \\
$\Omega_X=\{ 0, 1, \cdots, n \}$ & $0$ & si $x \notin \Omega_X$ & & & \\ \hline
Poisson $Po(\lambda)$ & $\frac{\lambda^x}{x!} e^{-\lambda}$ & si $x\in \Omega_X$ & 
 $\lambda$ & $\lambda$ & \\
$\Omega_X=\{ 0, 1, \cdots \}$ & $0$ & si $x \notin \Omega_X$ & & & \\ \hline
Geom\'etrica $Ge(p)$ & $(1-p)^{x-1} p$ & si $x\in \Omega_X$
 & $\frac{1}{p}$ & $\frac{1-p}{p^2}$ & \\
$\Omega_X=\{ 1, 2, \cdots \}$ & $0$ & si $x \notin \Omega_X$ & & & \\ \hline
Geom\'etrica $Ge(p)$ & $(1-p)^x p$ & si $x\in \Omega_X$ & 
$\frac{1-p}{p}$ & $\frac{1-p}{p^2}$ &  
$F_X(x)=\begin{cases}
1-(1-p)^{k+1} & x \in [k, k+1), \\
 & k \in \Omega_X \\
0 & x < 0
\end{cases}$
\\
$\Omega_X=\{ 0, 1, \cdots \}$ & $0$ & si $x \notin \Omega_X$ & 
 & & 
\\ \hline
Uniforme ${\cal U}(a, b)$ & $\frac{1}{b-a}$ & si $x \in [a, b]$ & 
$\frac{b+a}{2}$ & $\frac{(b-a)^2}{12}$ & 
$F_X(x)=\begin{cases} 
\frac{x-a}{b-a} & x \in [a, b] \\
0 & x < a \\
1 & x > b
\end{cases}$ \\
$\Omega_X=[a, b]$ & 0 & si $x \notin [a, b]$ &  & & \\ \hline
Gaussiana $X(\mu, \sigma^2)$ & & & $\mu$ & $\sigma^2$ & $Z\sim N(0, 1)$ normal est\'andar \\
$\Omega_X=\mathbb{R}$ & & & &  & $F_Z(-z)=1-F_Z(z)$ \\
 & & & &  & $F_X(x)=F_Z(\frac{x-\mu}{\sigma})$ \\ \hline
\end{tabular}


\vskip 0.7 cm
\textbf{Estad\'{\i}sticos usuales}
\vskip 0.2 cm

\begin{tabular}{c|c|c|cl}
Par\'ametro  & Esperanza & Varianza & Distribuci\'on  & \\
muestral &  &  & de probabilidad & \\
(estad\'{\i}stico) & & & & \\ 
\hline
$\bar{X}$ & $E(\bar{X})=\mu$ & $\mathrm{Var}(\bar{X})=\frac{\sigma^2}{n}$ & 
$\bar{X} \sim N(\mu, \frac{\sigma^2}{n})$ & poblaci\'on normal, $\sigma$ conocido \\
& & & $\frac{\bar{X}-\mu}{\hat{s}_X / \sqrt{n}} \sim t_{n-1}$ & 
poblaci\'on normal, $\sigma$ desconocido, $n \leq 30$ \\
& & & 
$\bar{X} \sim N(\mu, \frac{\hat{s}_X^2}{n})$ & 
$\sigma$ desconocido, $n > 30$ \\
& & & & \\
$\hat{s}_X^2$ & $E(\hat{s}_X^2)=\sigma^2$ & $\mathrm{Var}(\hat{s}_X^2)=\frac{2\sigma^4}{n-1}$ & 
$\frac{n-1}{\sigma^2}\hat{s}_X^2 \sim \chi^2_{n-1}$ & poblaci\'on normal \\
& & & & \\
$\hat{p}_X$ & $E(\hat{p}_X)=p$ & $\mathrm{Var}(\hat{p}_X)=\frac{p(1-p)}{n}$ &
$\hat{p}_X \sim N(p, \frac{p(1-p)}{n})$ & $n > 30$ \\
 & & & $\hat{p}_X \sim t_{n-1}$ & poblaci\'on normal, $n \leq 30$ 
\end{tabular}

\vskip 0.7 cm
\textbf{Intervalos de confianza usuales}
\vskip 0.2 cm

\begin{tabular}{l|ll}
Par\'ametro muestral & Intervalo de confianza & \\
\hline
& & \\
Media & $\displaystyle \bar{X} \pm z_{\alpha/2} \frac{\sigma}{\sqrt{n}}$ & la poblaci\'on
sigue una normal y
$\sigma$ es conocido \\
& & \\
 & $\displaystyle \bar{X} \pm t_{n-1, \alpha/2} \frac{\hat{s}_X}{\sqrt{n}}$ & la poblaci\'on
sigue una normal, $\sigma$ desconocido  \\
& & y $n \leq 30$\\
& & \\
& $\displaystyle \bar{X} \pm z_{\alpha/2} \frac{\hat{s}_X}{\sqrt{n}}$ & si $n > 30$ \\
& & \\
& & \\
Varianza & $\displaystyle \left[ \frac{n-1}{\chi^2_{n-1, 1-\alpha/2}} \hat{s}_X^2,
 \frac{n-1}{\chi^2_{n-1, \alpha/2}} \hat{s}_X^2 \right]$ & la poblaci\'on sigue una normal 
\\
& & \\
& & \\
Proporci\'on & $\displaystyle \hat{p}_X \pm z_{\alpha/2} \sqrt{\frac{\hat{p}_X (1-\hat{p}_X)}{n}}$ &
si $n > 30$ \\
& & \\
& & 
\end{tabular}


\end{document}
