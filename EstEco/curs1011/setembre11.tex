\documentclass[a4paper,10pt]{article}
%\usepackage[active]{srcltx}      	%% necesario para pasar del dvi al tex%
\usepackage[spanish]{babel}
\usepackage[latin1]{inputenc}
%\usepackage{sw20res1}
\usepackage{amsmath}
\usepackage{amsfonts}
\usepackage{amssymb}
\usepackage{graphicx}
\usepackage{enumerate}
\setlength{\topmargin}{-2.5cm}	%%formato de pagina que ocupa todo
\setlength{\textwidth}{16cm}
\setlength{\textheight}{26cm}
\setlength{\oddsidemargin}{-1cm}

\pagestyle{empty}
\newcounter{prbcont}
\stepcounter{prbcont}
\setcounter{prbcont}{0}
\newtheorem{problema}[prbcont]{Problema}

\begin{document}

\begin{center}
\textbf{{\large {Examen de Estad\'{\i}stica Econ�mica. Septiembre 2011}}}

\vspace{0.3cm}

\textbf{Contesta a tres de los cuatro problemas propuestos.\\}
\textbf{Utiliza hojas separadas para problemas diferentes.\\}
\textbf{Todos los problemas punt\'uan lo mismo.\\}
\textbf{Se penalizar\'a a quien entregue los cuatro problemas.\\}
\vspace{0.3cm}
\end{center}


\begin{problema}
Una cadena de hamburgueser\'ias regala dos tipus de mu\~necos con su men\'u infantil: 
el osito Big King y el cerdito MacWhopper. Un $40\%$ de los men\'us vienen con el
primero de los mu\~necos y un $60\%$ con el segundo. Adem\'as, 7 de cada 15 ositos
y 9 de cada 20 cerditos traen como premio un helado.
\begin{enumerate}[a)]
\item Identifica los sucesos m\'as relevantes y las probabilidades asignadas en el enunciado.
Distingue claramente las probabilidades condicionadas de las no condicionadas.
\item ?`Cu\'al es la probabilidad de que al pedir un men\'u infantil nos toque como premio un helado?
\item Si nos toca un helado de premio, ?`cu\'al es la probabilidad de que el men\'u contenga el
cerdito MacWhopper?
\item Si pedimos 10 men\'us infantiles, ?`cu\'al es la probabilidad de que 3 o m\'as tengan premio?
\end{enumerate} 
\end{problema}



\vskip 0.5cm
\begin{problema}
Una empresa de transportes realiza el $85\%$ de las entregas urgentes en un plazo de 24h.
De un conjunto de 20 env\'\i os se desea calcular:
\begin{enumerate}[a)]
\item La probabilidad de que al menos 15 de ellos lleguen a su destino en un plazo de 24h; 
\item La probabilidad de que menos de 3 de ellos lleguen a su destino en m�s de 24h;
\item La probabilidad de que m\'as de 5 y menos de 9 env\'\i os lleguen a su destino en m�s de 24h;
\item El servicio de entrega urgente cuesta 30 euros por env\'\i o. En el caso de que un env\'\i o tarde
m\'as de 24h en entregarse la empresa se compromete a pagar una cantidad $Q$ al cliente en concepto de 
indemnizaci\'on. ?`Cu\'al debe ser el valor m\'aximo de $Q$ para que el servicio no genere p\'erdidas
(el valor esperado de las p\'erdidas sea cero)?
\end{enumerate}
\end{problema}
 

\vskip 0.5cm
\begin{problema}
El n\'umero de billetes de metro de la l\'\i nea Palma-UIB vendidos diariamente
sigue una distribuci\'on aproximadamente normal. Se toma una muestra aleatoria de
64 observaciones y el n\'umero medio de billetes vendidos es 100 i la desviaci\'on
t\'\i pica 25. Adem\'as, en 20 de estos 64 d\'\i as los trenes sufrieron retrasos de
m\'as de 10 minutos.

\noindent
Se pide:
\begin{enumerate}[a)]
\item Calcular un intervalo de confianza del $98\%$ para la media 
de los billetes vendidos diariamente. 
\item Calcular un intervalo de confianza del $98\%$ para la varianza 
de los billetes vendidos diariamente. 
\item Calcular un intervalo de confianza del $95\%$ para la probabilidad 
de que un dia cualquiera los trenes sufran retrasos de m\'as de 10 minutos.
\end{enumerate}
\end{problema}

\begin{problema}
Un profesor muy exigente afirma que aprueba a m\'as de un $30\%$ de sus alumnos
(es decir, suspenden menos del $70\%$). No obstante los alumnos sospechan que
el porcentaje de suspensos es mayor. Para verificar la validez de la afirmaci\'on
del profesor se toman los resultados de las 10 \'ultimas convocatorias del examen,
y las medias de suspensos (en $\%$) son:

\[  70 \qquad 67 \qquad 78 \qquad 80 \qquad 72 \qquad 69 \qquad 75 \qquad 71 \qquad 68 \qquad 77 \]

Se pide hacer un contraste unilateral de hip\'otesis para aceptar o rechazar 
la afirmaci\'on del profesor, para un nivel de significaci\'on del $1\%$.
Se deben definir, de manera razonada, las hip\'otesis nula y alternativa.
\end{problema}



\newpage
\textbf{Variables aleatorias usuales}
\vskip 0.2 cm

\begin{tabular}{|c|cl|c|c|l|}
V.A. (X) & $f_X(x)$ & & $E(X)$ & $Var(X)$ & Otras propiedades \\
\hline
Binomial $B(n, p)$ & $\binom{n}{x} p^x (1-p)^{n-x}$ & si $x\in \Omega_X$ &
 $np$ & $np(1-p)$ & \\
$\Omega_X=\{ 0, 1, \cdots, n \}$ & $0$ & si $x \notin \Omega_X$ & & & \\ \hline
Poisson $Po(\lambda)$ & $\frac{\lambda^x}{x!} e^{-\lambda}$ & si $x\in \Omega_X$ & 
 $\lambda$ & $\lambda$ & \\
$\Omega_X=\{ 0, 1, \cdots \}$ & $0$ & si $x \notin \Omega_X$ & & & \\ \hline
Geom\'etrica $Ge(p)$ & $(1-p)^{x-1} p$ & si $x\in \Omega_X$
 & $\frac{1}{p}$ & $\frac{1-p}{p^2}$ & \\
$\Omega_X=\{ 1, 2, \cdots \}$ & $0$ & si $x \notin \Omega_X$ & & & \\ \hline
Geom\'etrica $Ge(p)$ & $(1-p)^x p$ & si $x\in \Omega_X$ & 
$\frac{1-p}{p}$ & $\frac{1-p}{p^2}$ &  
$F_X(x)=\begin{cases}
1-(1-p)^{k+1} & x \in [k, k+1), \\
 & k \in \Omega_X \\
0 & x < 0
\end{cases}$
\\
$\Omega_X=\{ 0, 1, \cdots \}$ & $0$ & si $x \notin \Omega_X$ & 
 & & 
\\ \hline
Uniforme ${\cal U}(a, b)$ & $\frac{1}{b-a}$ & si $x \in [a, b]$ & 
$\frac{b+a}{2}$ & $\frac{(b-a)^2}{12}$ & 
$F_X(x)=\begin{cases} 
\frac{x-a}{b-a} & x \in [a, b] \\
0 & x < a \\
1 & x > b
\end{cases}$ \\
$\Omega_X=[a, b]$ & 0 & si $x \notin [a, b]$ &  & & \\ \hline
Gaussiana $X(\mu, \sigma^2)$ & & & $\mu$ & $\sigma^2$ & $Z\sim N(0, 1)$ normal est\'andar \\
$\Omega_X=\mathbb{R}$ & & & &  & $F_Z(-z)=1-F_Z(z)$ \\
 & & & &  & $F_X(x)=F_Z(\frac{x-\mu}{\sigma})$ \\ \hline
\end{tabular}


\vskip 0.7 cm
\textbf{Estad\'{\i}sticos usuales}
\vskip 0.2 cm

\begin{tabular}{c|c|c|cl}
Par\'ametro  & Esperanza & Varianza & Distribuci\'on  & \\
muestral &  &  & de probabilidad & \\
(estad\'{\i}stico) & & & & \\ 
\hline
$\bar{X}$ & $E(\bar{X})=\mu$ & $\mathrm{Var}(\bar{X})=\frac{\sigma^2}{n}$ & 
$\bar{X} \sim N(\mu, \frac{\sigma^2}{n})$ & poblaci\'on normal, $\sigma$ conocido \\
& & & $\frac{\bar{X}-\mu}{\hat{s}_X / \sqrt{n}} \sim t_{n-1}$ & 
poblaci\'on normal, $\sigma$ desconocido, $n \leq 30$ \\
& & & 
$\bar{X} \sim N(\mu, \frac{\hat{s}_X^2}{n})$ & 
$\sigma$ desconocido, $n > 30$ \\
& & & & \\
$\hat{s}_X^2$ & $E(\hat{s}_X^2)=\sigma^2$ & $\mathrm{Var}(\hat{s}_X^2)=\frac{2\sigma^4}{n-1}$ & 
$\frac{n-1}{\sigma^2}\hat{s}_X^2 \sim \chi^2_{n-1}$ & poblaci\'on normal \\
& & & & \\
$\hat{p}_X$ & $E(\hat{p}_X)=p$ & $\mathrm{Var}(\hat{p}_X)=\frac{p(1-p)}{n}$ &
$\hat{p}_X \sim N(p, \frac{p(1-p)}{n})$ & $n > 30$ \\
 & & & $\hat{p}_X \sim t_{n-1}$ & poblaci\'on normal, $n \leq 30$ 
\end{tabular}

\vskip 0.7 cm
\textbf{Intervalos de confianza usuales}
\vskip 0.2 cm

\begin{tabular}{l|ll}
Par\'ametro muestral & Intervalo de confianza & \\
\hline
& & \\
Media & $\displaystyle \bar{X} \pm z_{\alpha/2} \frac{\sigma}{\sqrt{n}}$ & la poblaci\'on
sigue una normal y
$\sigma$ es conocido \\
& & \\
 & $\displaystyle \bar{X} \pm t_{n-1, \alpha/2} \frac{\hat{s}_X}{\sqrt{n}}$ & la poblaci\'on
sigue una normal, $\sigma$ desconocido  \\
& & y $n \leq 30$\\
& & \\
& $\displaystyle \bar{X} \pm z_{\alpha/2} \frac{\hat{s}_X}{\sqrt{n}}$ & si $n > 30$ \\
& & \\
& & \\
Varianza & $\displaystyle \left[ \frac{n-1}{\chi^2_{n-1, 1-\alpha/2}} \hat{s}_X^2,
 \frac{n-1}{\chi^2_{n-1, \alpha/2}} \hat{s}_X^2 \right]$ & la poblaci\'on sigue una normal 
\\
& & \\
& & \\
Proporci\'on & $\displaystyle \hat{p}_X \pm z_{\alpha/2} \sqrt{\frac{\hat{p}_X (1-\hat{p}_X)}{n}}$ &
si $n > 30$ \\
& & \\
& & 
\end{tabular}


\end{document}
