\documentclass{report}
\usepackage[catalan]{babel}
\usepackage[latin1]{inputenc}   % Permet usar tots els accents i carï¿œters llatins de forma directa.
\usepackage{enumerate}
\usepackage{amsfonts, amscd, amsmath, amssymb}

\setlength{\textwidth}{16.5cm}
\setlength{\textheight}{27cm}
\setlength{\oddsidemargin}{-0.3cm}
\setlength{\evensidemargin}{0.25cm} \addtolength{\headheight}{\baselineskip}
\addtolength{\topmargin}{-3cm}

\newcommand\Z{\mathbb{Z}}
\newcommand\R{\mathbb{R}}
\newcommand\N{\mathbb{N}}
\newcommand\Q{\mathbb{Q}}
\newcommand\K{\Bbbk}
\newcommand\C{\mathbb{C}}

\begin{document}

\begin{center}
\textsc{Examen Fonaments Matem�tics II
Telem\`{a}tica\\
setembre 2011}
\end{center}

\vspace{0.5 cm}
\noindent
\textbf{�lGEBRA LINEAL}

\vspace{0.5 cm}
\noindent\textbf{P1.-}

\begin{enumerate}[a)]

\item
Calculau, sense aplicar la regla de Sarrus i sense desenvolupar pels elements
d'una fila o columna, el determinant seg�ent:

\[
\left| 
\begin{array}{cccc}
1 & 1 & 1 & \lambda \\
1 & 1 & \lambda & 1 \\
1 & \lambda & 1 & 1 \\
\lambda & 1 & 1 & 1
\end{array}
\right|
\]

\ \hfill{\textbf{ 3 pt.}}

\item 
Discutiu i resoleu el sistema:

\[
\left.
\begin{array}{ccccccc}
x & + & y & + & z & = & \lambda \\
x & + & y & + & \lambda z & = & 1 \\
x & + & \lambda y & + & z & = & 1 \\
\lambda x & + & y & + & z & = & 1 
\end{array}
\right\}
\]

\ \hfill{\textbf{ 7 pt.}}

\end{enumerate}

\vspace{0.5 cm}
\noindent\textbf{P2.-}
Considerau les seg�ents successions recurrents:

\[
\begin{array}{ccccccc}
a_n & = & 4a_{n-1} & - & 4b_{n-1} & + & 6c_{n-1} \\
b_n & = & 3a_{n-1} & - & 4b_{n-1} & + & 6c_{n-1} \\
c_n & = &  a_{n-1} & - & 2b_{n-1} & + & 3c_{n-1}
\end{array}
\]

\noindent
on $a_0=2$, $b_0=-2$ i $c_0=1$.

\begin{enumerate}[a)]
\item Trobau la matriu $A$ tal que 
$\left( \begin{array}{c} a_n \\ b_n \\ c_n \end{array} \right) = A \left( \begin{array}{c} a_{n-1} \\ b_{n-1} \\ c_{n-1} \end{array} \right)$

\ \hfill{\textbf{ 1 pt.}}

\item Calculau els valors propis de $A$.
\ \hfill{\textbf{ 2 pt.}}

\item Trobau els espais propis associats a cada valor propi.
\ \hfill{\textbf{ 3 pt.}}

\item Indicau si la matriu �s diagonalitzable i per qu�.
\ \hfill{\textbf{ 1 pt.}}

\item Calculau les sucessions $a_n$, $b_n$ i $c_n$.
\ \hfill{\textbf{ 2 pt.}}

\item Si $f: \R^3 \rightarrow \R^3$ �s una aplicaci� lineal tal que 
$f(x, y, z)=(x, y, -2x-4y-z)$ indicau si $(0, 0, \sqrt{5})$ �s
un vector propi. En cas afirmatiu indicau el valor propi corresponent.
\ \hfill{\textbf{ 1 pt.}}

\end{enumerate}

\vspace{0.5 cm}
\noindent\textbf{P3.-} Definim el seg�ent producte escalar sobre $\R^3$:
\[
\langle (x_1, x_2, x_3), (y_1, y_2, y_3) \rangle = 2x_1y_1+x_2y_2+2x_3y_3
\]

\begin{enumerate}[a)]
\item Demostrau que �s un producte escalar.
\ \hfill{\textbf{ 2 pt.}}

\item Trobau la matriu associada al producte escalar respecte a la base can�nica.
\ \hfill{\textbf{ 1 pt.}}

\item Sigui $S=\{ (x, y, z) \in \R^3 | x+y-3z=0 \}$. Trobau una base ortonormal de $S$.
\ \hfill{\textbf{ 2 pt.}}

\item Trobau una base de $S^{\perp}$.
\ \hfill{\textbf{ 2 pt.}}

\item Calculau la projecci� ortogonal sobre $S$ del vector $(-1, 3, 2)$.
\ \hfill{\textbf{ 2 pt.}}

\item Trobau l'angle que forma el vector $(-1, 3, 2)$ amb l'espai vectorial $S$.
\ \hfill{\textbf{ 1 pt.}}

\end{enumerate}

\vspace{0.5 cm}
\noindent
\textbf{PROBABILITAT}

\vspace{0.5 cm}
\noindent\textbf{P4.-}
Una empresa d'electr�nica A fabrica xips i se sap que un $5\%$ dels xips que fabrica
s�n defectuosos.

\begin{enumerate}
\item[a)] En una capsa tenim 200 xips fabricats per A. N'agafam 10.
Quina �s la probabilitat que 3 d'aquests xips siguin defectuosos?
\ \hfill{\textbf{ 2 pt.}}
\end{enumerate}

A m�s de A, hi ha dues empreses m�s, B i C, que tamb� fabriquen xips. 
Se sap que s�n defectuosos un $10\%$ dels que fabrica B i un $3\%$
dels de C. Una tenda d'electr�nica compra 1000 xips a A, 500 a B i 500 a C.
Es demana:

\begin{enumerate}
\item[b)] Del total de xips que compra la tenda agafam un a l'atzar.
Quina �s la probabilitat que sigui defectu�s?
\ \hfill{\textbf{ 2 pt.}}
\item[c)] Si el xip que hem agafat �s defectu�s, quina �s la probabilitat
que l'hagi fabricat C?
\ \hfill{\textbf{ 2 pt.}}
\item[d)] Quina �s la probabilitat que el xip sigui defectu�s i l'hagi
fabricat B?
\ \hfill{\textbf{ 2 pt.}}
\end{enumerate}

\noindent
(Plantejament i notaci�: \textbf{ 2 pt.})


\vspace{0.5 cm}
\noindent\textbf{P5.-}
Una empresa immobili\`aria confecciona una oferta de lloguer d'habitatges. 
L'empresa sap per experi\`encia que en la situaci\'o actual \'unicament el 40\% dels pisos es lloguen realment.
\begin{enumerate}[a)]
\item  Quina \'es la probabilitat que es lloguin exactament 6 pisos si la immobili\`aria posa 9 pisos en lloguer?
\ \hfill{\textbf{ 2 pt.}}
\item  Quina \'es la probabilitat que es lloguin com a m\`axim 6 pisos si la immobili\`aria posa 10 pisos en lloguer?
\ \hfill{\textbf{ 2 pt.}}
\item  Quina \'es la probabilitat que es lloguin entre 12 i 15 pisos (tots dos inclosos) si la immobili\`aria posa 20 pisos en lloguer?
\ \hfill{\textbf{ 2 pt.}}
\item Quin �s el nombre de pisos que espera llogar si posa 90 pisos en lloguer.
\ \hfill{\textbf{ 2 pt.}}
\end{enumerate}
(Plantejament i notaci�: \textbf{ 2 pt.})
%
%
%Una petita barca amb capacitat per a 15 persones (a part de la tripulaci\'o) fa viatges diaris al Parc Natural de Cabrera. El patr\'o de la barca sap que 1 de cada 5 reserves fallen, per la qual cosa cada dia f\`a 17 reserves.
%\begin{enumerate}[a)]
%\item Quina \'es la probabilitat que un dia qualsevol no hagi espai a la barca per dur a totes les persones que han fet una reserva (overbooking)? 
%\ \hfill{\textbf{ 2 pt.}}
%\item Quina \'es la probabilitat que un dia qualsevol no s'ompli la barca?
%\ \hfill{\textbf{ 2 pt.}}
%\item Quina \'es la probabilitat que un dia qualsevol viatgin en la barca entre 10 i 12 persones (incloses)?
%\ \hfill{\textbf{ 2pt.}}
%\item Quina ha de \'esser la capacitat m\'{\i}nima de la barca per a que la probabilitat d'overbooking sigui inferior al $3\%$. 
%\ \hfill{\textbf{ 2 pt.}}
%\end{enumerate}
%(Plantejament i notaci�: \textbf{ 2 pt.})


\vspace{0.5 cm}
\noindent\textbf{P6.-}
L'usuari d'un p�rquing aparca el seu cotxe durant un temps variable $X$
que es pot modelar amb una variable aleat�ria Gaussiana de mitjana $100$
i desviaci� t�pica $15$ (unitats en minuts).
El p�rquing li costa una quantitat fixa de 50 c�ntims d'euro m�s
1 c�ntim per minut aparcat.
\begin{enumerate}[a)]
\item Quina �s la probabilitat que el p�rquing li costi menys de 1,20 euros?
\ \hfill{\textbf{ 2 pt.}}

\item Quina �s la probabilitat que el p�rquing li costi m�s de 2 euros?
\ \hfill{\textbf{ 2 pt.}}

\item Quina �s la probabilitat que el cotxe estigui aparcat entre $60$ i $90$ minuts?
\ \hfill{\textbf{ 2 pt.}}

\item $75\%$ �s el percentatge de vegades que el cotxe estigui aparcat menys de quants minuts?
\ \hfill{\textbf{ 2 pt.}}

\end{enumerate}
(Plantejament i notaci�: \textbf{ 2 pt.})

\vskip 5cm
\hrule

\vspace{0.25 cm}

\noindent Duraci\'o de l'examen 4 hores.\newline

\end{document}
