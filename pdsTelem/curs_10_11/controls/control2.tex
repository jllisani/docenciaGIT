\documentclass[12pt]{report}
\usepackage[catalan]{babel}
\usepackage[latin1]{inputenc}   % Permet usar tots els accents i car�ters llatins de forma directa.
\usepackage{enumerate}
\usepackage{amsfonts, amscd, amsmath, amssymb}
\usepackage[pdftex]{graphicx}

\setlength{\textwidth}{16cm}
\setlength{\textheight}{24.5cm}
\setlength{\oddsidemargin}{-0.3cm}
\setlength{\evensidemargin}{0.25cm} \addtolength{\headheight}{\baselineskip}
\addtolength{\topmargin}{-3cm}

\newcommand\Z{\mathbb{Z}}
\newcommand\R{\mathbb{R}}
\newcommand\N{\mathbb{N}}
\newcommand\Q{\mathbb{Q}}
\newcommand\K{\Bbbk}
\newcommand\C{\mathbb{C}}

\newcounter{exctr}
\newenvironment{exemple}
{ \stepcounter{exctr} 
\hspace{0.2cm} 
\textit{Exemple  \arabic{exctr}: }
\it
\begin{quotation}
}{\end{quotation}}

\pagestyle{empty}

\begin{document}

\begin{center}
\textbf{\Large Processament Digital del Senyal.\\ Control 2. Curs 2010-11}
\end{center}

\vskip 1cm
\noindent
\textbf{P1.} Considerau el sistema LTI causal descrit per l'equaci� en 
difer�ncies finites seg�ent:
\[
y[n]=x[n]-x[n-1]+\frac{1}{4}x[n-2]-\frac{1}{2}y[n-1]
\]

\noindent
Responeu a les seg�ents q�estions utilitzant la transformada Z i les 
seves propietats:
\begin{enumerate}[a)]
\item Calculau la funci� de transfer�ncia del sistema $H(z)$.
\item Dibuixau el diagrama de zeros i pols i discutiu l'estabilitat del sistema.
\item Calculau la resposta impulsional $h[n]$ i la ROC de $H(z)$.
\item Calculau la resposta del sistema al senyal d'entrada seg�ent:
\[
x[n]=n^2 \left( \frac{1}{2} \right)^{n+1} u[n-1]
\]
 


\end{enumerate}







\end{document}