\documentclass[12pt]{report}
\usepackage[catalan]{babel}
%\usepackage[latin1]{inputenc}   % Permet usar tots els accents i car�ters llatins de forma directa.
\usepackage[utf8]{inputenc}  
\usepackage{enumerate}
\usepackage{amsfonts, amscd, amsmath, amssymb}
\usepackage[pdftex]{graphicx}

\setlength{\textwidth}{16cm}
\setlength{\textheight}{24.5cm}
\setlength{\oddsidemargin}{-0.3cm}
\setlength{\evensidemargin}{0.25cm} \addtolength{\headheight}{\baselineskip}
\addtolength{\topmargin}{-3cm}

\newcommand\Z{\mathbb{Z}}
\newcommand\R{\mathbb{R}}
\newcommand\N{\mathbb{N}}
\newcommand\Q{\mathbb{Q}}
\newcommand\K{\Bbbk}
\newcommand\C{\mathbb{C}}

\newcounter{exctr}
\newenvironment{exemple}
{ \stepcounter{exctr} 
\hspace{0.2cm} 
\textit{Exemple  \arabic{exctr}: }
\it
\begin{quotation}
}{\end{quotation}}

\pagestyle{empty}

\begin{document}

\begin{center}
\textbf{\Large Fonaments i Aplicacions del Processament Digital dels Senyals.\\ Control 2. Curs 2011-12}
\end{center}

\vskip 1cm
\noindent
\textbf{P1.} Un sistema LTI causal s'excita amb un senyal 
\[
x[n]=\frac{\sqrt{2}}{2} (\sqrt{2})^n \cos(\frac{\pi}{2}(n-1)) u[n-1]
\]
i s'obt\'e una sortida
\[
y[n]=n 4^{n+1} u[n]
\]

Es demana:
\begin{enumerate}[a)]
\item Calculau la transformada Z del senyal d'entrada.
\item Calculau la transformada Z del senyal de sortida.
\item Calculau la funci\'o de transfer\`encia del sistema.
\item Calculau la resposta impulsional del sistema.
\item Dibuixau la ROC de la funci� de transfer\`encia.
\item Dibuixau el diagrama de zeros i pols.
\item Discutiu l'estabilitat del sistema.
\end{enumerate}



\end{document}