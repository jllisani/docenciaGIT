\documentclass{article}
\usepackage[catalan]{babel}
%\usepackage[latin1]{inputenc}   % Permet usar tots els accents i car�ters llatins de forma directa.
\usepackage[utf8]{inputenc}   % Permet usar tots els accents i car�ters llatins de forma directa.
\usepackage{enumerate}
\usepackage{amsfonts, amscd, amsmath, amssymb}
\usepackage[pdftex]{graphicx}

\setlength{\textwidth}{16cm}
\setlength{\textheight}{24.5cm}
\setlength{\oddsidemargin}{-0.3cm}
\setlength{\evensidemargin}{0.25cm} \addtolength{\headheight}{\baselineskip}
\addtolength{\topmargin}{-3cm}

\newcommand\Z{\mathbb{Z}}
\newcommand\R{\mathbb{R}}
\newcommand\N{\mathbb{N}}
\newcommand\Q{\mathbb{Q}}
\newcommand\K{\Bbbk}
\newcommand\C{\mathbb{C}}

\newcounter{exctr}
\newenvironment{exemple}
{ \stepcounter{exctr} 
\hspace{0.2cm} 
\textit{Exemple  \arabic{exctr}: }
\it
\begin{quotation}
}{\end{quotation}}


\begin{document}

\textbf{\Large Tema 1. Introducci\'o al Processament Digital del Senyal.}

\vskip 0.2 cm
\noindent
\textbf{\large Senyals}

\vskip 0.2 cm
\noindent
Un \textbf{senyal} \'es el conjunt de valors num\`erics resultants de mesurar un 
fenomen f\'isic que varia amb el temps, l'espai o qualsevol altre par\`ametre.

\vskip 0.2 cm
\noindent
Exemples:
\begin{itemize}
\item La mesura del corrent el\`ectric que mostra un oscil.loscopi.
\item Les variacions de la pressi\'o de l'aire quan una persona parla.
\item Un electrocardiograma.
\item Una fotografia.
\end{itemize}

En aquest curs ens limitarem a estudiar els senyals que varien respecte a un \'unic par\`ametre,
que en general ser\`a el temps, per tant els senyals estaran formats per una 
\textit{seq\"u\`encia de valors}. 
En funci\'o de com siguin aquesta seq\"u\`encia i aquest conjunt de valors poder classificar els
senyals de la seg\"uent manera:

\vskip 0.5 cm

\begin{center}

\begin{tabular}{c|c|c|}
 & Seq\"u\`encia cont\'inua & Seq\"u\`encia discreta \\ & & \\ \hline  & & \\

\begin{tabular}{c}
Conjunt continu \\ \\
de valors 
\end{tabular}

&
  
\begin{tabular}{c}
Senyal Anal\`ogic 
\\ \\
\begin{minipage}{5cm}\includegraphics[width=5cm]{signalanalog.png}\end{minipage}
\end{tabular}

& 

\begin{tabular}{c}
Senyal en temps discret 
\\ \\
\begin{minipage}{5cm}\includegraphics[width=5cm]{signaldisctime.png}\end{minipage}
\end{tabular}

\\ & & \\ \hline & & \\

\begin{tabular}{c}
Conjunt discret \\ \\
de valors 
\end{tabular}

 &

\begin{minipage}{5cm}\includegraphics[width=5cm]{signaldiscvalues.png}\end{minipage}                 

& 

\begin{tabular}{c}
Senyal digital
\\ \\
\begin{minipage}{5cm}\includegraphics[width=5cm]{signaldigital.png}\end{minipage}
\end{tabular}

\\
\hline

\end{tabular}

\end{center}


\vskip 0.5 cm
\noindent
Per \textbf{tractament o processament de senyal} entenem qualsevol operaci\'o que permet generar, rebre,
enviar o modificar un senyal. Fins als anys 60 la majoria de senyals amb qu\`e treballaven els enginyers eren 
senyals anal\`ogics (per exemple senyals de radar) que es tractaven amb circuits electr\`onics anal\`ogics 
(basats en v\`alvules i transistors). Amb l'aparici\'o dels ordinadors i els circuits digitals es va passar 
a treballar amb senyals digitals, que s\'on els \'unics que poden tractar els ordinadors.
L'\'us de la tecnologia digital permet molta m\'es flexibilitat en el tractament de senyals.
El \textbf{processament digital de senyals} fa refer\`encia a les modificacions que afecten els senyals digitals.

\vskip 0.2 cm
\noindent
\textbf{Notaci\'o}

\vskip 0.2 cm
\noindent
Matem\`aticament els senyals es modelen com a funcions: $f: A \rightarrow B$, on els conjunts $A$ i $B$ 
depenen del tipus de senyal:

\vskip 0.3 cm

\begin{center}
\begin{tabular}{c|c|c|}
 & 
\begin{tabular}{c} Seq\"u\`encia cont\'inua \\ $A=\R$ \end{tabular}
 & 
\begin{tabular}{c} Seq\"u\`encia discreta \\ $A=\Z$ \end{tabular} 

\\ & & \\ \hline  & & \\

\begin{tabular}{c}
Conjunt continu \\ \\
de valors \\ \\
$B=\R$ o $B=\C$
\end{tabular}

& 

\begin{tabular}{c}
Senyal anal\`ogic \\ \\
Notaci\'o: $f(t)$ 
\end{tabular}

& 

\begin{tabular}{c}
Senyal en temps discret \\ \\
Notaci\'o: $f(nT)$ 
\end{tabular}

\\ & & \\ \hline  & & \\

\begin{tabular}{c}
Conjunt discret \\ \\
de valors \\ \\
$B=\{v_0, v_1, \cdots, v_k, \cdots \}$ \\ \\
on $v_i \in \R$ o $v_i \in \C$
\end{tabular}

& 


& 

\begin{tabular}{c}
Senyal digital \\ \\
Notaci\'o: $f[n]$ 
\end{tabular}

\\
\hline

\end{tabular}
\end{center}

\vskip 0.3 cm
\noindent
Si $B$ est\`a format per valors reals deim que el senyal \'es un \textbf{senyal real},
si est\`a format per valors complexes parlam de \textbf{senyal complexe}.


\vskip 0.3 cm
\noindent
\textbf{Conversi\'o anal\`ogica-digital}

\vskip 0.2 cm
\noindent
Un senyal anal\`ogic es pot transformar en digital mitjantzant la
\textbf{digitalitzaci\'o}, que compren un proc\'es de \textbf{mostreig}
i un proc\'es de \textbf{quantitzaci\'o} (veure figura \ref{conversioAD}). 
La conversi\'o inversa (de
digital a anal\`ogic) tamb\'e \'es possible i el senyal obtingut \'es
id\`entic a l'original sempre que la quantitzaci\'o i el mostreig 
verifiquin unes certes condicions.

\begin{figure}[htbp]
\centering
\includegraphics[width=14cm]{conversioAD.png} 
\caption{Conversi\'o anal\`ogica-digital}
\label{conversioAD}
\end{figure}

\vskip 0.3 cm
\noindent
\textbf{\large Senyals digitals}

\vskip 0.2 cm
\noindent
En aquest curs treballarem principalment amb senyals digitals, els quals es poden 
descriure mitjan\c{c}ant una seq\"u\`encia de nombres o mitjan\c{c}ant una f\'ormula.

\vskip 0.2 cm
\noindent
Exemples:
\begin{itemize}
\item $x[n]=\{ \cdots, 5, \underline{7}, 15, 11, \cdots \}$ (seq\"u\`encia infinita)
\item $x[n]=\{2, 9, \underline{-4}, -3, 5, 17 \}$  (seq\"u\`encia finita)
\item $x[n]=(-1)^n + 2$  (seq\"u\`encia infinita)
\end{itemize}

\vskip 0.2 cm
\noindent
\textbf{Alguns senyals importants}:
\begin{itemize}
\item \textbf{Delta de Dirac (impuls unitari)}: $\delta[n]=\begin{cases} 1 & \text{si } n=0 \\ \\ 0 & \text{altrament} \end{cases}$
\item \textbf{Esgla\'o unitari:} $u[n]=\begin{cases} 1 & \text{si } n \geq 0 \\ \\ 0 & \text{altrament} \end{cases}$
\item \textbf{Senyal sinuso\"idal}:
\[
x[n]=A \cos(\omega n + \theta) \qquad \quad  \text{o}  \qquad \quad x[n]=A \sin(\omega n + \theta)
\]

\noindent
$A$ s'anomena amplitud, $\omega$ \'es la freq\"u\`encia angular ($\omega=2\pi f$) i $\theta$ \'es
la fase del senyal.

\item \textbf{Senyal exponencial complexe}:
\[
x[n]=a^n , \qquad a \in \C
\]

\noindent
Observem que, donat que $a$ \'es un nombre complexe, llavors $a=r e^{i\theta}$ i per tant
\[
x[n]=a^n = r^n e^{i \theta n} = r^n (\cos(\theta n) + i \sin(\theta n)) = r^n \cos(\theta n) + i r^n \sin(\theta n)=x_R[n]+i x_I[n]
\]
\noindent
on $x_R$ i $x_I$ s\'on senyals sinuso\"idals.

\end{itemize}

\vskip 0.2 cm
\noindent
\textbf{Classificaci\'o dels senyals}. 

\vskip 0.2 cm
\noindent
Els senyals digitals es poden classificar seguint diversos
criteris:
\begin{itemize}
\item Senyals peri\`odics i no peri\`odics. El senyal $x[n]$ es diu \textbf{peri\`odic}
si verifica la seg\"uent condici\'o:
\[
x[n]=x[n+N] \qquad \forall n \in \Z \qquad \qquad \text{per a algun $N \in \Z$} 
\]
\item Senyals parells i senars.

\noindent
$x[n]$ \'es parell si $x[n]=x[-n]$

\noindent
$x[n]$ \'es senar o imparell si $x[n]=-x[-n]$

\vskip 0.2 cm
\noindent
Propietat: donat un senyal qualsevol $x[n]$, $x_P[n]=\frac{1}{2} (x[n]+x[-n])$ \'es
un senyal parell, $x_I[n]=\frac{1}{2} (x[n]-x[-n])$ \'es un senyal senar i, a m\'es,
$x[n]=x_P[n]+x_I[n]$.

\item Senyals d'energia i de pot\`encia.

\noindent
$x[n]$ \'es un \textbf{senyal d'energia finita} si 
\[
E=\sum_{-\infty}^{+\infty} |x[n]|^2 < \infty
\]

\noindent
$x[n]$ \'es un \textbf{senyal de pot\`encia finita} si 
\[
P=\lim_{N \rightarrow \infty} \frac{1}{2N+1} \sum_{-N}^{+N} |x[n]|^2 < \infty
\]

\item Senyals causals i anticausals.

\noindent
$x[n]$ \'es \textbf{causal} si $x[n]=0$ $\forall n < 0$

\noindent
$x[n]$ \'es \textbf{anticausal} si $x[n]=0$ $\forall n \geq 0$


\item Senyals deterministes i aleatoris. Els senyals \textbf{deterministes} s\'on aquells
els valors dels quals s\'on coneguts sense cap incertesa. 
Els senyals \textbf{aleatoris}, en canvi, evolucionen amb el temps d'una manera
impredictible. Aquest senyals es modelen com a \textbf{processos aleatoris}
que, en general, es consideren estacionaris i erg\`odics.
\end{itemize}


\newpage
\noindent
\textbf{Operacions b\`asiques amb senyals}.
\begin{itemize}
\item Reflexi\'o ($R$). $y[n]=R \, x[n] = x[-n]$ 

\item Translaci\'o ($T_k$). $y[n]=T_k \, x[n] = x[n-k]$

\item Delmaci\'o ($D_k$). $y[n]=D_k \, x[n]=x[k n]$

\item Producte per un escalar. $y[n]=k \cdot x[n]$

\item Suma de senyals. $y[n]=x_1[n] + x_2[n]$

\item Producte de senyals. $y[n]=x_1[n] \cdot x_2[n]$
 
\end{itemize}

\vskip 0.2 cm
\noindent
Aquestes operacions es poden representar gr\`aficament mitjan\c{c}ant \textbf{diagrames de blocs}:

\vskip 0.4 cm
\begin{center}
\begin{tabular}{cc}
Reflexi\'o & Delmaci\'o \\
\begin{minipage}{5cm}\includegraphics[width=5cm]{diagramareflexio.png}\end{minipage} & 
\begin{minipage}{5cm}\includegraphics[width=5cm]{diagramadelmacio.png}\end{minipage}
\end{tabular}
\end{center}

\vskip 0.4 cm
\begin{center}
\begin{tabular}{cc}
Translaci\'o unit\`aria a la dreta & Translaci\'o unit\`aria a l'esquerra  \\
 (retard unitari) & (avan\c{c} unitari) \\
\begin{minipage}{5cm}\includegraphics[width=5cm]{diagramaretard.png}\end{minipage} & 
\begin{minipage}{5cm}\includegraphics[width=5cm]{diagramaavanc.png}\end{minipage}
\end{tabular}
\end{center}

\vskip 0.4 cm
\begin{center}
\begin{tabular}{ccc}
Suma & Producte & Producte per escalar \\
\begin{minipage}{5cm}\includegraphics[width=5cm]{diagramasuma.png}\end{minipage} & 
\begin{minipage}{5cm}\includegraphics[width=5cm]{diagramaproducte.png}\end{minipage} & 
\begin{minipage}{5cm}\includegraphics[width=5cm]{diagramaproducteesc.png}\end{minipage} 
\end{tabular}
\end{center}

\vskip 0.3 cm
\noindent
\textbf{Descomposici\'o d'un senyal en deltes de Dirac}

\vskip 0.2 cm
\noindent
Qualsevol senyal $x[n]$ es pot escriure com una combinaci\'o de deltes
de Dirac mitjan\c{c}ant les operacions de suma, producte per escalars i
translaci\'o, segons la f\`ormula seg\"uent:
\[
x[n]=\sum_{k=-\infty}^{+\infty} x[k] \delta[n-k]
\]


\vskip 0.3 cm
\noindent
\textbf{Convoluci\'o de senyals} 

\vskip 0.2 cm
\noindent
La convoluci\'o \'es una operaci\'o entre dos senyals que es denota amb el s\'imbol $*$
i es defineix de la seg\"uent manera:

\[
y[n]=x_1[n] * x_2[n]=\sum_{k=-\infty}^{+\infty} x_1[k] x_2[n-k] 
\]

\noindent
Propietats de la convoluci\'o:
\begin{itemize}
\item $x_1[n] * x_2[n] = x_2[n] * x_1[n]$
\item $x_1[n] * (x_2[n] * x_3[n])=(x_1[n] * x_2[n]) * x_3[n]$
\item $x_1[n] * (x_2[n] + x_3[n]) = x_1[n] * x_2[n] + x_1[n] * x_3[n]$
\item Si $x_1[n]$ t\'e una durada $M_1$ i $x_2[n]$ una durada $M_2$, llavors
$x_1[n] * x_2[n]$ t\'e una durada $M_1+M_2-1$.
\end{itemize}

\vskip 0.2 cm
\noindent
\textbf{Observaci\'o}: $x[n]=x[n] * \delta[n]$


\end{document}