\documentclass[a4paper,12pt]{article}
\usepackage{a4wide}
\usepackage[latin1]{inputenc}
\usepackage[catalan]{babel}
%\usepackage{t1enc}
\usepackage{amssymb,amsthm,amsmath,xspace,color}
\usepackage[dvips]{graphicx}
\usepackage{enumerate}
%\usepackage{graphicx}
\let\phi\varphi
%\usepackage[dvips,matrix,curve,arrow,rotate]{xy}
%\CompileMatrices
%\SelectTips{cm}{12}
%\input defs.tex
\newcommand\Z{\mathbb{Z}}
\newcommand\R{\mathbb{R}}
\newcommand\N{\mathbb{N}}
\newcommand\Q{\mathbb{Q}}
\newcommand\K{\Bbbk}
\newcommand\C{\mathbb{C}}
\newcommand\cT{{\cal T}}
\newcommand\bR{\mathbb{R}}
\newcommand\ROC{\text{ROC}}
\newcommand\cZ{{\cal Z}}

\def\figura#1{\begin{figure}[htbp]\centering\input{#1}\end{figure}\par}
%
\title{Processament Digital del Senyal \\ Problemes Tema 5}
%\author{Gabriel Cardona Juanals}
\date{}
\def\fl{\text{fl}}
\def\pt{\text{pt}}
\def\fm{\text{fm}}

\setlength{\textheight}{24.5cm}
\addtolength{\topmargin}{-3cm}


\begin{document}

\maketitle{}

\begin{enumerate}
\item Determinau la magnitud i la fase  de $H(\omega)$ per al filtre seg�ent (filtre de mitjana de tres punts):
\[
y[n]=\frac{1}{3} (x[n+1]+x[n]+x[n-1])
\]
\noindent
De quin tipus de filtre es tracta (passa-alt, passa-baix, passa-banda, passa tot o banda eliminada)?
\item Considerau el sistema causal definit per la seg�ent equaci� en difer�ncies:
\[
y[n]=ay[n-1]+ (1-a) x[n] \qquad 0 < a < 1
\]

Determinau la sortida del sistema a l'entrada per al cas $a=0,9$.
\[
x[n]=5+12\sin(\frac{\pi}{2}n)-20\cos(\pi n + \frac{\pi}{4} )
\]

\item Per a cadascuna de les seg�ents respostes impulsionals, digueu
  si correspon a un filtre de fase lineal generalitzada. En tal cas,
  trobeu $H(\omega)$ i els par�metres $\alpha$ i $\beta$.
  \figura{probl4.exc1.pstex_t}
%  \newpage{}
\item S'interconnecten els sistemes donats per les respostes
  impulsionals $h_1$, $h_2$ i $h_3$ de les dues maneres que segueix:
  \figura{probl4.exc3.pstex_t}
  Les respostes impulsionals venen donades per:
  %\newpage{}
  \figura{probl4.exc4.pstex_t}
  Digueu si els sistemes resultants tenen o no fase lineal
  generalitzada.
\item Per a cadascun dels seg�ents diagrames de zeros i pols (i les
  ROC corresponents), digueu si s�n certes o falses:
  \begin{enumerate}
  \item El sistema t� fase zero o fase lineal generalitzada.
  \item El sistema t� invers estable.
  \end{enumerate}
  \figura{probl4.exc2.pstex_t}
%\item Digueu si �s cert o no que si la funci� de transfer�ncia $H(z)$
%  d'un sistema t� pols fora del zero (i l'infinit), aleshores no pot
%  ser de fase lineal generalitzada.
%\item Considereu la classe de filtres FIR reals amb fase lineal
%  generalitzada. Proveu que l'amplitud de la resposta en freq��ncia �s
%  com a la taula seg�ent:
%  \begin{center}
%    \begin{tabular}{lc}
%      Tipus & $A(\omega)$ \\\hline
%      I & $\displaystyle\sum_{n=0}^{M/2} a[n] \cos(\omega n)$ \\[10pt]
%      II & $\displaystyle\sum_{n=1}^{(M+1)/2} b[n] \cos(\omega (n-1/2))$\\[10pt]
%      III & $\displaystyle\sum_{n=1}^{M/2} c[n] \sin(\omega n)$ \\[10pt]
%      IV & $\displaystyle\sum_{n=1}^{(M+1)/2} b[n] \sin(\omega (n-1/2))$\\[10pt] \hline
%    \end{tabular}
%  \end{center}
%  Digueu en cada cas com es relaciona $a[n],b[n],c[n],d[n]$ amb la resposta
%  impulsional $h[n]$.
%\item Digueu, per a cada tipus de sistema FIR de fase generalitzada,
%  si pot correspondre a un filtre passa-baix,  passa-alt, passa banda
%  o banda eliminada.
\item Sigui $h_l[n]$ la resposta impulsional d'un sistema passa-baix
  FIR de fase 
  lineal generalitzada. Vegeu que el sistema definit per
  $h_h[n]=(-1)^nh_l[n]$ �s passa-alt, FIR i de fase lineal
  generalitzada. Si volem que $h_h$ sigui de tipus I o II (�s a dir,
  sim�tric) de quin tipus ha de ser $h_l$?
\item Un sistema causal t� per funci� de transfer�ncia:
  $$H(z)=\frac{(1-0.5z^{-1})(1+4z^{-2})}{(1-0.64z^{-2})}.$$
  \begin{enumerate}
  \item Trobeu l'expressi� per a un sistema de fase m�nima $H_{\fm,1}$ i
    un sistema passa tot $H_\pt$ de manera que la composici� d'aquests
    ens doni el sistema original.
  \item Trobeu  l'expressi� per a un sistema de fase m�nima $H_{\fm,2}$ i
    un sistema FIR de fase lineal generalitzada $H_\fl$ de manera que
    la composici� d'aquests 
    ens doni el sistema original.
  \end{enumerate}
\item 
  Sigui un sistema de fase m�nima, amb funci� de transfer�ncia
  $H_\fm(z)$ tal que
  $$H_\fm(z)H_\pt(z)=H_\fl(z),$$
  on $H_\pt$ i $H_\fl$ corresponen, respectivament a un sistema
  passa-tot i un sistema causal de fase lineal generalitzada. Qu� en
  podem dir dels zeros de $H_\fm$?
\item Un sistema FIR de fase lineal generalitzada t� resposta
  impulsional real $h[n]$ que s'anul�la per a $n<0$ i $n\ge 8$. Aquest
  sistema t� un zero a $z=0.8e^{j\pi/4}$ i un altre zero a
  $z=-2$. Trobeu $H(z)$.
\end{enumerate}
\end{document}
%%% Local Variables: 
%%% mode: latex
%%% TeX-master: t
%%% End: 
