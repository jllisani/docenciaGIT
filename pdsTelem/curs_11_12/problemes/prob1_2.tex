\documentclass[a4paper,12pt]{article}
%\usepackage[latin1]{inputenc}
\usepackage[utf8]{inputenc}
\usepackage[catalan]{babel}
\usepackage{amssymb,amsthm,amsmath,xspace,color}
\usepackage[dvips]{graphicx}
\let\phi\varphi
\usepackage[dvips,matrix,curve,arrow,rotate]{xy}
%\CompileMatrices
%\SelectTips{cm}{12}
%\input defs.tex

\newcommand\Z{\mathbb{Z}}
\newcommand\R{\mathbb{R}}
\newcommand\N{\mathbb{N}}
\newcommand\Q{\mathbb{Q}}
\newcommand\K{\Bbbk}
\newcommand\C{\mathbb{C}}
\newcommand\cT{{\cal T}}
\newcommand\bR{\mathbb{R}}

%
\title{Processament Digital del Senyal \\ Problemes Tema 1 i 2}
%\author{Gabriel Cardona Juanals}
\date{}

\begin{document}

\maketitle{}

\begin{enumerate}
\item Sigui $x$ el senyal donat per
  $$x[n]=\begin{cases}
    1+\frac n3,& \text{si $-3\le n\le -1$,}\\
    1 & \text{si $0\le n\le 3$,}\\
    0 & \text{altrament}.
  \end{cases}
  $$
  \begin{enumerate}
  \item Dibuixeu el senyal $x$.
  \item Dibuixeu el senyal resultant de:
    \begin{enumerate}
    \item Primer reflexar el senyal $x$ i després retardar-lo $4$
      mostres.
    \item Primer retardar-lo $4$ mostres i després reflexar-lo.
    \end{enumerate}
  \item Dibuixeu els senyals $x[-n-4]$ i $x[-n+4]$ i compareu-los amb
  els resultats anteriors.
  \end{enumerate}
\item Considereu el senyal donat per
  $$x=(1,\underline{1},1,1,\frac12,\frac12).$$
  Dibuixeu els senyals $x[n]$, $x[n-2]$, $x[4-n]$, $x[n+2]$,
  $x[n]u[2-n]$, $x[n-1]\delta[n-3]$, $x[n^2]$, la part parell de $x$,
  la part senar de $x$.
\item Demostreu que:
  \begin{enumerate}
  \item $\delta[n]=u[n]-u[n-1]$.
  \item $u[n]=\sum_{k=0}^\infty
    \delta[n-k]$. 
  \end{enumerate}
\item Demostreu que la descomposició d'un senyal en la seva part
  parell i la seva part senar és única.
\item Sigui $x$ un senyal que pren valors reals. Demostreu que la
  potència de $x$ és igual a la suma de la potència de la seva part
  parell i la potència de la seva part senar. Proveu el resultat
  anàleg per a l'energia.
\item Sigui $\cT$ un sistema lineal. Demostreu que si l'entrada al
  sistema és idènticament nul (totes les mostres iguals a $0$),
  aleshores el senyal de sortida és idènticament nul.
\item Pels sistemes següents, digueu si tenen les propietats
  d'estabilitat, causalitat, linealitat i invariança amb el temps.
  \begin{enumerate}
  \item $(\cT x)[n]=x[n]g[n]$, on $g:\Z\to\bR$ és una funció arbitrària.
  \item $(\cT x)[n]=\sum_{k=0}^n x[k]$.
  \item $(\cT x)[n]=\sum_{k=n-2}^{n+2} x[k]$.
  \item $(\cT x)[n]=x[n-n_0]$, amb $n_0\in\Z$ fixat.
  \item $(\cT x)[n]=\cos(x[n])$.
  \item $(\cT x)[n]=x[n]\cos(\omega_0n)$, amb $\omega_0$ fixat.
  \item $(\cT x)[n]=e^{x[n]}$.
  \item $(\cT x)[n]=ax[n]+b$, amb $a,b\in\bR$ fixats.
  \item $(\cT x)[n]=x[-n]$.
  \item $(\cT x)[n]=|x[n]|$.
  \item $(\cT x)[n]=\lfloor x[-n] \rfloor$, on $\lfloor t\rfloor$ és
    la part entera de $t$, és a dir, el major enter menor o igual que $t$.
  \item $(\cT x)[n]=x[n]+|x[n]|$.
  \end{enumerate}
\item Dos sistemes $\cT_1$ i $\cT_2$ es connecten en serie formant un
  nou sistema $\cT$. Digueu si les següents afirmacions són certes o
  falses; demostreu les certes i doneu contraexemples per a les
  falses.
  \begin{enumerate}
  \item Si $\cT_1$ i $\cT_2$ són lineals, aleshores $\cT$ és lineal.
  \item Si $\cT_1$ i $\cT_2$ són invariants amb el temps, aleshores
    $\cT$ és invariant amb el temps. 
  \item Si $\cT_1$ i $\cT_2$ són causals, aleshores $\cT$ és causal.
  \item Si $\cT_1$ i $\cT_2$ són LTI, aleshores $\cT$ és LTI.
  \item Si $\cT_1$ i $\cT_2$ són no lineals, aleshores $\cT$ és no lineal.
  \item Si $\cT_1$ i $\cT_2$ són estables, aleshores $\cT$ és estable.
  \end{enumerate}
  Digueu si afecta l'ordre en que s'interconnecten els dos sistemes
  per al cas que els sistemes $\cT_1$ i $\cT_2$ siguin lineals,
  invariants amb el temps i LTI.
\item Per als parells de senyals que segueixen, calculeu-ne la
  convolució:
  \begin{enumerate}
  \item $h[n]=(\underline{0},1)$, $x[n]=(\underline{2},1)$.
  \item $h[n]=(\underline{2},-1)$, $x[n]=(\underline{-1},2,1)$.
  \item $h[n]=(\underline{1},1,1,1,1)$,
    $x[n]=(\underline{0},0,1,1,1,1,1,1,0,0,0,1,1,1,1,1,1)$.
  \item $h[n]=(1,2,\underline{1},1)$, $x[n]=(\underline{1},-1,0,0,1,1)$.
  \end{enumerate}
\item Calculeu la convolució dels senyals $h[n]=a^n u[n]$ i $x[n]=b^n
  u[n]$; distingiu els casos $a=b$ i $a\neq b$.
\item Sigui $\cT$ un sistema LTI estable. Demostreu les següents
  afirmacions. 
  \begin{enumerate}
  \item Si el senyal d'entrada és periòdic de periode $T$, aleshores
    el senyal de sortida és periòdic de periode $T$.
  \item Si el senyal d'entrada és fitat i tendeix a una constant, és a
    dir, $\lim_{n\to\infty}x[n]=a$, aleshores el senyal de sortida és
    fitat i tendeix a una constant. Calculeu explícitament aquesta
    constant. 
  \end{enumerate}
\item Sigui $\cT$ un sistema LTI amb resposta impulsional donada per
  $$h[n]=\begin{cases}
    a^n, & \text{si $n\ge 0$,}\\
    0,&\text{altrament.} 
  \end{cases}
  $$
  Determineu la sortida quan s'excita amb el graó unitat
  $u[n]$. Estudieu l'estabilitat del sistema en funció del paràmetre
  $a$.
\item Sabem que la resposta impulsional d'un sistema s'anul·la per a
  $n<n_1$ i per a $n>n_2$. Si s'excita amb un senyal que s'anul·la per
  a $n<n_3$ i per a $n>n_4$, demostreu que existeixen constants $n_5$
  i $n_6$ de manera que la sortida s'anul·la per a $n<n_5$ i per a
  $n>n_6$. Determineu $n_5$ i $n_6$ en funció de $n_1,n_2,n_3,n_4$.
\item Considereu el sistema de la figura:



  \begin{figure*}[htbp]
    \centering
    \input{probl1.exc1.pstex_t}
  \end{figure*}
  \begin{enumerate}
  \item Doneu la resposta impulsional del sistema en funció de
    $h_1,h_2,h_3,h_4$.
  \item Feu el cas particular
    $h_1[n]=(\underline{\frac12},\frac14,\frac12)$,
    $h_2[n]=h_3[n]=(n+1)u[n]$, $h_4[n]=\delta[n-2]$.
  \item Trobeu la resposta del sistema de l'apartat anterior quan
    s'excita amb el senyal $x[n]=\delta[n+2]+3\delta[n-1]-4\delta[n-3]$. 
  \end{enumerate}

\item Dibuixeu la forma directa i la forma canònica dels sistemes
  següents donats per equacions en diferències finites.
  \begin{enumerate}
  \item $2y[n]+y[n-1]-4y[n-3]=x[n]+3x[n-5]$.
  \item $y[n]=y[n-1]+x[n]-x[n-1]+2x[n-2]-3x[n-4]$.
  \end{enumerate}

\end{enumerate}

%\bibliography{genere2}
%\bibliographystyle{alpha}

\end{document}

%%% Local Variables: 
%%% mode: latex
%%% TeX-master: t
%%% End: 
