\documentclass[a4paper,12pt]{article}
%\usepackage[latin1]{inputenc}
\usepackage[utf8]{inputenc}
\usepackage[catalan]{babel}
%\usepackage{t1enc}
\usepackage{amssymb,amsthm,amsmath,xspace,color}
\usepackage[dvips]{graphicx}
\let\phi\varphi
\usepackage[dvips,matrix,curve,arrow,rotate]{xy}
%\CompileMatrices
%\SelectTips{cm}{12}
%\input defs.tex
%
\newcommand\Z{\mathbb{Z}}
\newcommand\R{\mathbb{R}}
\newcommand\N{\mathbb{N}}
\newcommand\Q{\mathbb{Q}}
\newcommand\K{\Bbbk}
\newcommand\C{\mathbb{C}}
\newcommand\cT{{\cal T}}
\newcommand\bR{\mathbb{R}}
\newcommand\ROC{\text{ROC}}
\newcommand\cZ{{\cal Z}}

\title{Processament Digital del Senyal \\ Problemes Tema 3}
%\author{Gabriel Cardona Juanals}
\date{}

\begin{document}

\maketitle{}

\begin{enumerate}
\item Determineu la transformada $\cZ$ i la ROC dels senyals següents:
  \begin{enumerate}
  \item $x[n]=(-2,1,0,3,\underline{0})$.
  \item $x[n]=(-2,1,0,3,\underline{3})$.
  \item $x[n]=(-2,1,\underline0,3)$.
  \item $x[n]=(\underline{3},1,-1)$.
  \end{enumerate}
\item Determineu la transformada $\cZ$ i la ROC dels senyals següents:
  \begin{enumerate}
  \item $x[n]=u[n]$.
  \item $x[n]=(1+n)u[n]$.
  \item $x[n]=(a^n+a^{-n})u[n]$.
  \item $x[n]=a^nu[n]-a^{-n}u[-n-1]$.
  \item $x[n]=\cos(\omega n) u[n]$.
  \item $x[n]=n\cos(\omega n) u[n]$.
  \item $x[n]=a^n\cos(\omega n)u[n]$.
  \item $x[n]=na^n\cos(\omega n)u[n]$.
  \item $x[n]=(n^2+n)a^{n-1}u[n-1]$.
  \end{enumerate}
  On $a$ i $\omega$ són paràmetres arbitraris.
\item Determineu la transformada $\cZ$ i la ROC dels senyals següents:
  \begin{enumerate}
  \item $x[n]=\begin{cases}
      \left(\frac13\right)^n, & \text{si $n\ge 0$} \\
      \left(\frac12\right)^{-n}, & \text{si $n<0$}
    \end{cases} $.
  \item $x[n]=\left(\frac13\right)^nu[n]+\left(\frac12\right)^{-n}u[-n-1]$.
  \end{enumerate}
\item Tenim uns senyals amb les propietats:
  \begin{itemize}
  \item $x_1$ és de durada finita i causal.
  \item $x_2$ és de durada finita i anticausal.
  \item $x_3$ és de durada finita i no és ni causal ni anticausal.
  \item $x_4$ és de durada infinita i causal.
  \item $x_5$ és de durada infinita i anticausal.
  \item $x_6$ és de durada infinita i no és ni causal ni anticausal.
  \end{itemize}
  Digueu quina d'aquestes ROC correspon a cada cas: (noteu que el
  quadrat exterior simbolitza l'infinit)

  \begin{figure*}[htbp]
    \centering
    \input{prob2.exc1.pstex_t}
    
  \end{figure*}
\item Trobeu la transformada $\cZ$ dels senyals
  \begin{enumerate}
  \item $y_1[n]=\sum_{k=-\infty}^n x[k]$.
  \item $y_2[n]=x[2n]$.
  \item $y_3[n]=\begin{cases} x[n/2] & \text{si $n$ és parell} \\
      0 &\text{altrament}
    \end{cases}$
  \end{enumerate}
  en funció de la transformada del senyal $x$.
\item Calculeu els senyals causals que tenen per transformada $\cZ$
  els següents:
  \begin{enumerate}
  \item $X(z)=\dfrac{1+3z^{-1}}{1+3z^{-1}+2z^{-2}}$.
  \item $X(z)=\dfrac{1}{1-z^{-1}+\frac12 z^{-2}}$.
  \item $X(z)=\dfrac{z^{-6}+z^{-7}}{1-z^{-1}}$.
  \item $X(z)=\dfrac{1+2z^{-2}}{1+z^{-2}}$.
  \end{enumerate}
\item D'un senyal tant sols coneixem l'expresió tancada de la
  transformada $\cZ$:
  $$X(z)=\frac{5z^{-1}}{(1-2z^{-1})(3-z^{-1})}.$$
  Digueu quantes possibles ROC diferents pot tenir i, per cadascuna
  d'aquestes, trobeu el senyal corresponent.
\item Sigui $x[n]$ un senyal amb transformada $X(z)$. Trobeu la
  transformada dels senyals:
  \begin{enumerate}
  \item $x^*[n]$, on $x^*$ és el complexe conjugat de $x$.
  \item $\Re(x[n])$.
  \item $\Im(x[n])$.
  \end{enumerate}
  Deduiu propietats de simetria per a senyals reals i imaginaris purs.
\item Un sistema LTI s'excita amb el senyal graó unitat $x[n]=u[n]$;
  la sortida que s'observa és $y[n]=(\frac12)^{n+1}u[n-1]$.
  \begin{enumerate}
  \item Trobeu la transformada de la resposta impulsional del sistema
    $H(z)$.
  \item Trobeu la resposta impulsional del sistema.
  \end{enumerate}
\item L'entrada a un sistema LTI és
  $x[n]=u[-n-1]+(\frac12)^nu[n]$. De la sortida tant sols coneixem la
  expresió tancada per a la transformada,
  $$Y(z)=\frac{-\frac12 z^{-1}}{(1-z^{-1})(1+z^{-1})},$$
  mentre que desconeixem la seva ROC.
  \begin{enumerate}
  \item Determineu la ROC de $Y(z)$.
  \item Determineu la resposta impulsional del sistema.
  \end{enumerate}
  \newpage{}
\item Sigui $x[n]$ un senyal causal.
  \begin{enumerate}
  \item Calculeu $\lim_{z\to\infty}X(z)$.
  \item Demostreu que si $X(z)$ ve donat per un quocient de polinomis,
    $X(z)=P(z)/Q(z)$, aleshores $\deg P\le \deg Q$.
  \end{enumerate}
\item Sigui $x[n]$ un senyal causal tal que la seva transformada $\cZ$
  té com a zeros $z_1=-1$ i  com a pols $p_1=\frac12$, $p_2=-\frac12$,
  $p_3=p_4^*=\frac12+j\frac12$.
  Determineu el diagrama de zeros i pols i la ROC dels senyals
  $x_1[n]=x[n-2]$, $x_2[n]=x[-n-2]$ i $x_3[n]=e^{jn\pi/3}x[n]$.
\item Disenyeu un sistema LTI causal tal que quan s'excita amb el
  senyal
  $$x[n]=\left(\frac12\right)^n
  u[n]-\frac14\left(\frac12\right)^{n-1}u[n-1]$$
  la sortida és
  $$y[n]=\left(\frac13\right)^nu[n].$$
  Seguiu els pasos:
  \begin{enumerate}
  \item Trobeu la transformada de la resposta impulsional necessària.
  \item Trobeu aquesta resposta impulsional.
  \item Trobeu una equació en diferències finites que implementi el
    sistema.
  \item Trobeu la realització canònica del sistema.
  \item Discutiu l'estabilitat del sistema.
  \end{enumerate}
\item Sigui $\cT$ un sistema causal caracteritzat per l'equació en
  diferències finites:
  $$y[n]+a_1y[n-1]+a_2y[n-2]=x[n].$$
  Feu servir la caracterització d'estabilitat d'un sistema causal en
  funció dels pols de $H(z)$ per trobar les condicions que han de
  complir els paràmetres $a_1$ i $a_2$ per tal que el sistema sigui
  estable.
\item Un sistema té 3 pols situats a $z=-3$, $-0.5$ i $2$ i un zero
  situat a $z=1$. 
  \begin{enumerate}
  \item Suposant que el sistema és estable, trobeu-ne la ROC.
  \item Suposant que el sistema és causal, trobeu-ne la ROC.
  \item És possible que un sistema estable i causal tingui aquesta
    configuració de zeros i pols?
  \end{enumerate}
  \newpage{}
\item D'un sistema LTI causal en coneixem el diagrama de zeros i pols,
on $r=1.5$ i $\theta=\pi/6$.:
  \begin{figure*}[htbp]
    \centering
    \input{prob2.exc3.pstex_t}
  \end{figure*}
  
  \begin{enumerate}
  \item Trobeu la funció de transferència sabent que $H(1)=1$.
  \item Trobeu la resposta impulsional.
  \item Discutiu l'estabilitat del sistema.
  \end{enumerate}
\item Demostreu que si $x[n]$ és un senyal causal, aleshores
  $$\lim_{z\to\infty}X(z)=x[0].$$
  Doneu un resultat d'aquest estil per a senyals anticausals.
\end{enumerate}

%\bibliography{genere2}
%\bibliographystyle{alpha}

\end{document}

%%% Local Variables: 
%%% mode: latex
%%% TeX-master: t
%%% End: 
