\documentclass[a4paper,12pt]{article}
\usepackage{a4wide}
%\usepackage[latin1]{inputenc}
\usepackage[utf8]{inputenc}
\usepackage[catalan]{babel}
%\usepackage{t1enc}
\usepackage{amssymb,amsthm,amsmath,xspace,color}
\usepackage[dvips]{graphicx}
%\usepackage{graphicx}
\let\phi\varphi
%\usepackage[dvips,matrix,curve,arrow,rotate]{xy}
%\CompileMatrices
%\SelectTips{cm}{12}
%\input defs.tex
\newcommand\Z{\mathbb{Z}}
\newcommand\R{\mathbb{R}}
\newcommand\N{\mathbb{N}}
\newcommand\Q{\mathbb{Q}}
\newcommand\K{\Bbbk}
\newcommand\C{\mathbb{C}}
\newcommand\cT{{\cal T}}
\newcommand\bR{\mathbb{R}}
\newcommand\ROC{\text{ROC}}
\newcommand\cZ{{\cal Z}}

\def\figura#1{\begin{figure}[htbp]\centering\input{#1}\end{figure}\par}
%
\title{Processament Digital del Senyal \\ Problemes Tema 4}
%\author{Gabriel Cardona Juanals}
\date{}

\begin{document}

\maketitle{}

\begin{enumerate}
\item Calculeu la transformada de Fourier dels senyal següents:
  \begin{enumerate}
  \item $x[n]=u[n]-u[n-6]$.
  \item $x[n]=2^nu[-n]$.
  \item $x[n]=(\alpha^n\sin(\omega_0n))u[n]$ amb $|\alpha|<1$.
  \item $x[n]=|\alpha|^n\sin(\omega_0n)$ amb $|\alpha|<1$.
  \end{enumerate}
\item Determineu els senyals que tenen com a transformades de Fourier
  les següents (pels senyals definits a intervals finits, s'entén que
  s'extenen per periodicitat):
  \begin{enumerate}
  \item $X(\omega)=\begin{cases}
      0,& \text{si $|\omega|<\omega_0$,}\\
      1,& \text{si $\omega_0 \leq |\omega|<\pi$.}
    \end{cases}$
  \item $X(\omega)=\cos^2(\omega)$.
  \item $X(\omega)=\begin{cases}
      0,& \text{si $|\omega|<\omega_1$,}\\
      1,& \text{si $\omega_1 \leq | \omega|\leq \omega_2$,}\\
      0,& \text{si $|\omega|>\omega_2$.}
    \end{cases}$
  \item
    $X(\omega)=\Pi\left(\dfrac\omega{\pi/8}\right)+
    \Pi\left(\dfrac\omega{3\pi/8}\right).$ $\qquad \qquad$ on $\qquad$
    $\Pi(\omega)=\begin{cases} 1 & \text{si } |\omega| \leq 1 \\ 0 & \text{altrament} \end{cases}$
  \end{enumerate}
\item Sigui $x[n]$ un senyal amb transformada $X(\omega)$. Expresseu
  en termes de $X(\omega)$ les transformades dels senyals:
  \begin{enumerate}
  \item $x^*[n]$.
  \item $x^*[-n]$.
  \item $y[n]=x[n]-x[n-1]$.
  \item $y[n]=\sum_{k=-\infty}^n x[k]$.
  \item $x[2n+1]$.
  \item $x[-2n]$.
  \item $x[n]*x[n-1]$.
  \item $e^{j \pi n/2}x[n+2]$.
  \item $x[n]\cos(3\pi n/10)$.
  \item $x[n]*x[-n]$.
  \end{enumerate}
\item Siguin $x[n]$ i $y[n]$ senyals discrets amb transformada de
  Fourier $X(\omega)$ i $Y(\omega)$, respectivament.
  \begin{enumerate}
  \item Trobeu un senyal en funció de $x$ i $y$ que tingui per
    transformada el producte $X(\omega)Y^*(\omega)$.
  \item Proveu que
    $$\sum_{n=-\infty}^\infty x[n]y^*[n]=\frac1{2\pi}\int_{-\pi}^\pi
    X(\omega)Y^*(\omega)\,d\omega.$$
  \item Determineu el valor de la suma
    $$\sum_{n=-\infty}^\infty \frac{\sin(\pi n/4)}{2\pi
      n}\frac{\sin(\pi n/6)}{5\pi n}.$$
  \end{enumerate}
\item Sigui $X(\omega)$ la transformada de Fourier del senyal $x[n]$
  de la figura.
  \figura{probl3.exc1.pstex_t}
  Resoleu les següents qüestions sense calcular explícitament
  $X(\omega)$. 
  \begin{enumerate}
  \item Trobeu el valor que pren $X(\omega)$ per $\omega=0$.
  \item Calculeu $\arg X(\omega)$.
  \item Avalueu $\int_{-\pi}^\pi X(\omega)\,d\omega$.
  \item Trobeu i dibueixeu el senyal que té per transformada de
    Fourier $\Re(X(\omega))$.
  \end{enumerate}
\item Sigui $x[n]$ un senyal amb transformada $X(\omega)$ i fixem $N$
  un enter positiu. Trobeu la
  transformada dels senyals següents en funció de $X(\omega)$:
  \begin{enumerate}
  \item
    $$y[n]=\begin{cases}
      x[n] & \text{si $N$ divideix $n$,} \\
      0 & \text{altrament.}
    \end{cases}
    $$
  \item
    $$y[n]=x[Nn].$$
  \item
    $$y[n]=\begin{cases}
      x[n/N] & \text{si $N$ divideix $n$,}\\
      0 & \text{altrament.}
    \end{cases}
    $$
  \end{enumerate}
  Dibueixeu aquestes transformades per al cas que $X(\omega)$ és com a la
  figura:
  \figura{probl3.exc2.pstex_t}
\item Un senyal discret
  $$x[n]=\frac{1}{T} \cos\left(\frac\pi4 n\right)$$
  s'ha obtingut per mostreig d'un senyal analògic
  $$x_a(t)=\cos(\Omega_0 t)$$
  amb taxa de mostreig de 1000 mostres per segon ($T$ denota el periode de mostreig). Determineu els
  possibles valors de $\Omega_0$.
\item Pel sistema de la figura, doneu-ne la sortida si l'entrada és
  $\delta[n]$ i el sistema $\cT$ és un sistema pas baix ideal amb
  funció de transferència definida a l'interval $[-\pi,\pi]$ per
  $$H(\omega)=\begin{cases}
    1 & \text{si $|\omega|<\pi/2$,} \\
    0 & \text{altrament,}
  \end{cases}$$
  i extesa per periodicitat a tot $\bR$.
  \figura{probl3.exc3.pstex_t}
\item Considereu el següent sistema on es representa un procés de
  mostreig seguit d'un sistema discret.
  \figura{probl3.exc4.pstex_t}
  Es té, a més, que
  \begin{align*}
    X_a(\Omega)&=0,\quad\text{si $\Omega\ge 2\pi\times10^4$,}\\
    x[n]&=x_a(nT)\\
    y[n]&=T\sum_{k=-\infty}^n x[k]
  \end{align*}
  \begin{enumerate}
  \item Doneu el màxim $T$ possible si volem evitar l'aliasing.
  \item Determineu $h[n]$.
  \item Doneu $\lim_{n\to\infty}y[n]$ en termes de $X(\omega)$.
  \end{enumerate}
\item Un senyal analògic complex té la transformada de Fourier de la figura.
  \figura{probl3.exc5.pstex_t}
  Aquest senyal es mostreja per obtenir $x[n]=x_a(nT)$.
  \begin{enumerate}
  \item Dibuixeu $X(\omega)$ per a $T=\pi/\Omega_2$ i $T=2\pi/\Omega_2$.
  \item Quin és el màxim $T$ que es pot fer servir a l'hora de
    mostrejar de manera que no existeixi el fenòmen d'aliasing? (és a
    dir, de manera que $x_a(t)$ es pugui recuperar a partir de
    $x[n]$).
  \item Repetiu els apartats anteriors pel senyal $x'_a$ amb
    $X'_a(\Omega)=X_a(\Omega)+X_a(-\Omega).$
  \end{enumerate}
\item Un senyal analògic $x_a(t)$ que no té components freqüencials per a
  $\Omega>\pi/T_1$ es mostreja amb periode $T_1$ resultant un senyal
  discret $x[n]$. Amb $x[n]$ es reconstrueix un senyal analògic
  $x'_a(t)$ fent
  servir un periode 
  entre mostres $T_2$. Pel cas general que $T_1\neq T_2$, expreseu
  $x'_a$ en funció de $x_a$. Hi ha diferències essencials entre els
  casos $T_1<T_2$ i $T_1>T_2$?
\item Considereu el sistema de la figura i
%  \figura{probl3.exc6.pstex_t}
  dibuixeu $X'_a(\Omega)$ pels casos següents:
  \begin{enumerate}
  \item $1/T_1=1/T_2=10^4$.
  \item $1/T_1=1/T_2=2\cdot10^4$.
  \item $1/T_1=2\cdot10^4$, $1/T_2=10^4$.
  \item $1/T_1=10^4$, $1/T_2=2\cdot10^4$.
  \end{enumerate}
  \figura{probl3.exc6.pstex_t}
%  \newpage{}
\item Considereu el sistema següent, que representa un sistema complet
  de processament digital de senyal analògic:
  \figura{probl3.exc7.pstex_t}
  On els blocs T$\to$D i D$\to$T simbolitzen, respectivament, el
  conversor de tren de deltas a discret i viceversa. Els blocs
  venen donats per
  \begin{align*}
    H(\omega)&=\Pi\left(\frac\omega{\pi/4}\right)\\
    H_r(\Omega)&=T\Pi\left(\frac\Omega{\pi/T}\right)
  \end{align*}
  mentre que el senyal d'entrada té transformada
  $$X_a(\Omega)=\Lambda\left(\frac\Omega{2\pi\cdot10^4}\right)$$
  \begin{enumerate}
  \item Suposant que $1/T=20\,\rm{kHz}$, dibuixeu $X_s(\Omega)$ i
    $X(\omega)$. 
  \item Determineu el conjunt de valors possibles de $T$ de manera que
    el sistema global és equivalent a un sistema pas baix ideal amb
    funció de transferència
    $$H_{eq}(\Omega)=\Pi\left(\frac\Omega{\Omega_c}\right),$$
    amb $\Omega_c$ a determinar.
  \item Pel conjunt trobat a l'apartat anterior, establiu la relació
    entre $\Omega_c$  i $1/T$.
  \end{enumerate}
  
  Nota:
  \begin{align*}
  \Pi(t)&=\begin{cases}
    1, &\text{si $|t|<1$,}\\  0,&\text{altrament,} 
  \end{cases},\\
   \Lambda(t)&=\begin{cases}
    1-|t|, &\text{si $|t|<1$,}\\  0,&\text{altrament,} 
  \end{cases},
\end{align*}
%\newpage{}
\item Un model de sistema digital de processament de senyal analògic
  ve donat per l'esquema de la figura.
  \figura{probl3.exc8.pstex_t}
  Suposant que el senyal d'entrada no té components freqüencials per a
  $\Omega>4000\pi$, trobeu el major $T$ possible i la resposta
  freqüencial del sistema discret, $H(\omega)$, de manera que
  $$Y_a(\Omega)=
  \begin{cases}
    |\Omega|X_a(\Omega) & \text{si $1000\pi < | \Omega | < 2000\pi$}, \\
    0 & \text{altrament.}
  \end{cases}
  $$
% \item Seguint amb la configuració del sistema anterior, es suposa ara
%   que $X_a(\Omega)=0$ si $|\Omega|>\pi/T$. Determineu el sistema
%   digital que fa possible implementar l'integrador:
%   $$y_a(t)=\int_{\tau=-\infty}^t x_a(\tau)\,d\tau.$$
\end{enumerate}
\end{document}
%%% Local Variables: 
%%% mode: latex
%%% TeX-master: t
%%% End: 
