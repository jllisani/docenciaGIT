\documentclass{article}
\usepackage[catalan]{babel}
%\usepackage[latin1]{inputenc}   % Permet usar tots els accents i car鐃諸ers llatins de forma directa.
\usepackage[utf8]{inputenc}   % Permet usar tots els accents i car鐃諸ers llatins de forma directa.
\usepackage{enumerate}
\usepackage{amsfonts, amscd, amsmath, amssymb}
\usepackage[pdftex]{graphicx}

\setlength{\textwidth}{16cm}
\setlength{\textheight}{24.5cm}
\setlength{\oddsidemargin}{-0.3cm}
\setlength{\evensidemargin}{0.25cm} \addtolength{\headheight}{\baselineskip}
\addtolength{\topmargin}{-3cm}

\newcommand\Z{\mathbb{Z}}
\newcommand\R{\mathbb{R}}
\newcommand\N{\mathbb{N}}
\newcommand\Q{\mathbb{Q}}
\newcommand\K{\Bbbk}
\newcommand\C{\mathbb{C}}


\begin{document}

\textbf{\Large Solucions Problemes Tema 3}

\vskip 1cm

\noindent
1.
\begin{enumerate}[(a)]
\item $-2z^4+z^3+3z$, ROC:$\C-\{\infty\}$
\item $3+3z+z^3-2z^4$, ROC:$\C-\{\infty\}$
\item $3z^{-1}+z-2z^2$, ROC:$\C-\{0, \infty\}$
\item $3+z^{-1}-z^{-2}$, ROC:$\C-\{0\}$
\end{enumerate}

\noindent
2.
\begin{enumerate}[(a)]
\item $\displaystyle \frac{1}{1-z^{-1}}$, ROC:$|z|>1$
\item $\displaystyle \frac{1}{(1-z^{-1})^2}$, ROC:$|z|>1$
\item $\displaystyle \frac{1}{1-az^{-1}}+\frac{1}{1-a^{-1}z^{-1}}$, 
ROC: $\begin{cases} |z|>1/|a| & \text{si } |a|\geq 1 \\ |z|>|a| & \text{si } |a|<1 \end{cases}$
\item $\displaystyle \frac{1}{1-az^{-1}}+\frac{1}{1-a^{-1}{z^{-1}}}$, 
ROC: $\begin{cases} \emptyset & \text{si } |a|\geq 1 \\ |a|<|z|<1/|a| & \text{si } |a|<1 \end{cases}$
\item $\displaystyle \frac{1-\cos\omega z^{-1}}{1-2\cos\omega z^{-1}+z^{-2}}$, ROC:$|z|>1$
\item $\displaystyle \frac{\cos\omega z^{-3}+\cos\omega z^{-1}-2z^{-2}}{(1-2\cos\omega z^{-1}+z^{-2})^2}$,
ROC:$|z|>1$
\item $\displaystyle \frac{1-a\cos\omega z^{-1}}{1-2a\cos\omega z^{-1}+a^2z^{-2}}$, ROC:$|z|>|a|$
\item $\displaystyle -z \frac{a\cos\omega z^{-2}(1-2a\cos\omega z^{-1}+a^2 z^{-2})-(1-a\cos\omega z^{-1})
(2a\cos\omega z^{-2}-2a^2 z^{-3})}{(1-2a\cos\omega z^{-1}+a^2 z^{-2})^2}$, ROC:$|z|>|a|$
\item $\displaystyle \frac{z^{-1}(1+az^{-1})}{(1-az^{-1})^3}+\frac{z^{-1}}{(1-az^{-1})^2}$, ROC:$|z|>|a|$
\end{enumerate}

\noindent
3.
\begin{enumerate}[(a)]
\item =(b) $\displaystyle \frac{1}{1-\frac{1}{3}z^{-1}} - \frac{1}{1-2z^{-1}}$, ROC:$\frac{1}{3}<|z|<2$
\end{enumerate}

\noindent
5.
\begin{enumerate}[(a)]
\item $\displaystyle \frac{X(z)}{1-z^{-1}}$
\item $\displaystyle \frac{X(z^{1/2})+X(-z^{1/2})}{2}$
\item $X(z^2)$
\end{enumerate}

\noindent 
7. 
\begin{description}
\item{Pas 1:    }
$
\displaystyle
X(z)=\frac{5z^{-1}}{(1-2z^{-1})(3-z^{-1})} = \frac{5z}{(z-2)(3z-1)}
$
\item{Pas 2:    }
$
\displaystyle
\frac{X(z)}{z}= \frac{5}{(z-2)(3z-1)} = \frac{A}{z-2} + \frac{B}{3z-1} = \text{(es calculen A i B)} = \frac{1}{z-2} + \frac{-3}{3z-1}
$
\item{Pas 3:    }
$
\displaystyle
X(z)=  \frac{z}{z-2} + \frac{-3z}{3z-1} = \frac{1}{1-2z^{-1}} - \frac{1}{1-\frac{1}{3}z^{-1}} = X_1(z) - X_2(z)
$
\newline
on $ \displaystyle X_1(z)=  \frac{1}{1-2z^{-1}} $, $ \displaystyle X_2(z)=\frac{1}{1-\frac{1}{3}z^{-1}}$. 
\newline
Per tant, $x[n]=x_1[n]-x_2[n]$
\item{Pas 4:}   possibles solucions (mirant taules de transformades):
\begin{enumerate}
\item $x_1[n]$ causal, $x_2[n]$ causal:
\begin{itemize}
\item $\displaystyle x_1[n]=2^n u[n]$, $ROC_1: |z| > 2$
\item $\displaystyle x_2[n]=\left(\frac{1}{3}\right)^n u[n]$, $ROC_2: |z| > 1/3$
\item $\displaystyle ROC = ROC_1 \cap ROC_2 = |z| > 2$, $\qquad$ $x[n]=2^n u[n]-(\frac{1}{3})^n u[n]$
\end{itemize}
\item $x_1[n]$ causal, $x_2[n]$ anticausal:
\begin{itemize}
\item $\displaystyle x_1[n]=2^n u[n]$, $ROC_1: |z| > 2$
\item $\displaystyle x_2[n]=-\left(\frac{1}{3}\right)^n u[-n-1]$, $ROC_2: |z| < 1/3$
\item $\displaystyle ROC = ROC_1 \cap ROC_2 = \emptyset$, $\qquad$ No és possible
\end{itemize}
\item $x_1[n]$ anticausal, $x_2[n]$ causal:
\begin{itemize}
\item $\displaystyle x_1[n]=- 2^n u[-n-1]$, $ROC_1: |z| < 2$
\item $\displaystyle x_2[n]=\left(\frac{1}{3}\right)^n u[n]$, $ROC_2: |z| > 1/3$
\item $\displaystyle ROC = ROC_1 \cap ROC_2 = 1/3 < |z| < 2$, $\qquad$ $x[n]=- 2^n u[-n-1]-(\frac{1}{3})^n u[n]$
\end{itemize}
\item $x_1[n]$ anticausal, $x_2[n]$ antcausal:
\begin{itemize}
\item $\displaystyle x_1[n]=- 2^n u[-n-1]$, $ROC_1: |z| < 2$
\item $\displaystyle x_2[n]=- \left(\frac{1}{3}\right)^n u[-n-1]$, $ROC_2: |z| < 1/3$
\item $\displaystyle ROC = ROC_1 \cap ROC_2 = |z| < 1/3$, $\qquad$ $x[n]=- 2^n u[-n-1]+(\frac{1}{3})^n u[-n-1]$
\end{itemize}
\end{enumerate}
\end{description}

\noindent 
14.
\begin{description}

\item{a)}
\begin{description}
\item{Pas 1:    }
$
\displaystyle
x[n]=\left(\frac{1}{2}\right)^n u[n] - \frac{1}{4} \left(\frac{1}{2}\right)^{n-1} u[n-1] = x_1[n] - \frac{1}{4} x_1[n-1]
$

on $x_1[n]=(\frac{1}{2})^n u[n]$

$
\displaystyle
X(z)=X_1(z) - \frac{1}{4} z^{-1} X_1(z)=(1-\frac{1}{4}z^{-1})X_1(z)=\text{(taules transformades)}=\frac{1-\frac{1}{4}z^{-1}}{1-\frac{1}{2}z^{-1}}
$, $\qquad$ $ROC=|z|>1/2$

\item{Pas 2:    }
$
\displaystyle
y[n]=\left(\frac{1}{3}\right)^n u[n]
$

$
\displaystyle
Y(z)=\frac{1}{1-\frac{1}{3}z^{-1}}
$, $\qquad$ $ROC=|z|>1/3$

\item{Pas 3:    }
$
\displaystyle
H(z)=\frac{Y(z)}{X(z)}=\frac{ \frac{1}{1-\frac{1}{3}z^{-1}} }{ \frac{1-\frac{1}{4}z^{-1}}{1-\frac{1}{2}z^{-1}} }=
\frac{1-\frac{1}{2}z^{-1}}{(1-\frac{1}{3}z^{-1})(1-\frac{1}{4}z^{-1})} = \frac{z(z-\frac{1}{2})}{(z-\frac{1}{3})(z-\frac{1}{4})}
$

\item{Pas 4:    }
$
\displaystyle
\frac{H(z)}{z}=\frac{z-\frac{1}{2}}{(z-\frac{1}{3})(z-\frac{1}{4})}=\frac{A}{z-\frac{1}{3}}+\frac{B}{z-\frac{1}{4}}=
\text{(es calculen A i B)}=\frac{-2}{z-\frac{1}{3}}+\frac{3}{z-\frac{1}{4}}
$

\item{Pas 5:    } 
$
\displaystyle
H(z)=\frac{-2z}{z-\frac{1}{3}}+\frac{3z}{z-\frac{1}{4}}
$

La ROC es calcula en l'apartat b).

\end{description}

\item{b)}
\begin{description}
\item{Pas 1:    } a partir del resultat de l'apartat a):
$
\displaystyle
H(z)=\frac{-2z}{z-\frac{1}{3}}+\frac{3z}{z-\frac{1}{4}}=-2\frac{1}{1-\frac{1}{3}z^{-1}}+3\frac{1}{1-\frac{1}{4}z^{-1}}=-2H_1(z)+3H_2(z)
$

on $\displaystyle H_1(z)=\frac{1}{1-\frac{1}{3}z^{-1}}$, $\displaystyle H_2(z)=\frac{1}{1-\frac{1}{4}z^{-1}}$.

Per tant, $h[n]=-2h_1[n]+3h_2[n]$

\item{Pas 2:}   solució causal (l'enunciat diu que és un sistema LTI causal) (mirant taules de transformades):

$
\displaystyle
h_1[n]=\left(\frac{1}{3}\right)^n u[n]
$, $ROC_1:|z|>1/3$

$
\displaystyle
h_2[n]=\left( \frac{1}{4} \right)^n u[n]
$, $ROC_2:|z|>1/4$

$
\displaystyle
h[n]=-2 \left(\frac{1}{3}\right)^n u[n] + 3 \left( \frac{1}{4} \right)^n u[n]
$, $ROC=ROC_1 \cap ROC_2:|z|>1/3$

\end{description}

\item{c)} A partir del resultat del pas 3 de l'apartat a):

$
\displaystyle
H(z)=\frac{Y(z)}{X(z)}= \frac{z(z-\frac{1}{2})}{(z-\frac{1}{3})(z-\frac{1}{4})}=\frac{z^2-\frac{1}{2}z}{z^2-\frac{7}{12}z+\frac{1}{12}}=
\frac{1-\frac{1}{2}z^{-1}}{1-\frac{7}{12}z^{-1}+\frac{1}{12}z^{-2}}
$
 
$
\displaystyle
Y(z) \left( 1-\frac{7}{12}z^{-1}+\frac{1}{12}z^{-2} \right) = X(z) \left(  1-\frac{1}{2}z^{-1} \right)
$

$
\displaystyle
Y(z) -\frac{7}{12}z^{-1} Y(z) +\frac{1}{12}z^{-2} Y(z)  = X(z) -\frac{1}{2}z^{-1} X(z)
$

Aplicant propietats de la transformada Z:

$
\displaystyle
y[n] -\frac{7}{12} y[n-1] +\frac{1}{12} y[n-2]  = x[n] -\frac{1}{2} x[n-1]
$

$
\displaystyle
y[n]= \frac{7}{12} y[n-1] - \frac{1}{12} y[n-2]  + x[n] -\frac{1}{2} x[n-1]
$


\item{d)} A partir del resultat del pas 3 de l'apartat a):

$
\displaystyle
H(z)= \frac{z(z-\frac{1}{2})}{(z-\frac{1}{3})(z-\frac{1}{4})}
$
 
Els zeros de $H(z)$ són les arrels del numerador, i els pols les arrels del denominador. Per tant:

zeros: $z_1=0$, $z_2=1/2$

pols: $p_1=1/3$, $p_2=1/4$


Com que el sistema és causal (ho diu l'enunciat) i $|p_1|<1$ i $|p_2|<1$, llavors el sistema és \textbf{estable}.

\end{description}


\noindent 
15.

\begin{description}
\item{Pas 1:}
$
\displaystyle
y[n]+a_1 y[n-1]+a_2 y[n-2] = x[n]
$

\item{Pas 2:} aplicam propietats de la transformada Z:

$
\displaystyle
Y(z)+a_1 z^{-1} Y(z)+a_2 z^{-2} Y(z) = X(z)
$

\item{Pas 3:}

$
\displaystyle
H(z)=\frac{Y(z)}{X(z)}=\frac{1}{1+a_1z^{-1}+a_2z^{-2}}=\frac{z^2}{z^2+a_1z+a_2}=\frac{z^2}{(z-p_1)(z-p_2)}
$

on

$\displaystyle p_1=\frac{-a_1 + \sqrt{a_1^2 - 4a_2}}{2}$

$\displaystyle p_2=\frac{-a_1 -  \sqrt{a_1^2 - 4a_2}}{2}$



\item{Pas 4:} el sistema és causal (ho diu l'enunciat), per ésser estable els pols han d'estar
a l'interior del cercle unitat ($|p_1| < 1$, $|p_2| < 1$).

Possibilitats:

\begin{enumerate}
\item $p_1$ i $p_2$ són reals i iguals: 

Això passa si $a_1^2 - 4a_2=0$, és a dir: $a_2=a_1^2/4$.

En aquest cas $p_1=p_2=\frac{a_1}{2}$.

Per tenir $|p_1|=|p_2| < 1$ haurà de passar $|\frac{a_1}{2}|<1$, i per tant $|a_1|<2$.

\item $p_1$ i $p_2$ són valors complexes conjugats: 

Això passa si $a_1^2 - 4a_2<0$, és a dir: $a_2 > a_1^2/4$.

En aquest cas:

$\displaystyle p_1=\frac{-a_1 + i \sqrt{-a_1^2 + 4a_2}}{2}$=$-\frac{a_1}{2} + i \frac{\sqrt{-a_1^2 + 4a_2}}{2} $

$\displaystyle p_1=\frac{-a_1 - i \sqrt{-a_1^2 + 4a_2}}{2}$=$-\frac{a_1}{2} - i \frac{\sqrt{-a_1^2 + 4a_2}}{2} $


$\displaystyle |p_1| = |p_2|=\sqrt{  \left( \frac{a_1}{2} \right)^2 + \left( \frac{\sqrt{-a_1^2 + 4a_2}}{2} \right)^2 }  = \sqrt{a_2} $

Per tenir $|p_1|=|p_2| < 1$ haurà de passar $\sqrt{a_2}<1$, i per tant $a_2<1$.


\item $p_1$ i $p_2$ són reals i diferents: no estudiam aquest cas


\end{enumerate}



\end{description}



\end{document}

