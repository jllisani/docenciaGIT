\documentclass{article}
\usepackage[catalan]{babel}
%\usepackage[latin1]{inputenc}   % Permet usar tots els accents i car�ters llatins de forma directa.
\usepackage[utf8]{inputenc}   % Permet usar tots els accents i car�ters llatins de forma directa.
\usepackage{enumerate}
\usepackage{amsfonts, amscd, amsmath, amssymb}
%\usepackage[pdftex]{graphicx}
\usepackage{graphicx}
\usepackage{longtable}
\usepackage{url}
\usepackage{hyperref}

\setlength{\textwidth}{16cm}
\setlength{\textheight}{24.5cm}
\setlength{\oddsidemargin}{-0.3cm}
\setlength{\evensidemargin}{0.25cm} \addtolength{\headheight}{\baselineskip}
\addtolength{\topmargin}{-3cm}

\newcommand\Z{\mathbb{Z}}
\newcommand\R{\mathbb{R}}
\newcommand\N{\mathbb{N}}
\newcommand\Q{\mathbb{Q}}
\newcommand\K{\Bbbk}
\newcommand\C{\mathbb{C}}

\newcounter{exctr}
\newenvironment{exemple}
{ \stepcounter{exctr} 
\hspace{0.2cm} 
\textit{Exemple  \arabic{exctr}: }
\it
\begin{quotation}
}{\end{quotation}}


\begin{document}

\textbf{\Large Pràctica Processament Digital de Senyal amb Matlab}


\vskip 0.3 cm
\noindent
\textbf{Exercici 1}

L'objectiu és estudiar l'efecte de l'aliasing en una gravació musical.
Per a això:
\begin{enumerate}[1)]
\item descarregau d'Internet (per exemple de http://www.mfiles.co.uk/classical-mp3.htm) alguna peça musical
\item si el format del fitxer no és .wav convertiu-lo a .wav amb http://media.io/
\item obriu el fitxer des de Matlab i convertiu-lo a mono si era estéreo. Quedau-vos només amb 4 o 5 segons de la peça
i guardau el nou senyal.
\item obriu el senyal i representau-lo. Representau també el mòdul de la seva transformada de Fourier
\item submostrejau el senyal amb diferents factors de submostreig, fins que l'efecte de l'aliasing sigui evident
\item representau la transformada de Fourier dels senyals obtinguts en l'apartat anterior i guardau els resultats en
format .wav
\end{enumerate}


\vskip 0.3 cm
\noindent
\textbf{Exercici 2}

L'objectiu és estudiar el filtratge del renou en un senyal 1D (senyal de so).
Per a això:
\begin{enumerate}[1)]
\item gravau 4 o 5 segons d'un senyal de veu contaminat per renou (per exemple, amb el telèfon mòvil gravau la televisió mentre teniu
un aparell amb un motor en marxa -un extractor, un assecador de cabells, etc-)
\item obriu el fitxer des de Matlab, representau-lo a nivell temporal i freqüencial
\item intentau eliminar el renou aplicant un filtre de mitjana de diferents tamanys. Representau els resultats a nivell temporal i freqüencial
\item intentau eliminar el renou aplicant un filtre de mediana de diferents tamanys. Representau els resultats a nivell temporal i freqüencial
\end{enumerate}

\vskip 0.3 cm
\noindent
\textbf{Exercici 3}

L'objectiu és modificar el contingut d'un fitxer de so.
Per a això:
\begin{enumerate}[1)]
\item descarregau-vos el fitxer de so  `sopractica.wav' disponible a Campus Extens i escoltau la peça musical
\item obriu el fitxer i representau-lo a nivell temporal i freqüencial
\item en la representació temporal observareu que hi ha una separació entre cada una de les notes de la peça musical.
En total hi ha 16 notes. L'objectiu de la pràctica és que cada alumne modifiqui el senyal per eliminar una nota de la peça, indicat a més
quina és la freqüència principal de la nota eliminada. Cada alumne té assignada una nota (la primera, la segona, etc. fins a la
darrera nota que és la número 16), l'assignació de notes està penjada de Campus Extens (`notesassignades.pdf')
\item es suggereix el següent procediment per eliminar la nota:
\begin{itemize}
\item observau, a nivell temporal, on es troba la nota a eliminar
\item copiau en un array la porció del senyal on es troba la nota
\item eliminau (posau a zero) la porció del senyal original on es troba la nota i guardau el resultat
\item dibuixau l'espectre del la porció copiada en el pas 2 i calculau a quina freqüència
contínua correspon el màxim de l'espectre
\end{itemize}
\end{enumerate}




\vskip 0.3 cm
\noindent
S'ha d'entregar un informe de la pràctica en format pdf on s'incloguin, per a cada exercici, 
les instruccions utilitzades, les figures generades i comentaris dels resultats. Els fitxers de so generats també s'han d'entregar.
Tots els fitxers s'han de comprimir en un únic fitxer amb format .zip i penjar-los de Campus Extens.



\end{document}