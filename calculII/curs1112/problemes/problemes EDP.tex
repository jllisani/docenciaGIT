\documentclass[12pt]{article}
\usepackage{fancyhdr}
\usepackage{makeidx}
\usepackage[catalan]{babel}
\usepackage{amsfonts}
\usepackage{amssymb}
\usepackage{amsthm}
\usepackage{amsmath}
\usepackage[all]{xy}
%\usepackage[ansinew]{inputenc} %%
\usepackage[utf8]{inputenc}
\usepackage[dvips]{epsfig}
\usepackage{color}
\usepackage{multicol}


\setlength{\textwidth}{16cm}
\setlength{\textheight}{24cm}
\setlength{\oddsidemargin}{-0.3cm}
\setlength{\topmargin}{-1.3cm}
% \setlength{\evensidemargin}{1cm}
% \addtolength{\headheight}{\baselineskip}
% \addtolength{\topmargin}{-3cm}


\newcommand{\sC}{{\cal C}}
\newcommand{\sF}{{\cal F}}
\newcommand{\sL}{{\cal L}}
\newcommand{\sU}{{\cal U}}
\newcommand{\sX}{{\cal X}}
\newcommand{\eop}{{\Box}}
% \newtheorem{corollary}{Corollary}
% \newtheorem{theorem}{Theorem}
% \newtheorem{lemma}{Lemma}
% \newtheorem{proposition}{Proposicio}
% \newtheorem{definition}{Definicio}
% \newtheorem{exercici}{}
%\newcommand{\ar}{A^{(r)}}
\newcommand{\HH}{{\bf H}}
\newcommand{\sS}{{\cal S}}
\newcommand{\Img}{\mbox{Img}}

\def\N{I\!\!N}
\def\R{I\!\!R}
\def\Z{Z\!\!\!Z}
\def\Q{O\!\!\!\!Q}
\def\C{I\!\!C}
%\def\text#1{\hbox{\rm #1}}
%\def\inter#1{{\buildrel {\circ}\over {#1}}}
%\def\defi{\sp2J\hskip1cm {\bf Definici{\'o}n}}

\newcount\problemes
\problemes=49

\def\probl{\advance\problemes by 1
\vskip 2ex\noindent{\bf \the\problemes \hbox{ } }}


\begin{document}

\pagenumbering{arabic}
\pagestyle{fancy}
\fancyhead{}
\fancyhead[RE,RO]{Exercicis Tema 5. Equacions en derivades parcials.}

%\pagestyle{empty}
%\parindent =0 pt
%{\bf Problemes de C{\`a}lcul II. Primer Telem{\`a}tica
%\hfill Bloc 5}

\vspace{0.4 cm}


\probl Trobau la soluci{\'o} general $u(x,y) $ de

\vspace{0.4 cm}
\begin{tabular}{lll}
a) $u_x=3x^2+y^2$ & \qquad\qquad b)$u_{xy}=x^2y$ & \qquad\qquad c) $u_{xxy}=1$ \\
& & \\
d) $u_x-2u=0$  &\qquad\qquad  e) $u_y+2yu=4xy$ &\qquad\qquad  f) $uu_{xy}-u_xu_y=0$
\end{tabular}

\vspace{0.7 cm}
\probl a) Trobau la soluci{\'o} general $u(x,y)$ de $u_{xx}-u=0$\\

b) Trobau la soluci{\'o} de l'equaci{\'o} de l'apartat anterior que
satisf{\`a} les condicions auxiliars
$$ u(0,y)=f(y)\,, \qquad u_x(0,y)=g(y)\,.$$


\vspace{0.4 cm}
\probl Donada l'equaci{\'o}

$$ 4y^2u_{xx}+2(1-y^2)u_{xy}-u_{yy}-\frac{2y}{1+y^2}(2u_x-u_y)=0$$

amb les condicions $\quad u(x,0)=f(x)\ $ i $\ u_y(x,0)=1\,, $

provau que
$$
u(x,y)=f(x-\frac{2}{3}y^3)+y+\frac{1}{3}y^3
$$
{\'e}s una soluci{\'o} del'EDP.



\vspace{0.7 cm}
\probl  Trobau la soluci{\'o} general de les seg{\"u}ents EDP's on
$u=u(x,y)$

\vspace{0.4 cm}
\begin{tabular}{lll}
a) $2u_x-3u_y=x$ &\qquad \qquad b) $3u_x-4u_y=x+e^x$ &\qquad\qquad c) $u_x+3u_y=9y^2$
\end{tabular}

\vspace{0.7 cm}
\probl Provau que l'EDP $\ u_x+u_y-u=0\ $ amb la condici{\'o}
$\ u(x,x)=\tan x\ $ no t{\'e} soluci{\'o}.

\vspace{0.7 cm}
\probl Quina forma ha de tenir $g(x)$ per a que el seg{\"u}ent
problema tingui soluci{\'o}?
$$ u_x+3u_y-u=1 \qquad u(x,3x)=g(x)$$
%Si $g$ t{\'e} la forma requerida, hi haur{\`a} m{\'e}s d'una soluci{\'o}?

\vspace{0.7 cm}
\probl Trobau la soluci{\'o} particular de les EDP's amb les
condicions donades

\vspace{0.4 cm}
\begin{tabular}{ll}
a) $\ xu_x+2yu_y=0\,,\quad u(x,\frac{1}{x})=x \ (x>0)$  &\qquad  b) $\ yu_x-4xu_y=0\,, \quad u(x,0)=x^4$ \\
\end{tabular}

\vspace{0.7 cm}
\probl Resoleu $\qquad u_t=2u_{xx} \qquad\quad 0< x <1\,, t> 0\qquad $ tal que\\

$\qquad u(0,t)=0\,,\ \  t>0\qquad u_x(1,t)=0\,,\ \ t>0\qquad  u(x,0)=\sin(\frac{3\pi x}{2})\,, \ \ x\in[0,1]$


\end{document}

\vspace{0.7 cm}
\probl a) Sigui $u(x,y)=\phi(\frac{y}{x})+x\varphi(\frac{y}{x})$,
demostrau que satisf{\`a}
$$x^2u_{xx}+2xyu_{xy}+y^2 u_{yy}=0$$

b) obteniu una soluci{\'o} de l'EDP anterior que satisfaci
$u(1,y)=\cos y$ i $u(1/2,y)=e^{-2y}$.
