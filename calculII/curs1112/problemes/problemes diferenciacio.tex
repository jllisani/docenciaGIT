\documentclass[12pt]{article}
\usepackage{fancyhdr}
\usepackage{makeidx}
\usepackage[catalan]{babel}
\usepackage{amsfonts}
\usepackage{amssymb}
\usepackage{amsthm}
\usepackage{amsmath}
\usepackage[all]{xy}
%\usepackage[ansinew]{inputenc} %%
\usepackage[utf8]{inputenc}
\usepackage[dvips]{epsfig}
\usepackage{color}
\usepackage{multicol}

%\usepackage[T1]{fontenc}
%\usepackage[latin1]{inputenc}
\setlength{\textwidth}{16cm}
\setlength{\textheight}{24cm}
\setlength{\oddsidemargin}{-0.3cm}
\setlength{\topmargin}{-1.3cm}
% \setlength{\evensidemargin}{1cm}
% \addtolength{\headheight}{\baselineskip}
% \addtolength{\topmargin}{-3cm}


\newcommand{\sC}{{\cal C}}
\newcommand{\sF}{{\cal F}}
\newcommand{\sL}{{\cal L}}
\newcommand{\sU}{{\cal U}}
\newcommand{\sX}{{\cal X}}
\newcommand{\eop}{{\Box}}
% \newtheorem{corollary}{Corollary}
% \newtheorem{theorem}{Theorem}
% \newtheorem{lemma}{Lemma}
% \newtheorem{proposition}{Proposicio}
% \newtheorem{definition}{Definicio}
% \newtheorem{exercici}{}
%\newcommand{\ar}{A^{(r)}}
\newcommand{\HH}{{\bf H}}
\newcommand{\sS}{{\cal S}}
\newcommand{\Img}{\mbox{Img}}

\def\N{I\!\!N}
\def\R{I\!\!R}
\def\Z{Z\!\!\!Z}
\def\Q{O\!\!\!\!Q}
\def\C{C\!\!\!\!I}

%\def\C{I\!\!C}
%\def\text#1{\hbox{\rm #1}}
%\def\inter#1{{\buildrel {\circ}\over {#1}}}
%\def\defi{\sp2J\hskip1cm {\bf Definici{\'o}n}}

\newcount\problemes
\problemes=23

\def\probl{\advance\problemes by 1
\vskip 2ex\noindent{\bf \the\problemes \hbox{ } }}


\begin{document}

\pagenumbering{arabic}
\pagestyle{fancy}
\fancyhead{}
\fancyhead[RE,RO]{Exercicis Tema 3. Funcions de vàries variables. Diferenciació.}



%\pagestyle{empty}
%\parindent =0 pt
%{\bf Problemes de C{\`a}lcul II. Primer Telem{\`a}tica
%\hfill Bloc 3}


\vspace{0.8 cm}
\probl
Calculau les derivades parcials de les seg{\"u}ents funcions

\vspace{0.6cm}
$\begin{array}{ll}
a)\;\;f(x,y,z)=\sin xy +\cos yz &b)\;\; f(x,y,z)=xe^{yz}\\
 & \\
c)\;\; f(x,y,z)=e^{xyz}\sin (xy) \cos(2xz)\hspace{1.3cm} &
d)\;\;f(t,u,v)=\sec (tu)+\arcsin (tv)\\
\end{array}$

\vspace{0.8 cm}
\probl
Calculau la matriu jacobiana de les seg{\"u}ents funcions:

\vspace{0.6cm}
$\begin{array}{ll} a)\;\; f(x,y)=(x^2y,xy,xy^2)\hspace{1.7cm} & b)\;\;
f(x,y,z)=(xe^{yz},ye^{xz},ze^{xy})
\end{array}$

\vspace{0.8 cm}
\probl
Calculau les derivades parcials de la funci{\'o}:
$$
f(x,y)=\begin{cases} \displaystyle{3x^2y\over x^2 +y^2} & si (x,y)\not= (0,0)\cr
& \cr
0 & si (x,y)=(0,0)\cr\end{cases}
$$

\vspace{0.8cm}
\probl
Estudiau l'exist{\`e}ncia de les derivades direccionals de la funci{\'o}
$f(x,y,z)=x\sqrt{y^2+z^2}$ en el punt $(0,0,0)$.

\vspace{0.8cm}
\probl
De les seg{\"u}ents funcions de $n$-variables calculau les seves derivades parcials:

\vspace{0.3cm}
a) $\; f(x_1,x_2,\dots , x_n)=\sqrt{x_1^2 + x_2^2 + \dots + x_n^2} $

\vspace{0.3cm}
b) $\;f(x_1,x_2,\dots , x_n)=\sin ( x_1 + 2x_2 + \dots + n x_n).$

\vspace{0.8cm}
\probl
Calculau $D_vf(P)$ on $v$ {\'e}s el vector unitari en la direcci{\'o}
$\vec{PQ}$. Utilitzau el fet que la funci{\'o} {\'e}s diferenciable.

\vspace{0.3cm}
a) $\;f(x,y,z)=x^2 +3xy+y^2 +z^2; \quad P=(1,0,2),\, Q=(-1,3,4).$

\vspace{0.3cm}
b) $\;f(x,y,z)=e^x\cos y + e^y\sin z;\quad P=(2,1,0),\, Q=(-1,2,2).$

\vspace{0.8cm}
\probl
Per cada una de les seg{\"u}ents funcions, trobau la direcci{\'o} en la qual la derivada direccional {\'e}s
m{\`a}xima:

\vspace{0.3cm}
a) $\; f(x,y)=\ln (x^2+2y^2)\ \ $ a $\ \ (1,-2)$ \qquad b) $\;f(x,y,z)=\sin xy -\cos
xz\ \ $ a $\ \ (\pi, 1, 1)$

\vspace{0.8cm}
\probl
Trobau l'equaci{\'o} del pla tangent a la superf{\'\i}cie donada en el punt
indicat
$$
\begin{array}{ll}
a)\ z=x^2 + 4y^2,\ \ (2,1,8)\ &  b)\ z= 5 + (x-1)^2 + (y+2)^2,\ \ (2,0,10)
\end{array}
$$

\probl
Siguin $f,\; g$ funcions reals de variable real derivables a tot $\R$, aleshores calculau
 $\displaystyle{\partial z\over \partial x}$  i
 $\displaystyle{\partial z\over \partial y}$ de les funcions:
$$
\begin{array}{ll}
a)\; z= f(x) + g(y)\qquad & b)\; z= f(x) g(y)\\
 & \\
c)\; z= f(xy)\; & d)\; z= f(ax + by).
\end{array}
$$

\vspace{0.8cm}
\probl
Utilitzau la regla de la cadena per calcular les derivades indicades:

%\vspace{0.3cm}
%a) $\; f(x,y)=x^y$, $x=t^2$, $y=\ln t;\qquad \displaystyle\frac{df}{dt}$
%
\vspace{0.3cm}
$\qquad f(x,y)=x^2+xy$, $x=ve^u$, $y=ue^v\qquad \displaystyle\frac{\partial f}{\partial u}\,,$
$\ \displaystyle\frac{\partial f}{\partial v}$

\vspace{0.8cm}
\probl
Trobau totes les derivades parcials de segon ordre de les funcions:
$$
\begin{array}{ll}
a)\; f(x,y)= x^2 y + x\sqrt{y},\qquad & b)\; f(x,y)= \sin (x+y) + \cos (x-y)
\end{array}
$$

\vspace{0.8cm}
\probl
Trobau, en cada cas, la derivada parcial indicada:
$$
\begin{array}{ll}
a)\; f(x,y)=x^2y^3 - 3x^4y;\, f_{xxx}\; & b)\; f(x,y,z)= x^5 + x^4y^4z^3 + yz^2;\, f_{xyz},\,
  f_{yxz},\, f_{zyx}
\end{array}
$$

%\vspace{0.8cm}
%\probl
%Sigui
%$$
%f(x,y)=\cases{\displaystyle{xy(x^2-y^2)\over x^2 + y^2} & si $(x,y)\not=(0,0)$ \cr
% & \cr
%0 & si $(x,y)=(0,0)$ \cr}
%$$
%
%a) Si $(x,y)\not=(0,0)$, calculau $f_x$ i $f_y$.
%
%\vspace{0.3cm}
%b) Provau que $f_x(0,0)=0=f_y(0,0)$.


\vspace{0.8cm}
\probl
Demostrau que la funci{\'o} $\ z=y\;\varphi (x^2 - y^2)\ $ satisf{\`a} l'equaci{\'o}
$$
{1\over x}{\partial z\over \partial x} + {1\over y}{\partial z\over
\partial y} = {z\over y^2}
$$



\vspace{0.8 cm}
\probl
Trobau els m{\`a}xims i m{\'\i}nims relatius de les funcions:

\vspace{0.6cm}
a) $f(x,y)=x^4+y^4-2(x-y)^2\qquad\qquad$ b) $f(x,y)=x^4+y^4+\displaystyle\frac{1}{x^4y^4}$

\vspace{0.8 cm}
\probl
Determinau els extrems absoluts de la funci{\'o}
$f(x,y)= xy ( 1 - x^2 - y^2)$ en el quadrat $[0,1]\times [0,1]$.

\end{document} 