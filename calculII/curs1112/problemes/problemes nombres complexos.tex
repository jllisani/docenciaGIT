\documentclass[12pt]{article}
\usepackage{fancyhdr}
\usepackage{makeidx}
\usepackage[catalan]{babel}
\usepackage{amsfonts}
\usepackage{amssymb}
\usepackage{amsthm}
\usepackage{amsmath}
\usepackage[all]{xy}
%\usepackage[ansinew]{inputenc} %%
\usepackage[utf8]{inputenc}
\usepackage[dvips]{epsfig}
\usepackage{color}
\usepackage{multicol}

%\usepackage[T1]{fontenc}
%\usepackage[latin1]{inputenc}
\usepackage{color}
\setlength{\textwidth}{16cm}
\setlength{\textheight}{24cm}
\setlength{\oddsidemargin}{-0.3cm}
\setlength{\topmargin}{-1.3cm}
% \setlength{\evensidemargin}{1cm}
% \addtolength{\headheight}{\baselineskip}
% \addtolength{\topmargin}{-3cm}


\newcommand{\sC}{{\cal C}}
\newcommand{\sF}{{\cal F}}
\newcommand{\sL}{{\cal L}}
\newcommand{\sU}{{\cal U}}
\newcommand{\sX}{{\cal X}}
\newcommand{\eop}{{\Box}}
% \newtheorem{corollary}{Corollary}
% \newtheorem{theorem}{Theorem}
% \newtheorem{lemma}{Lemma}
% \newtheorem{proposition}{Proposicio}
% \newtheorem{definition}{Definicio}
% \newtheorem{exercici}{}
%\newcommand{\ar}{A^{(r)}}
\newcommand{\HH}{{\bf H}}
\newcommand{\sS}{{\cal S}}
\newcommand{\Img}{\mbox{Img}}

\def\N{I\!\!N}
\def\R{I\!\!R}
\def\Z{Z\!\!\!Z}
\def\Q{O\!\!\!\!Q}
\def\C{C\!\!\!\!I}

%\newtheorem{nota}{Nota}[subsection]

%\def\C{I\!\!C}
%\def\text#1{\hbox{\rm #1}}
%\def\inter#1{{\buildrel {\circ}\over {#1}}}
%\def\defi{\sp2J\hskip1cm {\bf Definici{\'o}n}}

\newcount\problemes
\problemes=0

\def\probl{\advance\problemes by 1
\vskip 2ex\noindent{\bf \the\problemes \hbox{ } }}


\begin{document}

\pagenumbering{arabic}
\pagestyle{fancy}
\fancyhead{}
\fancyhead[RE,RO]{Exercicis Tema 1. Els nombres complexos}



%\pagestyle{empty}
%
%\parindent =0 pt
%{\bf Problemes de C{\`a}lcul II. Primer Telem{\`a}tica
%\hfill Tema 1. Nombres complexos.}


\vspace{0.6 cm}

\probl
 Si $\> z = 1+2i \> $ i $\> \omega = 3-i, \> $ comprovau les propietats seg\"uents:
$$
\begin{array}{lcl}
\mbox{\bf a)} \;\;\overline{\overline{z\vphantom{+}}} = z & \;\; &
\mbox{\bf b)} \;\;\overline{z + \omega} =
\overline{z\vphantom{+}} + \overline{\omega\vphantom{+}} \\
\\
\mbox{\bf c)} \;\; \overline{z \cdot \omega\vphantom{+}} =
\overline{z\vphantom{+}} \cdot \overline{\omega\vphantom{+}} &
\;\; & \mbox{\bf d)} \;\; z\cdot \overline{z\vphantom{+}} \geq 0
\end{array}
$$


\vspace{0.4cm}
\probl
Si $\> z = 3-4i \> $ i $\> \omega = -6+8i, \> $ comprovau les
propietats seg{\"u}ents:
$$
\begin{array}{lcl}
\mbox{\bf a)}\;\; \mid \overline{z\vphantom{+}} \mid \> = \> \mid
z \mid  & \;\; &

\mbox{\bf b)} \;\; \mid z \cdot \omega \mid \> = \> \mid z \mid \>
\cdot \> \mid \omega \mid \\
\\
\mbox{\bf c)}  \;\; \mid \mbox{Re} \> z \mid \> \leq \> \mid z \mid &
\;\; & \mbox{\bf d)} \;\; \mid \mbox{Im} \> z \mid \> \leq \> \mid z
\mid
\\
\\
\mbox{\bf e)} \;\; \mid z + \omega \mid \> \leq \> \mid
z \mid \> + \> \mid \omega\mid
\end{array}
$$


\vspace{0.4cm}
\probl
Calculau el m{\`o}dul i l'argument dels nombres complexos
seg{\"u}ents:
$$2\sqrt{3}-2i, \ \ 5i, \ \ -\sqrt{3}-i, \ \ -4+4\sqrt{3}i, \ \ 1+i$$


\vspace{0.4cm}
\probl
Representau gr{\`a}ficament i expressau en forma exponencial i en
forma trigonom{\`e}trica els nombres complexos seg{\"u}ents:
$$\begin{array}{lclcl}
\mbox{\bf a)} \;\; z = i & \;\; & \mbox{\bf b)} \;\; z = -2-2i &
\;\; &
\mbox{\bf c)} \;\; z = 2\\
\\
\mbox{\bf d)} \;\; z =1+i & \;\; & \mbox{\bf e)} \;\; z=1-i &  \;\; &
\mbox{\bf f)} \;\; z = -1+i\\
\end{array}$$


\vspace{0.4cm}
\probl
Representau gr{\`a}ficament i expressau en forma bin{\`o}mica els nombres
complexos seg{\"u}ents expressats en forma exponencial:
$$\begin{array}{lclcl}
\mbox{\bf a)} \;\; z = 3 \; e^{i\frac{\pi}{4}} & \;\; & \mbox{\bf
b)} \;\; z = 2 \; e^{i\frac{\pi}{3}} & \;\; &
\mbox{\bf c)} \;\; z = 6 \; e^{i\frac{5\pi}{3}}\\
\\
\mbox{\bf d)} \;\; z = e^{i\frac{\pi}{6}} & \;\; & \mbox{\bf e)}
\;\; z = 8 \; e^{i\frac{\pi}{2}} & \;\; &
\mbox{\bf f)} \;\; z = 4 \; e^{i \pi}\\
\end{array}$$


\vspace{0.4cm}
\probl
Resoleu les equacions  seg{\"u}ents en $\; \C$ i descomponeu en
factors:
$$\begin{array}{lcl}
\mbox{\bf a)} \;\; 4x^2+48x+169 =0 & \;\; & \mbox{\bf b)} \;\;
4x^2-12x+25 = 0 \\
\\
\mbox{\bf c)} \;\; x^2+49=0 & \;\; & \mbox{\bf d)} \;\; x^2+16=0
\\
\\
\mbox{\bf e)} \;\; x^2+2x-5=0 & \;\; &
\mbox{\bf f)} \;\; 3x^2-x-10=0\\
\end{array}$$


\vspace{0.4cm}
\probl
Si $\> z = 3+2i \> $ i $\> \omega = 5-i, \> $ comprovau les
propietats seg{\"u}ents:

$$\begin{array}{lcl} \mbox{\bf a)} \;\;\displaystyle e^{z + \omega}
= e^{z} \cdot e^{\omega} &\;\;& \mbox{\bf b)} \;\; \displaystyle
e^{-z} = {1 \over e^z} \\
\\
\mbox{\bf c)} \;\;\displaystyle \overline{e^z} = e^{\overline{z}}
&\;\; & \mbox{\bf d)}  \;\;\displaystyle e^z = e^{z+2 \pi i}\\

\end{array}
$$


%\newpage

\vspace{0.4cm}
\probl
Provau que si $\> z_1 = 2-2i, z_2 = 4+5i, \> $ aleshores $\>
\displaystyle (z_1 \cdot z_2)^2 = z_1^2 \cdot z_2^2$


\vspace{0.4cm}
\probl
Efectuau les operacions  seg{\"u}ents amb nombres complexos:
$$\begin{array}{lcl}
\mbox{\bf a)} \;\;  (1-i+i^2)(2i-1)^2 & \;\; & \mbox{\bf b)} \;\;
\displaystyle (2+3i)^5 \\
\\
\mbox{\bf c)} \;\; \displaystyle {i^{15}-i^{16} \over 2-i} & \;\;
& \mbox{\bf
d)} \;\; \displaystyle {{(1+i+...+i^{62})} \over 2-i}\\
\end{array}$$


\vspace{0.4cm}
\probl
Determinau el conjunt de tots els $\> x,y \in \R \> $ tals que: $$
\mbox{\bf a)} \;\;x+iy \> = \> \mid x-iy \mid \quad\quad \mbox{\bf
b)} \;\;  x+iy = (x-iy)^2 \quad\quad \mbox{\bf c)}
\;\;\displaystyle x+iy = \sum_{k=0}^{100} i^k $$


\vspace{0.4cm}
\probl
Determinau els conjunts seg{\"u}ents: $$ \mbox{\bf a)}\;\;\{z\in\C
\; : \; 1+e^z=0\} \qquad \mbox{\bf b)}\;\;\{z\in\C \; : \; {1\over
e}-e^z=0\} \qquad \mbox{\bf c)}\;\;\{z\in\C \; : \; 1+i-e^z=0\}
$$


\vspace{0.4cm}
\probl
Determinau les arrels complexes
$$\sqrt[3]{-5i}, \ \ \sqrt[4]{-\sqrt{3}+i}, \ \ \sqrt[5]{4-4\sqrt{3}i},
 \ \ \sqrt[6]{1+i}, \sqrt[3]{-1}$$


\vspace{0.4cm}
\probl
Determinau tots els $z\in \C$ tals que $z^4+i=0$ i tots els $z\in
\C$ tals que $z^8 = 1$.


\vspace{0.4cm}
\probl
Sigui $z = 1+i$. Calculau els conjunts de valors de ${\bigl(\root
n\of {z}\bigr)}^m$ i $\root n\of {z^m}$ per als casos seg{\"u}ents:

$$\hbox{a)}\ m=4,\ n=2\qquad\hbox{b)}\ m=3,\ n=2$$

\textbf{Nota.} Siguin $m$ i $n$ dos nombres naturals i $z$ un nombre complex
distint de 0. Llavors el nombre de valors distints de
${\bigl(\root n\of {z}\bigr)}^m$ {\'e}s ${n\over d},$ on
$d=\hbox{mcd} (m,n)$.




\end{document}
