\documentclass[12pt]{report}
\usepackage[catalan]{babel}
%\usepackage[latin1]{inputenc}   % Permet usar tots els accents i car�ters llatins de forma directa.
\usepackage[utf8]{inputenc}  
\usepackage{enumerate}
\usepackage{amsfonts, amscd, amsmath, amssymb}
\usepackage[pdftex]{graphicx}

\setlength{\textwidth}{16cm}
\setlength{\textheight}{24.5cm}
\setlength{\oddsidemargin}{-0.3cm}
\setlength{\evensidemargin}{0.25cm} \addtolength{\headheight}{\baselineskip}
\addtolength{\topmargin}{-3cm}

\newcommand\Z{\mathbb{Z}}
\newcommand\R{\mathbb{R}}
\newcommand\N{\mathbb{N}}
\newcommand\Q{\mathbb{Q}}
\newcommand\K{\Bbbk}
\newcommand\C{\mathbb{C}}

\newcounter{exctr}
\newenvironment{exemple}
{ \stepcounter{exctr} 
\hspace{0.2cm} 
\textit{Exemple  \arabic{exctr}: }
\it
\begin{quotation}
}{\end{quotation}}

\pagestyle{empty}

\begin{document}

\begin{center}
\textbf{\Large Càlcul II.\\ Control 2. Curs 2011-12}
\end{center}

\vskip 0.3cm
\noindent
\textbf{P1.} Per a la següent funció, trobau la direcció en què la derivada direccional és màxima
i el valor de la derivada direccional en aquesta direcció.

\[
f(x, y, z)=\sin xy - \cos xz \quad \text{a } \quad (\pi, 1, 1)
\]

\vskip 0.3cm
\noindent
\textbf{P2.} Determinau els extrems absoluts de la funció $f(x, y)=xy(1-x^2-y^2)$ en el
quadrat $[0, 1] \times [0, 1]$.



\end{document}