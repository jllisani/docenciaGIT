\documentclass[12pt]{report}
\usepackage[catalan]{babel}
%\usepackage[latin1]{inputenc}   % Permet usar tots els accents i car�ters llatins de forma directa.
\usepackage[utf8]{inputenc}  
\usepackage{enumerate}
\usepackage{amsfonts, amscd, amsmath, amssymb}
\usepackage[pdftex]{graphicx}

\setlength{\textwidth}{16cm}
\setlength{\textheight}{24.5cm}
\setlength{\oddsidemargin}{-0.3cm}
\setlength{\evensidemargin}{0.25cm} \addtolength{\headheight}{\baselineskip}
\addtolength{\topmargin}{-3cm}

\newcommand\Z{\mathbb{Z}}
\newcommand\R{\mathbb{R}}
\newcommand\N{\mathbb{N}}
\newcommand\Q{\mathbb{Q}}
\newcommand\K{\Bbbk}
\newcommand\C{\mathbb{C}}

\newcounter{exctr}
\newenvironment{exemple}
{ \stepcounter{exctr} 
\hspace{0.2cm} 
\textit{Exemple  \arabic{exctr}: }
\it
\begin{quotation}
}{\end{quotation}}

\pagestyle{empty}

\begin{document}

\begin{center}
\textbf{\Large Càlcul II.\\ Examen Setembre. Curs 2011-12}
\end{center}

\vskip 0.3cm
\noindent
L'examen consta de 5 blocs de preguntes, cada un d'ells amb dues opcions. Heu de contestar \textbf{només una} de les opcions
en cada bloc.

\vskip 0.5cm
\begin{enumerate}
\item[\textbf{Bloc 1}]. 

\begin{enumerate}
\item[
\textbf{Opció 1.}] Determinau el conjunt de tots els $x, y \in \R$ tals que $x+iy=(x-iy)^2$
\item[
\textbf{Opció 2.}] Determinau el conjunt següent: $\{ z \in \C : 1+i-e^{z}=0 \}$
\end{enumerate}

\vskip 0.3cm

\item[\textbf{Bloc 2}]. 

\begin{enumerate}
\item[
\textbf{Opció 1.}]  Estudiau la continuïtat en el punt $(1, 2)$ de la següent funció:
\[
f(x, y)=\begin{cases} \frac{3-x-y}{3+x-2y}  & \text{si } \quad (x, y) \neq (1, 2) \\ 0 & \text{si } \quad (x, y)=(1, 2) \end{cases}
\]
\item[
\textbf{Opció 2.}] Calculau els límits iterats, els límits segons les rectes que passen per $(2, 3)$ i el límit en $(2, 3)$
de la següent funció:
$
f(x, y)=x^2y^2-2xy^5+3y
$
\end{enumerate}

\vskip 0.3cm

\item[\textbf{Bloc 3}].

\begin{enumerate}
\item[
\textbf{Opció 1.}]  Demostrau que la funció $z=y \varphi(x^2-y^2)$ satisfà l'equació:
$
\displaystyle
\frac{1}{x} \frac{\partial z}{\partial x} + \frac{1}{y} \frac{\partial z}{\partial y}=\frac{z}{y^2}
$
\item[
\textbf{Opció 2.}]  Trobau els màxims i mínims relatius de la funció:
$
\displaystyle
f(x, y)=x^4+y^4+\frac{1}{x^4y^4}
$
\end{enumerate}

\vskip 0.3cm


\item[\textbf{Bloc 4}]. 

\begin{enumerate}
\item[
\textbf{Opció 1.}]  Resoleu l'integral següent i dibuixau el recinte d'integració:
\[
\int_0^1 \int_1^{e^x} (x+y) dy dx
\]
\item[
\textbf{Opció 2.}]   Resoleu l'integral següent i dibuixau el recinte d'integració:
\[
\iint_D xy dxdy
\]
on $D=\{ (x, y) : \quad y \leq 1-\frac{1}{2}x , \quad y \leq 1+\frac{1}{2}x, \quad -1 \leq x \leq 1 \}$.
\end{enumerate}

\vskip 0.3cm


\item[\textbf{Bloc 5}].

\begin{enumerate}
\item[
\textbf{Opció 1.}] Trobau la solució general $u(x, y)$ de la següent EDP i comprovau el resultat:
\[
u_{xy}=x^2y
\]
\item[
\textbf{Opció 2.}]   Trobau la solució general $u(x, y)$ de la següent EDP i comprovau el resultat:
\[
3u_x-4u_y=x+e^x
\]
\end{enumerate}

\end{enumerate}



\end{document}