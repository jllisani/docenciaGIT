\documentclass[12pt]{article}
%\usepackage[T1]{fontenc}
%\usepackage[latin1]{inputenc}
\setlength{\textwidth}{16cm}
\setlength{\textheight}{24cm}
\setlength{\oddsidemargin}{-0.3cm}
\setlength{\topmargin}{-1.3cm}
% \setlength{\evensidemargin}{1cm}
% \addtolength{\headheight}{\baselineskip}
% \addtolength{\topmargin}{-3cm}


\newcommand{\sC}{{\cal C}}
\newcommand{\sF}{{\cal F}}
\newcommand{\sL}{{\cal L}}
\newcommand{\sU}{{\cal U}}
\newcommand{\sX}{{\cal X}}
\newcommand{\eop}{{\Box}}
% \newtheorem{corollary}{Corollary}
% \newtheorem{theorem}{Theorem}
% \newtheorem{lemma}{Lemma}
% \newtheorem{proposition}{Proposicio}
% \newtheorem{definition}{Definicio}
% \newtheorem{exercici}{}
\newcommand{\ar}{A^{(r)}}
\newcommand{\HH}{{\bf H}}
\newcommand{\sS}{{\cal S}}
\newcommand{\Img}{\mbox{Img}}

\def\N{I\!\!N}
\def\R{I\!\!R}
\def\Z{Z\!\!\!Z}
\def\Q{O\!\!\!\!Q}
\def\C{C\!\!\!\!I}

%\def\C{I\!\!C}
%\def\text#1{\hbox{\rm #1}}
%\def\inter#1{{\buildrel {\circ}\over {#1}}}
%\def\defi{\sp2J\hskip1cm {\bf Definici{\'o}n}}

\newcount\problemes
\problemes=14

\def\probl{\advance\problemes by 1
\vskip 2ex\noindent{\bf \the\problemes \hbox{ } }}


\begin{document}

\pagestyle{empty}

\parindent =0 pt
{\bf Problemes de C{\`a}lcul II. Primer Telem{\`a}tica
\hfill Bloc 2}

\vspace{0.8 cm}

\vspace{0.7cm}
\probl
Sigui $f(x,y,z)=x^2 y e^{2x} + ( x + y - z)^2$, calculau:
$$
\begin{array}{lll}
a) \; f(0,0,0) &\;\; b)\; f(1,-1,1) & \;\; c)\; f(-1,1,-1)\cr
& & \\
d) \; \displaystyle{d\over dx} f(x,x,x)\hspace{0.7cm} &\;\; e)\displaystyle\;{d\over dy} f(1,y,1)\hspace{0.7cm} &\;\;
f)\; \displaystyle{d\over dz} f(1,1,z^2)
\end{array}
$$

\vspace{0.7 cm}
\probl
Trobau el domini de les seg{\"u}ents funcions reals:

\vspace{0.3 cm}
$\begin{array}{ll} a)\;\;  f(x,y)=\ln(1+xy) & \qquad b)\;\;
f(x,y)=\displaystyle\frac{\sqrt{1-x^2}}{\sqrt[3]{1-y}}\\
 & \\

c)\;\;  f(x,y)=\displaystyle e^{x+1\over y-2} & \qquad d)\;\;  f(x,y)=\displaystyle{1\over \sqrt{9-x^2-y^2}}\\
\end{array}$


\vspace{0.7 cm}
\probl
Descriviu les corbes de nivell de les seg{\"u}ents funcions:

$$
a)\;\; f(x,y)=x^2 -y^2\hspace{1.7cm} b)\;\; f(x,y)=x^2 -y\hspace{1.7cm}  c)\;\; f(x,y)=x^2+2y^2
$$


\vspace{0.7 cm}
\probl
Trobau la superf{\'\i}cie de nivell de $\ f(x,y,z)=C\ $ per al valor de $C$ donat:

\vspace{0.3 cm}
$\begin{array}{l}
f(x,y,z)=x^2 + z^2\qquad \mbox{per a}\ \ C=1\,. \\
\end{array}$


\vspace{0.7cm}
\probl Trobau el l{\'\i}mit, si existeix, o mostrau que no existeix  el l{\'\i}mit
de les seg{\"u}ents funcions en els punts indicats:
$$
\begin{array}{ll}
 a) \; \lim\limits_{(x,y)\to (2,3)} \left( x^2y^2 - 2xy^5 + 3y\right) &\qquad\;\;
 b) \;  \displaystyle\lim\limits_{(x,y)\to (0,0)} {x^2y^3 + x^3y^2 -5 \over 2 - xy}\cr
&\\
 c) \; \displaystyle\lim\limits_{(x,y)\to (0,0)} {x-y\over x^2 + y^2} & \qquad\;\;
 d) \; \displaystyle\lim\limits_{(x,y)\to (0,0)} {8x^2y^2\over x^4 + y^4} \cr
&\\
 e) \;  \displaystyle\lim\limits_{(x,y)\to (0,0)} {xy\over \sqrt{x^2+y^2}} &\qquad\;\;
 f) \; \displaystyle\lim\limits_{(x,y)\to (0,0)} {x^2 + y^2 \over \sqrt{x^2+y^2 +1} - 1}  \cr
\end{array}
$$


\vspace{0.7cm}
\probl
Donada la funci{\'o}
$$
\ f(x,y)=\left\{
\begin{array}{cc}
\displaystyle{2xy\over 4x^2+y^2} &\hspace{0.7cm}  \hbox{$(x,y)\not=(0,0)$} \\
& \\
0 &\hspace{0.7cm}  \hbox{$(x,y)=(0,0)$} \\
\end{array}
\right.
$$

calculau, en el cas que existeixin, els l{\'\i}mits iterats i el l{\'\i}mit de la funci{\'o}.%



\vspace{0.7cm}
\probl
Utilitzau coordenades polars per calcular els seg{\"u}ents l{\'\i}mits:
$$
a) \;\; \lim\limits_{(x,y)\to (0,0)} {x^3 + y^3\over x^2 + y^2}
\qquad\qquad\qquad
b)\;\; \lim\limits_{(x,y)\to (0,0)} {(x^2 + y^2)\ln (x^2+y^2)}
$$


\vspace{0.7cm}
 \probl
Trobau $h(x,y)= g\left( f(x,y)\right)$ i el conjunt on $h$ {\'e}s cont\'{\i}nua:

$$
\begin{array}{ll}
a) \;\; g(t) = e^{-t}\cos t\hspace{1.5cm}& f(x,y)=x^4+x^2y^2 + y^4\cr
\\
b) \;\; g(t) = \displaystyle{\sqrt{t} -1\over \sqrt{t} +1}& f(x,y)=x^2 -y\cr
\\
c) \;\; g(z) =\sin z & f(x,y)= y\ln x
\end{array}
$$


\vspace{0.7 cm}
\probl
Estudiau la continu{\"\i}tat en el punt $\ (1,2) \ $ de la seg{\"u}ent funci{\'o} $f:\R^2\to
\R$:\\

$$
f(x,y)=\cases{\displaystyle{3-x-y\over 3+x-2y} & \qquad $(x,y)\not=(1,2)$\cr
 & \cr
0 & \qquad $(x,y)=(1,2)$\cr}
$$






\end{document}



\iffalse
\fi
