\documentstyle[12pt]{article}


\setlength{\textwidth}{16cm}
\setlength{\textheight}{24cm}
\setlength{\oddsidemargin}{-0.3cm}
\setlength{\topmargin}{-1.3cm}
% \setlength{\evensidemargin}{1cm}
% \addtolength{\headheight}{\baselineskip}
% \addtolength{\topmargin}{-3cm}


\newcommand{\sC}{{\cal C}}
\newcommand{\sF}{{\cal F}}
\newcommand{\sL}{{\cal L}}
\newcommand{\sU}{{\cal U}}
\newcommand{\sX}{{\cal X}}
\newcommand{\eop}{{\Box}}
% \newtheorem{corollary}{Corollary}
% \newtheorem{theorem}{Theorem}
% \newtheorem{lemma}{Lemma}
% \newtheorem{proposition}{Proposicio}
% \newtheorem{definition}{Definicio}
% \newtheorem{exercici}{}
\newcommand{\ar}{A^{(r)}}
\newcommand{\HH}{{\bf H}}
\newcommand{\sS}{{\cal S}}
\newcommand{\Img}{\mbox{Img}}

\def\N{I\!\!N}
\def\R{I\!\!R}
\def\Z{Z\!\!\!Z}
\def\Q{O\!\!\!\!Q}
\def\C{I\!\!C}
%\def\text#1{\hbox{\rm #1}}
%\def\inter#1{{\buildrel {\circ}\over {#1}}}
%\def\defi{\sp2J\hskip1cm {\bf Definici{\'o}n}}

\newcount\problemes
\problemes=38

\def\probl{\advance\problemes by 1
\vskip 2ex\noindent{\bf \the\problemes \hbox{ } }}


\begin{document}

\pagestyle{empty}

\parindent =0 pt
{\bf Problemes de C{\`a}lcul II. Primer Telem{\`a}tica
\hfill Bloc 4}

\vspace{1 cm}

\probl Calculau les seg{\"u}ents integrals iterades:

\begin{tabular}{ll}
 & \\
a) $\displaystyle\int_{-1}^1\left(\int_0^1(x^4y+y^2)\, dy\right)dx$
\hspace{2cm} &
b) $\displaystyle\int_{0}^{\frac{\pi}{2}}\left(\int_0^1(y\cos x+2)\, dy\right)dx$\\
 & \\ 
 & \\
c) $\displaystyle\int_0^1\left(\int_0^1xye^{x+y}\, dy\right)dx$  &
d) $\displaystyle\int_{-1}^{0}\left(\int_1^2-x\ln y\, dy\right)dx$\\
\end{tabular}

\vspace{0.4 cm} 
\probl Calculau les seg{\"u}ents integrals dobles on
$\;R=[0,2]\times[-1,0]$:

\begin{tabular}{ll}
 & \\
a) $\displaystyle\int_R(x^2y^2+x)\; dx\,dy$ \hspace{2cm} &
b) $\displaystyle\int_R|y|\cos {\frac{\pi}{4}x}\, dx\,dy$\\
\end{tabular}


\vspace{0.7 cm}
\probl Calculau les integrals iterades i dibuixau la
regi{\'o} $D$ determinada pels l{\'\i}mits.

\begin{tabular}{ll}
 & \\
a) $\displaystyle\int_0^1\int_0^{x^2}\; dydx$ \hspace{2cm} &
b) $\displaystyle\int_1^2\int_{2x}^{3x+1}\;dy dx$ \\
 & \\
c) $\displaystyle\int_0^1\int_1^{e^x}(x+y)\;dy dx$ &
d) $\displaystyle\int_0^1\int_{x^3}^{x^2} y\;dy dx$ \\
\end{tabular}


\vspace{0.7 cm}
\probl Calculau el volum limitat per la gr{\`a}fica de
$\;f(x,y)=1+2x+3y\,,$ el rectangle $\;R=[1,2]\times[0,1]\;$ i les
quatre cares verticals.


\vspace{0.7 cm}
\probl Calculau el volum d'un graner que t{\'e} una base
rectangular de 20 m per 40 m, i parets verticals de 4m d'altura al
costat que fa 20 m i 6 m d'altura a l'altre costat. El terrat {\'e}s
pla.


\vspace{0.7 cm}
\probl En les seg{\"u}ents integrals, canviau l'ordre
d'integraci{\'o} i calculau la integral

\vspace{0.7 cm} a) $\displaystyle\int_0^1\int_x^1 xy\;dydx$
\hspace{2.4cm} b)$\displaystyle\int_0^{\pi\over 2}\int_0^{\cos x}
\cos x \; dydx$ 


\vspace{0.7 cm} 
\probl Intentau generalitzar a integrals triples lo
que sabeu d'integrals dobles per calcular

\vspace{0.7 cm} a) $\displaystyle\int_W e^{-xy}y\;dV$  \hspace{2cm}
b) $\displaystyle\int_W e^{x+y}z\;dV\qquad$ on $\ \ \ W=[0,1]\times
[0,1]\times [0,1]$.

\vspace{0.7 cm}
\probl Determinau l'{\`a}rea d'una el.lipse amb semieixos
de longitud $a$ i $b$. 

\vspace{0.7 cm}
\probl Sigui $D$ la regi{\'o} donada com el conjunt dels
$(x,y)$ on $1\leq x^2+y^2\leq 2$ i $y\geq 0$. Calculau

$$\int_D
(1+xy)dxdy
$$ 

\vspace{0.7 cm}
\probl Trobau el volum de la regi{\'o} dins la superf{\'\i}cie
$\ z=x^2+y^2\,,\ $ entre $\ z=0\ $ i $\ z=10$. 


\vspace{0.7 cm} 
\probl Trobau el volum del s{\`o}lid fitat per les
superf{\'\i}cies $\ x^2+2y^2=2\,,\ z=0\ $ i $\ x+y+2z=2\,.$ 



\end{document}


